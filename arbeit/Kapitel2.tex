\section{Riemanns zweiter Beweis}
\label{sec:riemann2}

\begin{defi}
	\label{def:zeta}
	Die Riemannsche Zeta-Funktion $\zeta(s)$  ist definiert als
	\begin{align}
		\zeta (s) = \sum_{n\in \N} \frac{1}{n^s}
	\end{align}
\end{defi}

\begin{satz}
	\label{form:funktionalgleichung}
	\begin{align}
		 \Xi(s) = \Xi(1-s)
	\end{align}
\end{satz}


\begin{defi}
	\label{def:xi}
	\begin{align}
		\Xi(s) := \Gamma_\infty(s) \zeta(s)
	\end{align}
\end{defi}

\begin{defi}
	\label{def:gamma_infty}
	\begin{align}
		\Gamma_\infty(s) := \pi^{-s/2} \Gamma(s/2)
	\end{align}
\end{defi}