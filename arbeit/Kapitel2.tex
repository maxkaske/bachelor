\section{Formeln und Definitionen}
\label{sec:riemann2}

\begin{defi}
	\label{def:zeta}
	Die Riemannsche Zeta-Funktion $\zeta(s)$  ist definiert als
	\begin{align}
		\zeta (s) = \sum_{n\in \N} \frac{1}{n^s}
	\end{align}
\end{defi}

\begin{satz}
	\label{form:funktionalgleichung}
	\begin{align}
		 \Xi(s) = \Xi(1-s)
	\end{align}
\end{satz}


\begin{defi}
	\label{def:xi}
	\begin{align}
		\Xi(s) := \Gamma_\infty(s) \zeta(s)
	\end{align}
\end{defi}

\begin{defi}
	\label{def:gamma_infty}
	\begin{align}
		\Gamma_\infty(s) := \pi^{-s/2} \Gamma(s/2)
	\end{align}
\end{defi}

\begin{satz}[Poisson Summenformel]
	\label{satz:poisson}
	F"ur $\kinf$ und $\tinf \in \kinf^*$, $\finf$ Schwartzfunktion, $|\tinf|_\infty := |\tinf|$ Absolutbetrag und $\finft$ Fourier-transformierte von $\finf$ gilt:
	\begin{align}
		\sum_{a \in \Z} \finf (a \tinf) = \frac{1}{|\tinf|_\infty} \sum_{a \in \Z} \finft \left( \frac{a} {\tinf} \right)
	\end{align}
\end{satz}

\begin{defi}[Fouriertransformation]
	\label{def:fourier}
	\begin{align}
		\finft(\xiinf) := \int_\R{e^{ 2\pi i (-\xinf\xiinf) }\finf(\xinf)d\xinf}
	\end{align}
\end{defi}