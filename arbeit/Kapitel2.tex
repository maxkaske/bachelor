\section{Riemanns Beweis der Funktionalgleichung}
\label{sec:riemann2}
%Riemanns klassischer BEweis
%Beweis der klassischen poisson
	\begin{defi}
		\label{def:zeta}
		Die Riemannsche Zeta-Funktion $\zeta(s)$  ist für $Re(s)>1$ definiert als
		\begin{align}
			\zeta (s) = \sum_{n\in \N} \frac{1}{n^s}
		\end{align}
	\end{defi}
	Sie kann meromorph auf ganz $\C$ fortgesetzt werden und erfüllt die Funktionalgleichung.
	\begin{satz}
		\label{eq:funktionalgleichung}
		\begin{align}
			 \Xi(s) = \Xi(1-s)
		\end{align}
	\end{satz}


	\begin{defi}
		\label{def:xi}
		\begin{align}
			\Xi(s) := \Gamma_\infty(s) \zeta(s)
		\end{align}
	\end{defi}



	\begin{defi}
		\label{def:gamma_infty}
		\begin{align}
			\Gamma_\infty(s) := \pi^{-s/2} \Gamma(s/2)
		\end{align}
	\end{defi}

	\begin{satz}[Poisson Summenformel]
		\label{satz:poisson}
		F"ur $\kinf$ und $\tinf \in \kinf^*$, $\finf$ Schwartzfunktion, $|\tinf|_\infty := |\tinf|$ Absolutbetrag und $\finft$ Fourier-transformierte von $\finf$ gilt:
		\begin{align}
			\sum_{a \in \Z} \finf (a \tinf) = \frac{1}{|\tinf|_\infty} \sum_{a \in \Z} \finft \left( \frac{a} {\tinf} \right)
		\end{align}
	\end{satz}

	\begin{defi}[Fouriertransformation]
		\label{def:fourier}
		\begin{align}
			\finft(\xiinf) := \int_\R{e^{ 2\pi i (-\xinf\xiinf) }\finf(\xinf)d\xinf}
		\end{align}
	\end{defi}

	%%% Gaussfunktion Fourier %%%
	\begin{satz}
		Die (archimedische) Gaussche Funktion
		\begin{align}
			g_\infty(\xinf) := e^{-\pi |\xinf|^2}
		\end{align}
		ist ihre eigene Fouriertransformierte.
	\end{satz}

	\begin{proof}
		Die Fouriertransformation von $g_\infty(x)$ ist definiert als
		\begin{align*}
			\hat{g}_\infty (\xi) = \int_{-\infty}^{\infty}{g(x)e^{-2\pi i x \xi}dx}
		\end{align*}
		Betrachten wir zunächst den Integranden etwas genauer sehen wir, dass wir dank
		\begin{align*}
			g(x)e^{-2\pi i x \xi} = e^{-\pi(x^2 +2 i x \xi - \xi ^2)}e^{-\pi \xi^2} = e^{-pi (x + i \xi)^2} g(\xi)
		\end{align*}
		die Fouriertransformierte $\hat{g} (\xi)$ umschreiben können zu
		\begin{align*}
			\hat{g}(\xi) = g(\xi) \int_{-\infty}^{\infty} {e^{-\pi(x+i\xi)^2}dx}
		\end{align*}
		Fuer den Beweis reicht es also zu zeigen, dass das verbleibende Integral gleich $1$ ist.
		Wir berechnen zun"achst
		\begin{align*}
			g(x)e^{-2\pi i x \xi} = e^{-\pi(x^2 +2 i x \xi - \xi ^2)}e^{-\pi \xi^2} = e^{-pi (x + i \xi)^2} g(\xi)
		\end{align*}
		und stellen erfreut fest, dass die Fouriertransformierte von $g$ gerade
		\begin{align*}
			\hat{g}(\xi) = g(\xi) \int_{-\infty}^{\infty} e^{-\pi(x+i\xi)^2}dx
		\end{align*}
		ist. Es reicht also zu zeigen, dass das zweite Integral 1 ist.\\
		Sei zun"achst $\gamma$ eine Kurve entlang des Rechtecks mit den Ecken $-R$, $R$, $R+i\eta$ und $-R+i\eta$. 
		Nach dem Cauchy Integralsatz gilt f"ur unsere ganze Funktion $g(z)$
		\begin{align*}
			0 = \int_{-R}^{R} {g(z)dz} + \int_{R}^{R+i\eta} {g(z)dz}  + \int_{R+i\eta}^{-R+i\eta} {g(z)dz}  + \int_{-R+i\eta}^{-R} {g(z)dz} 
		\end{align*}
		Weiter gilt $|g(z)|=e^{-\pi (R^2 - y^2)}$ f"ur $z=\pm R + i y$ und $0\leq y \leq \eta$ und so verschwinden das zweite und vierte Integral f"ur $R\rightarrow \infty$. 
		Nach Umstellen der verbleibenden Integrale und genauen hinsehen stellen wir fest, dass
		\begin{align*}
			\int_\R{e^{-\pi (x + i\xi)^2}} = \int_\R {e^{-\pi x^2} = 1}
		\end{align*}
	\end{proof}

