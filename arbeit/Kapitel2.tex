\section{Riemanns Beweis der Funktionalgleichung}
\label{sec:riemann2}
%Riemanns klassischer BEweis
%Beweis der klassischen poisson
	In Bernhard Riemanns 1859 veröffentlichten \glqq Ueber die Anzahl der Primzahlen unter einer gegebenen Grösse\grqq
	
	\begin{defi}
		\label{def:zeta}
		Die Riemannsche Zeta-Funktion $\zeta(s)$  ist für $Re(s)>1$ definiert als
		\begin{align}
			\zeta (s) = \sum_{n\in \N} \frac{1}{n^s} 
		\end{align}
	\end{defi}
	Ausgehend von der Formel
	\begin{align*}
		\zeta (s) = = \prod_{p \text{ prim}} \frac{1}{1-p^{-s}}
	\end{align*}
	
	
	Sie kann meromorph auf ganz $\C$ fortgesetzt werden und erfüllt die Funktionalgleichung.
	%\begin{satz}
		%\label{eq:funktionalgleichung}
		%\begin{align}
			 %\Xi(s) = \Xi(1-s)
		%\end{align}
	%\end{satz}


	%\begin{defi}
		%\label{def:xi}
		%\begin{align}
			%\Xi(s) \coloneqq  \Gamma_\infty(s) \zeta(s)
		%\end{align}
	%\end{defi}



	%\begin{defi}
		%\label{def:gamma_infty}
		%\begin{align}
			%\Gamma_\infty(s) \coloneqq  \pi^{-s/2} \Gamma(s/2)
		%\end{align}
	%\end{defi}

	\begin{satz}[Poisson Summenformel]
		\label{satz:poisson}
		F"ur jede Schwartzfunktion $f:\R \to \C$ und deren Fouriertransformation $\hat{f}:\R \to \C$ gilt:
		\begin{align}
			\sum_{a \in \Z} \finf (a) = \sum_{a \in \Z} \hat{f}(a)
		\end{align}
	\end{satz}

	\begin{defi}[Klassische Fouriertransformation]
		\label{def:fourier}
		
		\begin{align}
			\finft(\xiinf) \coloneqq  \int_\R{e^{ 2\pi i (-\xinf\xiinf) }\finf(\xinf)d\xinf}
		\end{align}
	\end{defi}
	
	\begin{satz}
		Die vervollst"andigte Riemann Zeta-Funktion erf"ullt die Funktionalgleichung
		\begin{align*}
			\Gamma(\frac{s}{2}) \pi^{-\frac{s}{2}} \zeta(s) = \Gamma(\frac{s-1}{2}) \pi^{-\frac{s-1}{2}} \zeta(s-1)
		\end{align*}
	\end{satz}
	\begin{proof}[Riemanns zweiter Beweis]
		\begin{align*}
			\Gamma(\frac{s}{2}) \pi^{-\frac{s}{2}} \zeta(s)
				&= \sum_{n=1}^{\infty}  \int_{0}^{\infty} n^{-s} \pi^{-s/2} t^{s/2} e^{-t} \frac{\dx[t]}{t}
				= \sum_{n=1}^{\infty}  \int_{0}^{\infty} \frac{t}{n^2\pi}^{s/2} e^{-t} \frac{\dx[t]}{t}\\
				&= \sum_{n=1}^{\infty}  \int_{0}^{\infty} t^{s/2} e^{-\pi n^2 t} \frac{\dx[t]}{t}
				= \frac{1}{2} \int_{0}^{\infty} (\Theta(t) - 1) t^{s/2}  \frac{\dx[t]}{t} \\
		\end{align*}
		Wir teilen das Integral auf in
		\begin{align*}
			\int_{0}^{\infty} (\Theta(t) - 1) t^{s/2}  \frac{\dx[t]}{t} = \int_{0}^{1} (\Theta(t) - 1) t^{s/2}  \frac{\dx[t]}{t} + \int_{1}^{\infty} (\Theta(t) - 1) t^{s/2}  \frac{\dx[t]}{t}
		\end{align*}
		und erhalten mit der Substitution $t \mapsto 1/t$ im ersten Summanden
		\begin{align*}
			\int_{0}^{1} (\Theta(t) - 1) t^{s/2}  \frac{\dx[t]}{t} 
				&= \int_{0}^{1} \Theta(t) t^{s/2}  \frac{\dx[t]}{t} - \frac{2}{s} 
				= \int_{1}^{\infty} \Theta(1/t) t^{-s/2}  \frac{\dx[t]}{t} - \frac{2}{s}
				= \int_{1}^{\infty} t^{1/2} \Theta(t) t^{-s/2}  \frac{\dx[t]}{t} - \frac{2}{s}\\
				&= \int_{1}^{\infty} t^{1/2} \Theta(t) t^{-s/2}  \frac{\dx[t]}{t} - \frac{2}{s}
				= \int_{1}^{\infty} (\Theta(t) - 1) t^{(1-s)/2}  \frac{\dx[t]}{t} - \frac{2}{s} - \frac{2}{1-s}
		\end{align*}
		Kombinieren wir beides, so erhalten wir
		\begin{align*}
			\Gamma(\frac{s}{2}) \pi^{-\frac{s}{2}} \zeta(s)
				= \frac{1}{2} \left( \int_{1}^{\infty} (\Theta(t) - 1) t^{s/2}  \frac{\dx[t]}{t} + \int_{1}^{\infty} (\Theta(t) - 1) t^{(1-s)/2}  \frac{\dx[t]}{t} \right)  - \frac{1}{s} - \frac{1}{1-s}
		\end{align*}
		und die Funktionalgleichung ist ersichtlich.
		
	\end{proof}
	
	%%% Gaussfunktion Fourier %%%
	%\begin{satz}
		%Die (archimedische) Gaussche Funktion
		%\begin{align}
			%g_\infty(\xinf) \coloneqq  e^{-\pi |\xinf|^2}
		%\end{align}
		%ist ihre eigene Fouriertransformierte.
	%\end{satz}
%
	%\begin{proof}
		%Die Fouriertransformation von $g_\infty(x)$ ist definiert als
		%\begin{align*}
			%\hat{g}_\infty (\xi) = \int_{-\infty}^{\infty}{g(x)e^{-2\pi i x \xi}dx}
		%\end{align*}
		%Betrachten wir zunächst den Integranden etwas genauer sehen wir, dass wir dank
		%\begin{align*}
			%g(x)e^{-2\pi i x \xi} = e^{-\pi(x^2 +2 i x \xi - \xi ^2)}e^{-\pi \xi^2} = e^{-pi (x + i \xi)^2} g(\xi)
		%\end{align*}
		%die Fouriertransformierte $\hat{g} (\xi)$ umschreiben können zu
		%\begin{align*}
			%\hat{g}(\xi) = g(\xi) \int_{-\infty}^{\infty} {e^{-\pi(x+i\xi)^2}dx}
		%\end{align*}
		%Fuer den Beweis reicht es also zu zeigen, dass das verbleibende Integral gleich $1$ ist.
		%Wir berechnen zun"achst
		%\begin{align*}
			%g(x)e^{-2\pi i x \xi} = e^{-\pi(x^2 +2 i x \xi - \xi ^2)}e^{-\pi \xi^2} = e^{-pi (x + i \xi)^2} g(\xi)
		%\end{align*}
		%und stellen erfreut fest, dass die Fouriertransformierte von $g$ gerade
		%\begin{align*}
			%\hat{g}(\xi) = g(\xi) \int_{-\infty}^{\infty} e^{-\pi(x+i\xi)^2}dx
		%\end{align*}
		%ist. Es reicht also zu zeigen, dass das zweite Integral 1 ist.\\
		%Sei zun"achst $\gamma$ eine Kurve entlang des Rechtecks mit den Ecken $-R$, $R$, $R+i\eta$ und $-R+i\eta$. 
		%Nach dem Cauchy Integralsatz gilt f"ur unsere ganze Funktion $g(z)$
		%\begin{align*}
			%0 = \int_{-R}^{R} {g(z)dz} + \int_{R}^{R+i\eta} {g(z)dz}  + \int_{R+i\eta}^{-R+i\eta} {g(z)dz}  + \int_{-R+i\eta}^{-R} {g(z)dz} 
		%\end{align*}
		%Weiter gilt $|g(z)|=e^{-\pi (R^2 - y^2)}$ f"ur $z=\pm R + i y$ und $0\leq y \leq \eta$ und so verschwinden das zweite und vierte Integral f"ur $R\rightarrow \infty$. 
		%Nach Umstellen der verbleibenden Integrale und genauen hinsehen stellen wir fest, dass
		%\begin{align*}
			%\int_\R{e^{-\pi (x + i\xi)^2}} = \int_\R {e^{-\pi x^2} = 1}
		%\end{align*}
	%\end{proof}

