	\begin{abstract}
		Inspiriert durch den Blogpost \glqq \emph{Tate’s proof of the functional equation}\grqq{} \cite{taoblog} von Terence Tao werde ich die Frage beantworten, woher der Gamma-Faktor $\Gamma(s/2)$ in der Funktionalgleichung der Riemannschen Zeta-Funktion kommt und warum genau dieser ben"otigt wird.
		Dazu vollziehen wir die unter dem Namen Tate's Thesis bekanntgewordene Doktorarbeit von John Tate f"ur den Spezialfall der rationalen Zahlen $\Q$ nach. 
		Es werden zun"achst die Grundlagen lokalkompakter Gruppen und Haar-Maßen behandelt, "uberspringen jedoch die f"ur den allgemeineren Fall ben"otigte Pontryagin-Dualit"at.
		Nach einem Ausflug in die Welt der $p$-adischen Zahlen etablieren Fourieranalysis auf den lokalen K"orpern $\Q_p$ und folgen Tate beim Beweis der Funktionalgleichung der lokalen Zeta-Funktionen und geben die Berechnung der f"ur den Beweis ben"otigten Funktionen.
		Um diese Ergebnisse in einer globalen Theorie zusammenzufassen f"uhren wir das eingeschr"ankte direkte Produkt ein und definieren damit die Adele- und Idelegruppe. 
		Diese Arbeit endet mit Tates Beweis der Funktionalgleichung globaler Zeta-Funktionen.
		Wir werden feststellen, dass dieser eine Verallgemeinerung von Riemanns klassischen Beweis ist und die Funktionalgleichung der Riemannsche Zeta-Funktion als ein Spezialfall auftritt.
	\end{abstract}