	\begin{abstract}
		In seiner einzigen Arbeit im Bereich der Zahlentheorie, zeigte Riemann, dass der Ausdruck $\pi^{-s/2}\Gamma(s/2)\zeta(s)$ unverändert bleibt, wenn man $s$ mit $1-s$ vertauscht.
		Wir gehen nun der Frage nach, woher der Faktor $\pi^{-s/2}\Gamma(s/2)$ kommt und warum er so nahtlos in diese Funktionalgleichung passt.
		Dazu besprechen wir die unter dem Namen \emph{Tate's Thesis} bekannte Doktorarbeit von John Tate im Spezialfall der rationalen Zahlen $\Q$.
		Es werden zunächst die nötigen Voraussetzungen im Bereich lokalkompakter Gruppen, $p$-adischer Zahlen und harmonischer Analysis behandelt.
		Anschließend übertragen wir diese Konzepte in die globale Theorie der Adele- und Idelegruppen.
		Dies führt zu Tates Beweis der Funktionalgleichung globaler Zeta-Funktionen, welcher eine Verallgemeinerung von Riemanns klassischen Beweis darstellt, jedoch ein wesentlich klareres Bild über das Zustandekommen der Gleichung liefert.
	\end{abstract}		