\section{Der Adele- und Idelering}
	In Zukunft m"ochten wir gerne auch unsere lokalen Ergebnisse 
	\subsection{Eingeschränktes Direktes Produkt}
		Wir beginnen mit einer
		\begin{defi}[Eingeschr"ankte direkte Produkt]
			Sei $J=\{v\}$ eine Indexmenge und f"ur jedes $v \in J$ sei $G_v$ eine lokal kompakte Gruppe. 
			Sei weiter $J_\infty \subseteq J$ eine endliche Teilmenge von $J$ und f"ur jedes $v \notin J_\infty$ sei $H_v\leq G_v$ eine kompakte offene Untergruppe. 
			Das \textit{eingeschr"ankte direkte Produkt} der $G_v$ bez"uglich $H_v$ ist definiert als 
			\begin{align*}
				\rdprod{v \in J} G_v := \{ (x_v): x_v \in G_v \text{ und } x_v \in H_v \text{ f"ur alle bis auf endlich viele } v \}.
			\end{align*}
		\end{defi}
		Sei $G$ das eingeschr"ankte direkte Produkt der obigen $G_v$ bez"uglich der $H_v$. 
		
		$G$ als topologische Gruppe zu realisieren, geben wir eine Umgebungsbasis der Identit"at an. Diese soll aus den Mengen der Form $\prod_{v \in J}{N_v}$ bestehen, wobei $N_v$ eine Umgebung der Identit"at von $G_v$ ist und zus"tszlich $N_v = H_v$ f"ur fast alle $v\in J$ gelten soll. Die dadurch induzierte Topologie auf $G$ unterscheidet sich im Allgemeinen von der Produkttopologie und wird als \textit{eingeschr"ankte Produkttopologie} bezeichnet.
		
		
	\subsection{Der Adelering $\Aq$}
	\subsection{Der Idelering $\Iq$}
	