\section{Adele und Idele}\label{sec:adeleidele} 
	F"ur die lokalen Funktionalgleichungen haben wir im wesentlichen zwei Sachen ben"otigt.
	Zum einen die Fouriertransformation: ein Integral "uber die additive Gruppe, welches als Funktion auf den additiven Charakteren definiert war.
	Zum anderen die Zeta-Funktion, gewissermaßen ein multiplikatives Gegenst"uck zur Fouriertransformation: eine Funktion mitunter auf den multiplikativen Quasi-Charakteren, definiert als Integral "uber der multiplikativen Gruppe.
	Dieses Zusammenspiel von additiven und multiplikativen Strukturen ziehen wir in den n"achsten zwei Kapiteln in die globale Welt des eingeschr"ankten direkten Produkts und beginnen mit der Betrachtung zweier lokalkompakter Gruppen: die additiven Gruppe der Adele und die multiplikative Gruppe der Idele.
\subsection{Die Gruppe der Adele}
	Wir definieren die \emph{Adelegruppe} $\A$ von $\Q$ als das eingeschr"ankte direkte Produkt "uber alle lokalkompakten Gruppen $\K_p^+$ bez"uglich der Untergruppen $\Zp$, d.h.
	\begin{align*}
		\A \coloneqq \rdprod{p\leq\infty} \Kpp = \{ (x_p): x_p \in \K_p^+ \text{ und } x_p \in \Zp \text{ f"ur fast alle } p<\infty\}
	\end{align*}
	und nennen ein beliebiges Element $x\in\A$ dieser Gruppe ein \emph{Adele}.
	Es gibt eine algebraische (aber nicht topologische) Einbettung
	\begin{align*}
		\K &\longrightarrow \A \\
		x &\longmapsto (x,x,x,\dots)
	\end{align*}
	der rationalen Zahlen in die Adele.
	Diese ist tats"achlich wohldefiniert, denn f"ur fast alle Stellen $p<\infty$ ist $\abs[x]_p = 1$ und damit $x\in \Zp$. 
	Es wird also keine Verwirrung stiften, wenn wir (algebraisch) $\K$ mit dieser Einbettung identifizieren.
	Wie sieht dann aber die Topologie aus?
	\begin{satz}~
		\begin{multicols}{2}
			\begin{enumerate}[label=\emph{(\roman*)}]
				\item $\K$ liegt diskret in $\A$.
				\item $\A/\K$ ist kompakt.
			\end{enumerate}
		\end{multicols}
	\end{satz}
	\begin{proof}
		Für (i) betrachten wir die offene Nullumgebung
		\begin{align*}
			U = \left(-\frac{1}{2}, \frac{1}{2}\right) \times \prod_{p<\infty}\Z_p  .
		\end{align*}
		Ist nun $r \in \K \cap U$, so gilt $\abs[r]_p \leq 1$ für alle $p < \infty$, also $r \in \Z$. 
		Nun ist aber $\abs[r]_\infty < \frac{1}{2}$ und damit muss schon $r=0$ gelten.
		Für einen beliebigen Punkt $x \in \K$ erhalten wir mit $x+U$ eine entsprechende offene Umgebung von $x$.
		
		Für (ii) zeigen wir, dass das Bild der Menge $K\coloneqq [0,1]\times \prod_{p<\infty} \Zp$ unter der Projektion $\rho:\A \to \A/\K$ schon ganz $\A/\Q$ ist. 
		Dann ist $\A/\Q$ als stetiges Bild des Kompaktums $K$ selber kompakt. 
		Sei $x \in \A$ beliebig und $S$ die  endliche Stellenmenge $\{ p : x_p \notin \Z_p\}$.
		Wählen wir ein $p\in S$, $p<\infty$ und schreiben
		\begin{align*}
			x_p = \sum_{k=-N}^\infty a_k p^k.
		\end{align*}
		Dann ist
		\begin{align*}
			x_p - \underbrace{\sum_{k=-N}^{-1}a_k p^k}_{=:r \in \K} \in \Z_p
		\end{align*}
		und für jede weitere endliche Stelle $q\not=p$ gilt
		\begin{align*}
			\abs[r]_q =\abs[\sum_{k=-N}^{-1}a_k p^k]_q \leq \max\left\{\abs[a_k p^k]_q\right\} \leq 1,
		\end{align*}
		also ist $r \in \Z_q$. 
		Ersetzen wir nun $x$ durch $x-r$, so reduziert sich die Stellenmenge $S$ zu $S\setminus\{p\}$.
		Dieses Argument setzen wir induktiv bis $S=\{\infty\}$ fort und erhalten damit ein $x$, welches sich nur um eine Zahl aus $\K$ von dem ursprünglichen Adel unterscheidet und selbst in $\R \times \prod_{p < \infty} \Z_p$ liegt.
		Nun können wir aber noch $x$ modulo $\Z$ nach $[0,1] \times \prod_{p < \infty} \Z_p$ verschieben und sind fertig.	
	\end{proof}
	
	Unter einer \emph{einfachen Funktion} auf $\A$ wollen wir nun Funktionen der Form $f = \prod_{p\leq\infty} f_p$ mit $f_p=\ind_\Zp$ f"ur fast alle Sellen $p$ verstehen.
	Solche haben bereits im vorherigen Kapitel im Kontext der Integration kennengelernt, daher "uberrascht uns das folgende Ergebnis nicht wirklich.
	\begin{satz}\label{satz:adeleidele:intA}
		Sei $f: \A \to \Komplex$ eine einfache integrierbare Funktion auf den Adelen. 
		Es gilt die Produktformel
		\begin{align*}
			\int_\A f(x) \dx = \prod_{p\leq \infty} \int_{\Kp} f_p(x_p) \dx_p
		\end{align*}
	\end{satz}
	\begin{proof}
		Dies ist eine Folgerung aus Satz \ref{prop:rdp:integrieren}, denn f"ur einfache Funktionen ist nach unserer Normierung das Produkt auf der rechten Seite endlich.
	\end{proof}
	
	Die Gruppe der Adele erh"alt über die komponentenweise Multiplikation eine Ringstruktur.
	Diese ist nicht nullteilerfrei, denn zum Beispiel gilt f"ur zwei verschiedene Stellen $p,q\leq\infty$, dass $\iota_p(1) \cdot \iota_q(1)=0$.
	Zudem ist in $\A^\times$ versehen mit der Teilraumtopologie die Inversenbildung nicht stetig, folglich bildet $\A^\times$ bez"uglich der Multiplikation keine topologische Gruppe.
	Um dieses Problem zu umgehen wiederholen wir in der n"achsten Untersektion  die Konstruktion, jedoch diesmal mit den multiplikativen Gruppen $\Kpx$.
	
		
\subsection{Die Gruppe der Idele}
		Analog zu den Adelen definieren wir die \emph{Idelegruppe} $\I$ von $\Q$ als das eingeschr"ankte direkte Produkt "uber alle lokalkompakten Gruppen $\Kpx$ bez"uglich der Untergruppen $\Zpx$, d.h.
		\begin{align*}
			\I = \rdprod{p\leq\infty} \Kpx = \{ (x_p): x_p \in \Kpx \text{ und } x_p \in \Zpx \text{ f"ur fast alle } p<\infty\}
		\end{align*}
		und nennen ein Element $x\in\I$ dieser Gruppe ein \emph{Idel}.
		Wieder gibt es die algebraische Einbettung
		\begin{align*}
			\K^\times &\longrightarrow \I \\
			x &\longmapsto (x,x,x,\dots)
		\end{align*}
		der rationalen Einheiten in die Idele, denn wie wir bereits bei den Adelen festgestellt haben, ist f"ur fast jede Stelle $\abs[x]_p = 1$ und somit $x\in \Zpx$.
		
		Wir haben uns bereits zweimal die lokalen Betr"age auf dem eigneschr"ankten direkten Produkt angeschaut. 
		Die Definition der Idele erlaubt es uns diese miteinander zu einem globalen Absolutbetrag zu multiplizeren.
		\begin{defi}
			Wir definieren den \emph{adelischen (globalen) Absolutbetrag} $\abs_\A: \I \to \R_+^\times$ durch das wohldefinierte Produkt
			\begin{align*}
				\abs[x]_\A = \prod_{p\leq \infty} \abs[x_p]_p
			\end{align*}
		\end{defi}
		Der Begriff \glqq Betrag\grqq{} sollte hier vorsichtig behandelt werden, denn streng genommen ist $\abs_\A$ auf keinem K"orper definiert.
		Wir wollen diese Unzul"anglichkeit jedoch verzeihen, denn es lassen sich mit dieser Abbildung Aussagen aus dem Lokalen in das Globale "ubertragen.
		\begin{lemma}
			F"ur das additive Haar-Maß $\dx$ auf $\A$ und jedes Idel $\lambda \in \I$ gilt
			\begin{align*}
				\dx[{(\lambda x)}] = \abs[\lambda]_\A\cdot \dx
			\end{align*}
		\end{lemma}
		\begin{proof}
			Der Beweis funktioniert analog zu und mit  Satz \ref{satz:lokal:translationDesMasses}.
			Wir betrachten dazu wieder die kompakte Menge $K=[0,1]\times \prod_{p<\infty} \Zp$. 
			Es ist $\lambda K = [0,\lambda_\infty] \times \prod_{p<\infty} \lambda_p\Zp$ wieder kompakt. Insbesondere sind fast alle Komponenten gleich $\Zp$ und wir rechnen nach Satz \ref{satz:adeleidele:intA}
			\begin{align*}
				\int_{\lambda K} \dx 	= \int_{[0,\lambda_\infty]}\dx_\infty \cdot \prod_{p<\infty} \int_{\lambda_p\Zp} \dx_p 
										= \abs[\lambda_\infty]_\infty \int_{[0,1]}\dx_\infty \cdot \prod_{p<\infty}\abs[\lambda_p]_p \int_{\Zp} \dx_p 
										= \abs[\lambda]_\A \int_{K}\dx,
			\end{align*}
			wobei wir in der dritten Gleichung Satz \ref{satz:lokal:translationDesMasses} benutzt haben.
		\end{proof}
		Halten wir noch eine weitere wichtige Aussage "uber diesen Betrag fest.
		\begin{satz}[Artins Produktformel]
			F"ur alle $x \in \K^\times$ gilt $\abs[x]_\A = 1$.
		\end{satz}
		\begin{proof}
			Aufgrund der Multiplikativit"at von $\abs_\A$ reicht es die Aussage f"ur Primzahlen $q$ zu zeigen. 
			$q$ hat aber an nur zwei Stellen nichtrivialen Betrag und daher
			\begin{align*}
				\abs[q]_\A = \abs[q]_\infty \cdot \abs[q]_q =q \cdot q^{-1} = 1.
			\end{align*}
		\end{proof}
		Analog zu $\Zpx = \{ x \in \Kp: \abs[x]_p = 1\}$ k"onnen wir auch auf den Idelen die Untergruppe
		\begin{align*}
			\I^1 = \{ x \in \I: \abs[x]_\A = 1\}
		\end{align*}
		der \emph{Betrag-Eins Idele} betrachten. 
		Die obige Produktformel sagt uns gerade, dass $\K^\times$ eine Teilmenge von $\I^1$ ist.
		Damit haben wir die n"otigen Definitionen zusammen um den folgenden Satz zu beweisen.
		\begin{satz}\label{satz:adeleidele:ideleiso}~
			\begin{enumerate}[label=\emph{(\roman*)}]
				\item $\K^\times$ liegt diskret in $\I$.
				\item $F\coloneqq \{1\}\times \prod_{p<\infty}\Zpx $ ist ein Fundamentalbereich der Gruppenwirkung von $\K^\times$ auf $\I^1$. 
					Genauer haben wir einen Isomorphismus topologischer Gruppen $F \cong \I^1/\K^\times$.
				\item $\I^1/\K^\times$ ist kompakt.
				\item Der adelische Betrag $\abs_\A$ induziert einen Isomorphismus topologischer Gruppen $\I \cong \I^1 \times \R^\times_+$.
			\end{enumerate}
		\end{satz}
		\begin{proof}
			(i) Wir w"ahlen ein beliebiges $0<\epsilon<1$ und setzen
			\begin{align*}
				U \coloneqq  (1-\epsilon, 1+\epsilon) \times \prod_{p<\infty} \Zpx.
			\end{align*}
			Dann ist $U$ offensichtlich eine offene Umgebung der $1$ in $\I$. 
			Wir wollen nun zeigen, dass $U$ keine weitere rationale Zahl enth"alt.
			Sei dazu $r\in \Kx \cap \I$.
			Das bedeutet, dass $r \in \Zpx$, also $\abs[r]_p=1$, f"ur alle Primzahlen $p<\infty$.
			Dann muss $r$ aber schon eine ganze Zahl sein. 
			F"ur diese gilt zudem $r\in (1-\epsilon, 1+\epsilon)$.
			Folglich ist $r=1$.
			
			(ii) Wir definieren die Abbildung 
			\begin{align*}
				\eta:\quad F	&\longrightarrow \I^1/\Kx\\
						x	&\longmapsto [x]
			\end{align*}
			und behaupten, dass dies ein Isomorphismus topologischer Gruppen ist.
			Zun"achst sehen wir, dass $\eta$ als Komposition von Inklusion in $\I^1$ und Projektion in den Quotienten wieder ein stetiger Gruppenhomomorphismus ist.
			Zudem k"onnen wir direkt eine Umkehrabbildung angeben durch 
			\begin{align*}
				\eta^{-1}:\quad \I^1/\Kx	&\longrightarrow F\\
				[x] &\longmapsto x_\infty^{-1}\cdot x
			\end{align*}
			Wir vergewissern uns, dass diese Abbildung wohldefiniert ist.
			Zum einen folgt aus $x\in \I^1$ schon, dass $x_\infty \in \Kx$, denn $\abs[x_\infty]_\infty = (\prod_{p<\infty}\abs[x_p]_p)^{-1} \in \Kx$.
			Aufgrund der Eindeutigkeit der Primfaktorzerlegung von $x_\infty$ und $\abs[x]_\A = 1$ muss dann schon $\abs[x_p/x_\infty]_p =1$ folgen.
			Also kann uns nur noch die Wahl des Repräsentanten Probleme machen.
			Wir stellen aber erfreut fest, dass $rx = (rx_p)$ abgebildet wird auf $((rx_\infty)^{-1}rx_p) = (x_\infty^{-1}x_p)$ f"ur beliebige $r\in\Kx$.
			Des weiteren ist sie stetig, denn sie l"asst sich schreiben als $\eta^{-1}(x) = \iota(\pi_\infty(x))^{-1} \cdot x$. 
			Damit ist dann $\eta$ also ein topologischer Isomorphismus.
			
			(iii) Dies ist eine einfache Folgerung aus (ii), da $F$ als Produkt kompakter Untergruppen wieder kompakt ist.
			
			(iv) Der Isomorphismus von $\I$ nach $\I^1\times \R_+^\times$  wird definiert durch die Abbildung $x \mapsto (\tilde{x}, \abs[x]_\A)$, wobei 
			\begin{align*}
				\tilde{x} = (\tilde{x}_p) = 
					\begin{cases}
						x_p& \text{falls } p<\infty\\
						\frac{x_\infty}{\abs[x]_\A}& \text{sonst}.
					\end{cases}
			\end{align*}
			Um zu sehen, dass die Abbildung ein topologischer Gruppenhomomorphismus ist, schreiben wir sie als $x \mapsto (\iota_\infty(\abs[x]_\A)^{-1} \cdot x, \abs[x]_\A)$.
			Die Umkehrabbildung ist dann gegeben durch $(x,y) \mapsto \iota_\infty(y) \cdot x$, welche wiederum stetig ist und wir somit einen Isomorphismus topologischer Gruppen haben.
		\end{proof}
		%Der Fundamentalbereich $F$ wird in unserer Behandlung von Tates Beweis noch eine wichtige Rolle spielen.
		