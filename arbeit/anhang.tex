\section{Beweis des Satzes von Ostrowski}
\begin{proof}[Beweis von Satz \ref{satz:padisch:ostrowski}]
		Sei $\abs$ ein beliebiger nicht-trivialer Absolutbetrag auf $\Q$. Wir untersuchen die zwei m"oglichen Fälle.
		
		Fall 1: $\abs$ ist archimedisch.
			Sei dann $n_0\in \N$ die kleinste natürliche Zahl mit $\abs[n_0] > 1$.
			Dann gibt es ein $\alpha \in \R^+$ mit $\abs[n_0]^{\alpha}=n_0$.
			Wir wollen zeigen, dass $\abs[n]=\abs[n]_\infty^\alpha$ für alle $n \in \N$ gilt. 
			Der allgemeine Fall für $\Q$ folgt dann aus den Eigenschaften des Betrags.
			Dazu bedienen wir uns eines kleinen Tricks: Für $n \in \N$ betrachten wir die Darstellung zur Basis $n_0$, d.h.
			\begin{align*}
				n = \sum_{i=0}^{k} a_i n_0^i
			\end{align*}
			mit $a_i \in \{0,1,\dots,n_0-1\}$, $a_k \neq 0$ und $n_0^k\leq n < n_0^{k+1}$. 
			Nehmen wir davon den Absolutbetrag und beachten, dass, nach unserer Wahl von $n_0$, $\abs[a_i]\leq 1$ gilt, so erhalten wir
			\begin{align*}
				\abs[n] \leq \sum_{i=0}^{k} \abs[a_i] n_0^{i\alpha}
					\leq \sum_{i=0}^{k} n_0^{i\alpha}
					\leq n_0^{k\alpha}\sum_{i=0}^{k} n_0^{-i\alpha}
					\leq n_0^{k\alpha}\sum_{i=0}^{\infty} n_0^{-i\alpha}
					 = n_0^{k\alpha} \frac{n_0^\alpha}{n_0^\alpha - 1}.
			\end{align*}
			Setzt man $C\coloneqq \frac{n_0^\alpha}{n_0^\alpha - 1}>0$, so sehen wir
			\begin{align*}
				\abs[n]\leq C n_0^{k\alpha}\leq C n^\alpha,
			\end{align*}
			für beliebige $n \in \N$, also insbesondere auch
			\begin{align*}
				|n^N|\leq C n^{N\alpha}.
			\end{align*}
			Ziehen wir auf beiden Seiten die $N$-te Wurzel und lassen $N$ gegen $\infty$ laufen, so konvergiert $\sqrt[N]{C}$ gegen $0$ und wir erhalten 
			\begin{align*}
				\abs[n] \leq n^\alpha
			\end{align*}
			Damit wäre die erste Hälfte geschafft. Gehen wir zurück zu unserer Basisdarstellung
			\begin{align*}
				n = \sum_{i=0}^{k} a_i n_0^i.
			\end{align*}
			Da $n < n_0^{k+1}$, erhalten wir die Abschätzung
			\begin{align*}
				n_0^{(k+1)\alpha}=|n_0^{k+1}| = |n + n_0^{k+1} - n| \leq \abs[n] + |n_0^{k+1} -n|.
			\end{align*}
			Mit dem Ergebnis aus der ersten Hälfte des Beweises und $n\geq n_0^k$ sehen wir
			\begin{align*}
				\abs[n] &\geq n_0^{(k+1)\alpha} - |n_0^{k+1} -n| 
					\geq n_0^{(k+1)\alpha} - (n_0^{k+1} -n)^\alpha
					\\&\geq n_0^{(k+1)\alpha} - (n_0^{k+1} -n_0^k)^\alpha
					=n_0^{(k+1)\alpha} \left(1 - \left(1 - \frac{1}{n_0}\right)\right)
					\\&> n^\alpha \left(1 - \left(1 - \frac{1}{n_0}\right)\right).
			\end{align*}
			Setzen wir wieder $C'\coloneqq \left(1 - \left(1 - \frac{1}{n_0}\right)\right) >0$ folgt analog zum ersten Teil, dass
			\begin{align*} 
				\abs[n]\geq n^\alpha
			\end{align*}
			und daher $\abs[n]=n^\alpha$. Damit haben wir gezeigt, dass $\abs$ äquivalent zum klassischen Absolutbetrag $\abs_\infty$ ist.
			
		Fall 2: $\abs$ ist nicht archimedisch.
			Dann ist $\abs[n_0]\leq 1$ für alle $n \in \N$ und, da $\abs$ nicht-trivial ist, muss es eine kleinste Zahl $n_0$ mit $\abs[n_0]<1$ geben. 
			Insbesondere muss $n_0$ eine Primzahl sein, denn sei $p \in \N$ ein Primteiler von $n_0$, also $n_0=p \cdot n'$ mit $n' \in \N$ und $n' < n$, dann gilt, nach unserer Wahl von $n_0$,
			\begin{align*}
				|p| = |p|\cdot |n'| =|p \cdot n'| = \abs[n_0] < 1.
			\end{align*}
			Folglich muss schon $p=n_0$ gelten. 
			Ziel wird es jetzt sein, zu zeigen, dass $\abs$ äquivalent zum $p$-adischen Absolutbetrag ist.
			Zunächst finden wir ein $\alpha \in \R^+$ mit $|p| = |p|_p^{\alpha} = \frac{1}{p^{\alpha}}$. 
			Sei als nächstes $n\in \Z$ mit $p \centernot\mid n$. 
			Wir schreiben
			\begin{align*}
				n= rp + s, r \in \Z, 0<s<p.
			\end{align*}
			Nach unserer Wahl von $p=n_0$ gilt $|s|=1$ und $|rp|<1$. 
			Es folgt 
			\begin{align*}
				\abs[n]=\text{max}\{|rp|,|s|\}=1.
			\end{align*}
			Sei $n\in \Z$ beliebig. 
			Wir schreiben $n=p^{v}n'$, mit $p \centernot\mid n'$, und sehen
			\begin{align*}
				\abs[n] = |p|^{v}|n'| = |p|^v = (|p|_p^{\alpha})^{v}=\abs[n]_p^{\alpha}.
			\end{align*}
			Mit den gleichen Überlegungen aus dem ersten Fall folgt damit die Behauptung.
	\end{proof}
\section{Beweis der Poisson Summenformel nach Tate}
	Für diesen Beweis wird unser Ausblick in die abstrakte harmonische Analysis nützlich.
	\begin{proof}[Beweis von Satz \ref{satz:tateproof:poisson}]
		Jede $\K$-invariante Funktion $\phi$ auf $\A$ induziert eine Funktion auf $\A/\K$, welche wir wieder mit $\phi$ bezeichnen.
		Wir k"onnen dann die Fouriertransformation von $\phi: \A/\K \rightarrow \Komplex$ als Funktion auf $\K$ betrachten, da $\K$ versehen mit der diskreten Topologie gerade mit den Pontryagin Dualen von $\A/\K$ identifiziert werden kann\footnote{vgl. \textcite{rama} Proposition 7-15}. 
		Dazu setzen wir
		\begin{align*}
			\hat{\phi}(\xi) = \int_{\A/\K}\phi(x)\Psi(\xi \cdot x)\overline{\dx},
		\end{align*}
		wobei $\overline{\dx}$ das \emph{Quotientenmaß} auf $\A/\K$ ist, welches von dem Maß $\dx$ auf $\A$ induziert wird. 
		Dieses Haar-Maß ist charakterisiert durch
		\begin{align}\label{eq:anhang:quotmeasure}
			\int_{\A/\K}\tilde{f}(x)\overline{\dx} =
			\int_{\A/\K}\sum_{\gamma \in \K}f(\gamma+x)\overline{\dx} =
			\int_{\A} f(x)\dx,
		\end{align}
		für alle $f\in L^1(\A)$. 
		Für eine genauere Behandlung dieses Maßes verweisen wir auf \textcite{knightly} Kapitel 7.2.
		
		Für den eigentlichen Beweis ben"otigen wir zwei kleine Ergebnisse.
		\begin{lemma}\label{lemma:anhang:tate1}
			Für jede Funktion $f \in S(\A)$ gilt:
			\begin{align*}
				\hat{f}|_\K = \hat{\tilde{f}}|_\K.
			\end{align*}
		\end{lemma}
		\begin{proof}
			Sei $\xi \in \K$ beliebig aber fest. 
			Wir beobachten zun"achst, dass wir aufgrund von $\Psi|_\K =1$,
			\begin{align*}
				\Psi(\xi x)= \Psi(\xi x)\Psi(\gamma \xi)=\Psi((\gamma + x) \xi)
			\end{align*}
			für alle $\gamma \in \K$ und $x\in \A$ haben. 
			Per Definition der abstrakten Fouriertransformation gilt
			\begin{align*}
				\hat{\tilde{f}}(\xi)	
					&= \int_{\A / \K} {\tilde{f}(x)\Psi(\xi x)\overline{\dx}} 
					 = \int_{\A / \K} \left(\sum_{\gamma \in \K}{f(\gamma + x)}\right)\Psi(-\xi x)\overline{\dx} =\\
					&= \int_{\A / \K} \left(\sum_{\gamma \in \K}{f(\gamma + x)}\Psi((\gamma + x)\xi)\right)\overline{\dx}
					 = \int_{\A} f(t)\Psi(\xi x)\dx = \hat{f}(\xi),
			\end{align*}
			wobei wir im vorletzten Schritt die oben besprochene Charakterisierung \eqref{eq:anhang:quotmeasure} des Quotientenmaßes $\overline{\dx}$ ausgenutzt haben.
		\end{proof}
		
		\begin{lemma}\label{lemma:anhang:tate2}
			Für jede Funktion $f \in S(\A)$ und jedes $x\in \K$ gilt
			\begin{align*}
				\tilde{f}(x) = \sum_{\gamma \in \K} {\hat{\tilde{f}}(\gamma)\overline{\Psi}(\gamma x)}.
			\end{align*}
		\end{lemma}
		\begin{proof}
			Wie wir eben bewiesen haben gilt $\hat{f}|_\K = \hat{\tilde{f}}|_\K$ und daher
			\begin{align*}
				\left| \sum_{\gamma \in \K} {\hat{\tilde{f}}(\gamma)\overline{\Psi}(\gamma x)}\right| = 
				\left| \sum_{\gamma \in \K} {\hat{f}(\gamma)\overline{\Psi}(\gamma x)}\right| 
				\leq \sum_{\gamma \in \K} {|\hat{f}(\gamma)|}
			\end{align*}
			unter Ausnutzen der Tatsache, dass $\Psi$ ein Charakter ist. Die rechte Seite der Gleichung ist normal konvergent, da $f \in S(\A)$. 
			Analog folgt, dass auch $\sum_{\gamma \in \K} {\hat{\tilde{f}}(\gamma)}$ normal konvergiert. 
			Also $\hat{\tilde{f}} \in L^1(\K)$ und
			\begin{align*}
				\sum_{\gamma \in \K} {\hat{\tilde{f}}(\gamma)\overline{\Psi}(\gamma x)}
			\end{align*}
			ist die Fouriertransformierte\footnote{Wir erinnern uns, dass auf der diskreten Topologie das Z"ahlmaß ein Haar-Maß ist.} von $\hat{\tilde{f}}$ ausgewertet am Punkt $-x$.
			Nach Umkehrformel aus Satz \ref{satz:topogroup:umkehrformel}, erhalten wir also
			\begin{align*}
				\tilde{f}(x) = \hat{\hat{\tilde{f}}}(-x) = \sum_{\gamma \in \K} {\hat{\tilde{f}}(\gamma)\overline{\Psi}(\gamma x)}
			\end{align*}
			und damit das Lemma.
		\end{proof}
		Zurück zum Beweis der Summenformel. 
		Mit Lemma \ref{lemma:anhang:tate2} und $x=0$ und anschließendem Anwenden von Lemma \ref{lemma:anhang:tate1}, erh"alt man
		\begin{align*}
			\tilde{f}(0) = 	\sum_{\gamma \in \K} \hat{\tilde{f}}(\gamma) \bar{\Psi}(0) =
							\sum_{\gamma \in \K} \hat{\tilde{f}}(\gamma) =
							\sum_{\gamma \in \K} \hat{f}(\gamma).
		\end{align*}
		Aber per Definition gilt gerade
		\begin{align*}
			\tilde{f}(0) = \sum_{\gamma \in \K}f(\gamma),
		\end{align*}
		also
		\begin{align*}
			\sum_{\gamma \in \K}f(\gamma) = \sum_{\gamma \in \K} \hat{f}(\gamma)
		\end{align*}
		und wir sind fertig.
	\end{proof}