\section{Appendix: Tates Beweis der Poisson-Summenformel}
\begin{satz}[Poisson Summenformel]
		Sei $f \in S(\A)$. Dann gilt:
		\begin{align}
			\sum_{\gamma \in \K} {f(\gamma + x)} = \sum_{\gamma \in \K}{\hat{f}(\gamma + x)}
		\end{align}
		f"ur alle $x \in \A$.
	\end{satz}
	\begin{proof}
		Jede $\K$-invariante Funktion $\phi$ auf $\A$ induziert eine Funktion auf $\A/\K$, welche wir wieder $\phi$ nennen.
		Wir k"onnen dann die Fouriertransformation von $\phi: \A/\K \rightarrow \C$ als Funktion auf $\K$ betrachten, da $\K$ gerade die duale Gruppe von $\A/\K$ ist. Dazu setzen wir
		\begin{align*}
			\hat{\phi}(x) = \int_{\A/\K}\phi(t)\Psi(tx)\overline{dt}
		\end{align*}
		wobei $\overline{dt}$ das Quotientenma\ss auf $\A/\K$ ist, welches von dem Ma\ss $dt$ auf $\A$ induziert wird. Dieses Haarma\ss ist charakterisiert durch
		\begin{align*}
			\int_{\A/\K}\tilde{f}(t)\overline{dt} =
			\int_{\A/\K}\sum{\gamma \in \K}f(\gamma+t)\overline{dt} =
			\int_{\A} f(t)dt
		\end{align*}
		f"ur alle stetigen Funktionen $f$ auf $\A$ mit geeigneten Konvergenzeigenschaften (z.b. $f\in S(\A)$). F"ur den eigentlichen Beweis ben"otigen wir zwei
		
		\begin{lemma}
			F"ur jede Funktion $f \in S(\A)$ gilt:
			\begin{align*}
				\hat{f}|_\K = \hat{\tilde{f}}|_\K.
			\end{align*}
		\end{lemma}
		\begin{proof}
			Sei $x \in \K$ beliebig aber fest. Wir beobachten zun"achst, dass wir wegen $\Psi|_\K =1$
			\begin{align*}
				\Psi(tx)= \Psi(tx)\Psi(\gamma x)=\Psi((\gamma + t) x)
			\end{align*}
			f"ur alle $\gamma \in \K$ und $t\in \A$ haben. Per Definition der Fouriertransformation
			\begin{align*}
				\hat{\tilde{f}}(x)	&= \int_{\A / \K} {\hat{f}(t)\Psi(tx)\overline{dt}} 
									 = \int_{\A / \K} \left(\sum_{\gamma \in \K}{f(\gamma + t)}\right)\Psi(tx)\overline{dt} =\\
									&= \int_{\A / \K} \left(\sum_{\gamma \in \K}{f(\gamma + t)}\Psi((\gamma + t)x)\right)\overline{dt}
									 = \int_{\A} f(t)\Psi(tx)dt = \hat{f}(x)
			\end{align*}
			wobei wir im vorletzten Schritt die oben besprochene Charakterisierung des Quotientenmaßes $\overline{dt}$ ausgenutzt haben.
		\end{proof}
		
		\begin{lemma}
			F"ur jede Funktion $f \in S(\A)$ und jedes $x\in \K$ gilt
			\begin{align*}
				\tilde{f}(x) = \sum_{\gamma \in \K} {\hat{\tilde{f}}(\gamma)\overline{\Psi}(\gamma x)}
			\end{align*}
		\end{lemma}
		\begin{proof}
			Wie wir eben bewiesen haben gilt $\hat{f}|_\K = \hat{\tilde{f}}|_\K$ und daher
			\begin{align*}
				\left| \sum_{\gamma \in \K} {\hat{\tilde{f}}(\gamma)\overline{\Psi}(\gamma x)}\right| = 
				\left| \sum_{\gamma \in \K} {\hat{f}(\gamma)\overline{\Psi}(\gamma x)}\right| 
				\leq \sum_{\gamma \in \K} {|\hat{f}(\gamma)|}
			\end{align*}
			unter Ausnutzen der Tatsache, dass $\Psi$ unit"ar ist. Die rechte Seite der Gleichung ist also normal konvergent, da $f \in S(\A)$. Analog folgt, dass auch $\sum_{\gamma \in \K} {\hat{\tilde{f}}(\gamma)}$ normal konvergiert. Wir erinnern uns, dass das Pontryagin Duale $\widehat{\A/\K}$ als topologische Gruppe isomorph zu $\K$\footnote{Achtung: Hier ist $Q$ versehen mit der diskreten Topologie gemeint} ist. Also $\hat{\tilde{f}} \in L^1(\K)$ und
			\begin{align*}
				\sum_{\gamma \in \K} {\hat{\tilde{f}}(\gamma)\overline{\Psi}(\gamma x)}
			\end{align*}
			ist die Fouriertransformierte\footnote{Wir erinnern uns, dass in diesem Fall das Z"ahlma\ss ein Haar-Ma\ss ist} von $\hat{\tilde{f}}$ ausgewertet am Punkt $-x$. Nach Fourierinversionsformel erhalten wir also
			\begin{align*}
				\tilde{f}(x) = \hat{\hat{\tilde{f}}}(-x) = \sum_{\gamma \in \K} {\hat{\tilde{f}}(\gamma)\overline{\Psi}(\gamma x)}
			\end{align*}
			und damit das Lemma.
		\end{proof}
		Zur"uck zum Beweis der Summenformel. Wir erhalten aufgrund des zweiten Lemmas mit $x=0$ und anschlie\ss enden Anwenden des Ersten
		\begin{align*}
			\tilde{f}(0) = 	\sum_{\gamma \in \K} \hat{\tilde{f}}(\gamma) \bar{\Psi}(0) =
							\sum_{\gamma \in \K} \hat{\tilde{f}}(\gamma) =
							\sum_{\gamma \in \K} \hat{f}
		\end{align*}
		Aber per Definition gilt gerade $\tilde{f}(0) = \sum_{\gamma \in \K}f(\gamma)$, also
		\begin{align*}
			\sum_{\gamma \in \K}f(\gamma) = \sum_{\gamma \in \K} \hat{f}
		\end{align*}
		und wir sind fertig.
	\end{proof}