

\documentclass[11pt]{article} % use larger type; default would be 10pt

 

\usepackage[utf8]{inputenc} % set input encoding (not needed with XeLaTeX)
\usepackage{amsmath}
\usepackage{amssymb}
\usepackage{setspace} 
\usepackage{rotating}
\usepackage{multicol}
\usepackage{xfrac}
\usepackage[ngerman]{babel}% deutsche Trennregeln
\usepackage[T1]{fontenc}% wichtig für Trennung von Wörtern mit Umlauten




%%% PAGE DIMENSIONS
\usepackage{geometry} % to change the page dimensions
\geometry{a4paper, left=30mm, right=40mm, top=25mm, bottom=20mm} 







%%% PACKAGES
\usepackage{booktabs} % for much better looking tables
\usepackage{array} % for better arrays (eg matrices) in maths
\usepackage{paralist} % very flexible & customisable lists (eg. enumerate/itemize, etc.)
\usepackage{verbatim} % adds environment for commenting out blocks of text & for better verbatim
\usepackage[table]{xcolor}    % loads also »colortbl«
\usepackage{epstopdf}
\usepackage{natbib}
\usepackage[bottom,hang]{footmisc}
\usepackage[percent]{overpic}
\usepackage{acronym}
\usepackage{amsfonts}
\usepackage{arydshln}
\usepackage{caption}
\usepackage{fancybox}
\usepackage{graphicx}
\usepackage{longtable}
\usepackage{lscape}
\usepackage{mdwlist}
\usepackage{mhequ}
\usepackage{multirow}
\usepackage{pdfpages}
\usepackage{pst-eps}
\usepackage{pst-plot}
\usepackage{pstricks-add}
\usepackage{subfigure}
\usepackage{subfloat}
\usepackage{textcomp}
\usepackage{times}
\usepackage{wasysym}
\usepackage{wrapfig}
\usepackage{xspace}


\usepackage{tikz}
\newcommand*\circled[1]{\tikz[baseline=(char.base)]{
            \node[shape=circle,draw,inner sep=2pt] (char) {#1};}}





%%% HEADERS & FOOTERS
\usepackage{fancyhdr} % This should be set AFTER setting up the page geometry
\pagestyle{fancy} % options: empty , plain , fancy
\renewcommand{\headrulewidth}{0pt} % customise the layout...



\lhead{}\chead{}\rhead{}
\lfoot{}\cfoot{\thepage}\rfoot{}




%%%%% SECTION TITLE APPEARANCE
%\usepackage{sectsty}
%\allsectionsfont{\rmfamily\mdseries\upshape} % (See the fntguide.pdf for font help)
%%% (This matches ConTeXt defaults)






%%% ToC (table of contents) APPEARANCE
\usepackage[nottoc]{tocbibind} % Put the bibliography in the ToC
\usepackage[titles,subfigure]{tocloft} % Alter the style of the Table of Contents
\renewcommand{\cftsecfont}{\rmfamily\mdseries\upshape}
\renewcommand{\cftsecpagefont}{\rmfamily\mdseries\upshape} % No bold!

%%% END Article customizations














\begin{document}



\parindent 0pt



\pagenumbering{Roman}



%%%Deckblatt
\thispagestyle{empty}



\begin{wrapfigure}{r}{0.3\textwidth}

\includegraphics[width=0.4\textwidth]{Bilder/logos.eps}



\end{wrapfigure}
$\;$\\
Universität Augsburg\\
Mathematisch-Naturwissenschaftlich-Technische Fakultät\\
Institut für Materials Resource Management\\
Prof. Dr. Andreas Rathgeber\\
$\;$\\
$\;$\\
$\;$\\
$\;$\\
$\;$\\
$\;$\\
$\;$\\
$\;$\\
\begin{spacing}{1.5}
\begin{center}
\huge\textbf{Titel der Arbeit}\\

$\;$\\


\Large BACHELORARBEIT\\
\Large zur Erlangung des akademischen Grades\\
\Large \glqq Bachelor of Science\grqq \\
\end{center}
$\;$\\
$\;$\\

\normalsize
\begin{tabbing}
Betreut von: \quad \=[Vorname und Name (mit akad. Grad)]\\


Vorgelegt von: \>Kaske, Maximilian\\
\>1291670\\
\>Mathematik, 9. Semester\\
\>[Telefon und E-Mail-Adresse - freiwillig]\\


Abgabetermin: \>[tt.mm.jjjj]
\end{tabbing}
\end{spacing}


\clearpage

\Large \textbf{Abstract}\\
\\
\normalsize
Bitte fügen Sie hier eine Kurzzusammenfassung Ihrer Arbeit ein (max. 200 Wörter).
\newpage

\begin{spacing}{1.5}


%%% Inhaltsverzeichnis
\tableofcontents
\clearpage


%%%Abbildungsverzeichnis
\listoffigures
\clearpage


%%%Tabellenverzeichnis
\listoftables
\clearpage




%%%Abkürzungsverzeichnis

\section*{Abkürzungsverzeichnis}




\begin{acronym}[SEPSEP]



\acro{USGS}{U.S. Geological Survey}




\end{acronym}

%%%"Abkürzungsverzeichnis" ins Inhaltsverzeichnis schreiben
\addcontentsline{toc}{section}{Abkürzungsverzeichnis}%


\clearpage



\pagenumbering{arabic}   



%%%%%%%%%%%%%%%%%%%Beginn



\section{Einleitung}
\label{sec:kapitel1}

Wenden wir Poissonsummenformel auf die Gaußsche Funktion $g_\infty(x_\infty) := e^{-\pi |x_\infty|^2}$ an erhalten wir
\begin{align}
	\sum_{n \in {\Z}} g_\infty( n x_\infty ) = 
\end{align}
Nach formaler Anwendung der Mellin Transformation auf $\Theta(x)$ erhalten wir
\begin{align}
	\int_{k_\infty^\times} \Theta_\infty(x_\infty) |x_\infty|_\infty^s d^\times x_\infty &= \\
	&=\int_{k_\infty^\times} \Theta_\infty(x_\infty) |x_\infty|_\infty^{1-s} d^\times x_\infty
\end{align}
nach einen CoV von $x = \frac{1}{y}$ und $dx = -\frac{1}{y^2}dy$ unter Beachtung der Integrationsgrenzen.
\clearpage





\section{Riemanns Beweis der Funktionalgleichung}
\label{sec:riemann2}
%Riemanns klassischer BEweis
%Beweis der klassischen poisson
	\begin{defi}
		\label{def:zeta}
		Die Riemannsche Zeta-Funktion $\zeta(s)$  ist für $Re(s)>1$ definiert als
		\begin{align}
			\zeta (s) = \sum_{n\in \N} \frac{1}{n^s}
		\end{align}
	\end{defi}
	Sie kann meromorph auf ganz $\C$ fortgesetzt werden und erfüllt die Funktionalgleichung.
	\begin{satz}
		\label{eq:funktionalgleichung}
		\begin{align}
			 \Xi(s) = \Xi(1-s)
		\end{align}
	\end{satz}


	\begin{defi}
		\label{def:xi}
		\begin{align}
			\Xi(s) := \Gamma_\infty(s) \zeta(s)
		\end{align}
	\end{defi}



	\begin{defi}
		\label{def:gamma_infty}
		\begin{align}
			\Gamma_\infty(s) := \pi^{-s/2} \Gamma(s/2)
		\end{align}
	\end{defi}

	\begin{satz}[Poisson Summenformel]
		\label{satz:poisson}
		F"ur $\kinf$ und $\tinf \in \kinf^*$, $\finf$ Schwartzfunktion, $|\tinf|_\infty := |\tinf|$ Absolutbetrag und $\finft$ Fourier-transformierte von $\finf$ gilt:
		\begin{align}
			\sum_{a \in \Z} \finf (a \tinf) = \frac{1}{|\tinf|_\infty} \sum_{a \in \Z} \finft \left( \frac{a} {\tinf} \right)
		\end{align}
	\end{satz}

	\begin{defi}[Fouriertransformation]
		\label{def:fourier}
		\begin{align}
			\finft(\xiinf) := \int_\R{e^{ 2\pi i (-\xinf\xiinf) }\finf(\xinf)d\xinf}
		\end{align}
	\end{defi}

	%%% Gaussfunktion Fourier %%%
	\begin{satz}
		Die (archimedische) Gaussche Funktion
		\begin{align}
			g_\infty(\xinf) := e^{-\pi |\xinf|^2}
		\end{align}
		ist ihre eigene Fouriertransformierte.
	\end{satz}

	\begin{proof}
		Die Fouriertransformation von $g_\infty(x)$ ist definiert als
		\begin{align*}
			\hat{g}_\infty (\xi) = \int_{-\infty}^{\infty}{g(x)e^{-2\pi i x \xi}dx}
		\end{align*}
		Betrachten wir zunächst den Integranden etwas genauer sehen wir, dass wir dank
		\begin{align*}
			g(x)e^{-2\pi i x \xi} = e^{-\pi(x^2 +2 i x \xi - \xi ^2)}e^{-\pi \xi^2} = e^{-pi (x + i \xi)^2} g(\xi)
		\end{align*}
		die Fouriertransformierte $\hat{g} (\xi)$ umschreiben können zu
		\begin{align*}
			\hat{g}(\xi) = g(\xi) \int_{-\infty}^{\infty} {e^{-\pi(x+i\xi)^2}dx}
		\end{align*}
		Fuer den Beweis reicht es also zu zeigen, dass das verbleibende Integral gleich $1$ ist.
		Wir berechnen zun"achst
		\begin{align*}
			g(x)e^{-2\pi i x \xi} = e^{-\pi(x^2 +2 i x \xi - \xi ^2)}e^{-\pi \xi^2} = e^{-pi (x + i \xi)^2} g(\xi)
		\end{align*}
		und stellen erfreut fest, dass die Fouriertransformierte von $g$ gerade
		\begin{align*}
			\hat{g}(\xi) = g(\xi) \int_{-\infty}^{\infty} e^{-\pi(x+i\xi)^2}dx
		\end{align*}
		ist. Es reicht also zu zeigen, dass das zweite Integral 1 ist.\\
		Sei zun"achst $\gamma$ eine Kurve entlang des Rechtecks mit den Ecken $-R$, $R$, $R+i\eta$ und $-R+i\eta$. 
		Nach dem Cauchy Integralsatz gilt f"ur unsere ganze Funktion $g(z)$
		\begin{align*}
			0 = \int_{-R}^{R} {g(z)dz} + \int_{R}^{R+i\eta} {g(z)dz}  + \int_{R+i\eta}^{-R+i\eta} {g(z)dz}  + \int_{-R+i\eta}^{-R} {g(z)dz} 
		\end{align*}
		Weiter gilt $|g(z)|=e^{-\pi (R^2 - y^2)}$ f"ur $z=\pm R + i y$ und $0\leq y \leq \eta$ und so verschwinden das zweite und vierte Integral f"ur $R\rightarrow \infty$. 
		Nach Umstellen der verbleibenden Integrale und genauen hinsehen stellen wir fest, dass
		\begin{align*}
			\int_\R{e^{-\pi (x + i\xi)^2}} = \int_\R {e^{-\pi x^2} = 1}
		\end{align*}
	\end{proof}







\clearpage
\section{Riemanns Beweis}
\label{sec:kapitel3}

Wenden wir die Poisson-Summenformel f\" ur $\kinf$ auf die Schwartz-Funktion $\ginf(\xinf):=e^{-\pi|\xinf|^2}$ an, sehen wir, dass die Thetafunktion
\begin{align}
	\Theta_\infty (\xinf):=\sum_{n\in\Z}{\ginf (n\xinf)} = 1+ 2\sum_{n=1}^\infty{e^{-\pi n^2 |\xinf|_\infty^2}}
\end{align}
die Funktionalgleichung
\begin{align}
	\label{eq:thetafunktional}
	\Theta_\infty (\xinf) = \frac{1}{|\xinf|_\infty} \Theta_\infty(\frac{1}{\xinf})
\end{align}

f\"ur $\xinf \in \kinf^\times :=\kinf\setminus{0}$. Da insbesondere $\Theta_\infty(x_\infty)-1$ f\"ur $\xinf \rightarrow \infty$ schnell f\"allt, sehen wir, dass $\Theta_\infty(\xinf) - 1 / \xinf$ schnell f\"allt wenn $\xinf \rightarrow 0$. 
Formal k\"onnen wir die Mellin-Transformation auf \eqref{eq:thetafunktional} anwenden und folgern
\begin{align}
	\label{eq:thetamellin}
	\int_{k_\infty^\times} \Theta_\infty(x_\infty) |x_\infty|_\infty^s d^\times x_\infty = \int_{k_\infty^\times} \Theta_\infty(x_\infty) |x_\infty|_\infty^{1-s} d^\times x_\infty
\end{align}

f\"ur beliebige $s$, wobei $d^\times x_\infty := \frac{dx_\infty}{|x_\infty|_\infty}$ das standard multiplikative Haarma{\ss} auf $\kinf^\times$ ist. Dies macht streng genommen keinen Sinn, da die beiden Integranden hier bei $0$ und $\infty$ divergieren (was letztendlich auf die Pole der Riemannschen Xi Funktion bei $s=0$ und $s=1$ zur\"uckgeht). Setzen wir hier trotzdem weiter an. Verwenden wir den Trafo $y := \pi n^2 t^2$ und erinnern uns an die Formeln der Riemannschen Xi Funktion und des Gamma-Faktors, erhalten wir
\begin{align}
	 \int_{k_\infty}e^{-\pi n^2 x_\infty^2} |x_\infty|_\infty^s d^\times x_\infty = \Gamma_\infty(s) n^{-s}
\end{align}

und somit formal
\begin{align}
	\int_{k_\infty} \Theta_\infty(x_\infty) |x_\infty|_\infty^s d^\times x_\infty = \int_{k_\infty} |x_\infty|^s d^\times x_\infty + 2\Gamma_\infty(s) \zeta(s)
\end{align}

Schmei{\ss}en wir das divergente Integral $\int_{k_\infty} |x_\infty|^s d^\times x_\infty$ weg und wenden \eqref{eq:thetamellin} erhalten wir rein formal die Funktionalgleichung \eqref{eq:funktionalgleichung}. Diese Berechnungen waren nur formeller Natur.

Dieser Teil soll die Arbeit abrunden und ein kurzes Fazit liefern.
\clearpage




\bibliographystyle{plain_literaturverzeichnis}
\bibliography{literatur}
\clearpage




\appendix

\section{Anhang}
\label{sec:anhang}

\large\textbf{Anhang 1: Beschreibung}\\
\normalsize
Sie können an dieser Stelle (fortlaufend nummeriert und jeweils mit Seitenumbruch getrennt) Inhalte einfügen, die zum Verständnis der Arbeit nicht kritisch sind, jedoch Teil des Gesamtwerks sein sollen.






\clearpage





\thispagestyle{empty}
\section*{Eidesstattliche Erklärung}
Ich versichere, dass ich die vorliegende Arbeit ohne fremde Hilfe und ohne Benutzung anderer als der angegebenen Quellen angefertigt habe, und dass die Arbeit in gleicher oder ähnlicher Form noch keiner anderen Prüfungsbehörde vorgelegen hat. Alle Ausführungen der Arbeit, die wörtlich oder sinngemäß übernommen wurden, sind als solche gekennzeichnet.\\
\linebreak
\linebreak
\flushleft
Kaske, Maximilian
\flushleft
Friedberg, der \today
\clearpage

\end{spacing}





























\end{document}

