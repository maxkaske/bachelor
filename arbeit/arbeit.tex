\documentclass[12pt]{article} %

%%% Input encoding
\usepackage[utf8]{inputenc}

%%% Deutsch bitte
\usepackage[ngerman]{babel}% deutsche Trennregeln
\usepackage[T1]{fontenc}% wichtig für Trennung von Wörtern mit Umlauten
\usepackage{lmodern}

%%% Bibliografie
\usepackage{csquotes}
\usepackage[
    backend=biber,
    style=numeric,
    sortlocale=de_DE,
    natbib=true,
	isbn=false,
    url=true, 
    doi=true,
    eprint=false
]{biblatex}
\addbibresource{arbeit.bib}

%%% SICHTBARE SEITENRAENDER
%\usepackage{showframe}

%%% PAGE DIMENSIONS

\usepackage[]{appendix}
% Listen
\usepackage{multicol}
\usepackage{enumitem} 

%%%%%%%%%%%%%%%%%%
% Mathe Packages %
%%%%%%%%%%%%%%%%%%

\usepackage{amsmath}
\usepackage{amsthm}
\usepackage{amssymb}
\usepackage{amsopn}
\usepackage{mathtools}
\usepackage{bbm}
\usepackage{centernot}

%footnotes
\usepackage[perpage]{footmisc}
\renewcommand{\thefootnote}{\fnsymbol{footnote}}


%Links


\usepackage{hyperref}
\hypersetup{
	colorlinks,
	citecolor=black,
	filecolor=black,
	linkcolor=black,
	urlcolor=black,
	plainpages=false,
}
\usepackage{bookmark}

\usepackage{geometry} % to change the page dimensions
\geometry{a4paper, left=30mm, right=30mm, top=25mm, bottom=25mm}

%%%%%%%%%%%%%%%%%%%%%%%%%
% Mathematische Symbole %
%%%%%%%%%%%%%%%%%%%%%%%%%

%Topologie
\newcommand{\Borel}{\mathcal{B}}

%masstheorie
\newcommand{\ind}{\mathbbm{1}}

\newcommand{\N}{\mathbb{N}}
\newcommand{\Z}{\mathbb{Z}}
\newcommand{\Zp}{{\Z_p}}
\newcommand{\Zpx}{\Z_p^\times}
\newcommand{\R}{\mathbb{R}}
\newcommand{\Komplex}{\mathbb{C}}
\newcommand{\Q}{\mathbb{Q}}
\newcommand{\A}{\mathbb{A}}
\newcommand{\I}{\mathbb{I}}

\newcommand{\Kpx}{\K_p^\times}
\newcommand{\Kpp}{\K_p^+}
\newcommand{\Sw}{S}
\newcommand{\Aq}{\A_\Q}
\newcommand{\Ak}{\A_\K}
\newcommand{\Iq}{\I_\Q}
\newcommand{\K}{\Q}
\newcommand{\Kx}{\K^\times}
\newcommand{\Kinf}{\K_\infty}
\newcommand{\Kinfx}{\K_\infty^\times}
\newcommand{\Kp}{{\K_p}}
\newcommand{\ginf}{g_\infty}
\newcommand{\tinf}{t_\infty}
\newcommand{\finf}{f_\infty}
\newcommand{\xinf}{x_\infty}
\newcommand{\xiinf}{\xi_\infty}
\newcommand{\finft}{\hat{f}_\infty}
\newcommand{\dedekind}{\mathcal{O}}%{\ensuremath{\scriptstyle\mathcal{O}}}
\newcommand{\sgn}{\text{sgn}}
\newcommand*\conj[1]{\overline{#1}}


%%%%%Integration

\newcommand{\dxx}[1][x]{d^\times {#1}}
\newcommand{\dxs}[1][x]{d^* {#1}}
\newcommand{\dxsp}[1][x]{d^* {#1}_p}
\newcommand{\dxxp}[1][x]{d^\times {#1}_p}
\newcommand{\dxxinfty}[1][x]{d^\times {#1}_\infty}
\newcommand{\dxp}[1][x]{d{#1}_p}
\newcommand{\dxinfty}[1][x]{d{#1}_\infty}
\newcommand{\dx}[1][x]{d{#1}}
\DeclareMathOperator{\Vol}{Vol}

%%% Eingeschraenkte direkte Produkt
\newcommand{\rdprod}[1]{\prod\limits_{#1}'}{%{\widehat{\prod\limits_{#1}}}

%%% BEtraege und norm
\DeclarePairedDelimiter\abso{\lvert}{\rvert}
\DeclarePairedDelimiter{\norm}{\lVert}{\rVert}

% Swap the definition of \abs* and \norm*, so that \abs
% and \norm resizes the size of the brackets, and the 
% starred version does not.
\makeatletter
\let\oldabs\abso
\def\abso{\@ifstar{\oldabs}{\oldabs*}}

\newcommand{\abs}[1][\,\cdot\,]{\abso{#1}}

%%%%%%%%%%%%
% Theoreme %
%%%%%%%%%%%%

\theoremstyle{definition}
\newtheorem{defi}{Definition}[section]
\newtheorem{bsp}[defi]{Beispiele}

\theoremstyle{plain}
\newtheorem{satz}[defi]{Satz}
\newtheorem{proposition}[defi]{Satz}
\newtheorem{lemma}[defi]{Lemma}
\newtheorem{korollar}[defi]{Korollar}


\begin{document}

%%%Deckblatt

\pagenumbering{Roman}
\begin{titlepage}
    \begin{center}
		\Large
        \textbf{Universit"at Augsburg}\\
		Mathematisch-Naturwissenschaftlich-Technische Fakultät\\
		Lehrstuhl für Algebra und Zahlentheorie\\
        \vspace{2cm}
		\includegraphics[width=0.3\textwidth]{siegel}\\
        \vspace{2cm}
		\LARGE
		\textbf{Tates Beweis der Funktionalgleichung der Riemannschen Zeta-Funktion}\\
        \vspace{2cm}
		\Large
        \textbf{Bachelorarbeit}\\[.15cm]
		\large
		zur Erlangung des akademischen Grades\\[.15cm]
		Bachelor of Science\\
		\vspace{3cm}
		\normalfont
		\normalsize
		\begin{tabular}{l l}
		
			Eingereicht von: & Maximilian Alexander Kaske \\ 
		
			Matrikelnummer: & 1291670 \\ 
		
			Studiengang: & Mathematik (B.Sc.) \\ 
		
			Betreut von:&  Prof. Dr. Marc Nieper-Wißkirchen\\ 
			%
			%2. Pr\"ufer &  Prof. Dr. Marco Hien\\ 
			
			Abgegeben am: & 20. September 2017\\ 
		
		\end{tabular} \\
    \end{center}
		%Sommersemester 2017\\
		%Matrikelnummer: 1291670\\
		%Adresse: Hagelm"uhlweg 14, 86316 Friedberg\\
		%E-Mail: maximilian.kaske@student.uni-augsburg.de\\
		%Abgabetermin: 20.09.2017
\end{titlepage}
\clearpage

%%% Zusammenfassung und Inhaltsverzeichnis
\pagenumbering{roman}
\setcounter{page}{2}
	\begin{abstract}
		Inspiriert durch den Blogpost \glqq \emph{Tate’s proof of the functional equation}\grqq{} \cite{taoblog} von Terence Tao werde ich die Frage beantworten, woher der Gamma-Faktor $\Gamma(s/2)$ in der Funktionalgleichung der Riemannschen Zeta-Funktion kommt und warum genau dieser ben"otigt wird.
		Dazu vollziehen wir die unter dem Namen Tate's Thesis bekanntgewordene Doktorarbeit von John Tate f"ur den Spezialfall der rationalen Zahlen $\Q$ nach. 
		Es werden zun"achst die Grundlagen lokalkompakter Gruppen und Haar-Maße behandelt, "uberspringen jedoch die f"ur den allgemeineren Fall ben"otigte Pontryagin-Dualit"at.
		Nach einem Ausflug in die Welt der $p$-adischen Zahlen etablieren Fourieranalysis auf den lokalen K"orpern $\Q_p$ und folgen Tate beim Beweis der Funktionalgleichung der lokalen Zeta-Funktionen und geben die Berechnung der f"ur den Beweis ben"otigten Funktionen.
		Um diese Ergebnisse in einer globalen Theorie zusammenzufassen f"uhren wir das eingeschr"ankte direkte Produkt ein und definieren damit die Adele- und Idelegruppe. 
		Diese Arbeit endet mit Tates Beweis der Funktionalgleichung globaler Zeta-Funktionen.
		Wir werden feststellen, dass dieser eine Verallgemeinerung von Riemanns klassischen Beweis ist und die Funktionalgleichung der Riemannsche Zeta-Funktion als ein Spezialfall auftritt.
	\end{abstract}
\clearpage
\tableofcontents
\clearpage

%%% Hauptteil
\pagenumbering{arabic}


\section{Von Riemann zu Tate}
	%\subsection{Historische Kontext}
	Im Jahr 1859 erschien mit \emph{\glqq Ueber die Anzahl der Primzahlen unter einer gegebenen Grösse\grqq} \cite{riemann1859ueber} Bernhard Riemanns erste und einzige ver"offentlichte Arbeit im Bereich der Zahlentheorie.
	Als Startpunkt nimmt Riemann die heute als \emph{Riemannsche Zeta-Funktion}\footnote{Obwohl diese Funktion heute nach Riemann benannt ist, war es Leonhard Euler der sich zuerst n"aher mit ihr besch"aftigte } $\zeta(s)$ bekannte Abbildung, welche f"ur $\Re(s)>1$ durch
	\begin{align*}
		\sum_{n\in \N} \frac{1}{n^s} = \prod_{p \text{ prim}} \frac{1}{1-p^{-s}}
	\end{align*}
	als absolut konvergente Reihe oder durch die \emph{Euler-Produktformel} dargestellt werden kann.
	Neben einer Vielzahl neuer Notationen, Definitionen und Ideen, darunter die bekannte \emph{Riemannsche Vermutung}, dass alle nicht-trivialen Nullstellen von $\zeta(s)$ den Realteil $1/2$ haben, gab er auch zwei Beweise der Funktionalgleichung
	\begin{align*}
		\Xi(s) = \Xi(1-s),
	\end{align*}
	wobei $\Xi(s) = \Gamma(s/2)\pi^{-s/2}\zeta(s)$ und $\Gamma(s)$\footnote{Riemann selbst benutze noch die durch Gauß definierte Notation $\Pi(s-1)=\Gamma(s)$} das bekannte Euler-Integral
	\begin{align*}
		\Gamma(s) = \int_0^\infty e^{-x} x^s \frac{\dx}{x}
	\end{align*}
	ist.
	%Wir m"ochten in dieser Arbeit zum e
	%und unter anderem die obige \emph{Euler-Produktformel} bewies.
	%Davon ausgehend
	%holomorphe Fortsetzung der Funktion,
	%befinden sich auch zwei Beweise der Funktionalgleichung
	%\begin{align*}
		%\Gamma\left(\frac{s}{2}\right)\pi^{-s/2}\zeta(s) = \Gamma\left(\frac{1-s}{2}\right)\pi^{-(1-s)/2}\zeta(1-s),
	%\end{align*}
	%wobei 
	%\begin{align*}
		%\Gamma(s) = \int_0^\infty e^{-x} x^s \frac{\dx}{x}
	%\end{align*}
	%die bekannte Gamma-Funktion in ihrer Integraldarstellung nach Euler ist.
	%die \emph{Riemann Hypothese}
	
\subsection{Der Beweis der Funktionalgleichung}
	In seinem ersten Beweis erh"alt Riemann zun"achst durch Anwendung von
	\begin{align}\label{eq:riemannstart}
		\Gamma(s)\frac{1}{n^s} = \int_0^\infty e^{-nx}x^{s}\frac{\dx}{x},
	\end{align}
	die Gleichung
	\begin{align*}
		\Gamma(s)\zeta(s) = \int_0^\infty \frac{x^{s}}{e^x-1} \frac{\dx}{x}.
	\end{align*}
	"Uber Wegintegration des Integrals
	\begin{align*}
		\int \frac{(-x)^{s-1}}{e^x-1}\dx
	\end{align*}
	etablierte er anschließend die meromorphe Fortsetzung der Zeta-Funktion (auch wenn nicht im Sinne von Weierstrass) und etablierte seine erste Form der Funktionalgleichung
	\begin{align*}
		\zeta(s) = \Gamma(s)(2\pi)^{s-1} 2 \sin(s\pi/2) \zeta(1-s).
	\end{align*}
	Riemann benutzte nun gel"aufige Identit"aten der Gamma-Funktion um dieses Ergebnis umzuformen:
	Der Ausdruck
	\begin{align*}
		\Xi(s) = \Gamma(s/2)\pi^{-s/2}\zeta(s)
	\end{align*}
	bleibt unver"andert, wie Riemann schreibt, \glqq wenn $s$ in $s-1$ verwandelt wird.\grqq{}
	
	Diese symmetrische Darstellung der Funktionalgleichung veranlasste Riemann nun dazu in Gleichung \eqref{eq:riemannstart} den Ausdruck $\Gamma(s/2)$ anstatt $\Gamma(s)$ f"ur die Grundlage eines weiteren Beweises zu betrachten.
	Diesen m"ochten wir uns im folgenden etwas genauer anschauen, wobei wir uns Konvergenzgedanken ganz im Geiste Riemanns aufsparen.
	\begin{satz}
		Die Riemannsche Zeta-Funktion $\zeta(s)$  kann zu einer meromorphen Funktion auf ganz $\Komplex$ fortgesetzt werden, welche die \emph{Funktionalgleichung}
		\begin{align*}
			\Gamma\left(\frac{s}{2}\right)\pi^{-s/2}\zeta(s) = \Gamma\left(\frac{1-s}{2}\right)\pi^{-(1-s)/2}\zeta(1-s)
		\end{align*}
		erf"ullt. Sie besitzt zwei einfache Pole bei $s=0$ und $s=1$.
	\end{satz}
	\begin{proof}[Beweis der Funktionalgleichung]
		Durch den Variablenwechsel $y=n^2\pi t^2$ in Eulers Integraldarstellung der Gamma-Funktion erhalten wir f"ur $\Re(s)>1$
		\begin{align*}
			%\int_{0}^{\infty} e^{-n^2\pi t}t^{s/2} \frac{\dx[t]}{t} 
				%= \int_0^{\infty} e^{-y} \pi^{-s/2}n^{-s} y^{s/2} {\dx[t][y]}{y} 
				%= \frac{1}{n^s} \pi^{-s/2}\Gamma\left(\frac{s}{2}\right)
			\frac{1}{n^s} \pi^{-s/2}\Gamma\left(\frac{s}{2}\right) 
				= \frac{1}{n^s} \pi^{-s/2} \int_0^{\infty} e^{-n^2\pi t} n^s\pi^{s/2} t^{s/2} \frac{\dx[t]}{t} 
				= \int_{0}^{\infty} e^{-n^2\pi t}t^{s/2} \frac{\dx[t]}{t}.
		\end{align*}
		Anschließendes aufsummieren und ausnutzen von Fubinis Integralgleichung ergibt die Formel
		\begin{align*}
			\Gamma\left(\frac{s}{2}\right)\pi^{-s/2}\zeta(s) 
				= \int_{0}^{\infty} \sum_{n=1}^{\infty}\left(e^{-n^2\pi t}\right)t^{s/2} \frac{\dx[t]}{t}
		\end{align*}
		Die rechte Seite entspricht gerade der $\Xi$-Funktion. Wir f"uhren nun die \emph{Thetafunktion}
		\begin{align*}
			\Theta(t) = \sum_{n\in \Z} e^{-\pi t n^2} = 1 + 2 \sum_{n\in\N}e^{-\pi t n^2}
		\end{align*}
		ein.
		Damit k"onnen wir obige Gleichung etwas vereinfacht darstellen mit
		\begin{align*}
			\Xi(s) 
				= \frac{1}{2}\int_{0}^{\infty}(\Theta(t)-1)t^{s/2} \frac{\dx[t]}{t}.
		\end{align*}
		%\begin{align*}
				%\Gamma(\frac{s}{2}) \pi^{-\frac{s}{2}} \zeta(s)
					%&= \sum_{n=1}^{\infty}  \int_{0}^{\infty} n^{-s} \pi^{-s/2} t^{s/2} e^{-t} \frac{\dx[t]}{t}
					%= \sum_{n=1}^{\infty}  \int_{0}^{\infty} \frac{t}{n^2\pi}^{s/2} e^{-t} \frac{\dx[t]}{t}\\
					%&= \sum_{n=1}^{\infty}  \int_{0}^{\infty} t^{s/2} e^{-\pi n^2 t} \frac{\dx[t]}{t}
					%= \frac{1}{2} \int_{0}^{\infty} (\Theta(t) - 1) t^{s/2}  \frac{\dx[t]}{t} \\
		%\end{align*}
		Teilen wir nun das Integral auf in
		\begin{align}
			\int_{0}^{\infty} (\Theta(t) - 1) t^{s/2}  \frac{\dx[t]}{t} = \int_{0}^{1} (\Theta(t) - 1) t^{s/2}  \frac{\dx[t]}{t} + \int_{1}^{\infty} (\Theta(t) - 1) t^{s/2}  \frac{\dx[t]}{t}.\label{eq:einleitung:riemann1}
		\end{align}
		%Wir m"ochten nun
		%und erhalten mit der Substitution $t \mapsto 1/t$ im ersten Summanden
		%\begin{align*}
			%\int_{0}^{1} (\Theta(t) - 1) t^{s/2}  \frac{\dx[t]}{t} 
				%&= \int_{0}^{1} \Theta(t) t^{s/2}  \frac{\dx[t]}{t} - \frac{2}{s}.
		%\end{align*}
		Wie ben"otigen nun die \emph{Theta-Transformationsformel}
		\begin{align}
			\Theta(t) = t^{-1/2} \Theta(1/t).\label{eq:einleitung:thetatrafo}
		\end{align}
		Um deren G"ultigkeit einzusehen ben"otigen man Zweierlei: 
		Zuerst erinnern wir uns an die \emph{klassische Fouriertransformation} $\hat{f}:\R\to\Komplex$ einer $L^1$-Funktion $f:\R\to\Komplex$ definiert durch
		\begin{align*}
			\hat{f}(\xi) = \int_\R f(x)e^{-2\pi i x \xi} \dx.
		\end{align*}
		Mit dieser Abbildung haben wir folgenden
		\begin{satz}[Klassische Poisson Summenformel]
			\label{satz:einleitung:poisson}
			F"ur jedes geeignete $f:\R \to \Komplex$ und deren Fouriertransformation $\hat{f}:\R \to \Komplex$ gilt:
			\begin{align}
				\sum_{a \in \Z} f(a) = \sum_{a \in \Z} \hat{f}(a) \label{eq:einleitung:poisson}
			\end{align}
		\end{satz}
		\begin{proof}
			In dieser klassischen Form zum Beispiel in Deitmar \cite{deitmar2010} Proposition 5.4.10.
		\end{proof}
		Als Zweites halten wir fest, dass die \emph{Fouriertransformierte} von $f_t(x) = e^{-\pi t x^2}$ gleich $\hat{f}_t(x)= t^{-1/2}f_{\frac{1}{t}}(x)$ ist.
		Einsetzen in die Poisson Summenformel \eqref{eq:einleitung:poisson} ergibt dann die Transformationsformel \eqref{eq:einleitung:thetatrafo}.
		Mit dieser kann man nun den ersten Summanden in Gleichung \eqref{eq:einleitung:riemann1} umformen zu
		\begin{align*}		
			\int_{0}^{1} \left(t^{-1/2}\Theta(1/t)-1\right) t^{s/2}  \frac{\dx[t]}{t} 
				= \int_{0}^{1}\Theta(1/t) t^{(s-1)/2}  \frac{\dx[t]}{t} - \frac{2}{s}
		\end{align*}
		Anschließend nutzen wir die Substitution $t \mapsto t^{-1}$ und rechnen
		\begin{align*}
			\int_{0}^{1}\Theta(1/t) t^{(s-1)/2}  \frac{\dx[t]}{t} - \frac{2}{s}
				&= \int_{1}^{\infty} \Theta(t) t^{(1-s)/2}  \frac{\dx[t]}{t} - \frac{2}{s}\\
				%&= \int_{1}^{\infty}  (\Theta(t) - 1) t^{(1-s)/2} + t^{(1-s)/2}  \frac{\dx[t]}{t} - \frac{2}{s}\\
				&= \int_{1}^{\infty} (\Theta(t) - 1) t^{(1-s)/2}  \frac{\dx[t]}{t} - \frac{2}{s} - \frac{2}{1-s}.
		\end{align*}
		Einsetzen in Gleichung von $\Xi(s)$ ergibt dann
		\begin{align*}
			\Xi(s)
				= \frac{1}{2} \left( \int_{1}^{\infty} (\Theta(t) - 1) t^{s/2}  \frac{\dx[t]}{t} + \int_{1}^{\infty} (\Theta(t) - 1) t^{(1-s)/2}  \frac{\dx[t]}{t} \right)  - \frac{1}{s} - \frac{1}{1-s}
		\end{align*}
		und wir sehen, dass dieser Ausdruck unver"andert bleibt wenn wir $s$ in $1-s$ verwandeln.
	\end{proof}

\subsection{Auf dem Weg zu Tate}
	Beide Beweise haben eine kleine Schw"ache: Sie starten bereits mit der Gamma-Funktion als einen notwendigen Faktor zur Bildung der Funktionalgleichung.
	Warum sie aber gerade so nahtlos zur Zeta-Funktion passt wird nicht ersichtlich. 
	Auftritt John Tate.
	
	Tate kommt aus einer langen Linie von Mathematikern deren Arbeit mehr oder weniger direkt durch Riemanns Ideen in  \emph{Ueber die Anzahl\dots} beeinflusst wurde.
	Unter der Aufsicht Emil Artins verfasste er 1950, fast 100 Jahre nach Riemann, seine Doktorarbeit \glqq Fourier Analysis in Number Fields and Hecke's Zeta-Functions\grqq{} \cite{tate}. Sie ist zum Beispiel in \cite{cassels1967algebraic} zu finden.
	In ihr bewies er die analytische Fortsetzung und Funktionalgleichung der Dedekind Zeta-Funktionen und Hecke L-Funktionen, eine Art Verallgemeinerung der Riemannschen Zeta-Funktion.
	Dieses Ergebnis war keineswegs neu und wurde 30 Jahre fr"uher bereits durch Erich Hecke gezeigt.
	Was Tates Doktorarbeit jedoch so besonders macht - und einer der Gr"unde daf"ur warum sie als  \glqq Tate's Thesis\grqq{} gewissen Kultstatus erreicht hat - ist die elegante Herangehensweise an Heckes Problemstellung in der globalen Sprache der \emph{Adele und Idele}.
	Sogar noch verbl"uffender: Ist Tates theoretischer Rahmen erstmal etabliert, so stimmen die einzelnen Beweisschritte gr"oßtenteils mit Riemanns klassischen zweiten Beweis überein.
	Es ergibt sich aber ein viel klareres Bild über das Zustandekommen der einzelnen Bestandteile der Funktionalgleichung.
	
	Ziel dieser Arbeit wird es nun sein Tates Doktorarbeit in ihrer einfachsten Form, d.h. im Fall des algebraischen K"orper $\Q$, Revue passieren zu lassen um nun die zentrale Frage
	\begin{quote}
		%\begin{flushright}
		\centering
		\textit{Woher kommt die Gamma-Funktion in der Funktionalgleichung der Riemannschen Zeta-Funktion?}
		%\end{flushright}
	\end{quote}
	zu beantworten. 
	%Wir werden daher Schritt f"ur Schritt die algebraischen, analytischen und topologischen Grundlagen f"ur das Verst"andnis von Tates Beweis erarbeiten.
	
	Wir beginnen dazu in Kapitel \ref{sec:topogroup} mit einer Einf"uhrung zu topologischen Gruppen, Lokalkompaktheit und Haar-Maße.
	Um den Rahmen dieser Arbeit nicht unn"otigen zu zerren werden wir allerdings Pontryagin Dualit"at und abstrakte Fourieranalyis - beides wichtige Grundlagen f"ur Tates Beweis in h"ohrer Allgemeinheit - nur in einem kurzen Ausblick behandeln.
	In Kapitel \ref{sec:padisch} wiederholen wir kurz die wichtigsten Begriffe und Eigenschaften zu Absolutbetr"agen. 
	Anschließend f"uhren wir die $p$-adischen Zahlen ein und untersuchen deren Eigenheiten.
	Um unser Auslassen der abstrakten Fourieranalysis wieder gut zu machen, definieren wir am Anfang von Kapitel \ref{sec:lokal} die Fouriertransformation auf den $p$-adischen Zahlen neu und zeigen, dass diese Definition der abstrakten entspricht. 
	Damit haben wir dann genug Grundlagen gesammelt um Tates erstes Ergebnis, die Funktionalgleichung lokaler Zeta-Funktion, zu beweisen.
	Wir schließen das Kapitel und die erste H"alfte der Arbeit mit der expliziten Berechnung f"ur den Beweis wichtiger Integrale.
	Es geht weiter mit der globale Betrachtung der Problemstellung in Kapitel \ref{sec:rdp} mit der wichtigen Theorie des eingeschr"ankten direkten Produkts.
	Wir stellen uns die Frage wie man auf dieser abstrakten Gruppe integriert und wie Charaktere aussehen.
	Daraufhin lernen wir in Kapitel \ref{sec:adeleidele} mit der Adele- und Idelegruppe $\A$ und $\I$ zwei konkrete Beispiele kennen.
	Kapitel \ref{sec:tateproof} behandelt nun die Fourieranalysis im globalen Kontext der Adele und Idele, beweisen die algebraische Variante des Satzes von Riemann-Roch und geben Tates vollen Beweis der Funktionalgleichung globaler Zeta-Funktion. 
	Zum Schluss runden wir die Arbeit ab und besprechen wie aus diesem Ergebnis direkt die Funktionalgleichung der Riemannschen Zeta-Funktion folgt.
	
	
	
	
	%Trotz der relativen K"urze von nur 10 Seiten beeinflussten Riemanns Ideen in \emph{Ueber die Anzahl\dots} die Arbeit vieler bedeutender Mathematiker, darunter Hadamard, von Mangoldt, de la Vallée Poussin, Landau, Hardy, Littlewood, Siegel, Polya, Jensen, Lindelöf, Bohr, Selberg, Artin und Hecke.
	%"Uber die letzten beiden kommen wir schließlich zu John Tate. 
	%Unter der Aufsicht Artins verfasste dieser 1950 seine Doktorarbeit\cite{tate} \glqq Fourier Analysis in Number Fields and Hecke's Zeta-Functions\grqq{}.
	%Das Ergebnis, die Funktionalgleichungen gewisser \glqq Hecke\grqq{} Zeta-Funktionen, wurde, wie der Name es vielleicht erahnen l"asst, bereits 30 Jahre fr"uher von Hecke gezeigt.
	%Tates Arbeit bietet dagegen einen elegante Umformulierung der Problemstellung von Hecke in einen m"achtigeren Kontext und erreichte damit als \glqq Tate's Thesis\grqq{}gewissen Kultstatus.
	



\clearpage

\section{Lokalkompakte Gruppen und harmonische Analysis}\label{sec:topogroup}
	Auch wenn es der Beweis Riemanns nicht direkt vermuten l"asst, die Fouriertransformation von Funktionen wird eine entscheidende Rolle im Beweis der Funktionalgleichung spielen.
	Wir werfen also zun"achst einen Blick auf die (abstrakte) harmonische Analysis, eine Verallgemeinerung der klassischen Fourieranalysis auf $\R$.
	Im Mittelpunkt stehen hier die lokalkompakten Gruppen.
	Auf diesen l"asst sich sehr nat"urlich ein zum klassischen Lebesgue-Maß analoges Maß definieren, wodurch wir Integrieren und auch Fourier-transformieren k"onnen.
	Wir werden in dieser Arbeit jedoch nicht den vollen Weg zu abstrakten Fourieranalysis gehen.
	Stattdessen definieren wir in Kapitel \ref{sec:lokal} eine eigene Fouriertransformation und geben hier im letzten Abschnitt nur einen kurzen Ausblick.
	F"ur die Behandlung des Stoffes halten wir uns dabei an Ramakrishnan und Valenza \cite{rama} Kapitel 1 und 3.
\subsection{Lokalkompakte Gruppen}
%topologische Gruppe: done
%charaktere:partly
%haar masse: done
%fouriertransformation:done
%pontryagin erwaehnen: done
	Am Anfang steht immer eine
	\begin{defi}
		Eine \emph{topologische Gruppe} ist eine (nicht unbedingt abelsche) Gruppe $G$ zusammen mit einer Topologie, welche die folgenden Eigenschaften erf"ullt:
			\begin{enumerate}[label=(\roman*)] % leftmargin=*, align=left, labelsep=1pt]
				\item Die Gruppenoperation
					\begin{align*}
						G \times G &\longrightarrow G\\
						(g,h) &\longmapsto gh
					\end{align*}
				stetig auf der Produkttopologie von $G \times G$.
				\item Die Umkehrabbildung
					\begin{align*}
						G &\longrightarrow G\\
						g &\longmapsto g^{-1}
					\end{align*}
					ist stetig auf G.
			\end{enumerate}
		Ein \emph{Homomorphismus topologischer Gruppen} zwischen $G_1$ und $G_2$ ist ein stetiger Gruppenhomomorphismus $G_1 \to G_2$.
		Ist dieser bijektiv und die Umkehrabbildung wieder ein Homomorphismus topologischer Gruppen, so sprechen wir von einem \emph{Isomorphismus topologischer Gruppen}.
	\end{defi}
	Man sieht sofort, dass jede Translation um ein beliebiges Gruppenelement einen Homeomorphismus $G \to G$ bildet.
	Die Topologie ist also \emph{translationsinvariant} in dem Sinne, dass f"ur alle $g \in G$ und jede Mengen $U$ die "Aquivalenzen
	\begin{align*}
		U \text{ ist offen} \Leftrightarrow gU = \{gu \in G: u\in U\} \text{ ist offen} \Leftrightarrow Ug = \{ug \in G: u\in U\} \text{ ist offen}
	\end{align*}
	gelten.
	Analog verh"alt es sich f"ur die Umkehrabbildung. 
	Sie ist ebenso ein Homeomorphismus und $U$ ist genau dann offen, wenn $U^{-1}=\{u^{-1}\in G: u \in U\}$ offen ist.
	
	Die Translationsinvarianz der Topologie hat nun einige nette Vorteile.
	Zum Beispiel wird bereits die ganze Topologie durch die Umgebungsbasis des neutralen Elements definiert.
	Durch Translation erhalten wir Umgebungsbasen beliebiger anderer Elemente und damit zwangsl"aufig die komplette Topologie.
	F"ur ein weiteres Beispiel erinnern wir uns an die Definition der Stetigkeit in topologischen R"aumen.
	Eine Abbildung $f: G_1 \to G_2$ zwischen zwei topologischen R"aumen heißt stetig, wenn f"ur alle $g\in G_1$ und jede offene Umgebung $U$ von $f(g)$ eine offene Umgebung $V$ von $g$ existiert, sodass $f(V)\subseteq U$.
	Sind $G_1$ und $G_2$ nun topologische Gruppen und ist $f$ ein (nicht unbedingt topologischer) Gruppenhomomorphismus reicht es jetzt die Stetigkeit in dem neutralen Element $e_1$ der Gruppe $G_1$ nachzuweisen.
	Denn ist $f$ stetig in $e_1$ und sei $g$ ein beliebiges weiteres Element der Gruppe, so ist f"ur jede offene Umgebung U von $f(g)$ ist $U'\coloneqq f(g)^{-1}U$ eine offene Umgebung des neutralen Elements $e_2$. 
	Wegen Stetigkeit in $e_1$ gibt es dann eine offene Umgebung $V'$ mit $f(V') \subseteq U'$.
	Nun ist aber $V\coloneqq  gV'$ eine offene Umgebung von $g$ und da $f$ ein Gruppenhomomorphismus ist, haben wir
	\begin{align*}
		f(V) = f(gV') = f(g)f(V') \subseteq f(g) U' =f(g)f(g)^{-1} U = U,
	\end{align*}
	also ist die Abbildung "uberall stetig.
	
	Haben wir wieder zwei topologische Gruppen $G_1$ und $G_2$, so ist deren direktes Produkt $G_1\times G_2$, wie man leicht feststellen kann, wieder eine topologische Gruppe.
	Dieses Ergebnis l"asst sich auch auf endliche, abz"ahlbare und sogar beliebige Mengen von Gruppen "ubertragen. 
	\begin{lemma}\label{lemma:topogroup:directproduct}
		Sei $I$ eine beliebige Indexmenge und $G_\nu$ eine topologische Gruppe f"ur alle $\nu\in I$. 
		Das direkte Produkt $G = \prod_{\nu\in I} G_\nu$ versehen mit der Produkttopologie und komponentenweiser Gruppenverkn"upfung ist wieder eine topologische Gruppe.
	\end{lemma}
	\begin{proof}
		Wir erinnern uns daran, dass eine Basis der Produkttopologie gegeben ist durch Rechtecke der Form
		\begin{align*}
			\prod_{\nu\in E} U_\nu \times \prod_{\nu\in I\setminus E}  G_\nu,
		\end{align*}
		wobei $E$ eine endliche Teilmenge von $I$ und jedes $U_\nu$ offen in $G_\nu$ ist. 
		Ohne Einschr"ankung sei also 
		\begin{align*}
			W = \prod_{\nu\in E} W_\nu \times \prod_{\nu\in I\setminus E}  G_\nu
		\end{align*}
		eine offene Umgebung von $gh = (g_\nu h_\nu)$. 
		Da die $G_\nu$ topologische Gruppen sind, finden wir f"ur  alle $\nu\in E$ offene Umgebungen $U_\nu$ und $V_\nu$ von $g_\nu$ und $h_\nu$,  sodass  $U_\nu V_\nu \subseteq W_\nu$. Wir behaupten nun, dass
		\begin{align*}
			(\prod_{\nu\in E} U_\nu \times \prod_{\nu\in I\setminus E}  G_\nu) \times (\prod_{\nu\in E} V_\nu \times \prod_{\nu\in I\setminus E}  G_\nu) \subseteq G \times G
		\end{align*}
		eine offene Umgebung von $(g, h) \in G \times G$ ist, deren Bild in $W$ liegt. 
		Der erste Aussage ist klar, beide Faktoren des Produkts sind offene Basiselemente der Topologie und enthalten jeweils $g$ bzw. $h$.
		Weiter ist das Bild unter Gruppenoperation gegeben durch
		\begin{align*}
			\prod_{\nu\in E} U_\nu V_\nu \times \prod_{\nu\in I\setminus E}  G_\nu,
		\end{align*}
		was nach obigen "Uberlegungen in $W$ liegt.
		Somit folgt die Stetigkeit der Gruppenverkn"upfung.
		Der Beweis f"ur die Stetigkeit der Umkehrabbildung funktioniert analog.
	\end{proof}
	%Hat man eine Topologie so kann man sich auch die Frage nach Kompaktheit stellen.
	%Leider sind die meisten topologischen Gruppen, die wir betrachten werden, jedoch nicht kompakt.
	
	\begin{defi}
		Ein topologischer Raum heißt \emph{lokalkompakt}, wenn jeder Punkt des Raumes eine kompakte Umgebung hat. 
		Eine \emph{lokalkompakte Gruppe} ist eine topologische Gruppe, die sowohl lokalkompakt als auch hausdorffsch ist. 
	\end{defi}
	Wir kennen bereits einige lokalkompakte Gruppen.
	\begin{bsp}~ 
		\begin{enumerate}[label=(\alph*)]
			\item Jede diskrete topologische Gruppe $G$, also eine Gruppe in der alle Teilmengen offen sind, ist lokalkompakt. 
				F"ur jedes $x \in G$ ist $\{x\}$ eine kompakte Umgebung von $x$.
			\item Die additive Gruppe $\R^+$ mit der gewohnten Topologie ist lokalkompakt. 
				Denn ist $x\in\R^+$, so ist $[x-\varepsilon, x+\varepsilon]$ f"ur $\varepsilon>0$ eine kompakte Umgebung von $x$. "
				Ahnlich verh"alt es sich f"ur die multiplikative Gruppe $\R^\times = \R \setminus\{0\}$.
			\item Analog kann man sich "uberlegen, dass die Gruppen $\Komplex^+$ und $\Komplex^\times$ lokalkompakt sind, wobei wir hier die abgeschlossenen B"alle $\conj{B_\varepsilon(x)}$ als kompakte Umgebung von $x$ haben.
		\end{enumerate}
	\end{bsp}
	Auch hier k"onnen wir uns das direkte Produkt zweier lokalkompakter Gruppen anschauen. 
	\begin{lemma}\label{satz:topo:lcaproduct}
		Seien $G_1$ und $G_2$ zwei lokalkompakte Gruppen. 
		Dann ist $G_1\times G_2$ wieder lokalkompakt. 
		Insbesondere ist also jedes endliche direkte Produkt lokalkompakter Gruppen lokalkompakt.
	\end{lemma}
	\begin{proof}
		Sei $(g_1,g_2) \in G_1\times G_2$. Wegen der Lokalkompaktheit von $G_1$ und $G_2$ finden wir kompakte Umgebungen $K_1$, $K_2$ von $g_1$ bzw. $g_2$. Dann ist aber $K_1 \times K_2$ eine kompakte Umgebung von $(g_1,g_2)$. 
		Weiter ist das direkte Produkt zweier Hausdorff-R"aume wieder hausdorffsch, wodurch $G_1\times G_2$ zu einer lokalkompakten Gruppe wird.
	\end{proof}
	Wir werden sp"ater in Lemma \ref{Lemma:lokalkompaktProd} sehen, kann diese Aussage nicht ohne Weiteres auf beliebig große direkte Produkte "ubertragen werden.

\subsection{Das Haar-Maß}
	Nun zu etwas Maßtheorie. 
	Wir beginnen mit einer kleinen Auffrischung der wichtigsten Konzepte. 
	Eine \emph{$\sigma$-Algebra} auf einer Menge $X$ ist eine Teilmenge $\Omega$ von $P(X)$, so dass
	\begin{enumerate}[label=(\roman*)]
		\item $X \in \Omega$
		\item Wenn $A \in \Omega$, dann $X\setminus A \in \Omega$.
		\item $\Omega$ ist abgeschlossen unter abz"ahlbarer Vereinigung, d.h. $\bigcup_{k=1}^{\infty} A_k \in \Omega$, falls $A_k \in \Omega$ f"ur alle $k$.
	\end{enumerate}
	Die Elemente in $\Omega$ werden \emph{messbar} genannt. 
	Aus den Axiomen l"asst sich leicht folgern, dass die leere Menge und abz"ahlbare Schnitte von messbaren Mengen wiederum messbar sind. 
	Weiter ist der Schnitt $\bigcap_n \Omega_n$ beliebiger Familien $\{\Omega_n\}$ von $\sigma$-Algebren auf X selbst wieder eine $\sigma$-Algebra.
	
	Eine Menge $X$ zusammen mit einer $\sigma$-Algebra $\Omega$ bilden den \emph{messbaren Raum} $(X, \Omega)$. 
	Ist X ein topologischer Raum, so k"onnen wir die kleinste $\sigma$-Algebra $\Borel$ betrachten, die alle offenen Mengen von $X$ enth"alt. 
	Die Elemente von $\Borel$ werden \emph{Borelmengen} von $X$ genannt.
	
	Ein \emph{Maß} auf einem beliebigen messbaren Raum $(X, \Omega)$ ist eine Funktion $\mu: \Omega \to [0, \infty]$ mit $\mu(\emptyset) = 0$ und die \emph{$\sigma$-additiv} ist. 
	Das bedeutet
	\begin{align*}
		\mu\left( \bigcup_{k=1}^{\infty} A_k\right) = \sum_{k=1}^{\infty} \mu (A_k)
	\end{align*}
	f"ur beliebige Familien $\{A_n\}_1^\infty$ von paarweise disjunkten Mengen in $\Omega$.
	Zusammen definiet dies den \emph{Maßraum} $(X, \Omega, \mu)$.
	Ist dieses Maß definiert auf der $\sigma$-Algebra der Borelmengen, so nennen wir es ein Borel-Maß.
	
	Ein wichtiges Ziel der Maßtheorie war es den Begriff des Integrals zu verallgemeinern.
	Wir geben eine kurze, stark vereinfachte Variante der Grundkonzepte der Integrationstheorie und verweisen auf Folland \cite{folland} Kapitel 2 f"ur eine vollst"andige Einf"uhrung.
	
	F"ur einen beliebigen Maßraum $(X, \Omega, \mu)$ geschieht dies "uber  dieso genannten \emph{Treppenfunktionen} auf $X$
	\begin{align*}
		f(x) = \sum_{k=1}^n \alpha_k \ind_{A_k} (x)
	\end{align*}
	mit $\alpha \in \Komplex$ und der \emph{charakteristischen Funktion} der messbaren Menge $A_k$
	\begin{align*}
		\ind_{A_k}(x) =
			\begin{cases}
				1 &\text{falls } x\in A_k\\
				0 &\text{sonst}.
			\end{cases}
	\end{align*}
	Das Integral dieser Funktionen wird definiert durch
	\begin{align*}
		\int_G f d\mu = \sum_{k=1}^n \alpha_k \mu(A_k),
	\end{align*}
	mit der Konvention, dass $0 \times \infty = 0$. 
	Eine M"oglichkeit dies auf andere Funktionen $f$ zu erweitern ist es Folgen $(f_n)$ von solchen Treppenfunktionen zu betrachten.
	Diese nennen wir \emph{$L^1$-Cauchy-Folge}, wenn sie eine Cauchy-Folge bez"uglich der Norm $\norm*{g}_{L^1}\coloneqq  \int_X \abs[g] d\mu$ ist. 
	Konvergiert die Folge zus"atzlich fast "uberall punktweise gegen eine Funktion $f:X \to \Komplex$, so definieren wir mit
	\begin{align*}
		\int f d\mu = \lim_{n\to \infty} \int f_n d\mu
	\end{align*}
	das Integral von $f$ "uber $X$. 
	
	Ist $Y \subseteq X$ messbar in X, so setzen wir $\int_Y f d\mu \coloneqq  \int \ind_Y f d\mu$ und definieren
	\begin{align*}
		\Vol(Y,d\mu) = \int_Y d\mu = \int \ind_Y d\mu = \mu(Y).
	\end{align*}
	Wir nennen eine Funktion $f$ integrierbar auf Y, wenn das Integral $\int_Y \abs[f]d\mu$ endlich ist.
	Allgemeiner heißt eine Funktion integrierbar, wenn sie auf X integrierbar ist und wir schreiben\footnote{Hier missbrauchen wir etwas die Notation des $L^1$-Raumes. (vgl. Folland \cite{folland} Seite 53)} dann $f\in L^1(X,\Omega, \mu)$. 
	Wenn es ist klar ist, "uber welchen Maßraum wir reden, lassen wir h"aufig auch die einfach $\sigma$-Algebra und Maß weg und schreiben dann $\dx$ f"ur das Maß, $\int_X f(x) \dx = \int f d\mu$ f"ur das Integral und $L^1(X)$ f"ur die integrierbaren Funktionen auf $X$. Damit beenden wir uns kurze Wiederholung.
	
	Sei nun $\mu$ ein Borel-Maß auf einem lokalkompakten, hausdorffschen Raum X und sei $E$ ein eine beliebige Borelmenge von $X$.
	Wir nennen $\mu$ von \emph{innen regul"ar} auf E, falls
	\begin{align*}
		\mu(E) = \sup\{\mu(K): K \subseteq E, K \text{ kompakt}\}
	\end{align*}
	Umgekehrt heißt $\mu$ von \emph{außen regul"ar} auf E, wenn
	\begin{align*}
		\mu(E) = \inf\{\mu(U): E \subseteq U, U \text{ offen}\}.
	\end{align*}
	
	\begin{defi}
		Ein \emph{Radon-Maß} auf $X$ ist ein Borel-Maß, das endlich auf kompakten Mengen, von innen regul"ar auf allen offenen Mengen und von außen regul"ar auf allen Borelmengen ist.
	\end{defi}
	Radon-Maße stehen in einem engen Zusammenhang mit positiven linearen Funktionalen auf dem Vektorraum der stetigen Funktionen mit kompakten Tr"ager $C_c(X)$.
	Dies sind lineare Abbildungen $I$ von $C_c(X)$ nach $\R$, so dass $I(f) \geq 0$ wenn $f\geq 0$.
	\begin{satz}[Rieszscher Darstellungssatz]
		Sei $I$ solch ein positives lineares Funktional auf dem Raum der stetigen Funktionen mit kompakten Tr"ager $C_c(X)$.
		Dann gibt es ein eindeutiges Radon-Maß $\mu$ auf $X$, so dass $I(f) = \int f d\mu$ f"ur alle $C_c(X)$.
	\end{satz}
	\begin{proof}[Beweisskizze]
		F"ur die Konstruktion des Maßes definiert man die Abbildungen
		\begin{align*}
			\mu(U) = \sup\{I(f): f\in C_c(X), 0\leq f\leq 1 \text{ und } \text{supp}(f) \subseteq U\}
		\end{align*}
		f"ur alle $U$ offen und
		\begin{align*}
			\mu^*(E) = \inf \{ \mu(U): U\supseteq E \text{ und } U \text{ offen}\}
		\end{align*}
		f"ur beliebige Teilmengen $E\subseteq X$. 
		Anschließend zeigt man
		\begin{enumerate}[label=(\roman*)]
			\item $\mu^*$ ist ein "außeres Maß.
			\item Jede offene Menge ist $\mu^*$-messbar.
			\item $\mu(K) = \inf\{I(f): f\in C_c(X), f\geq \ind_K\}$ f"ur alle kompakten Mengen $K \subseteq X$.
			\item $I(f) = \int f d\mu$ f"ur alle $f\in C_c(X)$.
		\end{enumerate}
		(i) und (ii) implizieren zusammen mit dem Satz von Caratheodory, dass $\mu$ ein Borel-Maß, welches von außen regul"ar ist.
		(iii) liefert uns die Endlichkeit auf kompakten Mengen und die Regularit"at von innen auf offenen Mengen und (iv) vollendet den Satz.
		F"ur den vollst"andigen Satz verweisen wir auf Folland \cite{folland} Kapitel 7 Satz 7.2.
	\end{proof}
	Dieser Satz ist ein wichtiger Grundstein f"ur viele Aussagen "uber Radon-Maße.
	Sind zum Beispiel $X$ und $Y$ zwei lokalkompakte Gruppen mit dazugeh"origen Radonmaßen $\mu$ und $\nu$ so ist im Allgemeinen das klassische Produktmaß $\mu \times \nu$ kein Borel- und daher kein Radon-Maß auf $X \times Y$.
	Wir definieren daher das Radonprodukt von $\mu$ und $\nu$ als das Radon-Maß welches durch das positive Funktional $I(f) = \int f d(\mu \times \nu)$ nach Rieszschen Darstellungssatz gegeben wird.
	Dieses Produkt wird f"ur uns in Kapitel \ref{sec:rdp} eine Rolle spielen.
	Danach k"onnen wir es wieder vergessen, denn erf"ullen $X$ und $Y$ das zweite Abzählbarkeitsaxiom so entspricht das Radonprodukt genau dem Produktmaß.
	F"ur eine ausf"urhliche Behandlung dieser Konzepte verweisen wir auf Folland \cite{folland} Kapitel 7.
	
	Damit kommen wir zum großen Ziel dieses Abschnitts.
	Sei nun $G$ eine lokalkompakte Gruppe und $\mu$ ein beliebiges Borel-Maß.
	Wir k"onnen uns dann anschauen, wie sich $\mu$ bez"uglich der Translation durch beliebige Gruppenelemente $g\in G$ verh"alt. 
	Gilt $\mu(gE) = \mu(E)$ f"ur jede Borelmenge, so nennen wir $\mu$ \emph{linksinvariant}.
	Analog heißt $\mu$ \emph{rechtsinvariant}, falls $\mu(Eg) = \mu(E)$.
	Ist $G$ abelsch fallen diese beiden Begriffe nat"urlich zusammen und wir sprechen nur von einem \emph{translationsinvarianten} Maß $\mu$.
	
	Nun haben wir alle wichtigen Grundlagen zusammen f"ur folgende wichtige
	\begin{defi}
		Ein \emph{linkes} bzw. \emph{rechtes} \emph{Haar-Maß} auf $G$ ist ein linksinvariantes bzw. rechtsinvariantes Radon-Maß, das auf nichtleeren offenen Mengen positiv ist. 
	\end{defi}
	Wir haben bereits einige solcher Haar-Maße kennengelernt:
	\begin{bsp}~ 
		\begin{enumerate}[label=(\alph*)]
			\item Ist $G$ wieder eine diskrete Gruppe, dann ist das Z"ahlmaß ein Haar-Maß.
			\item F"ur $G=\R^+$ definiert das Lebesgue-Maß $\dx$ ein Haar-Maß.
			\item F"ur $G=\R^\times$ wird durch $\mu(E)\coloneqq \int_{\R^\times} \ind_{E} \frac{1}{\abs[x]}dx$ ein Haar-Maß, wie wir sp"ater in Satz \ref{satz:lokal:multiplikativesmass}sehen werden.
		\end{enumerate}
	\end{bsp}
	
	Diese Beispiele sind keine Ausnahmen.
	Einer der Hauptgr"unde, warum wir uns mit lokalkompakten Gruppen besch"aftigen ist n"amlich folgender Satz.
	\begin{satz}[Existenz und Eindeutigkeit des Haar-Maß]
	\label{satz:topo:haarmeasure}
		Sei $G$ eine lokalkompakte Gruppe. Dann existiert ein linksinvariantes Haar-Maß auf $G$. Dieses ist eindeutig bis auf skalares Vielfaches.
	\end{satz}
	\begin{proof}
		Der etwas l"angere Beweis befindet sich in Ramakrishnan und Valenza \cite{rama} Kapitel 1 und benutzt den oben vorgestellten Rieszschen Darstellungssatz.
		Ziel ist es dabei, ein links-invariantes lineares Funktional auf $C_c(G)$ zu konstruieren.
	\end{proof}
	Dieser stellt quasi die bestm"ogliche Situation dar, die man sich erhoffen kann. 
	Ist n"amlich $\mu$ ein Haar-Maß und $c>0$ eine Konstante, so heralten wir durch $c\cdot \mu$ wieder ein Haar-Maß. 
	Wir beenden diesen Abschnitt mit einem kleinen Lemma, welches uns einige bekannte Eigenschaften des Lebesgue-Maß auf Haar-Maße verallgemeinert.
	\begin{lemma}F"ur jede integrierbare Funktion $f\in L^{1}(G)$ gilt
		\begin{enumerate}[label=\emph{(\roman*)}]
			\item $\int_{G} f(yx)d\mu(x) =  \int_{G} f(x)d\mu(x)$
			\item $\int_{G} f(x^{-1})d\mu(x) = \int_{G} f(x)d\mu(x)$
		\end{enumerate}
	\end{lemma}
	\begin{proof}
		(i) folgt leicht aus der Definition des Integrals und der Translationsinvarianz des Maßes, denn $\ind_A(yx) = \ind_{y^{-1}A}(x)$ und $\Vol({y^{-1}A}, \dx) = \Vol(A,\dx)$.
		
		F"ur (ii) "uberlegen wir uns zun"achst, dass $\tilde{\mu}(E)\coloneqq  \mu(E^{-1})$ ein weiteres Haar-Maß auf $G$ definiert. 
		Nach der Eindeutigkeit unterscheiden sich beide Maße nur um eine Konstante $c > 0$. Wir wollen zeigen, dass $c=1$ ist.
		Sei dazu $K$ eine kompakte Umgebung der $1$. Dann gibt es eine offene Umgebung $U$ der $1$ mit $G \subseteq K$. Definieren wir nun $S \coloneqq  KK^{-1}$, so ist $S$ kompakt, $U \subseteq S$ und es gilt $0 < \mu(U) \leq \mu(S)<\infty$. 
		Es folgt $ c\cdot \mu(S) = \tilde{\mu}(S) = \mu(S^{-1}) =\mu(S)$ und damit $c=1$. Das Haar-Maß ist also invariant unter der Umkehrabbildung. Der Rest folgt dann aus der Definition des Integrals.
	\end{proof}
	
\subsection{Charaktere und Quasi-Charaktere}
	F"ur die Fouriertransformation auf beliebigen lokalkompakten Gruppen definieren zu k"onnen kommt nicht um die sogenannten Charaktere herum.
	Auch wenn wir sie in diesem Kontext nicht (explizit) brauchen, werden sie f"ur uns eine wichtige Rolle im Beweis der Funktionalgleichung spielen.
	\begin{defi}
		Ein \emph{Quasi-Charakter} einer topologischen Gruppe $G$ ist ein stetiger Gruppenhomomorphismus von $G$ in die multiplikative Gruppe $\Komplex^\times$ der komplexen Zahlen.
		Ein \emph{Charakter} ist ein Quasi-Charakter, dessen Bild auf dem komplexen Einheitskreis $S^1 =\{z\in\Komplex: \abs[z]=1\}$ liegt.
	\end{defi}
	In der Literatur werden Quasi-Charaktere h"aufig auch einfach nur als Charaktere bezeichnet. Was wir in dieser Arbeit unter einem Charakter verstehen, wird dann unit"arer Charakter genannt.
	Beginnen wir mit ein paar Beispielen.
	\begin{bsp}~
		\begin{enumerate}[label=(\alph*)]
			\item F"ur jede topologische Gruppe $G$ ist die Abbildung $g\mapsto 1$ ein Charakter, der sogenannte \emph{triviale Charakter}. 
				Er ist der einzige konstante Charakter, denn f"ur jeden Gruppenhomomorphismus $\chi$ gilt bekanntlich $\chi(1) = 1$.
			\item Ein nicht-triviales  Beispiel f"ur einen Charakter ist die Abbildung $t \mapsto \exp(i t)$ von der additiven Gruppe $\R^+$ in den Einheitskreis $S^1$.
			\item  Die bekannte Abbildung $\exp: \Komplex^+ \to \Komplex^\times$ von der additiven Gruppe in die multiplikative Gruppe der komplexen Zahlen ist ein (Quasi-Charakter, jedoch kein Charakter.
		\end{enumerate}
	\end{bsp}
	
	Werfen wir einen Blick auf die Quasi-Charaktere kompakter Gruppen.
	\begin{lemma}\label{Lemma:trivialerCharAufKompakt}
		Sei $K$ eine kompakte Gruppe mit Haar-Maß $\dx$ und $\chi: K \to \Komplex^\times$ ein Quasi-Charakter. Dann gilt
		\begin{enumerate}[label=\emph{(\roman*)}]
				\item $\chi$ ist bereits ein Charakter.
				\item F"ur das Integral von $\chi$ "uber $K$ gilt
					\begin{align*}
						\int_K \chi(x) \dx = 
							\begin{cases}
								\text{\emph{Vol}}(K, \dx),	& \text{\emph{falls} } \chi\equiv 1\\
								0,					& \text{\emph{ansonsten.}}
							\end{cases}
					\end{align*}
		\end{enumerate}
	\end{lemma}
	\begin{proof}
		F"ur (i) sei $x$ ein beliebiges Element von $K$. 
		Sei $H$ der Abschluss der von $x$ erzeugten Untergruppe von $K$.
		Damit ist $H$ selbst eine Untergruppe von $K$ und als abgeschlossene Teilmenge eines Kompaktums kompakt.
		Da $\chi$ ein stetiger Gruppenhomomorphismus ist, muss $\chi(H)$ eine kompakte Untergruppe von $\Komplex^\times$ sein.
		Diese liegen aber gerade alle auf $S^1$ und die Behauptung folgt.
		
		Nun zu (ii): Der erste Fall ist klar. 
		Im zweiten Fall gibt es ein $x_0 \in K$ mit $\chi(x_0) \not=1$ und mit Translationsinvarianz daher
		\begin{align*}
			\int_K \chi(x)\dx = \int_K\chi(x_0x)\dx = \chi(x_0)\int_K\chi(x)\dx.
		\end{align*}
		Umstellen und Division durch $\chi(x_0) - 1 \not=0$ ergibt $\int_K \chi(x)dx = 0$.
	\end{proof}
	Im weiteren Verlauf dieser Arbeit werden wir (Quasi-)Charaktere noch gr"undlicher untersuchen.
	
\subsection{Ausblick: Fouriertransformation und Pontryagin-Dualit"at}
	Zum Ende dieses Kapitels blicken wir etwas "uber den Tellerrand hinaus.
	Dieser Abschnitt ist f"ur das Verst"andis der kommenden Beweise dieser Arbeit absolut optional, bietet aber Einblick die n"otigen Abstraktionen, die in Tates eigener Argumentation zum tragen kamen.
	Wir schneiden daher die f"ur Tates Dokorabeit wichtigsten Grundlagen der abstrakten harmonischen Analysis an und versprechen in den sp"ateren Kapiteln zu zeigen, dass unsere eigenen Definitionen mit denen in diesem Abschnitt vertr"aglich sind.
	
	Beginnen wir mit der Charaktergruppe auf einer beliebigen topologischen Gruppe $G$.
	Die unit"aren Charaktere auf $G$ werden mit punktweiser Multiplikation $\chi\psi (x) \coloneqq \chi(x) \psi (x)$ selbst wieder eine Gruppe, welche als die \emph{duale Gruppe $\hat{G}$} bezeichnet wird.
	Wir statten $\hat{G}$ mit einer Topologie aus:
	Sei $K$ eine kompakte Teilmenge von $G$ und $V$ eine Umgebung der $1\in S^1$.
	Dann wird durch die Teilmengen
	\begin{align*}
		W(K, V) = \{ \chi\in \hat{G}: \chi(K)\} \subseteq V
	\end{align*}
	eine Umgebungsbasis des trivialen Charakters definiert und induziert damit eine Topologie: die sogenannte \emph{kompakt-offen Topologie}
	Damit wird $\hat{G}$ zu einer topologischen Gruppe.
	Man hat nun folgenden Satz
	\begin{satz} Sei $G$ eine abelsche topologische Gruppe. Dann gelten die folgenden Aussagen:
		\begin{enumerate}[label=\emph{(\roman*)}]
			\item Ist $G$ diskret, so ist $\hat{G}$ kompakt.
			\item Ist $G$ kompakt, so ist $\hat{G}$ diskret.
			\item Ist $G$ lokalkompakt, so ist auch $\hat{G}$ lokalkompakt.
		\end{enumerate}
	\end{satz}
	Aussage (iii) sieht verd"achtig aus.
	Eine erste Vermutung w"are, dass $G$ isomorph zu $\hat{G}$ sein k"onnte.
	Dieser Verdacht stimmt im Allgemeinen leider nicht. 
	Daf"ur haben wir aber das n"achstbeste Ergebnis:
	\begin{satz}[Pontryagin Dualit"at]
		Jede lokalkompakte Gruppe $G$ ist kanonisch isomorph zu ihrem Doppel-Dual $\hat{\hat{G}}$.
		Der Isomorphismus topologischer Gruppen $\alpha$ ist gegeben durch die Auswertungsabbildung $\alpha(y)(\chi) = \chi(y)$.
	\end{satz}
	\begin{proof}
		Der etwas l"angere Beweis ist zum Beispiel zu finden in Ramakrishnan und Valenza \cite{rama} Kapitel 3, Theorem 3-20.
	\end{proof}
	
	F"ur den Beweis wird ein klassisches Konzept nun verallgemeinert.
	Wir wissen bereits, dass jede abelsche lokalkompakte Gruppe $G$ ein eindeutiges Haar-Maß $\dx$ besitzt.
	Damit k"onnen wir auf $G$ integrieren und definieren das abstrakte Analogon zur klassischen Fouriertransformation.
	\begin{defi}[Fouriertransformation]
		Sei $f\in L^1(G)$. Wir definieren dann die \emph{Fouriertransformation} $\hat{f}: \hat{G} \to \Komplex$ von $f$ durch die Formel
		\begin{align*}
			\hat{f}(\chi) = \int_G f(x)\conj{\chi}(x) dx
		\end{align*}
		f"ur alle $\chi\in \hat{G}$.
	\end{defi}
	Diese Formel macht Sinn, denn f"ur alle $x \in G$ hat $\chi(x)$ den Betrag $1$. 
	Ist also $f$ integrierbar, so ist es auch das Produkt im Integranden.
	Da $\hat{G}$ selber wieder lokalkompakt ist, besitzt das Duale ein Haar-Maß $\dx[\chi]$ und es macht Sinn die Fouriertransformation auf $\hat{G}$ betrachten.
	Durch geeignete Normierung der verwendeten Maße gelangen wir zu folgendem Satz der unter Anderem in Tates Beweis der verallgemeinerten Poisson-Summenformel eine wichtige Rolle spielt.
	\begin{satz}[Fourier-Umkehrformel f"ur lokalkompakte Gruppen]
		Es gibt ein Haar-Maß $\dx[\chi]$ auf $\hat{G}$, so dass alle $f \in L^1(G)$ stetig mit $\hat{f} \in L^1(\hat{G})$, die Gleichung
		\begin{align*}
			f(x) = \int_{\hat{G}} \hat{f}(\chi) \chi(x) \dx[\chi]
		\end{align*}
		erf"ullen. Insbesondere ist also $\hat{\hat{f}}(x) = f(-x)$.
	\end{satz}
	\begin{proof}
		Siehe zum Beispiel Folland \cite{folland} Kapitel 4, Satz 4.32 oder Ramakrishnan und Valenza \cite{rama} Kapitel 3, Satz 3-9.
	\end{proof}

	
	

\clearpage

\section{Lokale Betrachtungen}
	\subsection{Exkurs: $p$-adische Zahlen}
	Sei $\K$ ein beliebiger K"orper und $\R_+ = \{x\in \R: x\geq 0\}$ die Menge der nicht-negativen reellen Zahlen.
	\begin{defi}
		Ein \emph{Absolutbetrag} auf $\K$ ist eine Abbildung
		\begin{align*}
			\abs:\K \longrightarrow \R_+
		\end{align*}
		welche die folgenden Bedingungen erf"ullt:
		\begin{enumerate}[label=(\roman*),leftmargin=1.5cm]
			\item $\abs[x] = 0 \Leftrightarrow x=0$ (Definitheit)
			\item $\abs[xy] = \abs[x]\abs[y]$ f"ur alle $x, y \in \K$ (Multiplikativit"at)
			\item $\abs[x+y] \leq \abs[x] + \abs[y]$ f"ur alle $x,y \in \K$ (Dreiecksungleichung)
		\end{enumerate}
		Wir nennen den Absolutbetrag $\abs$ nicht-archimedisch, wenn er zus"atzlich die st"arkere Bedingung
		\begin{enumerate}[label=(\roman*)$'$,leftmargin=1.5cm]
			\setcounter{enumi}{2}
			\item $\abs[x+y] \leq \max\{\abs[x],\abs[y]\}$ f"ur alle $x, y \in \K$ (versch"arfte Dreiecksungleichung)
		\end{enumerate}
		erf"ullt. Anderenfalls sagen wir der Absolutbetrag ist archimedisch.
	\end{defi}
	Wir m"ochten zun"achst ein paar allgeimg"ultige Eigenschaften von Absolutbetr"agen im folgenden Lemma festhalten.
	\begin{lemma}
		F"ur beliebige Absolutbetr"age $\abs$ auf $\K$ und Elemente $x\in \K$ gilt:
		\begin{enumerate}[label=(\roman*),leftmargin=1.5cm]
			\item $\abs[1] = 1$
			\item $\abs[-1] = 1$
			\item Falls $\abs[x^n]=1$, dann $\abs[x]=1$
			\item $\abs[-x]$ = $\abs[x]$ 
		\end{enumerate}
	\end{lemma}
	
	Betrachten wir nun den K"orper $\K = \Q$ der rationalen Zahlen. Sei $x\in \Q^\times$ eine beliebige rationale Zahl. Dann existiert eine eindeutige (bis auf Reihenfolge) Primfaktorzerlegung
	\begin{align*}
		x = \prod_{p} p^{v_p},
	\end{align*}
	wobei das Produkt "uber alle Primzahlen $p \in \N$ geht und $v_p \in \Z$ f"ur fast alle $p$ gleich $0$ ist. Legen wir uns auf ein $p$ fest, so erm"oglicht sich die 
	\begin{defi}
		F"ur beliebige $x \in \Q$ sei der \emph{p-adische Absolutbetrag} von $x$ gegeben durch
		\begin{align*}
			\abs[x]_p = p^{-v_p}
		\end{align*}
		f"ur $x\neq 0$ und $v_p \in \Z$ wie oben. Durch $\abs[0]_p := 0$ vervollst"andigen wir die Definition.
	\end{defi}
	\begin{lemma}
		$\abs_p$ ist ein nicht-archimedischer Absolutbetrag auf $\Q$
	\end{lemma}
	\begin{proof}
		
	\end{proof}
	
	\subsection{Lokale Fourieranalysis}
		F"ur die unendliche Stelle $p=\infty$ definieren wir die \emph{Schwartz-Bruhat Funktion} als eine komplexwertige, glatte Funktion $f$, die f"ur alle nicht-negativen ganzen Zahlen $n$ und $m$ die Bedingung
		\begin{align*}
			\sup_{x\in \K_\infty}\abs[x^n\frac{d^m}{dx^m}f(x)] < \infty
		\end{align*}
		erf"ullt.\footnote{Hier ist mit $\abs$ der komplexe Absolutbetrag gemeint.}
		F"ur die endlichen Stellen $p<\infty$ definieren wir eine \emph{Schwartz-Bruhat Funktion} als eine lokal konstante Funktion mit kompakten Träger.
		Die Menge aller solcher Funktionen bilden einen komplexen Vektorraum, den wir mit $\Sw(\K_p)$ bezeichnen. 
		Im Fall $p<\infty$ erkennt man leicht, dass $\Sw(\Kp)\subseteq L^1(\Kp)$. 
		F"ur $p=\infty$ gilt nach obiger Bedingung $(\abs[1]+\abs[x^2])\abs[f(x)] \leq C$, also $\abs[f(x)]\leq C(1+x^2)^{-1}$ und $(1+x^2)^{-1} \in L^1(\K\infty)$
		
		\begin{bsp}~ 
			\begin{enumerate}[label=(\roman*)]
				\item Im Fall $p=\infty$ ist die Funktion $f_k = x^k e^{-x^2}$ f"ur jedes $k\in\N_0$ in $\Sw(\K_\infty)$. 
				Die Ableitungen $\frac{d^m}{dx^m} f_k(x)$ sind von der Form $p(x)e^{-x^2}$, wobei $p(x)$ ein Polynom ist. 
				Aus der Analysis ist dann bekannt, dass $\abs[x^n p(x)e^{-x^2}]$ f"ur jedes $n\in \N_0$ beschränkt ist.
				\item Im Fall $p<\infty$ sind offensichtlich die charakteristischen Funktionen kompakter Mengen in $\Sw(\Kp)$. 
				Beispiele f"ur Kompakta sind Mengen der Form $a+p^k\Zp$ mit $a\in \K$ und $k\in \Z$.
			\end{enumerate}
		\end{bsp}
		
		\begin{lemma}\label{lemma:padischSBF}
			Jede Funktion $f\in \Sw(\Kp)$, $p<\infty$, ist eine endliche Linearkombination von charakteristischen Funktionen der Form $\ind_{a+p^k\Zp}$, wobei $a\in \K$ und $k\in \Z$
		\end{lemma}
		\begin{proof}
			Sei $f \in \Sw(\Kp)$. 
			Da $f$ lokal konstant ist, ist f"ur jedes $z\in\C$ das Urbild $f^{-1}(z)$ offen in $\Kp$. 
			Also ist $f^{-1}(0)$ offen, folglich $\Kp \setminus f^{-1}(0)$ abgeschlossen und daher schon $\text{supp}(f) = \Kp \setminus f^{-1}(0)$. 
			Per Definition hat die Schwartz-Bruhat Funktion $f$ kompakten Tr"ager, also ist $\Kp \setminus f^{-1}(0)$ kompakt. 
			Diese Menge wird von den offenen Mengen $f^{-1} (x)$ mit $x\not= 0$ "uberdeckt, wovon nach Kompaktheit schon endlich viele reichen.
			$f$ hat somit endliches Bild. Weiter ist jede offene Menge $f^{-1} (x)$ eine disjunkte Vereinigung offener B"allen in $\Kp$. 
			Diese haben aber genau die gesuchte Form $a+p^k\Zp$ wie oben. 
			Aufgrund der Kompaktheit, reichen wieder endliche viele solcher B"alle. Damit folgt auch schon das Lemma.
		\end{proof}
		\begin{lemma}
			Sei $f \in \Sw(\Kp)$.
			\begin{enumerate}[label=\emph{(\alph*)}]
				\item Ist $g(x)=f(x)e_p(ax)$ mit $a\in\Kp$, dann gilt $\hat{g}(x) = \hat{f}(x-a)$.
				\item Ist $g(x)=f(x-a)$ mit $a\in\Kp$, dann gilt $\hat{g}(x) = \hat{f}(x-a)e_p(-ax)$.
				\item Ist $g(x)=f(\lambda x)$ mit $\lambda \in\Kp^\times$, dann gilt $\hat{g}(x) =\frac{1}{\abs[\lambda]_p} \hat{f}(\frac{x}{\lambda})$.
			\end{enumerate}
		\end{lemma}
		\begin{proof}
			(a) und (b) sind einfache Folgerungen aus der Definition mit der Multiplikativit"at von $e_p$ und der Translationsinvarianz des Haar-Maß. 
			Bei (c) spielt unsere Normeriung des Absolutbetrags eine Rolle, denn mit dem Variablenwechsel $y\mapsto \lambda^{-1}y$ erhalten wir
			\begin{align*}
				\hat{g}(x) = \int_{\Kp} f(\lambda y) e_p(-xy)dy = \frac{1}{\abs[\lambda]_p} \int_{\Kp} f(y) e_p(-x\lambda^{-1}y)dy = \frac{1}{\abs[\lambda]_p} \hat{f}\left(\frac{x}{\lambda}\right)
			\end{align*}
		\end{proof}
		\begin{satz}\label{Satz:fourierumkehrformel}
			Ist $p\leq\infty$ und $f\in\Sw(\Kp)$, so ist $\hat{f} \in \Sw(\Kp)$ und es gilt die Umkehrformel
			\begin{align*}
				\hat{\hat{f}}(x) = f(-x)
			\end{align*}
		\end{satz}
		\begin{proof}
			Betrachten wir zuerst den Fall $p<\infty$. Wie wir eben in Lemma \ref{lemma:padischSBF} gesehen haben haben, ist jede Funktion in $\Sw(\Kp)$ eine Linearkombination von Funktionen der Form $f = \ind_{a+p^k\Zp}$. Es reicht also die Aussage f"ur solche $f$ zu zeigen.
			Sei dazu $h:= \ind_{\Zp}$. Wir zeigen $\hat{h} = h$ durch folgende Rechnung
			\begin{align*}
				\hat(h)(x) = \int_\Kp h(y) e_p (-xy) dy_p = \int_\Zp e_p(-xy) dy_p.
			\end{align*}
			Nun ist $\chi(y):=e_p(-xy)$ ein Charakter auf $\Zp$ und genau dann trivial, wenn $x\in\Zp$. 
			Weiter ist $\Zp$ kompakt. 
			Nach Lemma \ref{Lemma:trivialerCharAufKompakt} und unserer Normierung von $dy_p$ folgt also
			\begin{align*}
				\hat(h)(x) = \text{Vol}(\Zp, dy_p) \ind_\Zp = \ind_\Zp = h(x)
			\end{align*}
			
			Wir f"uhren nun folgende Operatoren auf $\Sw(\Kp)$ ein
			\begin{align*}
				L_a f(x) = f(x-a), M_\lambda f(x) = f(\lambda x),
			\end{align*}
			wobei $a \in \Kp$ und $\lambda \in \Kp^\times$. 
			Nun k"onnen wir $f$ schreiben als $L_a M_{p^{-k}}h$. 
			Es folgt
			\begin{align*}
				\hat{f} = (L_a M_{p^{-k}}h)\widehat{\phantom{x}} = \Omega_{-a}p^{k}M_{p^k}\hat{h}=\Omega_{-a}p^{-k}M_{p^k}h.
			\end{align*}
			Also ist $\hat{f} (x) = e_p(-ax)p^k\ind_{p^{-k}\Zp}(x)$. 
			Der Charakter $e_p$ ist lokal konstant, $\hat{f}$ als das Produkt lokal konstanter Funktionen selbst wieder lokal konstant und damit in $\Sw(\Kp)$. 
			Damit haben wir den ersten Teil der Aussage gezeigt.
			
			F"ur den zweiten Teil sehen wir
			\begin{align*}
				\hat{\hat{f}} = (L_a M_{p^{-k}}h)\widehat{\widehat{\phantom{x}}} = L_{-a} (M_{p^k}h)\widehat{\widehat{\phantom{x}}}=L_{-a}M_{p^k}\hat{h} =L_{-a}M_{p^k}h,
			\end{align*}
			also $\hat{\hat{f}} (x) = \ind_{-a+p^k\Zp} (x) = \ind_{a+p^k\Zp} (-x) = f(-x)$. 
			Hier haben wir $p^k\Zp = - p^k\Zp$ ausgenutzt. 
			Damit haben wir die Umkehrformel f"ur den $p$-adischen Fall gezeigt. 
			F"ur $p=\infty$ ist die Formel bereits aus der klassischen Fourieranalysis bekannt.
		\end{proof}
	
	\begin{satz}[Ostrowski]
		Jeder nicht-triviale Absolutbetrag auf $\Q$ ist "aquivalent zu einem der Absolutbetr"age $\abs_p$, wobei $p$ entweder eine Primzahl ist oder $p=\infty$.
	\end{satz}
	\begin{proof}
		Sei $\abs$ ein beliebiger nicht-trivialer Absolutbetrag auf $\Q$. Wir untersuchen die zwei m"oglichen F"alle.
		\begin{enumerate}[align=left, leftmargin=0cm, labelsep=0cm, label=\alph*)\ ]
		\item $\abs$ ist archimedisch.
			Sei dann $n_0\in \N$ die kleinste nat"urliche Zahl mit $\abs[n_0] > 1$.
			Dann gibt es ein $\alpha \in \R^+$ mit $\abs[n_0]^{\alpha}=n_0$.
			Wir wollen nun zeigen, dass $\abs[n]=\abs[n]_\infty^\alpha$ f"ur alle $n \in \N$ gilt. Der allgemeine Fall f"ur $\Q$ folgt dann aus den Eigenschaften des Betrags.
			Dazu bedienen wir uns eines kleinen Tricks: F"ur $n \in \N$ nehmen wir die Darstellung zur Basis $n_0$, d.h.
			\begin{align*}
				n = \sum_{i=0}^{k} a_i n_0^i
			\end{align*}
			mit $a_i \in \{0,1,\dots,n_0-1\}$, $a_k \neq 0$ und $n_0^k\leq n < n_0^{k+1}$. Nehmen wir davon den Absolutbetrag und beachten, dass $\abs[a_i]\leq 1$ nach unserer Wahl von $n_0$ gilt, so erhalten wir
			\begin{align*}
				\abs[n] \leq \sum_{i=0}^{k} \abs[a_i] n_0^{i\alpha}
					\leq \sum_{i=0}^{k} n_0^{i\alpha}
					\leq n_0^{k\alpha}\sum_{i=0}^{k} n_0^{-i\alpha}
					\leq n_0^{k\alpha}\sum_{i=0}^{\infty} n_0^{-i\alpha}
					 = n_0^{k\alpha} \frac{n_0^\alpha}{n_0^\alpha - 1}.
			\end{align*}
			Setzt man nun $C:=\frac{n_0^\alpha}{n_0^\alpha - 1}>0$, so sehen wir
			\begin{align*}
				\abs[n]\leq C n_0^{k\alpha}\leq C n^\alpha
			\end{align*}
			f"ur beliebige $n \in \N$, also insbesondere auch
			\begin{align*}
				|n^N|\leq C n^{N\alpha}.
			\end{align*}
			Ziehen wir nun auf beiden Seiten die $N$-te Wurzel und lassen $N$ gegen $\infty$ laufen, so konvergiert $\sqrt[N]{C}$ gegen $0$ und wir erhalten 
			\begin{align*}
				\abs[n] \leq n^\alpha
			\end{align*}
			Damit w"are die erste H"alfte geschafft. Gehen wir nun z"uruck zu unserer Basisdarstellung
			\begin{align*}
				n = \sum_{i=0}^{k} a_i n_0^i.
			\end{align*}
			Da $n < n_0^{k+1}$ erhalten wir die Absch"atzung
			\begin{align*}
				n_0^{(k+1)\alpha}=|n_0^{k+1}| = |n + n_0^{k+1} - n| \leq \abs[n] + |n_0^{k+1} -n|.
			\end{align*}
			 mit dem Ergebnis aus der ersten H"alfte des Beweises und $n\geq n_0^k$ sehen wir
			\begin{align*}
				\abs[n] &\geq n_0^{(k+1)\alpha} - |n_0^{k+1} -n| 
					\geq n_0^{(k+1)\alpha} - (n_0^{k+1} -n)^\alpha
					\\&\geq n_0^{(k+1)\alpha} - (n_0^{k+1} -n_0^k)^\alpha
					=n_0^{(k+1)\alpha} \left(1 - \left(1 - \frac{1}{n_0}\right)\right)
					\\&> n^\alpha \left(1 - \left(1 - \frac{1}{n_0}\right)\right).
			\end{align*}
			Setzen wir wieder $C':=\left(1 - \left(1 - \frac{1}{n_0}\right)\right) >0$ folgt analog zum ersten Teil, dass
			\begin{align*} 
				\abs[n]\geq n^\alpha
			\end{align*}
			und daher $\abs[n]=n^\alpha$. Damit haben wir gezeigt, dass $\abs$ "aquivalent zum klassischen Absolutbetrag $\abs_\infty$ ist.
		\item $\abs$ ist nicht archimedisch.
			Dann ist $\abs[n_0]\leq 1$ f"ur alle $n \in \N$ und, da $\abs$ nicht-trivial ist, muss es eine kleinste Zahl $n_0$ geben mit $\abs[n_0]<1$. Insbesondere muss $n_0$ eine Primzahl sein, denn sei $p \in \N$ ein Primteiler von $n_0$, also $n_0=p \cdot n'$ mit $n' \in \N$ und $n' < n$, dann gilt nach unserer Wahl von $n_0$
			\begin{align*}
				|p| = |p|\cdot |n'| =|p \cdot n'| = \abs[n_0] < 1.
			\end{align*}
			Folglich muss schon $p=n_0$ gelten. Ziel wird es jetzt nat"urlich sein zu zeigen, dass $\abs$ "aquivalent zum $p$-adischen Absolutbetrag ist.
			Zun"achst finden wir ein $\alpha \in \R^+$ mit $|p| = |p|_p^{\alpha} = \frac{1}{p^{\alpha}}$ . Sei als n"achstes $n\in \Z$ mit $p \not | n$. Wir schreiben
			\begin{align*}
				n= rp + s, r \in \Z, 0<s<p
			\end{align*}
			Nach unserer Wahl von $p=n_0$ gilt $|s|=1$ und $|rp|<1$. Es folgt $\abs[n]=\text{max}\{|rp|,|s|\}=1$. Sei nun $n\in \Z$ beliebig. Wir schreiben $n=p^{v}n'$ mit $p\not | n'$ und sehen
			\begin{align*}
				\abs[n] = |p|^{v}|n'| = |p|^v = (|p|_p^{\alpha})^{v}=\abs[n]_p^{\alpha}.
			\end{align*}
			Mit den gleichen "Uberlegungen aus dem ersten Fall folgt damit die Behauptung.
		\end{enumerate}
	\end{proof}
\clearpage

\section{Lokale...}
\subsection{... K"orper}
	Wir betrachten im Folgenden alle 
	%Q_p topologischer Ring: done
	%Skalierung der Mengen durch multiplikation: done
	%Haarmass auf Q_p^\times: done
	%normierung des mult masses
	%annulus
	\begin{satz}\label{satz:QpIstLokalKompakt}
		F"ur alle Stellen $p\leq\infty$ gilt
		\begin{enumerate}[label=\emph{(\roman*)}]
		\item $(\Kp, +)$ ist eine lokalkompakte Gruppe.
		\item $(\K_p^\times, \cdot)$ ist eine lokalkompakte Gruppe.
		\end{enumerate}
	\end{satz}
	\begin{proof}
	%%% hier normierung des haar masses auf Qp
	
		Zu (i): Die Stetigkeit der Addition und der Negierung sind eigentlich direkte Folgen der Konstruktion der Vervollst"andigung.
		Als kleine Auffrischung zeigen wir sie trotzdem.
		Seien $x_n \to x$ und $y_n \to y$ konvergente Folgen in $\Kp$. 
		Es ist zu zeigen, dass $x_n+y_n$ gegen $x+y$ konvergiert. 
		Nach der Dreiecksungleichung gilt
		\begin{align*}
			\abs[(x_n+y_n) - (x+y)]_p = \abs[(x_n - x) + (y_n - y)]_p \leq \abs[(x_n - x)]_p + \abs[(y_n - y)]_p.
		\end{align*}
		Die rechte Seite konvergiert gegen $0$, also ist die linke eine Nullfolge und die Stetigkeit der Addition folgt.
		"Ahnlich zeigen wir $-x_n \to -x$, denn $\abs[(-x_n) - (-x)]_p = \abs[x_n-x]_p$.
		Damit ist $\Kp$ eine topologische Gruppe. 
		
		Da die Topologie von einer Metrik induziert wird, ist diese hausdorffsch.
		Wir m"ussen folglich nur noch die Lokalkompaktheit zeigen.
		Wegen der Stetigkeit der Addition reicht es dazu eine kompakte Umgebung der $0$ zu finden, denn sei $K$ eine kompakte Nullumgebung, so ist $a+K$ f"ur beliebige $a\in \Kp$ eine kompakte Umgebung von $a$.
		Wir behaupten, dass die Umgebung
		\begin{align*}
			\Zp = \{x \in \Kp : \abs[x]_p \leq 1\}
		\end{align*}
		kompakt ist.
		F"ur $p=\infty$ ist $\Z_\infty = [-1,1]$ also kompakt.
		F"ur $p<\infty$ gen"ugt es zu zeigen, dass $\Zp$ vollst"andig und totalbeschr"ankt ist. 
		Ersteres folgt daraus, dass $\Zp$ als abgeschlossene Menge des vollst"andigen Raumes $\Kp$ selber vollst"andig ist.
		Eine Menge heißt totalbeschr"ankt, wenn wir sie f"ur jedes $\varepsilon > 0$ mit endlich vielen $\varepsilon$-B"allen "uberdecken k"onnen.
		Wir k"onnen uns auf $\varepsilon$ der Form $p^{-k}$, $k\geq 0$ beschr"anken, da unsere Metrik nur den diskreten Wertebereich $p^\Z$ hat.
		Es ist $p^{k+1}\Zp$ eine Untergruppe von $\Zp$ und f"ur den Index gilt $[\Zp : p^{k+1}\Zp] = p^{k+1}$. Das bedeutet, dass die $p^{-k}$-B"alle
		\begin{align*}
			a + p^{k+1}\Zp = \{y\in \Zp : \abs[y-a]_p \leq p^{-k-1}\} = \{y\in \Zp : \abs[y-a]_p < p^{-k}\} = B_{p^{-k}}(a)
		\end{align*}
		mit $a=0\dots p^{k+1}-1$ die Menge $\Zp$ "uberdecken. Als vollst"andige und totalbeschr"ankte Menge ist $\Zp$ somit kompakt.
		
		
		Zu (ii): $\K_p^\times = \Kp\setminus \{0\}$ ist ein offener Teilraum von $\Kp$ und somit selbst wieder hausdorffsch und lokalkompakt.
		F"ur die Stetigkeit seien $x_n \to x$ und $y_n \to y$ zwei konvergente Folgen in $\K_p^\times$. 
		Wir zeigen, dass $x_ny_n$ gegen $xy$ konvergiert.
		Dies folgt aus der Absch"atzung
		\begin{align*}
			\abs[x_n y_n - xy]_p 
			&= \abs[x_n y_n -x_ny + x_n y- xy]_p  \\
			&\leq \abs[x_n y_n -x_ny]_p + \abs[x_n y- xy]_p \\
			&= \underbrace{\abs[x_n]_p}_{\text{beschr"ankt}} \underbrace{\abs[y_n -y]_p}_{\to 0} + \underbrace{\abs[x_n- x]_p}_{\to 0} \abs[y]_p \to 0
		\end{align*}
		Weiter zur Invertierung. Wegen
		\begin{align*}
			\abs[\frac{1}{x_n} - \frac{1}{x}]_p = \frac{\abs[x_n - x]_p}{\abs[x_nx]_p} \to 0
		\end{align*}
		folgt, dass $x_n^{-1}$ gegen $x^{-1}$ konvergiert und die Stetigkeit ist gezeigt.
	\end{proof}
	
	\begin{satz}\label{satz:multiplikativesHaarMass}
		F"ur jede messbare Menge $A\subset \Kp$ und jedes $x\in \Kp$ gilt
		\begin{align*}
			\mu(xA) = \abs[x]_p \mu(A).
		\end{align*}
		Insbesondere folgt f"ur jedes $f \in L^1(\Kp)$ und $x\not= 0$:
		\begin{align*}
			\int_\Kp f(x^{-1}y)d\mu(y) = \abs[x]_p \int_\Kp f(y)d\mu(y).
		\end{align*}
	\end{satz}
	\begin{proof}
		Sei $x\in \K_p^\times$. Die Funktion $\mu_x$ definiert durch
		\begin{align*}
			\mu_x (A) = \mu(xA)
		\end{align*}
		definiert wieder ein Haar-Maß auf $\Kp$ und unterscheidet sich daher nur durch Skalierung mit einer positiven Konstante $c>0$ von $\mu$.
		Ziel wird es nun sein $c=\abs[x]_p$ zu zeigen. 
		Dies ist klar im Fall $p=\infty$. 
		F"ur $p<\infty$ reicht es dank unserer Normierung $\mu(x\Zp) = \abs[x]_p$ zu zeigen.
		Sei dazu $\abs[x]_p = p^{-k}$.
		Dann ist $x=p^ky$ mit $y\in \Zp$ und $x\Zp = p^k\Zp$.
		Daher reduziert sich unsere Betrachtung auf $\mu(p^k\Zp) = p^{-k}$.
		Beginnen wir mit dem Fall $k\geq 0$. Dann ist $p^k\Zp$ eine Untergruppe von $\Zp$ und f"ur den Index gilt $[\Zp : p^k\Zp] = p^k$.
		Wir haben also eine disjunkte Zerlegung $\Zp = \bigsqcup_{a=0}^{p^{k}-1} a + p^k\Zp$.
		Aus der Translationsinvarianz folgern wir dann
		\begin{align*}
			1 = \mu (\Zp) = \sum_{a=0}^{p^{k}-1} \mu(a + p^k\Zp) =\sum_{a=0}^{p^{k}-1} \mu(p^k\Zp) = p^k \mu(p^k\Zp).
		\end{align*}
		Die Behauptung folgt dann durch einfaches Umformen. 
		Im anderen Fall $k<0$ ist umgekehrt $\Zp$ eine Untergruppe von $p^k\Zp$ mit Index $[p^k\Zp:\Zp]= p^{-k}$ und die Behauptung folgt analog.
	\end{proof}
	
	\begin{satz}\label{satz:lokal:multiplikativesmass}
		Ist $dx$ ein additives Haar-Maß auf $\Kp$, so definiert $\frac{dx}{\abs[x]_p}$ ein multiplikatives Haar-Maß $d^\times x$ auf $\K_p^\times$.
		Insbesondere gilt dann f"ur alle $g \in L^1(\K_p^\times)$
		\begin{align*}
			\int_{\K_p^\times} g(x) d^\times x = \int_{\Kp \setminus \{0\}} g(x) \frac{dx}{\abs[x]_p}
		\end{align*}
	\end{satz}
	\begin{proof}
		Wir haben bereits gezeigt, dass $\K_p^\times$ eine lokalkompakte Gruppe ist und folglich ein Haar-Maß besitzt.
		Wenn wir nun ein positives, lineares Funktional auf $C_c(\K_p^\times)$ angeben, erhalten wir nach Rieszschen Darstellungssatz ein Radonmaß, welches diesem Funktional entspricht. 
		Ist nun $g \in C_c(\K_p^\times)$, so ist $g\abs_p^{-1} \in C_c(\Kp\setminus\{0\})$. 
		Dies ist in der Tat eine eins-zu-eins Zuweisung.
		Wir definieren nun das Funktional
		\begin{align*}
			\Phi(g) = \int_{\Kp \setminus \{0\}} g(x) \frac{dx}{\abs[x]_p}.
		\end{align*}
		Dieses ist offensichtlich ein positives, nicht-triviales, lineares Funktional auf $ C_c(\K_p^\times)$. Es ist translationsinvariant, denn
		\begin{align*}
			\int_{\Kp \setminus \{0\}} g(y^{-1}x) \frac{dx}{\abs[x]_p} = \int_{\Kp \setminus \{0\}} g(x) \frac{\abs[y]_pdx}{\abs[yx]_p} = \int_{\Kp \setminus \{0\}} g(x) \frac{dx}{\abs[x]_p},
		\end{align*}
		folglich kommt es von einem Haar-Maß $d^\times x$. 
		Der zweite Teil der Behauptung folgt aus der Tatsache, dass die Funktionen in $C_c(\K_p^\times)$ dicht in $L^1(\K_p^\times)$ liegen.
		Beim "Ubergang zum Grenzwert erhalten wir die Gleichung auf Funktionen in $L^1$.
	\end{proof}
\subsection{... Fourieranalysis}
%hier den additiven charakter besprechen und kurz bedeutung erklaeren: done
%selbstdual:ist eigentlich fourierumkehrformel: done
%fouriertransformation: done
%schwartz-bruhat und deren form: done
%fourierumkehrformel: done
		F"ur die unendliche Stelle $p=\infty$ definieren wir die \emph{Schwartz-Bruhat Funktion} als eine komplexwertige, glatte Funktion $f$, die f"ur alle nicht-negativen ganzen Zahlen $n$ und $m$ die Bedingung
		\begin{align*}
			\sup_{x\in \K_\infty}\abs[x^n\frac{d^m}{dx^m}f(x)] < \infty
		\end{align*}
		erf"ullt.
		F"ur die endlichen Stellen $p<\infty$ definieren wir eine \emph{Schwartz-Bruhat Funktion} als eine lokal konstante Funktion mit kompakten Tr?ger.
		Die Menge aller solcher Funktionen bilden einen komplexen Vektorraum, den wir mit $\Sw(\K_p)$ bezeichnen. 
		Im Fall $p<\infty$ erkennt man leicht, dass $\Sw(\Kp)\subseteq L^1(\Kp)$. 
		F"ur $p=\infty$ gilt nach obiger Bedingung $(\abs[1]+\abs[x^2])\abs[f(x)] \leq C$, also $\abs[f(x)]\leq C(1+x^2)^{-1}$ und $(1+x^2)^{-1} \in L^1(\K\infty)$
		
		\begin{bsp}~ 
			\begin{enumerate}[label=(\roman*)]
				\item Im Fall $p=\infty$ ist die Funktion $f_k = x^k e^{-x^2}$ f"ur jedes $k\in\N_0$ in $\Sw(\K_\infty)$. 
				Die Ableitungen $\frac{d^m}{dx^m} f_k(x)$ sind von der Form $p(x)e^{-x^2}$, wobei $p(x)$ ein Polynom ist. 
				Aus der Analysis ist dann bekannt, dass $\abs[x^n p(x)e^{-x^2}]$ f"ur jedes $n\in \N_0$ beschr"ankt ist.
				\item Im Fall $p<\infty$ sind offensichtlich die charakteristischen Funktionen kompakter Mengen in $\Sw(\Kp)$. 
				Beispiele f"ur Kompakta sind Mengen der Form $a+p^k\Zp$ mit $a\in \K$ und $k\in \Z$.
			\end{enumerate}
		\end{bsp}
		
		\begin{lemma}\label{lemma:padischSBF}
			Jede Funktion $f\in \Sw(\Kp)$, $p<\infty$, ist eine endliche Linearkombination von charakteristischen Funktionen der Form $\ind_{a+p^k\Zp}$, wobei $a\in \K$ und $k\in \Z$
		\end{lemma}
		\begin{proof}
			Sei $f \in \Sw(\Kp)$. 
			Da $f$ lokal konstant ist, ist f"ur jedes $z\in\C$ das Urbild $f^{-1}(z)$ offen in $\Kp$. 
			Also ist $f^{-1}(0)$ offen, folglich $\Kp \setminus f^{-1}(0)$ abgeschlossen und daher schon $\text{supp}(f) = \Kp \setminus f^{-1}(0)$. 
			Per Definition hat die Schwartz-Bruhat Funktion $f$ kompakten Tr"ager, also ist $\Kp \setminus f^{-1}(0)$ kompakt. 
			Diese Menge wird von den offenen Mengen $f^{-1} (x)$ mit $x\not= 0$ "uberdeckt, wovon nach Kompaktheit schon endlich viele reichen.
			$f$ hat somit endliches Bild. Weiter ist jede offene Menge $f^{-1} (x)$ eine disjunkte Vereinigung offener B"allen in $\Kp$. 
			Diese haben aber genau die gesuchte Form $a+p^k\Zp$ wie oben. 
			Aufgrund der Kompaktheit, reichen wieder endliche viele solcher B"alle. Damit folgt auch schon das Lemma.
		\end{proof}
		
		Bevor wir mit etwas klassischer Fourieranalysis beginnen wollen wir den Kreis zur abstrakten Variante schließen.
		Daf"ur definieren wir z"unachst f"ur jede Stelle $p$ einen nicht-trivialen, unit"aren Charakter $e_p$ auf der additiven Gruppe $\K_p^+$ und zeigen anschließend, dass wir jeden beliebigen Charakter $\psi \in \widehat{\K}_p^+$ durch $e_p$ darstellen k"onnen.
		\begin{defi}
			Wir definieren den \emph{Standardcharakter} $e_p:\K_p^+ \to S^1$ wie folgt:
			\begin{itemize}[itemindent=3em]
				\item [$p=\infty$:] $e_p(x) = \exp(-2\pi i x)$
				\item [$p<\infty$:] Hier wird die Definition etwas aufwendiger. Zun"achst haben wir eine nat"urliche Projektion $\Kp \twoheadrightarrow \Kp/\Zp$.
				Die "Aquivalenzklassen von $\Kp/\Zp$ werden nach unseren "Uberlegungen zur Potenzreihendarstellung eindeutig repräsentiert durch Zahlen der Form $\sum_{k=-n}^{-1} a_kp^k$ mit $a_k \in \Z$ und $0\leq a_k\leq p-1$.
				Wir definieren den stetigen Homomorphismus $\Kp/\Zp \to \K/\Z$ indem wir diese Repräsentanten als summen in $\K$ interpretieren. 
				Zu guter Letzt schicken wir diese Summe in $\K/\Z$ durch $e: \$\K/\Z \to S^1, x \mapsto \exp(2\pi i x)$ in den Einheitskreis. 
				Alle diese Abbildungen sind stetige Gruppenhomomorphismen, bilden also selber wieder einen stetigen Gruppenhomomorphismus, den wir mit $e_p$ bezeichnen.
				Interpretieren wir die Potenzreihenentwicklung der $p$-adischen Zahlen als nicht unbedingt konvergente Summe in $\Q$, so k"onnen wir (etwas unsch"on) schreiben
				\begin{align*}
					e_p\left(\sum_{k=-n}^{\infty} a_kp^k\right) = \exp\left(2\pi i \sum_{k=-n}^{\infty} a_kp^k\right) = \exp\left(2\pi i \sum_{k=-n}^{-1} a_kp^k\right)
				\end{align*}
			\end{itemize}
		\end{defi}
		Sei $\psi$ ein unitärer Charakter auf $\K_p^+$, $p<\infty$ und $U$ eine offene Umgebung der $1 \in S^1$, die nur die triviale Untergruppe $1$ enth"alt. 
		Aufgrund der Stetigkeit von $\psi$ gibt es dann eine offene Umgebung $V$ der $0\in\K_p^+$ mit $\psi(V)\subseteq U$. 
		Ohne Einschr"ankung ist $V$ eine Untergruppe der Form $p^k\Zp$ f"ur ein $k\in\Z$. 
		Dann ist aber $\psi(V)$ eine Untergruppe von $S^1$ und somit gleich $1$.
		Die kleinste Untegruppe $p^k\Zp$ nennen wir den \emph{Konduktor} von $\psi$.
	
		
		\begin{lemma}
			Jeder unit"are Charakter $\psi: \Kp \to S^1$ ist von der Form $x \mapsto e_p(ax)$ f"ur ein $a \in \Kp$.
		\end{lemma}
		\begin{proof}
			Der Beweis teilt sich auf in die F"alle $p=\infty$ und $p<\infty$.
			Zuerst zu $p=\infty$: Sei $\psi: \K_\infty \to S^1$ ein Charakter.
			Wegen der Stetigkeit von $\psi$ existiert ein $\varepsilon > 0$, so dass $\psi((-\varepsilon, \varepsilon)) \subset \{z\in S^1: \Re(z)>0\}$.
			W"ahlen wir $\varepsilon$ noch etwas kleiner k"onnen wir sogar $\psi([-\varepsilon, \varepsilon)]) \subset \{z\in S^1: \Re(z)>0\}$ garantieren.
			Wir definieren nun $a$ als das eindeutig bestimmte Element aus $[-\frac{1}{4\varepsilon},\frac{1}{4\varepsilon}]$
			\footnote{Also der Logarithmuszweig, dass $-\pi/2<\varepsilon a<\pi/2$}, so dass $\psi (\varepsilon) = \exp(2\pi i a \varepsilon)$.
			Als n"achstes behaupten wir, dass auch
			\begin{align*}
				\psi \left(\frac{\varepsilon}{2}\right) =  \exp\left(2\pi i  a \frac{\varepsilon}{2}\right)
			\end{align*}
			gilt.
			Wegen $\psi (\frac{\varepsilon}{2})^2 = \psi (\varepsilon) = \exp(2\pi i a \varepsilon)$ ist $\psi (\frac{\varepsilon}{2}) = \pm \exp(2\pi i  a \frac{\varepsilon}{2})$.
			Da aber $\psi (\frac{\varepsilon}{2})$ nach der Wahl von $\varepsilon$ positiven Realteil hat, kommt nur $ \exp(2\pi i  a \frac{\varepsilon}{2})$ in Frage.
			Durch Iteration des Arguments erhalten wir $\psi (\frac{\varepsilon}{2^n}) = \exp(2\pi i  a \frac{\varepsilon}{2^n})$ f"ur $n\in \N_0$.
			
			Setzen wir jetzt $\varepsilon = 2^{-n_0}$ f"ur ein geeignetes  $n_0\in\N$.
			Dann ist $\varepsilon^{-1}$ eine nat"urliche Zahl und f"ur beliebige $k\in\Z$ haben wir
			\begin{align*}
				\psi \left(\frac{k} {2^{n}}\right) &= \psi \left(\frac{k\varepsilon} {2^{n}\varepsilon}\right) 
										= \psi \left(\frac{\varepsilon} {2^{n}}\right) ^{\frac{k}{\varepsilon}} 
										\\&= \exp\left(2\pi i  a \frac{\varepsilon}{2^n}\right) ^{\frac{k}{\varepsilon}}
										= \exp\left(2\pi i  a \frac{k}{2^n}\right)
			\end{align*}
			Die Menge aller $\frac{k}{2^n}$ mit $k\in \Z$ und $n\in \N_0$ liegt nun aber dicht in $\R$ und wir k"onnen aus der Stetigkeit $\chi(x) = \exp(2\pi i a x)$ schließen.
			Um die Eindeutigkeit von $a$ zu sehen, reicht es die Ableitung  von $x \mapsto \exp(2\pi i a x)$ zu berechnen und an der Stelle $x=0$ auszuwerten.
			Der Wert betr"agt gerade $2\pi i a$.
			
			Kommen wir nun zum Fall $p<\infty$. 
			Sei $\psi$ ein unit"arer Charakter auf $\K_p^+$ und sei $p^k\Zp$ dessen Konduktor.
			F"ur $k\leq0$ gilt offensichtlich $\psi(\Zp)=1$.
			Ohne Beschr"ankung der Allgemeinheit betrachten wir nur solche Charaktere, denn im Fall $k>0$ k"onnen wir auch den Charakter $x\mapsto \psi(p^kx)$ betrachten und die Aussage folgt aus $\psi(p^kx) = \psi(x)^{(p^k)}$.
			
			Wir suchen nun ein geeignetes $a\in\Kp$. 
			Dabei f"allt uns zun"achst auf, dass der Charakter bereits eindeutig durch seine Werte fur $p^{-k}$, $k \in \N$ bestimmt wird.
			Da $\psi(\Zp)=1$ gilt n"amlich
			\begin{align*}
				\psi \left( \sum_{k=-n}^\infty x_k p^k \right) = \sum_{k=-n}^{-1} \psi \left(p^k \right)^{x_k}.
			\end{align*}
			Es reicht also ein geeignetes $a$ f"ur diese Potenzen zu finden.
			Schauen wir uns $\psi{p^{-1}}$ genauer an, so erkennen wir, dass dies eine $p$-te Einheitswurzel sein muss.
			Damit ist $\psi(p^{-1}) = \exp(2\pi i \frac{a_1}{p})$ f"ur ein eindeutig bestimmtes nat"urliches $0\leq a_1 \leq p - 1$.
			Analog argumentieren wir auch $\psi(p^{-k}) = \exp(2\pi i \frac{a_k}{p^k})$ mit $0\leq a_k \leq p^k - 1$.
			Zudem gilt 
			\begin{align*}
				\exp\left(2\pi i \frac{a_{k+1}}{p^{k}}\right) = \exp\left(2\pi i \frac{a_{k+1}}{p^{k+1}}\right)^{p}= \psi(p^{-k-1})^p = \psi(p^{-k}) = \exp\left(2\pi i \frac{a_k}{p^k}\right).
			\end{align*}
			F"ur unsere Folge heißt das aber gerade $a_k \equiv a_{k+1} \pmod{p^k}$.
			Nach unseren "Uberlegunen zu den Potenzreihen definiert eine solche Folge aber gerade eine eindeutig bestimmte $p$-adische Zahl $a$ mit $a \equiv a_k \pmod{p^k}$ und es gilt
			\begin{align*}
				e_p\left( \frac{a}{p^{k}} \right) = \exp\left( 2\pi i \frac{a_k}{p^k} \right) = \psi \left( \frac{1}{p^k} \right)
			\end{align*}
			Damit sind wir dann aber auch schon fertig.
		\end{proof}
		Der Beweis liefert uns sogar einen kleinen Bonus.
		\begin{korollar}\label{kor:lokal:charTrivialZp}
			Wirkt $\chi:\Kp\to S^1$ im endlichen Fall trivial auf $\Zp$, so gilt $\chi(x) = e_p(ax)$ mit $a\in \Zp$.
		\end{korollar}
		\begin{proof}
			Wie wir im Beweis der Potenzreihendarstellung gesehen haben, konvergiert eine solche Folge von rationalen Zahlen $a_k$ gegen einen Wert aus $\Zp$.
		\end{proof}
		Somit k"onnen wir auch mit gutem Gewissen Folgendes definieren.
		\begin{defi}[Lokale Fouriertransformation]
			Sei $f\in L^1(\Kp)$. Wir definieren dann die \emph{Fouriertransformation} $\hat{f}: \Kp \to \C$ von $f$ durch die Formel
		\begin{align*}
			\hat{f}(\xi) = \int_{\Kp} f(x)e_p(-\xi x)  \dx
		\end{align*}
		f"ur alle $\xi \in \Kp$.
		\end{defi}
		Diese Definition entspricht im Fall $p=\infty$ gerade der klassischen Fouriertransformation bis auf eventuelle Normierung.
		Von daher m"ochten wir auch einige klassische Ergebnisse festhalten und beweisen.
		\begin{lemma}
			Sei $f \in L^1(\Kp)$.
			\begin{enumerate}[label=\emph{(\alph*)}]
				\item Ist $g(x)=f(x)e_p(ax)$ mit $a\in\Kp$, dann gilt $\hat{g}(x) = \hat{f}(x-a)$.
				\item Ist $g(x)=f(x-a)$ mit $a\in\Kp$, dann gilt $\hat{g}(x) = \hat{f}(x-a)e_p(-ax)$.
				\item Ist $g(x)=f(\lambda x)$ mit $\lambda \in\Kp^\times$, dann gilt $\hat{g}(x) =\frac{1}{\abs[\lambda]_p} \hat{f}(\frac{x}{\lambda})$.
			\end{enumerate}
		\end{lemma}
		\begin{proof}
			(a) und (b) sind einfache Folgerungen aus der Definition mit der Multiplikativit"at von $e_p$ und der Translationsinvarianz des Haar-Maß. 
			Bei (c) spielt unsere Normeriung des Absolutbetrags eine Rolle, denn mit dem Variablenwechsel $y\mapsto \lambda^{-1}y$ erhalten wir
			\begin{align*}
				\hat{g}(x) = \int_{\Kp} f(\lambda y) e_p(-xy)dy = \frac{1}{\abs[\lambda]_p} \int_{\Kp} f(y) e_p(-x\lambda^{-1}y)dy = \frac{1}{\abs[\lambda]_p} \hat{f}\left(\frac{x}{\lambda}\right)
			\end{align*}
		\end{proof}
		Beschr"anken wir uns jetzt auf die Fouriertransformation von Schwartz-Bruhat Funktionen.
		Diese besitzen, wie wir noch feststellen werden, besonders gutes Konvergenzverhalten. 
		So ist zum Beispiel folgendes Ergebnis bereits aus der Fourieranalysis in $\R$ bekannt und kann jetzt auf die $p$-adischen Zahlen "ubertragen werden.
		\begin{satz}\label{satz:lokal:umkehrformel}
			Ist $p\leq\infty$ und $f\in\Sw(\Kp)$, so ist $\hat{f} \in \Sw(\Kp)$ und es gilt die \emph{Umkehrformel}
			\begin{align*}
				\hat{\hat{f}}(x) = f(-x)
			\end{align*}
		\end{satz}
		\begin{proof}
			Betrachten wir zuerst den Fall $p<\infty$. Wie wir eben in Lemma \ref{lemma:padischSBF} gesehen haben haben, ist jede Funktion in $\Sw(\Kp)$ eine Linearkombination von Funktionen der Form $f = \ind_{a+p^k\Zp}$. Es reicht also die Aussage f"ur solche $f$ zu zeigen.
			Sei dazu $h:= \ind_{\Zp}$. Wir zeigen $\hat{h} = h$ durch folgende Rechnung
			\begin{align*}
				\hat{h}(x) = \int_\Kp h(y) e_p (-xy) dy_p = \int_\Zp e_p(-xy) dy_p.
			\end{align*}
			Nun ist $\chi(y):=e_p(-xy)$ ein Charakter auf $\Zp$ und genau dann trivial, wenn $x\in\Zp$. 
			Weiter ist $\Zp$ kompakt. 
			Nach Lemma \ref{Lemma:trivialerCharAufKompakt} und unserer Normierung von $dy_p$ folgt also
			\begin{align*}
				\hat(h)(x) = \text{Vol}(\Zp, dy_p) \ind_\Zp = \ind_\Zp = h(x)
			\end{align*}
			
			Wir f"uhren nun folgende Operatoren auf $\Sw(\Kp)$ ein
			\begin{align*}
				L_a f(x) = f(x-a), M_\lambda f(x) = f(\lambda x),
			\end{align*}
			wobei $a \in \Kp$ und $\lambda \in \Kp^\times$. 
			Nun k"onnen wir $f$ schreiben als $L_a M_{p^{-k}}h$. 
			Es folgt
			\begin{align*}
				\hat{f} = (L_a M_{p^{-k}}h)\widehat{\phantom{x}} = \Omega_{-a}p^{k}M_{p^k}\hat{h}=\Omega_{-a}p^{-k}M_{p^k}h.
			\end{align*}
			Also ist $\hat{f} (x) = e_p(-ax)p^k\ind_{p^{-k}\Zp}(x)$. 
			Der Charakter $e_p$ ist lokal konstant, $\hat{f}$ als das Produkt lokal konstanter Funktionen selbst wieder lokal konstant und damit in $\Sw(\Kp)$. 
			Damit haben wir den ersten Teil der Aussage gezeigt.
			
			F"ur den zweiten Teil sehen wir
			\begin{align*}
				\hat{\hat{f}} = (L_a M_{p^{-k}}h)\widehat{\widehat{\phantom{x}}} = L_{-a} (M_{p^k}h)\widehat{\widehat{\phantom{x}}}=L_{-a}M_{p^k}\hat{h} =L_{-a}M_{p^k}h,
			\end{align*}
			also $\hat{\hat{f}} (x) = \ind_{-a+p^k\Zp} (x) = \ind_{a+p^k\Zp} (-x) = f(-x)$. 
			Hier haben wir $p^k\Zp = - p^k\Zp$ ausgenutzt. 
			Damit haben wir die Umkehrformel f"ur den $p$-adischen Fall gezeigt. 
			F"ur $p=\infty$ ist die Formel bereits aus der klassischen Fourieranalysis bekannt.
		\end{proof}
		
\subsection{... Funktionalgleichung}
%unverzweigte charactere: done
%deren form: done
%lokale funktionalgleichung: done
%FILLER
	Die Einheiten $K_p^\times$ der lokalen K"orper $\Kp$ k"onnen dargestellt werden als direktes Produkt $\dedekind_p^\times \times V(\Kp)$, wobei $\dedekind_p^\times$ die Untergruppe der Elemente von $\K_p^\times$ mit Absolutbetrag $1$ und 
	\begin{align*}
		V(\Kp) := \abs[\K_p^\times]
	\end{align*}
	der Wertebereich des Absolutbetrags auf den Einheiten ist. 
	Wir haben n"amlich einen stetigen Homomorphismus $\tilde{\cdot}: x \mapsto \frac{x}{\abs[x]_p}$ von $\K_p^\times$ nach $\dedekind_p^\times$.\\
	F"ur $p=\infty$ ist $\dedekind_p^\times = \{-1, 1\}$ und $V(\K_p) = \R_+^\times$. 
	Jedes $x \in \Kp$ hat gerade Form $x=\text{sgn}(x)\abs[x]_p$, denn $\tilde{x}$ ist gerade die Signumsabbildung.\\
	Wenn $p<\infty$ ist $\dedekind_p^\times = \Z_p^\times$, $V(\K_p) = p^\Z$ und wir k"onnen jedes Element $x \in \K_p^\times$ schreiben als $x = \abs[x]_p\tilde{x}$.
	Es wird nun von Interesse sein, wie die multiplikativen Charaktere auf die Untergruppe $\dedekind_p^\times$ wirken. Dazu zun"achst eine kleine Definition.
	\begin{defi}
		Ein Charakter $\chi \in \text{Hom}_\text{cont}(\K_p^\times, \C^\times)$ ist \emph{unverzweigt}, wenn er trivial auf die Untergruppe $\dedekind_p^\times$ wirkt.
	\end{defi}
	Die unverzweigten Charaktere haben eine recht einfache Form, was folgendes Lemma zeigt.
	\begin{lemma}
		Jeder unverzweigte Charakter $\chi$ auf $\K_p^\times$ hat die Form $\chi(x) = \abs[x]_p^s$ mit $s\in\C$.
	\end{lemma}
	\begin{proof}
		Es ist klar, dass Funktionen dieser Form tats"achlich unverzweigte Quasi-Charaktere sind.
		Umgekehrt sei $\chi$ ein unverzweigter Quasi-Charakter. Dann gilt $\chi(x) = \chi(\abs[x]_p \tilde{x}) = \chi(\abs[x]_p)$.
		Dadurch induziert $\chi$ eine stetige Abbildung auf dem Wertebereich $V(\Kp)$. Wir zeigen, dass diese Abbildung gerade die Form $t\mapsto t^s$ hat.
		
		Sei zuerst $p=\infty$, also $V(\Kp) = \R_+^\times$. Wir definieren $s:= \log(\chi(e))$, also $\chi(e) = e^s$.
		Induktiv l"asst sich nun leicht $\chi(e^n) = e^{ns}$ f"ur ganze Zahlen $n\in\Z$ zeigen. 
		Analog zeigt man 
		\begin{align*}
			\chi(e^{\frac{n}{m}})^m = \chi(e^{m\frac{n}{m}}) =\chi(e^n) = e^{ns},
		\end{align*}
		woraus
		\begin{align*}
			\chi(e^{\frac{n}{m}}) = \left(\chi(e^{\frac{n}{m}})^m\right)^{\frac{1}{m}} = (e^{ns})^\frac{1}{m} = e^{\frac{n}{m}s}
		\end{align*}
		folgt, so dass wir $\chi(e^q) = e^{qs}$ f"ur alle rationalen Zahlen $q\in\Q$ haben. 
		Wegen Stetigkeit gilt nach "Ubergang zu Grenzwerten $\chi(e^r) = e^{rs}$ f"ur alle reellen $r \in \R$, also $\chi(t)=t^s$ f"ur alle $t\in \R_+^\times$.
		
		Der Fall $p<\infty$ ist etwas leichter. Wir definieren dieses mal $s:=\frac{\log(\chi(p))}{\log(p)}$, so dass $\chi(p) = p^s$. Da der Wertebereich aber gerade $p^\Z$ war, folgt die Behauptung sofort.
	\end{proof}
	%%% Filler: allgemeine charaktere sehen dann so aus blabla
	\begin{satz}\label{satz:lokal:stdchar}
		Jeder Charakter $\chi$ von $\K_p^\times$ hat die Form
		\begin{align*}
			\chi(x) = \mu(\tilde{x})\abs[x]_p^s,
		\end{align*}
		wobei $\mu$ ein unit"arer Charakter auf $\dedekind_p^\times$, $\tilde\cdot$ der stetige Homomorphismus von $\K_p^\times$ nach $\dedekind_p^\times$ und $s\in\C$ ist.
	\end{satz}
	\begin{proof}
		Es ist wieder klar, dass $\mu(\tilde{\cdot})\abs_p^s$ tats"achlich ein Charakter ist. 
		Betrachten wir nun einen beliebigen Charakter $\chi$ und definieren $\mu$ als die Einschr"ankung von $\chi$ auf $\dedekind_p^\times$. 
		Da die Untergruppe $\dedekind_p^\times$ kompakt und $\mu$ eine stetige Abbildung nach $\C^\times$ ist, muss $\mu(\dedekind_p^\times)$ eine kompakte Untergruppe von $\C^\times$ sein und ist damit in $S^1$ enthalten. 
		Folglich ist $\mu$ ein unit"arter Charakter auf $\dedekind_p^\times$.
		Damit definiert der stetige Homomorphismus $x\mapsto \chi(x)\mu(\tilde{x})^{-1}$ einen unverzweigten Charakter auf $\K_p^\times$, hat also nach vorherigem Lemma die Form $\chi(x)\mu(\tilde{x})^{-1} = \abs[x]_p^s$ f"ur ein $s\in\C$. Der Satz folgt sofort.
	\end{proof}
	Aus $\abs[\mu(\tilde{x})\abs[x]_p^s] = \abs[x]_p^\sigma$ folgt, dass der Realteil $\sigma=\text{Re}(s)$ eindeutig bestimmt ist. Er wird auch \emph{Exponent} des Charakters $\chi$ genannt.
		%%%Filler: wie sehen charaktere auf R aus (vllt auch Qp) wann sind sie aquivalent was ist eine klasse von charakteren
	Wir kommen zum ersten Ergbnis aus Tates Doktorarbeit
	
	\begin{satz}[Lokale Funktionalgleichung]
		Sei $f_p \in \Sw(K_p)$ und $\chi = \mu \abs_p^s$. Sei weiter $\sigma = \text{Re}(s)$. Dann gelten die folgenden Aussagen:
		\begin{enumerate}[label=\emph{(\roman*)}]
			\item $Z(f,\chi) = Z(f, \mu, s)$ ist holomorph und absolut konvergent f"ur $\sigma > 0$
			\item Auf dem Streifen $0 < \sigma < 1$ haben wir eine Funktionalgleichung
				\begin{align*}
					Z(\hat{f}, \check{\chi}) = \gamma(\chi, e_p, dx) Z(f,\chi),
				\end{align*}
				wobei $\gamma(\chi, \psi, dx)$ unabh"angig von $f$ und meromorph als Funktion in $s$ ist. Damit besitzt $Z(f,\chi)$ eine meromorphe Fortsetzung auf ganz $\C$
		\end{enumerate}
	\end{satz}
	\begin{proof}
		(i) Es reicht im Allgemeinen zu zeigen, dass das Integral
		\begin{align*}
			\int_{\Kp \setminus \{0\}} \abs[f] \cdot \abs[x]_p^{\sigma-1} dx
		\end{align*}
		endlich ist, denn das Haar-Maß $d^\times x$ ist ein konstantes vielfaches von $dx$.
		
		Sei zun"achst $p=\infty$. 
		Wir k"onnen $\abs[f_p]$  absch"atzen durch ein skalares Vielfaches von $\frac{1}{1+\abs[x]_\infty^n}$, wobei $n \in \N$ mit $n > \sigma$.
		Zusammen haben wir dann
		\begin{align*}
			\int_{\Kp \setminus \{0\}} \abs[f] \cdot \abs[x]_p^{\sigma-1} dx \leq C\cdot \int_{\Kp} \frac{\abs[x]_p^{\sigma-1}}{1+\abs[x]_\infty^n}dx < \infty.
		\end{align*}
		%%%EVTL TODO: Erklaerung warum endlich: Dazu aufteilen in kompaktum K und R\K. Einmal stetig kompakt also beschr, andere mal abschaetzen durch 1/x^2
		Im endlichen Fall
		
		%%%TODO
		
		(ii) Wir folgen Tate und beweisen ein kleines Lemma.
		\begin{lemma}
			F"ur alle Charaktere $\chi$ mit Exponenten $0<\sigma<1$ und beliebige Funktionen $f,g \in \Sw(\Kp)$ gilt:
			\begin{align*}
				Z(f, \chi) Z(\hat g, \check{\chi}) = Z(\hat f, \check{\chi}) Z(g, \chi) 
			\end{align*}
		\end{lemma}
		\begin{proof}
			Nach (i) haben wir absolute Konvergenz der Integrale f"ur Exponenten $\sigma > 0$. 
			Zudem ist $\check{\chi} = \abs[x]_p\chi^{-1} = \abs[x]_p^{1-s} \mu^{-1}$, also haben wir in diesem Fall Konvergenz f"ur $\sigma <1$.
			Damit sind die obigen Zeta-Funktionen wohldefiniert auf dem Streifen den wir betrachten.
			Wir schreiben das Produkt als Doppelintegral "uber $\K_p^\times \times \K_p^\times$
			\begin{align*}
				Z(f, \chi) Z(\hat g, \check{\chi}) 
				&= \iint\limits_{\K_p^\times \times \K_p^\times} f(x)\chi(x) \hat{g}(y)\chi^{-1}(y)\abs[y]_p d^\times (x,y) \\
				&= \iint\limits_{\K_p^\times \times \K_p^\times} f(x) \hat{g}(y)\chi(xy^{-1})\abs[y]_p d^\times (x,y)
			\end{align*}
			Das Integral ist invariant unter der Translation $(x,y)\mapsto (x,xy)$ und wir erhalten
			\begin{align*}
				\iint\limits_{\K_p^\times \times \K_p^\times} f(x) \hat{g}(xy)\chi(y^{-1})\abs[xy]_p d^\times (x,y).
			\end{align*}
			Nach Fubini ist das wiederum gleich
			\begin{align*}
				\int_{\K_p^\times} \left( \int_{\K_p^\times} f(x) \hat{g}(xy) d^\times x \right) \chi(y^{-1})\abs[y]_p d^\times y.
			\end{align*}
			Wir m"ussen also nur noch zeigen, dass das innere Integral symmetrisch in $f$ und $g$ ist.
			Dazu erinnern wir uns, dass $d^\times x = c \frac{dx}{\abs[x]_p}$ und nach der Definition der Fouriertransformation daher
			\begin{align*}
				\int_{\K_p^\times} f(x) \hat{g}(xy) d^\times x  
				= c \int_{\Kp}  \int_{\Kp} f(x) g(z) e_p(-xyz) dz dx = \int_{\K_p^\times} g(z) \hat{f}(zy) d^\times z,
			\end{align*}
			wobei wieder Fubini das Vertauschen der Reihenfolge bei der Integration erlaubt.
		\end{proof}
		Damit sind wir auch schon fast mit dem eigentlichen Beweis fertig. Wir fixieren eine geeignete Schwartz-Bruhat Funktion $g\in \Sw(\Kp)$ und setzen
		\begin{align*}
			\gamma(\chi, e_p, dx) := \frac{Z(\hat g, \check{\chi})}{Z(g, \chi)}.
		\end{align*}
		Aus dem Lemma folgt, dass dieser Quotient sicherlich unabh"angig von der Wahl von $g$ ist, und, dass
		\begin{align*}
			Z(\hat f, \check{\chi}) = \gamma(\chi, e_p, dx) Z(f, \chi).
		\end{align*}
		%%%%TODO: FILLER laber darueber dass es solche funktionen gibt, dass diese meromorph sind und dass wir die faktoren explizit ausrechnen
	\end{proof}
\subsection{... Berechnungen}

\subsubsection{Der Fall \texorpdfstring{$p = \infty$}{p gleich unendlich}}
	Wir betrachten zuerst die Klasse $\chi = \abs_\infty^s$ und nehmen die Schwartz-Bruhat Funktion
	\begin{align*}
		f(x) = e^{-\pi x^2}.
	\end{align*} 
	Wir behaupten, dass $f$ ihre eigene Fouriertransformierte ist. Dazu rechnen wir
	\begin{align*}
		\hat{f}(\xi) 	&= \int_{\K_\infty} e^{-\pi x^2} e_\infty(-x\xi)dx 
					= \int_{\K_\infty} e^{-\pi x^2} e^{2\pi ix\xi}dx
					\\&= \int_{\K_\infty} e^{-\pi (x^2 - 2ix\xi - \xi^2)} e^{-\pi \xi^2}dx
					= f(\xi) \int_{\K_\infty} e^{-\pi (x - i\xi)^2} dx.
	\end{align*}
	Es gen"ugt also zu zeigen, dass das Integral gleich $1$ ist. Dazu nutzen wir das bekannte Integral $\int_{\K_\infty} e^{-\pi x^2} dx = 1$ und etwas Kontourintegration.
	Sei $\gamma$ das Rechteck von $-r$ nach $r$ auf der reellen Achse, dann runter zu $r-i\xi$, horizontal zu $-r-i\xi$ und wieder zur"uck zu $-r$.
	Da $f$ eine ganze Funktion ist, gilt $\int_\gamma f(z) dz = 0$. 
	Die Integrale an der linken und rechten Seite des Rechtecks konvergieren gegen $0$ wenn $r$ anw"achst, denn f"ur $z = \pm r - iy$ und $0\leq y\leq \xi$ gilt
	\begin{align*}
		\abs[f(z)] = \abs[e^{-\pi (\pm r-iy)^2}] = e^{-\pi (r^2 - y^2)}
	\end{align*}
	und wir haben die Absch"atzung
	\begin{align*}
		\abs[\int_{\pm r}^{\pm r-i\xi} f(z)dz] = \abs[\int_{0}^{\xi} f(\pm r - iy)dy] \leq e^{-\pi r^2} \int_{0}^{\xi} e^{\pi y^2}dy \xrightarrow[]{r\to \infty} 0.
	\end{align*}
	Folglich muss schon $\int_{\K_\infty} f(x-i\xi) dx = \int_{\K_\infty} f(x)dx = 1$ gelten und wir sind fertig mit der Fouriertransformation.
	
	Nun zu den Zeta-Funktionen:
	\begin{align*}
		Z(f, \chi) = Z(f, \abs_\infty^s) = \int_{\Kinfx} f(x) \abs[x]_p^s \dxx 
							= \int_{\R^\times} e^{-\pi x^2} \abs[x]_\infty^s \dxx 
							= 2 \int_0^\infty e^{-\pi x^2} x^{s-1} \dx
	\end{align*}
	Wir benutzen den Trafo $u =\pi x^2 \Rightarrow du = 2\pi^{1/2}u^{1/2}$ und erhalten
	\begin{align*}
		Z(f, \chi) &= \int_0^\infty e^{-u}(u\pi^{-1})^\frac{s-1}{2} \pi^{-\frac{1}{2}} u^{-\frac{1}{2}} du\\	
							&= \pi^{-\frac{s}{2}} \int_0^\infty e^{-u} u^{\frac{s}{2} -1}du = \pi^{-\frac{s}{2}} \Gamma\left(\frac{s}{2}\right)
	\end{align*}
	Mit dem gleichen Argumentation rechnen wir auch
	\begin{align*}
		Z(\hat{f}, \check{\chi}) = Z(f, \abs_\infty^{1-s}) = \pi^{-\frac{1-s}{2}} \Gamma\left(\frac{1-s}{2}\right).
	\end{align*}
	Jetzt k"onnen wir endlich den versprochenen Faktor
	\begin{align*}
		\gamma(\abs_\infty^s, e_\infty, \dx) = \frac{\pi^{-\frac{1-s}{2}} \Gamma\left(\frac{1-s}{2}\right)}{\pi^{-\frac{s}{2}} \Gamma\left(\frac{s}{2}\right)}
	\end{align*}
	angeben und sehen, dass dieser auf dem Streifen $0<\sigma<1$ als Quotient holomorpher Funktionen meromorph ist.

	Nun zur zweiten und auch schon letzten Klasse $\chi = \sgn \abs_\infty^s$ von multiplikativen Charakteren auf $\Kinf$. 
	Wir w"ahlen die Funktion 
	\begin{align*}
		f_\pm (x) = x e^{-\pi x^2} \in \Sw(\K_\infty)
	\end{align*}
	und bemerken zun"achst die Beziehung $f_\pm(x) = (-2\pi)^{-1} f'(x) $.
	Damit k"onnen wir die Fouriertransformation schnell aus einem Ergebnis der klassischen Fourieranalysis gewinnen.
	Denn wir haben
	\begin{align*}
		\hat{f}_\pm(\xi) 	&= \int_{-\infty}^\infty f_\pm (x) e_\infty(-x\xi)\dx
							 = \int_{-\infty}^\infty (-2\pi)^{-1} f'(x) e_\infty(-x\xi)\dx\\
							&= \left[ (-2\pi^{-1} f(x) e_\infty(-x\xi)  \right]_{-\infty}^\infty 
								- \int_{-\infty}^\infty -2\pi^{-1} f(x) \cdot (-2\pi i \xi) e_\infty(-x\xi)\dx&\\
							&= 0 - i \xi \hat{f}(\xi) = -i \xi f(\xi) = -i f_\pm(\xi).
	\end{align*}
	Wir berechnen die Zeta-Funktionen
	\begin{align*}
		Z(f_\pm, \chi) 	&= Z(f, \sgn\abs_\infty^s) 
						= \int_{\Kinfx} f_\pm(x) \sgn(x)\abs[x]_\infty^s \dxx \\
						&= \int_{\Kinfx} x f(x) \sgn(x) \abs[x]_\infty^s \dxx
						= \int_{\Kinfx} f(x) \abs[x]_\infty^{s+1} \dxx \\
						&= Z(f, \abs_\infty^{s+1}) = \pi^{-\frac{s+1}{2}}\Gamma\left(\frac{s+1}{2}\right)
	\end{align*}
	und mit $\check{\chi} = \sgn^{-1}\abs_\infty^{1-s} = \sgn\abs_\infty^{1-s} $
	\begin{align*}
		Z(\hat{f}_\pm, \chi) 	&= Z(-i f_\pm, \sgn\abs_\infty^{1-s}) 
						= -i \int_{\Kinfx} f_\pm(x) \sgn(x)\abs[x]_\infty^{1-s} \dxx \\
						&= -i \int_{\Kinfx} x f(x) \sgn(x) \abs[x]_\infty^{1-s} \dxx
						= -i \int_{\Kinfx} f(x) \abs[x]_\infty^{2-s} \dxx \\
						&= -i Z(f, \abs_\infty^{2-s}) = -i \pi^{-\frac{2-s}{2}}\Gamma\left(\frac{2-s}{2}\right).
	\end{align*}
	Damit haben wir den Faktor
	\begin{align*}
		\gamma(\sgn\abs_\infty^s, e_\infty, \dx) = i\frac{\pi^{-\frac{2-s}{2}} \Gamma\left(\frac{2-s}{2}\right)}{\pi^{-\frac{s+1}{2}} \Gamma\left(\frac{s+1}{2}\right)}
	\end{align*}
	der nach der gleichen Begr"undung wie oben meromorph ist.
\subsubsection{Der Fall \texorpdfstring{$p < \infty$}{p kleiner unendlich}}
	Wir beginnen wieder mit dem unverzweigten Fall $\chi = \abs_p$ und definieren unsere Schwartz-Bruhat Funktion
	\begin{align*}
		f_0(x) = \ind_{\Zp}(x).
	\end{align*}
	Wie auch schon im archimedischen Fall werden wir feststellen, dass $f_0$ ihre eigene Fouriertransformierte ist.
	Wir rechnen
	\begin{align*}
		\hat{f}_0 (\xi) = \int_{\Kp} f_0(x) e_p(-x\xi) \dx = \int_{\Zp} e_p(-x\xi) \dx.
	\end{align*}
	Nun wissen wir, dass $\Zp$ kompakt ist und der Charakter $e_p(-x\xi)$ genau dann trivial auf $x\in\Zp$ wirkt, wenn auch $\xi \in \Zp$.
	In diesem Fall entpricht das Integral nach Lemma \ref{Lemma:trivialerCharAufKompakt} gerade dem Volumen von $\Zp$ bez"uglich $\dx$.
	Ansonsten verschwindet es.
	Mit unserer Normierung des Haar-Maßes folgt dann
	\begin{align*}
		\hat{f}_0 (\xi) = \int_{\Zp} e_p(-x\xi) \dx = \text{Vol}(\Zp, \dx) \ind_\Zp(\xi) = \ind_\Zp (\xi) = f_0(\xi).
	\end{align*}
	F"ur die Berechnungen der Zeta-Funktionen erinnern wir uns an die Beobachtung $\Zp\setminus\{0\} = \bigcup_{k=0}^\infty p^k\Zpx$ als disjunkte Vereinigung.
	\begin{align*}
		Z(f_0, \chi) 	&= Z(f_0, \abs_p^s) 
						= \int_{\Kpx} f_0(x) \abs[x]_p^s \dxx 
						= \int_{\Zp\setminus\{0\}} \abs[x]_p^s \dxx 
						\\&= \sum_{k=0}^{\infty} \int_{p^k\Zpx} \abs[x]_p^s \dxx
						= \sum_{k=0}^{\infty} \int_{\Zpx}  \abs[p^kx]_p^s \dxx
						\\&= \sum_{k=0}^{\infty} \int_{\Zpx}  p^{-ks} \dxx
						= \sum_{k=0}^{\infty} p^{-ks} \text{Vol}(\Zpx, \dxx)
						= \frac{1}{1-p^{-s}}
	\end{align*}
	und analog
	\begin{align*}
		Z(\hat{f}_0, \check{\chi}) 	= Z(f_0, \abs_p^{1-s})	= \frac{1}{1-p^{s-1}}.
	\end{align*}
	Der Faktor hat damit die Form
	\begin{equation*}
		\gamma(\abs_p^s, e_p, \dx) = \frac{1-p^{-s}}{1-p^{s-1}}
	\end{equation*}
	und ist insbesondere somit holomorph im betrachteten Streifen.
	
	Nun kommen wir zum verzweigten Fall $\chi = \mu \abs_p^s$.
	Bevor wir allerdings mit den eigentliche Berechnungen anfangen, schauen wir uns den unit"aren Charakter $\mu:\Zpx\to S^1$ etwas genauer an.
	W"ahlen wir eine offene Umgebung $U$ der $1 \in S^{1}$, die nur die triviale Untergruppe enth"alt, so finden wir aufgrund der Stetigkeit von $\mu$ eine offene Umgebung $V$ der $1 \in \Zpx$ mit $\mu(V)\subseteq U$.
	Diese enthalten aber stets eine Untergruppe der Form $1+p^n\Zp$.
	Da $\mu$ aber ein Gruppenhomomorphismus ist, muss diese Untergruppe bereits auf $1$ abgebildet werden.
	Es gibt also f"ur jeden Charakter $\chi = \mu \abs_p^s$ ein kleinstes $n\in\N$ mit $\mu(1+p^{n}\Zp) = 1$.
	Wir nennen dann $p^n$ den \emph{Konduktor von $\chi$}.
	Analog l"asst sich auch der Konduktor eines additiven Charakters definieren.
	Mit Hilfe des Konduktors $p^n$ definieren wir nun die Schwartz-Bruhat Funktion
	\begin{align*}
		f_n(x) = e_p(x)\ind_{p^{-n}\Zp}(x).
	\end{align*}
	Die Berechnung der Fouriertransformation erfolgt "ahnlich zum unverzweigten Fall:
	\begin{align*}
		\hat{f}_n(\xi) 	= \int_{\Kp} f_n(x) e_p(-x\xi)\dx 
						= \int_{p^{-n}\Zp} e_p\left(x(1-\xi)\right)\dx
	\end{align*}
	Der Charakter $\psi(x) = e_p(x(1-\xi))$ wirkt genau dann trivial auf $p^{-n}\Zp$, wenn $1-\xi \in p^n\Zp$, oder "aquivalent $\xi \in 1+p^n\Zp$.
	Es folgt 
	\begin{align*}
		\hat{f}_n(\xi) 	= \text{Vol}(p^{-n}\Zp, dx) \ind_{1+p^n\Zp}(\xi) =p^n \ind_{1+p^n\Zp}(\xi).
	\end{align*}
	blablabla zeta funktion
	\begin{align*}
		Z(f_n, \chi) &= Z(f_n, \mu\abs_p^s) 	
											= \int_{\Kpx} f_n(x) \mu(\tilde{x}) \abs[x]_p^s \dxx
											= \int_{p^{-n}\Zp\setminus\{0\}} e_p(x) \mu(\tilde{x}) \abs[x]_p^s \dxx
											\\&= \sum_{k=-n}^\infty  \int_{p^k\Zpx} e_p(x) \mu(\tilde{x}) \abs[x]_p^s \dxx
											= \sum_{k=-n}^\infty  \int_{\Zpx} e_p(p^k x) \mu(\widetilde{p^k x}) \abs[p^kx]_p^s \dxx
											\\&= \sum_{k=-n}^\infty p^{-ks} \int_{\Zpx} e_p(p^k x) \mu(x) \dxx.
	\end{align*}
	Ein Integral der Form $g(\omega, \lambda) = \int_{\Zpx} \omega(x)\lambda(x) \dxx$ mit multiplikativen Charakter $\omega: \Zpx \to S^1$ und additiven Charakter $\lambda: \Zp \to S^1$ wird \emph{Gauß-Summe} genannt.
	Mit $e_{p,k} (x) = e_p(p^kx)$ haben wir dann
	\begin{align}\label{eq:ZetaSumme}
		Z(f_n, \chi) = \sum_{k=-n}^\infty p^{-ks} g(\mu,e_{p,k}).
	\end{align}
	F"ur die weitere Berechnung definieren wir $U_k:= 1+p^k\Zp$ und beweisen ein kleines Lemma "uber Gauß-Summen.
	\begin{lemma}\label{lemma:gausssumme}
		Seien $\gamma$ und $\lambda$ wie oben. 
		Seien $n$ und $r$ die kleinsten Zahlen mit $\gamma(1+p^n\Zp) = 1$ und $\lambda(p^r\Zp)=1$, d.h. $p^n$ und $p^r$ sind die Konduktoren von $\gamma$ bzw. $\lambda$.
		Es gelten folgende Aussagen:  
		\begin{enumerate}[label=(\roman*)]
			\item Wenn $n>r$, dann $g(\omega,\lambda) = 0$. \label{lemma:gausssummei}
			\item Wenn $n=r$, dann $\abs[g(\omega,\lambda]^2 = c^2p^{-n}$.
			\item Wenn $n<r$, dann $\abs[g(\omega,\lambda]^2 = c^2\left[p^{-n} - p^{-r}\right]$.
		\end{enumerate}
	\end{lemma}
	%Wir werden Aussage (iii) hier nicht brauchen und verweisen daher f"ur den Beweis auf Ramakrishnan \cite{rama} Lemma 7-4.
	\begin{proof}
		F"ur (i) zerlegen wir $\Zpx$ in die Nebenklassen, die von $U_r=1+p^r\Zp$ erzeugt werden.
		Ein generisches Element aus $aU_r$ hat die Form $a(1+p^rb)$, wobei $a$ ein Repräsentant der Nebenklasse und $b$ aus $\Zp$ ist. 
		Wir haben $\lambda(a(1+p^rb)) = \lambda(a)\lambda(p^rab) = \lambda(a)$ nach der Definition des Konduktors.
		Daher %%%REpresentantensystem ueberlegen
		\begin{align*}
			g(\omega,\lambda) = \sum_{aU_r}  \int_{aU_r} \omega(x)\lambda(x) \dxx = \sum_{aU_r} \omega(a)\lambda(a) \int_{U_r} \omega(x)\dxx.
		\end{align*}
		Da aber $n>r$ wirkt $\omega$ nicht trivial auf $U_r$ und somit verschwindet das Integral.
		
		Weiter zur Aussage (ii) und (iii): Sei also $n\leq r$. Wir rechnen
		\begin{align*}
			\abs[g(\omega,\lambda)]^2 	&= \int_{\Zpx} \omega(x)\lambda(x)\dxx \cdot \conj{\int_{\Zpx} \omega(x)\lambda(x)\dxx}\\
										&= \int_{\Zpx} \int_{\Zpx} \omega(xy^{-1})\lambda(x-y) \dxx\dxx[y]\\
										&= \int_{\Zpx} \omega(x)h(x)\dxx
		\end{align*}
		wobei wir im letzten Schritt zum einen die Translation $x \mapsto xy$ und zum anderen die Funktion
		\begin{align*}
			h(x) = \int_{\Zpx} \lambda(xy-y)) \dxx[y] = c \int_{\Zpx} \lambda(y(x-1)) \frac{\dx[y]}{\abs[y]_p} = c \int_{\Zpx} \lambda(y(x-1))\dx[y]
		\end{align*}
		eingef"uhrt haben. Wegen $\Zpx = \Zp - p\Zp$ k"onnen wir das Integral weiter aufspalten.
		\begin{align*}
			h(x) =  c \int_{\Zp - p\Zp} \lambda(y(x-1))\dx[y] = c \int_{\Zp} \lambda(y(x-1))\dx[y] - c \int_{p\Zp} \lambda(y(x-1))\dx[y].
		\end{align*}
		Nun haben wir wieder den Fall von Lemma \ref{Lemma:trivialerCharAufKompakt}. 
		$y\mapsto \lambda(y(x-1))$ ist trivial auf $\Zp$ genau dann wenn $x-1 \in p^r\Zp$, d.h. wenn $x \in U_r$.
		"Ahnlich verh"alt es sich mit $y\mapsto \lambda(y(x-1))$ auf $p\Zp$, wobei dieser genau dann trivial ist, wenn $x\in U_{r-1}$.
		Es gilt also
		\begin{align*}
			h(x) 	=&  c \Vol(\Zp,\dx)\ind_{U_r} - c \Vol(p\Zp, \dx) \ind_{U_{r-1}} \\
					=& c \Vol(\Zp,\dx)\ind_{U_r} - c p^{-1} \Vol(\Zp, \dx) \ind_{U_{r-1}}
		\end{align*}
		Einf"ugen in $\abs[g(\omega,\lambda)]^2$ ergibt dann
		\begin{align*}
			\abs[g(\omega,\lambda)]^2 	&= \int_{\Zpx} \omega(x)h(x)\dxx \\
										&= c\Vol(\Zp,\dx) \int_{U_r} \omega(x)\dxx - c p^{-1}\Vol(\Zp, \dx) \int_{U_{r-1}} \omega(x)\dxx.
		\end{align*}
		Im Fall (ii) haben wir $n=r$. Damit ist der erste Integrand trivial, der zweite jedoch nicht.
		Folglich verschwindet das zweite Integral.
		Wir haben
		\begin{align*}
			\abs[g(\omega,\lambda)]^2 =  c\Vol(\Zp,\dx)\Vol(U_n,\dxx)
		\end{align*}
		und sind somit fertig.
		Im Fall (iii) ist $n<r$. Beide Integranden sind trivial und es folgt mit
		\begin{align*}
			\abs[g(\omega,\lambda)]^2 =  c\Vol(\Zp,\dx)\Vol(U_r,\dxx) - cp^{-1}\Vol(\Zp, \dx)\Vol(U_{r-1}, \dxx)
		\end{align*}
		die Behauptung. Damit ist das Lemma bewiesen.
	\end{proof}
	Zur"uck zur Berechnung der Zeta-Funktion.
	Der multiplikative Charakter $\mu$ hat den Konduktor $p^n$, w"ahrend die additiven Charaktere $e_{p,k}(x) = e_p(p^kx)$ offensichtlich den Konduktor $p^{-k}$ haben.
	Nach Lemma \ref{lemma:gausssumme} (i) verschwinden in \ref{eq:ZetaSumme} fast alle Summanden und wir erhalten
	\begin{align*}
		Z(f_n, \chi) = \sum_{k=-n}^\infty p^{-ks} g(\mu,e_{p,k}) = p^{ns} g(\mu,e_{p,-n})
	\end{align*}
	Die verbleibende Gauß-Summe konvergiert dann nach Aussage (ii) des Lemmas.
	
	F"ur die Berechnung der zweiten Zeta-Funktion bemerken wir zun"achst, dass $\mu^{-1}= 1/\mu = \conj{\mu}/(\mu \conj{\mu}) = \conj{\mu}$ den gleichen Konduktor wie $\mu$ hat.
	\begin{align*}
		Z(\hat{f}_n, \check{\chi}) 	&= Z(\hat{f}_n, \conj{\mu}\abs_p^{1-s})
									= p^n \int_{1+p^n\Zp}  \conj{\mu}(\tilde{x}) \abs[x]_p^{1-s} \dxx
									\\&= p^n \int_{1+p^n\Zp} \dxx
									= p^n c \int_{p^n\Zp} \dx
									= c.
	\end{align*}
	Zu guter Letzt der erhalten wir den holomorphen Faktor
	\begin{align*}
		\gamma(\mu\abs_p^s, e_p, \dx) = \frac{c}{p^{ns} g(\mu,e_{p,-n})} = \frac{cp^{-ns} \conj{g(\mu,e_{p,-n})}}{c^2p^{-n}} = c^{-1} p^{n(1-s)} \conj{g(\mu,e_{p,-n})}
	\end{align*}
	%%%TODO: volumen berechnunen fuer U_n einmal explizit am anfang.
\clearpage

\section{Eingeschr"ankte direkte Produkt abstrakter Gruppen}
	Ziel dieser Sektion ist es, die n"otigen Grundlagen zu schaffen um alle bisherigen lokalen fourieranalytischen Berechnung in einer einzigen globalen fourieranalytischen Berechnung zu vereinen. 
	Wir werden erkl"aren, warum der Ans"atze mit bekannten Objekten schiefgehen und f"uhren dann das eingeschr"ankte direkte Produkt und die dazugeh"orige Topologie ein.
	Anschließend schauen wir uns die Besonderheiten der Charaktere und der Integration an. 
	Wir halten uns dabei an Ramakrishnans und Valenzas Definitionen in \cite{rama}, die sich wiederum an Tates Eigenen orientieren.
	F"ur eine etwas alternative Definition siehe zum Beispiel Deitmar \cite{deitmar2010}
\subsection{Definitionen}\label{kapitel:RDP}
		Beginnen wir mit einer (endlichen, abz"ahlbaren, "uberabz"ahlbaren) Indexmenge $I$ und zu jedem $\nu\in I$ haben wir eine lokalkompakte Gruppen $G_\nu$.
		Gesucht ist zun"achst ein Objekt $G$, welches alle $G_\nu$ umfasst und auf dem wir Fourieranalysis betreiben k"onnen.
		$G$ sollte also eine lokalkompakte Gruppe sein.
		Ein erster Ansatz wäre natürlich das direkte Produkt der Gruppen $G_\nu$ zu bilden, allerdings zeigt folgendes Lemma, dass dieser Versuch im allgemeinen Fall fehlschlagen wird.
		\begin{lemma}\label{Lemma:lokalkompaktProd}
			Sei $I$ eine Indexmenge und $X_\nu$ ein lokalkompakter Hausdorff-Raum für alle $\nu \in I$. Der Raum $X:=\prod_{\nu \in I} X_\nu$ ist genau dann lokalkompakt, wenn fast alle $X_\nu$ kompakt sind.
		\end{lemma}
		Bevor wir den Beweis nach Deitmar \cite{deitmar2010} geben noch eine kurze Beobachtung: Ist $X$ kompakt, so ist auch jedes $X_\nu$ kompakt als Bild von $X$ unter den stetigen Projektionen $\pi_\nu:X \to X_\nu$.
		\begin{proof}
			Sei $E \subseteq I$ eine endliche Teilmenge und für jedes $\nu\in E$ sei $U_\nu \in X_\nu$ eine offene Menge. Wir betrachten die offenen Rechtecke
			\begin{align*}
				\prod_{\nu\in E} U_\nu \times \prod_{\nu\in I\setminus E} X_\nu,
			\end{align*}
			welche eine Basis der Produkttopologie bilden. 
			Ist $X$ lokalkompakt, so gibt es ein offenes Rechteck, dessen Abschluß kompakt ist. 
			Folglich sind fast alle $X_\nu$ kompakt. 
			Die Rückrichtung ist eine Folgerung des Satzes von Tychonov, der besagt, dass das direkte Produkt beliebiger Familien kompakter Mengen wieder kompakt ist, und der Tatsache, dass endliche Produkte lokalkompakter Räume wieder lokalkompakt sind (vgl. Lemma \ref{satz:topo:lcaproduct}).
		\end{proof}
		Das direkte Produkt lokalkompakter Gruppen liefert uns daher im Allgemeinen keine neue lokalkompakte Gruppe. 
		Wir sehen jetzt aber, welche Einschr"ankung n"otig ist, damit wir doch lokalkompakt werden.
		%%%%%%%%%%%%%%%%%%
		%   DEFINITION   %
		%%%%%%%%%%%%%%%%%%
		\begin{defi}[Eingeschränkte direkte Produkt]
			Sei $I=\{v\}$ eine Indexmenge und für jedes $v \in I$ sei $G_v$ eine lokalkompakte Gruppe. 
			Sei weiter $I_\infty \subseteq I$ eine endliche Teilmenge von $I$ und für jedes $v \notin I_\infty$ sei $H_v\leq G_v$ eine kompakte offene Untergruppe. 
			Das \emph{eingeschränkte direkte Produkt} der $G_v$ bezüglich $H_v$ ist definiert als 
			\begin{align*}
				G = \rdprod{v \in I} G_v := \{ (x_v): x_v \in G_v \text{ und } x_v \in H_v \text{ für alle bis auf endlich viele } v \}.
			\end{align*}
			mit komponentenweiser Verknüpfung. Die Topologie auf $G$ ist gegeben durch die \emph{eingeschränkte Produktopologie}. 
			Diese wird erzeugt durch die Basis der \emph{eingeschränkten offenen Rechtecke}
			\begin{align*}
				\prod_{\nu\in E} U_\nu \times \prod_{\nu\in I\setminus E} H_\nu,
			\end{align*}
			wobei $E \subset I$ eine endliche Teilmenge mit $I_\infty \subset E$ und $U_\nu \in G_\nu$ offen für alle $\nu\in E$ ist.
		\end{defi}
		G ist damit (gruppentheoretisch) zwischen der direkten Summe und dem direkten Produkt der Komponenten $G_\nu$ anzusiedeln.
		Topologisch entspricht die eingeschränkte Produkttopologie jedoch nicht der vom direkten Produkt induzierten Teilraumtopologie.
		Sie ist im Allgemeinen feiner, was zur Folge hat, dass die \emph{Projektion}
		\begin{align*}
			\pi_\nu: G &\to G_\nu\\
					g&\mapsto g_\nu
		\end{align*}
		auf die $\nu$-te Komponenten von $G$ eine stetige Abbildungen ist.
		
		Schauen wir uns einige Untergruppen des eingeschr"ankten direkten Produkts an.
		Die Gruppen $G_\nu$ k"onnen "uber die stetige \emph{Inklusion} 
		\begin{align*}
			\iota_\nu: G_\nu &\to G \\
					g &\mapsto (\dots,1,g,1,\dots)
		\end{align*}
		auf nat"urliche Weise in $G$ eingebettet werden und bilden damit eine Familie von abgeschlossenen Untergruppen.
		
		Sei $S$ eine endliche Teilmenge von $I$, die $I_\infty$ enth"alt. 
		Wir definieren die offene Untergruppe
		\begin{align*}
			G_S := \prod_{\nu\in S}G_\nu \times \prod_{\nu\in I\setminus S} H_\nu
		\end{align*}
		von $G$. 
		Bezüglich der Produkttopologie sind diese $G_S$ nach Lemma \ref{lemma:direktesProduktTopologischerGruppen} und Lemma \ref{Lemma:lokalkompaktProd} selbst wieder lokalkompakte Gruppen.
		Die Produkttopologie stimmt aber mit der durch $G$ induzierten Teilraumtopologie "uberein und die $G_S$ bilden eine Familie von lokalkompakten Untergruppen in $G$
		Nun liegt jeder Punkt $x \in G$ in einer Untergruppe dieser Form und daher folgt sofort, dass $G$ selbst eine lokalkompakte Gruppe ist.
		Damit haben wir einen geeignet Kandidaten gefunden, der uns hoffentlich die globale Kalkulation erm"oglicht.
		
		Zum Abschluss dieser Untersektion richten wir unseren Blick auf die kompakten Mengen des eingeschr"ankten direkten Produkts.
		\begin{satz}%Kompakte Mengen in G_S
			Eine Teilmenge $Y$ von $G$ hat genau dann kompakten Abschluss, wenn $Y \subseteq \prod{K_\nu}$ für eine Familie von kompakten Teilmengen $K_\nu \subseteq G_\nu$ mit $K_\nu = H_\nu$ für fast alle Indizes $i$.
		\end{satz}
		\begin{proof}
			Die Rückrichtung ist klar, denn jede abgeschlossene Teilmenge eines kompakten Raumes ist wieder kompakt. 

			Für die Hinrichtung sei nun $K$ der Abschluss von $Y$ und kompakt in $G$. 
			Da die Untergruppen $G_S$ eine offene "Uberdeckung von $G$ bilden, gibt es eine endliche Familie $\{G_{S_n}\}$, die $K$ überdecken. 
			Wir können sogar noch mehr sagen. 
			Da die die $S_k$ endlich sind, ist $S = \bigcup S_k$ endlich, also wird $K$ sogar von nur einem $G_S$ überdeckt. 
			Sei $K_\nu$ das Bild von $K$ der natürlichen Einbettung nach $G_\nu$. 
			Da die Topologie auf $G_S$ gerade der Produkttopologie entspricht und $K\subseteq G_S$ ist diese Abbildung stetig und $K_\nu$ damit kompakt als stetiges Bild einer kompakten Menge. 
			Außerdem ist $K_\nu \subseteq H_\nu$ für alle $i\notin S$, sodass wir hier $K_\nu$ durch die kompakten $H_\nu$ ersetzen können. 
			Dann ist $Y\subseteq K \subseteq \prod{K_\nu}$ und wir sind fertig.
		\end{proof}
		\begin{korollar}
			Jede kompakte Teilmenge $K$ von $G$ liegt in einer der Untergruppen $G_S$.
		\end{korollar}
 
\subsection{(Quasi-)Charaktere}%TODO Lemma genauer, beweis schoener
		Machen wir uns einige Gedanken "uber die (Quasi-)Charaktere auf dem eingeschr"ankten direkten Produkt.
		Jede stetige Abbildung $f$ auf $G$ induziert durch die Inklusion $\iota_\nu: G_\nu \to G$ eine stetige Abbildung  $f_\nu = f \circ \iota_\nu$ auf der Komponente $G_\nu$.
		Analog ist $f_\nu$ ein Homomorphismus auf $G_\nu$, wenn $f$ selbst ein Homomorphismus auf $G$ ist.
		Damit lassen sich die Quasi-Charaktere auf $G$ wie folgt charakterisieren.
		\begin{lemma}\label{lemma:rdp:char}
			Sei $\chi:G \to \C^\times$ ein (Quasi-)Charakter. 
			Dann sind alle $\chi_\nu: G_\nu \to \C^\times$ (Quasi-)Charaktere und wirken fast alle trivial auf $H_\nu$
			Folglich haben wir fast "uberall $\chi_\nu (g_\nu) = 1$ mit $g=(g_\nu)\in G$ und es gilt die Produktformel
			\begin{align*}
				\chi(g) = \prod_\nu \chi_\nu(g_\nu).
			\end{align*}
		\end{lemma}
		\begin{proof}
			Das alle $\chi_\nu$ (Quasi-)Charaktere sind folgt aus unseren Vor"uberlegungen.
			Wir müssen also nur noch zeigen, dass fast alle $\chi_\nu$ trivial auf die Untergruppen $H_\nu$ wirken. 
			Dazu wählen wir uns eine offene Umgebung $V$ der $1$ in $\C^\times$, die nur die triviale Untergruppe $\{1\}$ enthält. 
			Aufgrund der Stetigkeit von $\chi$ finden wir eine offene Umgebung $U=\prod_\nu U_\nu$ der $1$ in $G$ mit $U_\nu = H_\nu$ für alle $i$ außerhalb einer endlichen Indexmenge $S$ und $\chi(U)\subseteq V$.
			Dann gilt aber
			\begin{align*}
				(\prod_{\nu\in S} 1) \times (\prod_{i \notin S} H_\nu) \subseteq U 
			\end{align*}
			und daher
			\begin{align*}
				\chi((\prod_{\nu\in S} 1) \times (\prod_{i \notin S} H_\nu)) \subseteq V 
			\end{align*}
			Die linke Seite ist aber als Bild einer Gruppe unter einem Homomorphismus selbst wieder eine Gruppe. 
			Nach unserer Wahl von $V$ folgt also
			\begin{align*}
				\chi((\prod_{\nu\in S} 1) \times (\prod_{i \notin S} H_\nu)) = \{1\}.
			\end{align*}
			Folglich $\chi_\nu (H_\nu) = \{1\}$ für alle $i\notin S$. 
			Damit ist aber klar, dass für jedes $g \in G$ das Produkt $\prod_\nu \chi_\nu(g_\nu)$ endlich ist und wegen $g = \prod_{\nu} \iota_\nu(g_\nu)$ genau $\chi(g)$ entspricht.
			
		\end{proof}
		Wir k"onnen das Lemma aber auch umdrehen und Quasi-Charaktere auf $G$ durch solche auf $G_\nu$ konstruieren.
		\begin{lemma}
			Seien $\chi_\nu: G_\nu \to \C^\times$ (Quasi-)Charaktere und nehmen wir an, dass fast alle trivial auf $H_\nu$ wirken.
			Dann ist
			\begin{align*}
				\chi(g) = \prod_\nu \chi_\nu(g_\nu)
			\end{align*}
			ein (Quasi-)Charakter auf $G$.
		\end{lemma}
		\begin{proof}
			Das Produkt $\chi$ ist sicherlich wohldefiniert und bildet ein Gruppenhomomorphismus auf $G$. 
			Es bleibt noch zu zeigen, dass $\chi$ stetig ist. 
			Da $G$ und $\C^\times$ topologische Gruppen sind, genügt es sich offene Umgebungen der $1$ anzuschauen. 
			Sei daher $U$ eine offene Umgebung der $1 \in \C^\times$.
			Sei $S$ die endliche Menge aller Indizes, so dass $\chi_\nu$ nicht trivial auf $H_\nu$ wirkt, und setze $n = \abs[S]$. 
			Wir finden eine weitere Umgebung $W$ der $1$ in $\C^\times$, so dass das Produkt von $n$ beliebigen Elementen aus $W$ wieder in $U$ liegt. 
			Da die $\chi_\nu$ stetig sind, finden wir offene Umgebungen $V_\nu$ der $1 \in G_\nu$ mit $\chi_\nu(V_\nu) \subseteq W$. 
			Für $\nu\in S$ können wir ohne Probleme $V_\nu = H_\nu$ setzen. 
			Dann ist $V = \prod_\nu V_\nu$ eine offene Umgebung der $1 \in G$ und für jedes $g \in V$ ist $\chi(g)$ das endliche Produkt von $n$ Faktoren aus $W$, also $\chi(g) \in V$.		
		\end{proof}
	
\subsection{Integration}
		Wie wir gesehen haben ist $G=\rdprod{\nu\in I} G_\nu$ eine lokalkompakte Gruppe, besitzt also nach Satz \ref{satz:topo:haarmeasure} ein Haar-Maß.
		Dieses m"ochten wir abh"angig von den lokalen Maßen normalisieren.
		\begin{satz}
			Sei $G=\rdprod{\nu\in I} G_\nu$ das eingeschränkte direkte Produkt einer Familie lokalkompakter Gruppen $G_\nu$ bezüglich der kompakten Untergruppen $H_\nu \subseteq G_\nu$.
			Bezeichne $dg_\nu$ das Haar-Maß auf $G_\nu$ mit der Normalisierung
			\begin{align*}
				\Vol(H_\nu, dg_\nu) = \int_{H_\nu} dg_\nu = 1
			\end{align*}
			für fast alle $\nu$. 
			Dann gibt es ein eindeutiges Haar-Maß $dg$ auf G, so dass für jede endliche Teilmenge $S\supseteq I_\infty$ der Indexmenge $I$ die Einschränkung $dg_S$ von $dg$ auf $G_S$ genau das Produktmaß ist.
		\end{satz}
		
		\begin{proof}
			Die im Satz angesprochene Normalisierung der $dg_\nu$ ist möglich, da per Definition die Untergruppen $H_\nu$ offen und kompakt sind und daher positives und endliches Maß haben.
			
			Sei $S$ nun eine beliebige Menge wie im Satz beschrieben und definiere $dg_S$ als das Produktmaß $dg_S :=\left(\prod_{s \in S}dg_\nu\right) \times dg^S$, wobei $dg^S$ das Haar-Maß auf der kompakten Gruppe $G^S:=\prod_{i \notin S} H_\nu$, sodass $\Vol(G^S, dg^S) = 1$. 
			F"ur die Existenz von $dg^S$ siehe zum Beispiel Folland \cite{folland} Kapitel 7, Satz 7.28. 
			Als endliches Produkt von Haar-Maßen ist $dg_S$ selbst wieder Haar-Maß und wir können das Maß $dg$ auf $G$ so normieren, dass dessen Einschränkung auf $G_S$ mit $dg_S$ übereinstimmt.
			Damit haben wir eine Normierung, zun"achst noch abh"angig von unserer Wahl der Indexmenge $S$.
			Das macht allerdings nichts.
			Denn sei $T\supseteq S$ eine weitere endliche Indexmenge. 
			Per Definition ist $G_S$ eine Untergruppe von $G_T$. 
			Wir müssen jetzt nur noch zeigen, dass die Einschränkung von $dg^T$ auf $G^S$ mit $dg^S$ übereinstimmt.
			Man erkennt, dass $G^S = \left(\prod_{\nu\in T \setminus S} H_\nu\right) \times G^T$.
			Daher bildet $\left(\prod_{\nu\in T \setminus S} dg_\nu\right) \times dg^T$ ein Haar-Maß, welches der kompakten Gruppe $G_S$ das oben geforderte Maß $1$ zuweist.
			Aus der Eindeutigkeit des Haar-Maßes auf (lokal)kompakten Gruppen folgt somit die Gleichheit zu $dg^S$.
			Sei nun $S'$ eine beliebige weitere Indexmenge, die $I_\infty$ enthält. 
			Das normierte Maß $dg$ wird auf $G_{S\cup S'}$ eingeschränkt zu einem Maß, welches ein konstantes Vielfaches von $dg_{S\cup S'}$ ist. 
			Da aber $G_S \subseteq G_{S\cup S'}$ muss diese Konstante $1$ sein, denn nach obigen "Uberlegungen ist die Einschränkung von $dg_{S\cup S'}$ auf $G_S$ gerade $dg_{S}$.
			Umgekehrt ist aber $dg_{S'}$ die die Einschränkung von $dg_{S\cup S'}$ auf $G_{S'}$, also ist die Normalisierung unabhängig von der Wahl unserer Indexmenge $S$.
		\end{proof}
		%Wir schreiben manchmal $\prod_{i} dg_\nu$ für das wie oben normierte Maß $dg$.
		Als n"achstes lernen wir, wie man einfache Funktionen auf $G$ bez"uglich $dg$ integriert.
		\begin{proposition}\label{prop:integrieren}
			Sei $G$ das eingeschränkte direkte Produkt mit dem induzierten Maß $dg$
			\begin{enumerate}[label=\emph{(\roman*)}]
				\item Sei $f \in L(G)$ eine integrierbare Funktion auf $G$. Dann gilt
					\begin{align*}
						\int_G f(g)dg = \lim_S \int_{G_S} f(g_S) dg_S,
					\end{align*}
					wobei $S$ über alle endlichen Indexmengen läuft, die $I_\infty$ enthalten.
				\item Sei $S_0$ ein beliebige endliche Indexmenge, die $I_\infty$ und alle $i$ enthält, für die $\text{Vol}(H_\nu, dg_\nu) \not= 1$. 
					Für jeden Index $i$ haben wir eine stetige Funktion $f_\nu$ auf $G_\nu$, so dass $f_\nu |_{H_\nu} = 1$ für alle $i \notin S_0$. 
					Wir definieren
					\begin{align*}
						f(g) = \prod_{i}f_{i}(g_\nu),
					\end{align*}
					für $g=(g_\nu) \in G$. 
					Dann ist $f$ wohldefiniert und stetig auf $G$. 
					Sind die $f_\nu$ sogar integrierbar und ist $S$ eine weitere endliche Indexmenge, die $S_0$ enthählt, haben wir
					\begin{align}\label{eq:satz:integration}
						\int_{G_S} f(g_S) dg_S = \prod_{\nu\in S}\Bigl(\int_{G_\nu} f_\nu (g_\nu)dg_\nu\Bigr).
					\end{align}
					Ist das Produkt
					\begin{align*}
						\prod_{\nu\in S}\Bigl(\int_{G_\nu} f_\nu (g_\nu)dg_\nu\Bigr)
					\end{align*}
					sogar endlich, dann ist $f$ insbesondere integrierbar und es gilt
					\begin{align*}
						\int_{G} f(g) dg = \prod_{\nu\in S}\Bigl(\int_{G_\nu} f_\nu (g_\nu)dg_\nu\Bigr).
					\end{align*}	
			\end{enumerate}
		\end{proposition}
		\begin{proof}
			(i) Wir halten z"unachst fest was "uberhaupt mit dem Ausdruck $\lim_S \phi(S) = \phi_0$ f"ur eine Funktion $\phi$ auf den endlichen Indexmengen $S$ mit Werten in einem abgeschlossenen topologischen Raum gemeint ist: Gegeben eine beliebige Umgebung $V$ von $\phi_0$, dann gibt es eine Indexmenge $S(V)$, sodass f"ur alle weiteren Indexmengen $S \supseteq S(V)$ der Wert $\phi(S)$ in $V$ liegt. Intuitiv ist also  $\lim_S \phi(S)$ der Limes von $\phi(S)$ "uber gr"oßer und gr"oßere $S$.
			
			Zur"uck zum Beweis der Aussage.
			Aus der Integrationstheorie (z.B. Folland \cite{folland} Kapitel 7 Korollar 7.13) ist bekannt, dass %vllt auch nicht TODO
			\begin{align*}
				\int_{G} f(g) = \sup_{K \text{ kompakt}} \left\{ \int_{K} f(g)dg \right\}.
			\end{align*}
			Da aber jedes solches $K$ in einer der Mengen $G_S$ liegt, folgt die Gleichung sofort.
			
			\noindent(ii) Aus der Bedingung $f_\nu |_{H_\nu} = 1$ folgt, dass das Produkt $f(g) = \prod_{i}f_{i}(g_\nu)$ für alle $g \in G$ endlich, und die Funktion damit wohldefiniert ist. Eine Umgebung von $g$ ist gegeben durch ein offenes beschränktes Rechteck. Diese liegen in einem der $G_S$ (versehen mit der Produkttopologie) und wir können ohne Einschränkung $S$ um alle Indizes $i$ mit $f_\nu|_{H_\nu}\not= 1$ vergrößern.
			Lokal betrachtet ist $f$ also ein endliches Produkt stetiger Funktionen auffassen und daher $f$ selber stetig.\\
			Für den anderen Teil der Behauptung sei $S$ nun eine Indexmenge nach den Bedingungen des Satzes. 
			Nach der Definition von $G_S$ und den Annahmen $f_\nu|_{H_\nu} = 1$, $\text{Vol}(H_\nu, dg_\nu) = 1$ für alle $i$ nicht in $S$, ist es klar, dass Gleichung \ref{eq:satz:integration} gilt, denn $dg_S$ war gerade das Produktmaß auf $G_S$. Nehmen wir nun an, dass das Produkt endlich ist. 
			Dann gilt aber nach (i) und Gleichung \ref{eq:satz:integration}
			\begin{align*}
				\prod_{i}\Bigl(\int_{G_\nu} f_\nu (g_\nu)dg_\nu\Bigr) = \lim_S \int_{G_S} f(g_S) dg_S = \int_{G} f(g) dg
			\end{align*}
			und wir sind fertig.
		\end{proof}
	Damit haben wir die wichtigsten Grundlagen etabliert.
\clearpage

\section{Der Adele- und Idelering}
	In der vorherigen Sektion haben wir uns die Lokalisierungen $\Kp$ im einzelnen angeschaut. Jetzt wollen wir einen Schritt weiter gehen und alle $\Kp$ auf einmal betrachten, indem wir sie in einem neuen Objekt einkapseln.
	\subsection{Eingeschränktes Direktes Produkt}
		\begin{lemma}\label{Lemma:lokalkompaktProd}
			Sei $I$ eine Indexmenge und $X_i$ ein lokalkompakter Hausdorff-Raum f"ur alle $i \in I$. Der Raum $X:=\prod_{i \in I} X_i$ ist genau dann lokalkompakt, wenn fast alle $X_i$ kompakt sind.
		\end{lemma}
		Wir geben den Beweis von Deitmar \cite{deitmar2010}:
		\begin{proof}
			Zun"achst eine Beobachtung: Ist $X$ kompakt, so ist auch jedes $X_i$ kompakt als Bild von $X$ unter der (stetigen) Projektion $\pi_i:X \to X_i$.
			Sei $E \subset I$ eine endliche Teilmenge und $U_i \in X_i$ eine offene Menge f"ur jedes $i \in E$. Wir betrachten die offenen Rechtecke
			\begin{align*}
				\prod_{i \in E} U_i \times \prod_{i \in I\setminus E} X_i,
			\end{align*}
			welche eine Basis der Produkttopologie bilden. Ist $X$ lokalkompakt, so gibt es ein offenes Rechteck, dessen Abschlu\ss kompakt ist. Folglich sind fast alle $X_i$ kompakt. Die R"uckrichtung ist eine Folgerung des Satzes von Tychonov, der besagt, dass das direkte Produkt beliebiger Familien kompakter Mengen wieder kompakt ist, und der Tatsache, dass endliche Produkte lokalkompakter R"aume wieder lokalkompakt sind.
		\end{proof}
		
		Das direkte Produkt lokalkompakter Gruppen liefert uns daher im Allgemeinen keine neue lokalkompakte Gruppe. Wir sehen nun aber, was wir zu tun haben damit doch eine runde Sache daraus wird und geben folgende
		\begin{defi}[Eingeschr"ankte direkte Produkt]
			Sei $I=\{v\}$ eine Indexmenge und f"ur jedes $v \in I$ sei $G_v$ eine lokalkompakte Gruppe. 
			Sei weiter $I_\infty \subseteq I$ eine endliche Teilmenge von $I$ und f"ur jedes $v \notin I_\infty$ sei $H_v\leq G_v$ eine kompakte offene Untergruppe. 
			Das \emph{eingeschr"ankte direkte Produkt} der $G_v$ bez"uglich $H_v$ ist definiert als 
			\begin{align*}
				G = \rdprod{v \in I} G_v := \{ (x_v): x_v \in G_v \text{ und } x_v \in H_v \text{ f"ur alle bis auf endlich viele } v \}.
			\end{align*}
			mit komponentenweiser Verkn"upfung. Die Topologie auf $G$ ist gegeben durch die \emph{eingeschr"ankte Produktopologie}. Diese wird erzeugt durch die Basis der \emph{eingeschr"ankten offenen Rechtecke}
			\begin{align*}
				\prod_{i \in E} U_i \times \prod_{i \in I\setminus E} H_i,
			\end{align*}
			wobei $E \subset I$ eine endliche Teilmenge mit $I_\infty \subset E$ und $U_i \in G_i$ offen f"ur alle $i \in E$ ist.
		\end{defi}
		G ist offensichtlich eine Untergruppe des direkten Produkts, die eingeschr"ankte Produkttopologie ist jedoch nicht die Teilraumtopologie.
		
		Wir f"uhren nun eine n"utzliche Familie von Untergruppen von G ein. Sei $S$ eine endliche Teilmenge von $I$ mit $I_\infty \in S$. Wir definieren die Untergruppe
		\begin{align*}
			G_S := \prod_{i \in S}G_i \times \prod_{i \in I\setminus S} H_i
		\end{align*}
		von $G$. Diese ist offensichtlich offen. Nach Lemma \ref{lemma:direktesProduktTopologischerGruppen} und Lemma \ref{Lemma:lokalkompaktProd} ist $G_S$ selbst wieder eine lokalkompakte Gruppe bez"uglich der Produkttopologie. Man sieht aber leicht, dass diese mit der durch $G$ induzierten Teilraumtopologie "ubereinstimmt. Da jeder Punkt $x \in G$ in einer Untergruppe dieser Form liegt folgt sofort, dass $G$ wieder eine lokalkompakte Gruppe ist.
		
		Abschließend m"ochten wir noch einen kleinen Satz festhalten.
		\begin{satz}
			Eine Teilmenge $Y$ von $G$ hat genau dann kompakten Abschluss, wenn $Y \subseteq \prod{K_i}$ f"ur eine Familie von kompakten Teilmengen $K_i \subseteq G_i$ mit $K_i = H_i$ f"ur fast alle Indizes $i$.
		\end{satz}
		\begin{proof}
			Die R"uckrichtung ist klar, denn jede abgeschlossene Teilmenge eines kompakten Raumes ist wieder kompakt. 
			F"ur die Hinrichtung sei nun $K$ der Abschluss von $Y$ und kompakt in $G$. 
			Da die Untergruppen $G_S$ eine offene "Uberdeckung von $G$ bilden, gibt es eine endliche Familie $\{G_{S_n}\}$, die $K$ "uberdecken. 
			Wir k"onnen sogar noch mehr sagen. Da die die $S_k$ endlich sind, ist $S = \bigcup S_k$ endlich, also wird $K$ sogar von nur einem $G_S$ "uberdeckt. 
			Sei $K_i$ das Bild von $K$ der nat"urlichen Einbettung nach $G_i$. 
			Da die Topologie auf $G_S$ gerade der Produkttopologie entspricht und $K\subseteq G_S$ ist diese Abbildung stetig und $K_i$ damit kompakt als stetiges Bild einer kompakten Menge. Außerdem ist $K_i \subseteq H_i$ f"ur alle $i\notin S$, sodass wir hier $K_i$ durch die kompakten $H_i$ ersetzen k"onnen. Dann ist $Y\subseteq K \subseteq \prod{K_i}$ und wir sind fertig.
		\end{proof}
		
		\subsection{Integration auf dem eingeschr"ankten Produkt}
		Wie wir gesehen haben ist $G=\rdprod{i \in I} G_i$ eine lokalkompakte Gruppe, besitzt also nach Satz \ref{satz:LCAMeasure} ein Haar-Maß. Wir wollen dieses geeignet normalisieren.
		\begin{satz}
			Sei $G=\rdprod{i \in I} G_i$ das eingeschr"ankte direkte Produkt einer Familie lokalkompakter Gruppen $G_i$ bez"uglich der Untergruppen $H_i \subseteq G_i$. Bezeichne $dg_i$ das Haar-Maß auf $G_i$ mit der Normalisierung
			\begin{align*}
				\int_{H_i} dg_i = 1
			\end{align*}
			f"ur fast alle $i \notin I_\infty$. Dann gibt es ein eindeutiges Haar-Maß $dg$ auf G, so dass f"ur jede endliche Teilmenge $S\supseteq I_\infty$ der Indexmenge $I$ die Einschr"ankung $dg_S$ von $dg$ auf $G_S$ genau das Produktmaß ist.
		\end{satz}
		
		\begin{proof}
			Wir vergewissern uns zun"achst, dass die Normalisierung der $dg_i$ m"oglich ist, da per Definiton die Untergruppen $H_i$ offen und kompakt sind und daher positives und endliches Maß haben.
			
			Sei $S$ nun eine beliebige Menge wie im Satz beschrieben und definiere $dg_S$ als das Produktm"aß $dg_S :=\left(\prod_{s \in S}dg_i\right) \times dg^S$, wobei $dg^S$ das Haar-Maß auf der kompakten Gruppe $G^S:=\prod_{i \notin S} H_i$ mit $\int_{G^S} dg^S = 1$ ist. 
			Siehe Folland \ref{folland} Kapitel 7, Satz 7.28 f"ur eine genauere Beschreibung des Maßes $dg^S$. 
			Als endliches Produkt von Haar-Maßen ist $dg_S$ selbst wieder Haar-Maß und wir k"onnen $dg$ normieren, dass dessen Einschr"ankung auf $G_S$ mit $dg_S$ "ubereinstimmt.
			Unsere Wahl von der Teilmenge war willkürlich, allerdings k"onnen wir zeigen, dass die gew"ahlte Normierung unabh"angig von $S$ ist. Sei dazu $T\supseteq S$ eine weitere endliche Indexmenge. 
			Per Definition ist $G_S$ eine Untergruppe von $G_T$. 
			Wir m"ussen jetzt nur noch zeigen, dass die Einschr"ankung von $dg^T$ auf $G^S$ mit $dg^S$ "ubereinstimmt.
			Man erkennt, dass $G^S = \left(\prod_{i \in T \setminus S} H_i\right) \times G^T$. Daher bildet $\left(\prod_{i \in T \setminus S} dg_i\right) \times dg^T$.
			ein Haar-Maß, welches der kompakten Gruppe $G_S$ das oben geforderte Maß $1$ zuweist. Aus der Eindeutigkeit des Haar-Maßes auf (lokal)kompakten Gruppen folgt somit die Gleichheit zu $dg^S$.
			Sei nun $S'$ eine beliebige weitere Indexmenge, die $I_\infty$ enth"alt. Das normierte Maß $dg$ wird auf $G_{S\cup S'}$ eingeschr"ankt zu einem Maß, welches ein konstantes Vielfaches von $dg_{S\cup S'}$ ist. 
			Da aber $G_S \subseteq G_{S\cup S'}$ muss diese Konstante $1$ sein, denn nach obigen "Uberlegungen ist die Einschr"ankung von $dg_{S\cup S'}$ auf $G_S$ gerade $dg_{S}$.
			Umgekehrt ist aber $dg_{S'}$ die die Einschr"ankung von $dg_{S\cup S'}$ auf $G_{S'}$, also ist die Normalisierung unabh"angig von der Wahl unserer Indexmenge $S$.
		\end{proof}
		Wir schreiben manchmal $\prod_{i} dg_i$ f"ur das wie oben normierte Maß $dg$.
		
		\begin{proposition}
			Sei $G$ das eingeschr"ankte direkte Produkt mit dem induzierten Maß $dg$
			\begin{enumerate}[label=(\roman*)]
				\item Sei $f \in L(G)$ eine integrierbare Funktion auf $G$. Dann gilt
					\begin{align*}
						\int_G f(g)dg = \lim_S \int_{G_S} f(g_S) dg_S,
					\end{align*}
					wobei $S$ "uber alle endlichen Indexmengen l"auft, die $I_\infty$ enthalten.
				\item Sei $S_0$ ein beliebige endliche Indexmenge, die $I_\infty$ und alle $i$ enth"alt, f"ur die $\text{Vol}(H_i, dg_i) \not= 1$. 
					F"ur jeden Index $i$ haben wir eine stetige Funktion $f_i$ auf $G_i$, so dass $f_i |_{H_i} = 1$ f"ur alle $i \notin S_0$. 
					Wir definieren
					\begin{align*}
						f(g) = \prod_{i}f_{i}(g_i),
					\end{align*}
					f"ur $g=(g_i) \in G$. 
					Dann ist $f$ wohldefiniert und stetig auf $G$. 
					Sind die $f_i$ sogar integrierbar und ist $S$ eine weitere endliche Indexmenge, die $S_0$ enth"ahlt, haben wir
					\begin{align*}
						\int_{G_S} f(g_S) dg_S = \prod_{i \in S}\Bigl(\int_{G_i} f_i (g_i)dg_i\Bigr).
					\end{align*}
					Ist das Produkt
					\begin{align*}
						\prod_{i \in S}\Bigl(\int_{G_i} f_i (g_i)dg_i\Bigr)
					\end{align*}
					sogar endlich, dann ist $f$ insbesondere integrierbar und es gilt
					\begin{align*}
						\int_{G} f(g) dg = \prod_{i \in S}\Bigl(\int_{G_i} f_i (g_i)dg_i\Bigr).
					\end{align*}	
			\end{enumerate}
		\end{proposition}
		
		\begin{proof}
			(i) Aus der Integrationstheorie ist bekannt, dass
			\begin{align*}
				\int_{G} f(g) = \lim_{K} \int_{K} f(g)dg,
			\end{align*}
			wobei der Limes"uber immer gr"oßer und gr"oßere kompakte Mengen  $K$ geht. Da aber jedes solches $K$ in einer der Mengen $G_S$ liegt, folgt die Gleichung sofort.
			
			(ii) Aus der Bedingung $f_i |_{H_i} = 1$ folgt, dass das Produkt $f(g) = \prod_{i}f_{i}(g_i)$ f"ur alle $g \in G$ endlich, und die Funktion damit wohldefiniert ist. Jedes $g$ hat eine Umgebung, die wiederum in einem der $G_S$ und wir k"onnen annehmen, dass $S$ alle Indizes $i$ mit $f_i|_{H_i}\not= 1$ enth"alt. %TODO
			Wir k"onnen also $f$ lokal als endliches Produkt stetiger Funktionen auffassen. Damit ist $f$ selber stetig.
		\end{proof}
%$G$ als topologische Gruppe zu realisieren, geben wir eine Umgebungsbasis der Identit"at an. Diese soll aus den Mengen der Form $\prod_{v \in J}{N_v}$ bestehen, wobei $N_v$ eine Umgebung der Identit"at $e$ von $G_v$ ist und zus"atszlich $N_v = H_v$ f"ur fast alle $v\in J$ gelten soll. Die dadurch induzierte Topologie auf $G$ unterscheidet sich im Allgemeinen von der Produkttopologie und wird als \emph{eingeschr"ankte Produkttopologie} bezeichnet.	
	\subsection{Der Adelering}
	\subsection{Der Idelering}
	
\clearpage

\section{Tates Beweis}
\subsection{Adelische Poisson Summenformel und der Satz von Riemann-Roch}
	\begin{satz}[Poisson Summenformel]\label{satz:adelic-poisson}
		Sei $f \in S(\Aq)$. Dann gilt:
		\begin{align}
			\sum_{\gamma \in \Q} {f(\gamma + x)} = \sum_{\gamma \in \Q}{\hat{f}(\gamma + x)}
		\end{align}
		f"ur alle $x \in \Aq$.
	\end{satz}
	\begin{proof}
		Jede $\Q$-invariante Funktion $\phi$ auf $\Aq$ induziert eine Funktion auf $\Aq/\Q$, welche wir wieder $\phi$ nennen.
		Wir k"onnen dann die Fouriertransformation von $\phi: \Aq/\Q \rightarrow \C$ als Funktion auf $\Q$ betrachten, da $\Q$ gerade die duale Gruppe von $\Aq/\Q$ ist. Dazu setzen wir
		\begin{align*}
			\hat{\phi}(x) = \int_{\Aq/\Q}\phi(t)\Psi(tx)\overline{dt}
		\end{align*}
		wobei $\overline{dt}$ das Quotientenma\ss auf $\Aq/\Q$ ist, welches von dem Ma\ss $dt$ auf $\Aq$ induziert wird. Dieses Haarma\ss ist charakterisiert durch
		\begin{align*}
			\int_{\Aq/\Q}\tilde{f}(t)\overline{dt} =
			\int_{\Aq/\Q}\sum{\gamma \in \Q}f(\gamma+t)\overline{dt} =
			\int_{\Aq} f(t)dt
		\end{align*}
		f"ur alle stetigen Funktionen $f$ auf $\Aq$ mit geeigneten Konvergenzeigenschaften (z.b. $f\in S(\Aq)$). F"ur den eigentlichen Beweis ben"otigen wir zwei
		
		\begin{lemma}
			F"ur jede Funktion $f \in S(\Aq)$ gilt:
			\begin{align*}
				\hat{f}|_\Q = \hat{\tilde{f}}|_\Q.
			\end{align*}
		\end{lemma}
		\begin{proof}
			Sei $x \in \Q$ beliebig aber fest. Wir beobachten zun"achst, dass wir wegen $\Psi|_\Q =1$
			\begin{align*}
				\Psi(tx)= \Psi(tx)\Psi(\gamma x)=\Psi((\gamma + t) x)
			\end{align*}
			f"ur alle $\gamma \in \Q$ und $t\in \Aq$ haben. Per Definition der Fouriertransformation
			\begin{align*}
				\hat{\tilde{f}}(x)	&= \int_{\Aq / \Q} {\hat{f}(t)\Psi(tx)\overline{dt}} 
									 = \int_{\Aq / \Q} \left(\sum_{\gamma \in \Q}{f(\gamma + t)}\right)\Psi(tx)\overline{dt} =\\
									&= \int_{\Aq / \Q} \left(\sum_{\gamma \in \Q}{f(\gamma + t)}\Psi((\gamma + t)x)\right)\overline{dt}
									 = \int_{\Aq} f(t)\Psi(tx)dt = \hat{f}(x)
			\end{align*}
			wobei wir im vorletzten Schritt die oben besprochene Charakterisierung des Quotientenmaßes $\overline{dt}$ ausgenutzt haben.
		\end{proof}
		
		\begin{lemma}
			F"ur jede Funktion $f \in S(\Aq)$ und jedes $x\in \Q$ gilt
			\begin{align*}
				\tilde{f}(x) = \sum_{\gamma \in \Q} {\hat{\tilde{f}}(\gamma)\overline{\Psi}(\gamma x)}
			\end{align*}
		\end{lemma}
		\begin{proof}
			Wie wir eben bewiesen haben gilt $\hat{f}|_\Q = \hat{\tilde{f}}|_\Q$ und daher
			\begin{align*}
				\left| \sum_{\gamma \in \Q} {\hat{\tilde{f}}(\gamma)\overline{\Psi}(\gamma x)}\right| = 
				\left| \sum_{\gamma \in \Q} {\hat{f}(\gamma)\overline{\Psi}(\gamma x)}\right| 
				\leq \sum_{\gamma \in \Q} {|\hat{f}(\gamma)|}
			\end{align*}
			unter Ausnutzen der Tatsache, dass $\Psi$ unit"ar ist. Die rechte Seite der Gleichung ist also normal konvergent, da $f \in S(\Aq)$. Analog folgt, dass auch $\sum_{\gamma \in \Q} {\hat{\tilde{f}}(\gamma)}$ normal konvergiert. Wir erinnern uns, dass das Pontryagin Duale $\widehat{\Aq/\Q}$ als topologische Gruppe isomorph zu $\Q$\footnote{Achtung: Hier ist $Q$ versehen mit der diskreten Topologie gemeint} ist. Also $\hat{\tilde{f}} \in L^1(\Q)$ und
			\begin{align*}
				\sum_{\gamma \in \Q} {\hat{\tilde{f}}(\gamma)\overline{\Psi}(\gamma x)}
			\end{align*}
			ist die Fouriertransformierte\footnote{Wir erinnern uns, dass in diesem Fall das Z"ahlma\ss ein Haar-Ma\ss ist} von $\hat{\tilde{f}}$ ausgewertet am Punkt $-x$. Nach Fourierinversionsformel erhalten wir also
			\begin{align*}
				\tilde{f}(x) = \hat{\hat{\tilde{f}}}(-x) = \sum_{\gamma \in \Q} {\hat{\tilde{f}}(\gamma)\overline{\Psi}(\gamma x)}
			\end{align*}
			und damit das Lemma.
		\end{proof}
		Zur"uck zum Beweis der Summenformel. Wir erhalten aufgrund des zweiten Lemmas mit $x=0$ und anschlie\ss enden Anwenden des Ersten
		\begin{align*}
			\tilde{f}(0) = 	\sum_{\gamma \in \Q} \hat{\tilde{f}}(\gamma) \bar{\Psi}(0) =
							\sum_{\gamma \in \Q} \hat{\tilde{f}}(\gamma) =
							\sum_{\gamma \in \Q} \hat{f}
		\end{align*}
		Aber per Definition gilt gerade $\tilde{f}(0) = \sum_{\gamma \in \Q}f(\gamma)$, also
		\begin{align*}
			\sum_{\gamma \in \Q}f(\gamma) = \sum_{\gamma \in \Q} \hat{f}
		\end{align*}
		und wir sind fertig.
	\end{proof}
	
	\begin{satz}[Riemann-Roch]
		Sei $x \in \Iq$ ein Idel von $\Q$ und sei $f\in S(\Aq)$. Dann
		\begin{align*}
			\sum_{\gamma \in \Q} {f(\gamma x)} = \frac{1}{|x|_{\Aq}}\sum_{\gamma \in \Q} {\hat{f}(\gamma x^{-1})}
		\end{align*}
	\end{satz}
	\begin{proof}
	Sei $x \in \Iq$ beliebig aber fest. F"ur beliebige $y \in \Aq$ definieren wir eine Funktion $h(y):=f(yx)$. Diese ist wieder in $S(\Aq)$ und erf"ullt damit die Poisson-Summenformel
		\begin{align*}
			\sum_{\gamma \in \Q}h(\gamma) = \sum_{\gamma \in \Q} \hat{h}.
		\end{align*}
		Berechnen wir allerdings die Fouriertransformation von $h$ erhalten wir mit Translation um $x^{-1}$
		\begin{align*}
			\hat(h){\gamma}) &= \int_{\Aq}h(y)\Psi(\gamma y)dy \\
							 &= \int_{\Aq}f(yx)\Psi(\gamma y)dy \\
							 &= \frac{1}{|x|_{\Aq}} \int_{\Aq}f(y)\Psi(\gamma y x^{-1})dy \\
							 &= \frac{1}{|x|_{\Aq}} \hat{f}(\gamma x^{-1}).
		\end{align*}
	\end{proof}
\clearpage

% Anhang

\begin{appendices}
\section{Beweis des Satz von Ostrowski}
\begin{proof}[Beweis von Satz \ref{satz:padisch:ostrowski}]
		Sei $\abs$ ein beliebiger nicht-trivialer Absolutbetrag auf $\Q$. Wir untersuchen die zwei m"oglichen Fälle.
		
		Fall 1: $\abs$ ist archimedisch.
			Sei dann $n_0\in \N$ die kleinste natürliche Zahl mit $\abs[n_0] > 1$.
			Dann gibt es ein $\alpha \in \R^+$ mit $\abs[n_0]^{\alpha}=n_0$.
			Wir wollen nun zeigen, dass $\abs[n]=\abs[n]_\infty^\alpha$ für alle $n \in \N$ gilt. Der allgemeine Fall für $\Q$ folgt dann aus den Eigenschaften des Betrags.
			Dazu bedienen wir uns eines kleinen Tricks: Für $n \in \N$ nehmen wir die Darstellung zur Basis $n_0$, d.h.
			\begin{align*}
				n = \sum_{i=0}^{k} a_i n_0^i
			\end{align*}
			mit $a_i \in \{0,1,\dots,n_0-1\}$, $a_k \neq 0$ und $n_0^k\leq n < n_0^{k+1}$. Nehmen wir davon den Absolutbetrag und beachten, dass $\abs[a_i]\leq 1$ nach unserer Wahl von $n_0$ gilt, so erhalten wir
			\begin{align*}
				\abs[n] \leq \sum_{i=0}^{k} \abs[a_i] n_0^{i\alpha}
					\leq \sum_{i=0}^{k} n_0^{i\alpha}
					\leq n_0^{k\alpha}\sum_{i=0}^{k} n_0^{-i\alpha}
					\leq n_0^{k\alpha}\sum_{i=0}^{\infty} n_0^{-i\alpha}
					 = n_0^{k\alpha} \frac{n_0^\alpha}{n_0^\alpha - 1}.
			\end{align*}
			Setzt man nun $C\coloneqq \frac{n_0^\alpha}{n_0^\alpha - 1}>0$, so sehen wir
			\begin{align*}
				\abs[n]\leq C n_0^{k\alpha}\leq C n^\alpha
			\end{align*}
			für beliebige $n \in \N$, also insbesondere auch
			\begin{align*}
				|n^N|\leq C n^{N\alpha}.
			\end{align*}
			Ziehen wir nun auf beiden Seiten die $N$-te Wurzel und lassen $N$ gegen $\infty$ laufen, so konvergiert $\sqrt[N]{C}$ gegen $0$ und wir erhalten 
			\begin{align*}
				\abs[n] \leq n^\alpha
			\end{align*}
			Damit wäre die erste Hälfte geschafft. Gehen wir nun zurück zu unserer Basisdarstellung
			\begin{align*}
				n = \sum_{i=0}^{k} a_i n_0^i.
			\end{align*}
			Da $n < n_0^{k+1}$ erhalten wir die Abschätzung
			\begin{align*}
				n_0^{(k+1)\alpha}=|n_0^{k+1}| = |n + n_0^{k+1} - n| \leq \abs[n] + |n_0^{k+1} -n|.
			\end{align*}
			 mit dem Ergebnis aus der ersten Hälfte des Beweises und $n\geq n_0^k$ sehen wir
			\begin{align*}
				\abs[n] &\geq n_0^{(k+1)\alpha} - |n_0^{k+1} -n| 
					\geq n_0^{(k+1)\alpha} - (n_0^{k+1} -n)^\alpha
					\\&\geq n_0^{(k+1)\alpha} - (n_0^{k+1} -n_0^k)^\alpha
					=n_0^{(k+1)\alpha} \left(1 - \left(1 - \frac{1}{n_0}\right)\right)
					\\&> n^\alpha \left(1 - \left(1 - \frac{1}{n_0}\right)\right).
			\end{align*}
			Setzen wir wieder $C'\coloneqq \left(1 - \left(1 - \frac{1}{n_0}\right)\right) >0$ folgt analog zum ersten Teil, dass
			\begin{align*} 
				\abs[n]\geq n^\alpha
			\end{align*}
			und daher $\abs[n]=n^\alpha$. Damit haben wir gezeigt, dass $\abs$ äquivalent zum klassischen Absolutbetrag $\abs_\infty$ ist.
			
		Fall 2: $\abs$ ist nicht archimedisch.
			Dann ist $\abs[n_0]\leq 1$ für alle $n \in \N$ und, da $\abs$ nicht-trivial ist, muss es eine kleinste Zahl $n_0$ geben mit $\abs[n_0]<1$. Insbesondere muss $n_0$ eine Primzahl sein, denn sei $p \in \N$ ein Primteiler von $n_0$, also $n_0=p \cdot n'$ mit $n' \in \N$ und $n' < n$, dann gilt nach unserer Wahl von $n_0$
			\begin{align*}
				|p| = |p|\cdot |n'| =|p \cdot n'| = \abs[n_0] < 1.
			\end{align*}
			Folglich muss schon $p=n_0$ gelten. Ziel wird es jetzt natürlich sein zu zeigen, dass $\abs$ äquivalent zum $p$-adischen Absolutbetrag ist.
			Zunächst finden wir ein $\alpha \in \R^+$ mit $|p| = |p|_p^{\alpha} = \frac{1}{p^{\alpha}}$. 
			Sei als nächstes $n\in \Z$ mit $p \centernot\mid n$. Wir schreiben
			\begin{align*}
				n= rp + s, r \in \Z, 0<s<p
			\end{align*}
			Nach unserer Wahl von $p=n_0$ gilt $|s|=1$ und $|rp|<1$. 
			Es folgt 
			\begin{align*}
				\abs[n]=\text{max}\{|rp|,|s|\}=1.
			\end{align*}
			Sei nun $n\in \Z$ beliebig. 
			Wir schreiben $n=p^{v}n'$ mit $p \centernot\mid n'$ und sehen
			\begin{align*}
				\abs[n] = |p|^{v}|n'| = |p|^v = (|p|_p^{\alpha})^{v}=\abs[n]_p^{\alpha}.
			\end{align*}
			Mit den gleichen Überlegungen aus dem ersten Fall folgt damit die Behauptung.
	\end{proof}
\section{Tates Beweis der Poisson-Summenformel}
	F"ur diesen Beweis wird unser Ausblick in die abstrakte harmonische Analysis n"utzlich.
	\begin{proof}[Beweis von Satz \ref{satz:tateproof:poisson}]
		Jede $\K$-invariante Funktion $\phi$ auf $\A$ induziert eine Funktion auf $\A/\K$, welche wir wieder mit $\phi$ bezeichnen.
		Wir k"onnen dann die Fouriertransformation von $\phi: \A/\K \rightarrow \Komplex$ als Funktion auf $\K$ betrachten, da $\K$ versehen mit der diskreten Topologie gerade mit den Pontryagin Dualen von $\A/\K$ identifiziert werden kann\footnote{vgl. Ramakrishnan und Valenza \cite{rama} Proposition 7-15}. 
		Dazu setzen wir
		\begin{align*}
			\hat{\phi}(\xi) = \int_{\A/\K}\phi(x)\Psi(\xi \cdot x)\overline{\dx}
		\end{align*}
		wobei $\overline{\dx}$ das \emph{Quotientenmaß} auf $\A/\K$ ist, welches von dem Maß $\dx$ auf $\A$ induziert wird. 
		Dieses Haar-Maß ist charakterisiert durch
		\begin{align}\label{eq:anhang:quotmeasure}
			\int_{\A/\K}\tilde{f}(x)\overline{\dx} =
			\int_{\A/\K}\sum_{\gamma \in \K}f(\gamma+x)\overline{\dx} =
			\int_{\A} f(x)\dx
		\end{align}
		f"ur alle $f\in L^1(\A)$. F"ur eine genauere Behandlung dieses Maßes verweisen wir auf Knightly und Li \cite{knightly} Kapitel 7.2.
		
		F"ur den eigentlichen Beweis ben"otigen wir zwei kleine Ergebnisse.
		\begin{lemma}\label{lemma:anhang:tate1}
			F"ur jede Funktion $f \in S(\A)$ gilt:
			\begin{align*}
				\hat{f}|_\K = \hat{\tilde{f}}|_\K.
			\end{align*}
		\end{lemma}
		\begin{proof}
			Sei $\xi \in \K$ beliebig aber fest. 
			Wir beobachten zun"achst, dass wir wegen $\Psi|_\K =1$
			\begin{align*}
				\Psi(\xi x)= \Psi(\xi x)\Psi(\gamma \xi)=\Psi((\gamma + x) \xi)
			\end{align*}
			f"ur alle $\gamma \in \K$ und $x\in \A$ haben. 
			Per Definition der abstrakten Fouriertransformation
			\begin{align*}
				\hat{\tilde{f}}(\xi)	
					&= \int_{\A / \K} {\tilde{f}(x)\Psi(\xi x)\overline{\dx}} 
					 = \int_{\A / \K} \left(\sum_{\gamma \in \K}{f(\gamma + x)}\right)\Psi(-\xi x)\overline{\dx} =\\
					&= \int_{\A / \K} \left(\sum_{\gamma \in \K}{f(\gamma + x)}\Psi((\gamma + x)\xi)\right)\overline{\dx}
					 = \int_{\A} f(t)\Psi(\xi x)\dx = \hat{f}(\xi)
			\end{align*}
			wobei wir im vorletzten Schritt die oben besprochene Charakterisierung \eqref{eq:anhang:quotmeasure} des Quotientenmaßes $\overline{\dx}$ ausgenutzt haben.
		\end{proof}
		
		\begin{lemma}\label{lemma:anhang:tate2}
			F"ur jede Funktion $f \in S(\A)$ und jedes $x\in \K$ gilt
			\begin{align*}
				\tilde{f}(x) = \sum_{\gamma \in \K} {\hat{\tilde{f}}(\gamma)\overline{\Psi}(\gamma x)}
			\end{align*}
		\end{lemma}
		\begin{proof}
			Wie wir eben bewiesen haben gilt $\hat{f}|_\K = \hat{\tilde{f}}|_\K$ und daher
			\begin{align*}
				\left| \sum_{\gamma \in \K} {\hat{\tilde{f}}(\gamma)\overline{\Psi}(\gamma x)}\right| = 
				\left| \sum_{\gamma \in \K} {\hat{f}(\gamma)\overline{\Psi}(\gamma x)}\right| 
				\leq \sum_{\gamma \in \K} {|\hat{f}(\gamma)|}
			\end{align*}
			unter Ausnutzen der Tatsache, dass $\Psi$ ein Charakter ist. Die rechte Seite der Gleichung ist also normal konvergent, da $f \in S(\A)$. Analog folgt, dass auch $\sum_{\gamma \in \K} {\hat{\tilde{f}}(\gamma)}$ normal konvergiert. 
			Also $\hat{\tilde{f}} \in L^1(\K)$ und
			\begin{align*}
				\sum_{\gamma \in \K} {\hat{\tilde{f}}(\gamma)\overline{\Psi}(\gamma x)}
			\end{align*}
			ist die Fouriertransformierte\footnote{Wir erinnern uns, dass auf der diskreten Topologie das Z"ahlmaß ein Haar-Maß ist.} von $\hat{\tilde{f}}$ ausgewertet am Punkt $-x$.
			Nach Umkehrformel aus Satz \ref{satz:topogroup:umkehrformel} erhalten wir also
			\begin{align*}
				\tilde{f}(x) = \hat{\hat{\tilde{f}}}(-x) = \sum_{\gamma \in \K} {\hat{\tilde{f}}(\gamma)\overline{\Psi}(\gamma x)}
			\end{align*}
			und damit das Lemma.
		\end{proof}
		Zur"uck zum Beweis der Summenformel. 
		Mit Lemma \ref{lemma:anhang:tate2} und $x=0$ und anschließendem Anwenden von Lemma \ref{lemma:anhang:tate1} erh"alt man
		\begin{align*}
			\tilde{f}(0) = 	\sum_{\gamma \in \K} \hat{\tilde{f}}(\gamma) \bar{\Psi}(0) =
							\sum_{\gamma \in \K} \hat{\tilde{f}}(\gamma) =
							\sum_{\gamma \in \K} \hat{f}(\gamma)
		\end{align*}
		Aber per Definition gilt gerade
		\begin{align*}
			\tilde{f}(0) = \sum_{\gamma \in \K}f(\gamma),
		\end{align*}
		also
		\begin{align*}
			\sum_{\gamma \in \K}f(\gamma) = \sum_{\gamma \in \K} \hat{f}(\gamma)
		\end{align*}
		und wir sind fertig.
	\end{proof}
\clearpage
\end{appendices}
 %Bibliografie
\printbibliography
\end{document}