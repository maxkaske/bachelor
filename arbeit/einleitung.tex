
\section{Von Riemann zu Tate}
	Im Jahr 1859 erschien mit \citetitle{riemann1859ueber} \cite{riemann1859ueber} Bernhard Riemanns erste und einzige veröffentlichte Arbeit im Bereich der Zahlentheorie.
	Als Startpunkt nimmt Riemann die heute als \emph{Riemannsche Zeta-Funktion}\footnote{Obwohl diese Funktion heute nach Riemann benannt ist, war es Leonhard Euler, der sich zuerst näher mit ihr beschäftigte.} $\zeta(s)$ bekannte Abbildung, welche für $\Re(s)>1$ durch
	\begin{align*}
		\sum_{n\in \N} \frac{1}{n^s} = \prod_{p \text{ prim}} \frac{1}{1-p^{-s}}
	\end{align*}
	als absolut konvergente Reihe oder durch die \emph{Euler-Produktformel} dargestellt werden kann.
	Neben einer Vielzahl neuer Notationen, Definitionen und Ideen, darunter die bekannte \emph{Riemannsche Vermutung}, dass alle nicht-trivialen Nullstellen von $\zeta(s)$ den Realteil $1/2$ haben, gab er auch zwei Beweise der Funktionalgleichung
	\begin{align*}
		\Xi(s) = \Xi(1-s),
	\end{align*}
	wobei $\Xi(s) = \Gamma(s/2)\pi^{-s/2}\zeta(s)$ und $\Gamma(s)$ das bekannte Euler-Integral
	\begin{align*}
		\Gamma(s) = \int_0^\infty e^{-x} x^s \frac{\dx}{x}
	\end{align*}
	bezeichnet\footnote{Riemann selbst benutze noch die durch Gauß definierte Notation $\Pi(s-1)=\Gamma(s)$.}.

\subsection{Die klassische Funktionalgleichung}
	In seinem ersten Beweis erhält Riemann zunächst durch Aufsummieren von
	\begin{align}\label{eq:riemannstart}
		\Gamma(s)\frac{1}{n^s} = \int_0^\infty e^{-nx}x^{s}\frac{\dx}{x},
	\end{align}
	die Gleichung
	\begin{align*}
		\Gamma(s)\zeta(s) = \int_0^\infty \frac{x^{s}}{e^x-1} \frac{\dx}{x}.
	\end{align*}
	Über Wegintegration des Integrals
	\begin{align*}
		\int \frac{(-x)^{s-1}}{e^x-1}\dx
	\end{align*}
	etablierte er anschließend die meromorphe Fortsetzung der Zeta-Funktion (wenn auch nicht im Sinne von Weierstrass) und zeigte die erste Form der Funktionalgleichung
	\begin{align*}
		\zeta(s) = \Gamma(s)(2\pi)^{s-1} 2 \sin(s\pi/2) \zeta(1-s).
	\end{align*}
	Riemann nutzte nun geläufige Identitäten der Gamma-Funktion, um dieses Ergebnis umzuformulieren:
	Der Ausdruck
	\begin{align*}
		\Xi(s) = \Gamma(s/2)\pi^{-s/2}\zeta(s)
	\end{align*}
	bleibt unverändert, wie Riemann schreibt, \glqq wenn $s$ in $s-1$ verwandelt wird.\grqq{}
	
	Diese symmetrische Darstellung der Funktionalgleichung veranlasste Riemann dazu, in Gleichung \eqref{eq:riemannstart} den Ausdruck $\Gamma(s/2)$ anstatt $\Gamma(s)$ für die Grundlage eines weiteren Beweises zu betrachten.
	Diesen möchten wir im folgenden etwas genauer besprechen, wobei wir uns Konvergenzgedanken ganz im Geiste Riemanns aufsparen.
	\begin{satz}\label{satz:einleitung:riemann}
		Die Riemannsche Zeta-Funktion $\zeta(s)$  kann zu einer meromorphen Funktion auf ganz $\Komplex$ fortgesetzt werden, welche die Funktionalgleichung
		\begin{align*}
			\Gamma\left(\frac{s}{2}\right)\pi^{-s/2}\zeta(s) = \Gamma\left(\frac{1-s}{2}\right)\pi^{-(1-s)/2}\zeta(1-s)
		\end{align*}
		erfüllt. Sie besitzt zwei einfache Pole bei $s=0$ und $s=1$.
	\end{satz}
	\begin{proof}[Beweis der Funktionalgleichung]
		Durch den Variablenwechsel $y=n^2\pi t^2$ in Eulers Integraldarstellung der Gamma-Funktion erhalten wir für $\Re(s)>1$
		\begin{align*}
			\frac{1}{n^s} \pi^{-s/2}\Gamma\left(\frac{s}{2}\right) 
				= \frac{1}{n^s} \pi^{-s/2} \int_0^{\infty} e^{-n^2\pi t} n^s\pi^{s/2} t^{s/2} \frac{\dx[t]}{t} 
				= \int_{0}^{\infty} e^{-n^2\pi t}t^{s/2} \frac{\dx[t]}{t}.
		\end{align*}
		Anschließendes aufsummieren und ausnutzen von Fubinis Integralgleichung ergibt die Formel
		\begin{align*}
			\Gamma\left(\frac{s}{2}\right)\pi^{-s/2}\zeta(s) 
				= \int_{0}^{\infty} \sum_{n=1}^{\infty}\left(e^{-n^2\pi t}\right)t^{s/2} \frac{\dx[t]}{t}.
		\end{align*}
		Die rechte Seite entspricht gerade der $\Xi$-Funktion. 
		Wir führen nun die \emph{Thetafunktion}
		\begin{align*}
			\Theta(t) = \sum_{n\in \Z} e^{-\pi t n^2} = 1 + 2 \sum_{n\in\N}e^{-\pi t n^2}
		\end{align*}
		ein.
		Damit können wir obige Gleichung etwas vereinfacht darstellen als
		\begin{align*}
			\Xi(s) 
				= \frac{1}{2}\int_{0}^{\infty}(\Theta(t)-1)t^{s/2} \frac{\dx[t]}{t}.
		\end{align*}
		Teilen wir nun das Integral auf in
		\begin{align}
			\int_{0}^{\infty} (\Theta(t) - 1) t^{s/2}  \frac{\dx[t]}{t} = \int_{0}^{1} (\Theta(t) - 1) t^{s/2}  \frac{\dx[t]}{t} + \int_{1}^{\infty} (\Theta(t) - 1) t^{s/2}  \frac{\dx[t]}{t}.\label{eq:einleitung:riemann1}
		\end{align}
		Als n"achstes benötigen wir die \emph{Theta-Transformationsformel}
		\begin{align}
			\Theta(t) = t^{-1/2} \Theta(1/t).\label{eq:einleitung:thetatrafo}
		\end{align}
		Um deren Gültigkeit einzusehen, benötigt man Zweierlei. 
		Zuerst erinnern wir uns an die \emph{klassische Fouriertransformation} $\hat{f}:\R\to\Komplex$ einer $L^1$-Funktion $f:\R\to\Komplex$ definiert durch
		\begin{align*}
			\hat{f}(\xi) = \int_\R f(x)e^{-2\pi i x \xi} \dx.
		\end{align*}
		Mit dieser Abbildung k"onnen wir folgenden Satz formulieren.
		\begin{satz}[Klassische Poisson Summenformel]
			\label{satz:einleitung:poisson}
			Für jede Schwartz Funktion $f:\R \to \Komplex$ und deren Fouriertransformation $\hat{f}:\R \to \Komplex$ gilt:
			\begin{align}
				\sum_{a \in \Z} f(a) = \sum_{a \in \Z} \hat{f}(a) \label{eq:einleitung:poisson}
			\end{align}
		\end{satz}
		\begin{proof}
			Siehe zum Beispiel \textcite{deitmar2010} Proposition 5.4.10.
		\end{proof}
		Als Zweites halten wir fest, dass die \emph{Fouriertransformierte} von $f_t(x) = e^{-\pi t x^2}$ gleich $\hat{f}_t(x)= t^{-1/2}f_{1/t}(x)$ ist.
		Einsetzen in die Poisson Summenformel \eqref{eq:einleitung:poisson} ergibt dann die Transformationsformel \eqref{eq:einleitung:thetatrafo}.
		Mit dieser kann man den ersten Summanden in Gleichung \eqref{eq:einleitung:riemann1} umformen zu
		\begin{align*}		
			\int_{0}^{1} \left(t^{-1/2}\Theta(1/t)-1\right) t^{s/2}  \frac{\dx[t]}{t} 
				= \int_{0}^{1}\Theta(1/t) t^{(s-1)/2}  \frac{\dx[t]}{t} - \frac{2}{s}.
		\end{align*}
		Anschließend nutzen wir die Substitution $t \mapsto t^{-1}$ und rechnen
		\begin{align*}
			\int_{0}^{1}\Theta(1/t) t^{(s-1)/2}  \frac{\dx[t]}{t} - \frac{2}{s}
				&= \int_{1}^{\infty} \Theta(t) t^{(1-s)/2}  \frac{\dx[t]}{t} - \frac{2}{s}\\
				%&= \int_{1}^{\infty}  (\Theta(t) - 1) t^{(1-s)/2} + t^{(1-s)/2}  \frac{\dx[t]}{t} - \frac{2}{s}\\
				&= \int_{1}^{\infty} (\Theta(t) - 1) t^{(1-s)/2}  \frac{\dx[t]}{t} - \frac{2}{s} - \frac{2}{1-s}.
		\end{align*}
		Einsetzen in die Gleichung von $\Xi(s)$ ergibt dann
		\begin{align*}
			\Xi(s)
				= \frac{1}{2} \left( \int_{1}^{\infty} (\Theta(t) - 1) t^{s/2}  \frac{\dx[t]}{t} + \int_{1}^{\infty} (\Theta(t) - 1) t^{(1-s)/2}  \frac{\dx[t]}{t} \right)  - \frac{1}{s} - \frac{1}{1-s}
		\end{align*}
		und wir sehen, dass dieser Ausdruck unverändert bleibt, wenn \glqq{}$s$ in $1-s$ verwandelt wird.\grqq{}
	\end{proof}

\subsection{Auf dem Weg zu Tate}
	Riemanns Beweise haben eine kleine Schwäche. Beide starten mit einem Ausdruck, in dem die Gamma-Funktion bereits vorkommt.
	Die Herkunft des Faktors und ein Grund, warum er zur Bildung der Funktionalgleichung so nahtlos an die Zeta-Funktion passt, wird nicht ersichtlich.
	Auftritt John Tate.
	
	Tate kommt aus einer langen Linie von Mathematikern, deren Arbeit mehr oder weniger direkt durch Riemanns Ideen in \emph{Ueber die Anzahl\dots} beeinflusst wurde.
	Unter der Aufsicht Emil Artins verfasste er 1950, fast 100 Jahre nach Riemann, seine Doktorarbeit\footnote{Sie ist zum Beispiel in \cite{cassels1967algebraic} zu finden.}  \citetitle{tate} \cite{tate}. 
	In ihr bewies er die analytische Fortsetzung und Funktionalgleichung der Dedekind Zeta-Funktionen und Hecke L-Funktionen, eine Art Verallgemeinerung der Riemannschen Zeta-Funktion.
	Dieses Ergebnis war keineswegs neu und wurde bereits 30 Jahre zuvor von Erich Hecke gezeigt.
	Was Tates Doktorarbeit so besonders macht -- und einer der Gründe dafür, warum sie als  \glqq Tate's Thesis\grqq{} gewissen Kultstatus erreicht hat -- ist die elegante Herangehensweise an Heckes Problemstellung in der globalen Sprache der \emph{Adele und Idele}.
	Sogar noch verblüffender: Ist Tates theoretischer Rahmen etabliert, so stimmen die einzelnen Beweisschritte größtenteils mit Riemanns klassischen zweiten Beweis überein.
	Es ergibt sich allerdings ein viel klareres Bild über das Zustandekommen der einzelnen Bestandteile der Funktionalgleichung.
	
	Ziel dieser Arbeit wird es nun sein, Tates Doktorarbeit in ihrer einfachsten Form, d.h. im Fall des algebraischen Körpers $\Q$, Revue passieren zu lassen, um die zentrale Frage
	\begin{quote}
		\centering
		\textit{Woher kommt die Gamma-Funktion in der Funktionalgleichung der Riemannschen Zeta-Funktion?}
	\end{quote}
	zu beantworten. 
	%Wir werden daher Schritt für Schritt die algebraischen, analytischen und topologischen Grundlagen für das Verständnis von Tates Beweis erarbeiten.
	
	Wir beginnen dazu in Kapitel \ref{sec:topogroup} mit einer Einführung zu topologischen Gruppen, Lokalkompaktheit und Haar-Maßen.
	Um den Rahmen dieser Arbeit nicht unnötigen zu zerren, werden wir allerdings Pontryagin Dualität und abstrakte Fourieranalyis -- beides wichtige Grundlagen für Tates Beweis in höhrer Allgemeinheit -- nur in einem kurzen Ausblick behandeln.
	In Kapitel \ref{sec:padisch} wiederholen wir kurz die wichtigsten Begriffe und Eigenschaften zu Absolutbeträgen. 
	Anschließend führen wir die $p$-adischen Zahlen ein und besch"aftigen uns etwas mit deren Eigenheiten.
	Als Ersatz zur harmonischen Fourieranalysis definieren wir am Anfang von Kapitel \ref{sec:lokal} die Fouriertransformation auf den $p$-adischen Zahlen neu und zeigen, dass unser eigener Ansatz mit der abstrakten Theorie übereinstimmt. 
	Damit werden wir genug Grundlagen gesammelt haben, um Tates erstes Ergebnis, die Funktionalgleichung lokaler Zeta-Funktion, zu beweisen.
	Wir schließen das Kapitel und die erste Hälfte der Arbeit mit der expliziten Berechnung für den Beweis wichtiger Integrale.
	In Kapitel \ref{sec:rdp} beginnen wir mit der Theorie des eingeschränkten direkten Produkts, um die Problemstellung in einen globalen Kontext zu übertragen.
	Wir stellen uns die Frage, wie man auf dieser abstrakten Gruppe integriert und welche Form die Quasi-Charaktere annehmen.
	Daraufhin lernen wir in Kapitel \ref{sec:adeleidele} mit den Gruppen der Adele $\A$ und Idele $\I$ zwei konkrete Beispiele für eingeschr"ankte dirkte Produkte kennen.
	Kapitel \ref{sec:tateproof} behandelt die Fourieranalysis im globalen Kontext der Adele und Idele. 
	Dort beweisen wir die zahlentheoretische Variante des Satzes von Riemann-Roch und geben Tates vollen Beweis der Funktionalgleichung globaler Zeta-Funktionen. 
	Zum Schluss runden wir die Arbeit ab und besprechen wie aus diesem Ergebnis direkt die Funktionalgleichung der Riemannschen Zeta-Funktion folgt.