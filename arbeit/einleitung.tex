
\section{Von Riemann zu Tate}
	%\subsection{Historische Kontext}
	Im Jahr 1859 erschien mit \emph{\glqq Ueber die Anzahl der Primzahlen unter einer gegebenen Grösse\grqq} Bernhard Riemanns erste und einzige ver"offentlichte Arbeit im Bereich der Zahlentheorie.
	Als Startpunkt nimmt Riemann die heute als \emph{Riemannsche Zeta-Funktion}\footnote{Obwohl diese Funktion heute nach Riemann benannt ist, war es Leonhard Euler der sich zuerst n"aher mit ihr besch"aftigte } $\zeta(s)$ bekannte Abbildung, welche f"ur $\Re(s)>1$ durch
	\begin{align*}
		\sum_{n\in \N} \frac{1}{n^s} = \prod_{p \text{ prim}} \frac{1}{1-p^{-s}}
	\end{align*}
	als absolut konvergente Reihe oder durch die \emph{Euler-Produktformel} dargestellt werden kann.
	Neben einer Vielzahl neuer Notationen, Definitionen und Ideen, darunter die bekannte \emph{Riemannsche Vermutung}, dass alle nicht-trivialen Nullstellen von $\zeta(s)$ den Realteil $1/2$ haben, gab er auch zwei Beweise der Funktionalgleichung
	\begin{align*}
		\Xi(s) = \Xi(1-s),
	\end{align*}
	wobei $\Xi(s) = \Gamma(s/2)\pi^{-s/2}\zeta(s)$ und $\Gamma(s)$\footnote{Riemann selbst benutze noch den durch Gauß definierte Notation $\Pi(s-1)=\Gamma(s)$} das bekannte Euler-Integral
	\begin{align*}
		\Gamma(s) = \int_0^\infty e^{-x} x^s \frac{\dx}{x}
	\end{align*}
	ist.
	%Wir m"ochten in dieser Arbeit zum e
	%und unter anderem die obige \emph{Euler-Produktformel} bewies.
	%Davon ausgehend
	%holomorphe Fortsetzung der Funktion,
	%befinden sich auch zwei Beweise der Funktionalgleichung
	%\begin{align*}
		%\Gamma\left(\frac{s}{2}\right)\pi^{-s/2}\zeta(s) = \Gamma\left(\frac{1-s}{2}\right)\pi^{-(1-s)/2}\zeta(1-s),
	%\end{align*}
	%wobei 
	%\begin{align*}
		%\Gamma(s) = \int_0^\infty e^{-x} x^s \frac{\dx}{x}
	%\end{align*}
	%die bekannte Gamma-Funktion in ihrer Integraldarstellung nach Euler ist.
	%die \emph{Riemann Hypothese}
	
\subsection{Der Beweis der Funktionalgleichung}
	In seinem ersten Beweis erh"alt Riemann zun"achst durch Anwendung von
	\begin{align}\label{eq:riemannstart}
		\Gamma(s)\frac{1}{n^s} = \int_0^\infty e^{-nx}x^{s}\frac{\dx}{x},
	\end{align}
	die Gleichung
	\begin{align*}
		\Gamma(s)\zeta(s) = \int_0^\infty \frac{x^{s}}{e^x-1} \frac{\dx}{x}.
	\end{align*}
	"Uber Wegintegration des Integrals
	\begin{align*}
		\int \frac{(-x)^{s-1}}{e^x-1}\dx
	\end{align*}
	etablierte er anschließend die meromorphe Fortsetzung der Zeta-Funktion (auch wenn nicht im Sinne von Weierstrass) und etablierte seine erste Form der Funktionalgleichung
	\begin{align*}
		\zeta(s) = \Gamma(s)(2\pi)^{s-1} 2 \sin(s\pi/2) \zeta(1-s).
	\end{align*}
	Riemann benutzte nun gel"aufige Identit"aten der Gamma-Funktion um dieses Ergebnis umzuformen:
	Der Ausdruck
	\begin{align*}
		\Xi(s) = \Gamma(s/2)\pi^{-s/2}\zeta(s)
	\end{align*}
	bleibt unver"andert, wie Riemann schreibt, \glqq wenn $s$ in $s-1$ verwandelt wird.\grqq{}
	
	Diese symmetrische Darstellung der Funktionalgleichung veranlasste Riemann nun dazu in Gleichung \eqref{eq:riemannstart} den Ausdruck $\Gamma(s/2)$ anstatt $\Gamma(s)$ f"ur die Grundlage eines weiteren Beweises zu betrachten.
	Diesen m"ochten wir uns im folgenden etwas genauer anschauen, wobei wir uns Konvergenzgedanken ganz im Geiste Riemanns aufsparen.
	\begin{satz}
		Die Riemannsche Zeta-Funktion $\zeta(s)$  kann zu einer meromorphen Funktion auf ganz $\Komplex$ fortgesetzt werden, welche die \emph{Funktionalgleichung}
		\begin{align*}
			\Gamma\left(\frac{s}{2}\right)\pi^{-s/2}\zeta(s) = \Gamma\left(\frac{1-s}{2}\right)\pi^{-(1-s)/2}\zeta(1-s)
		\end{align*}
		erf"ullt. Sie besitzt zwei einfache Pole bei $s=0$ und $s=1$.
	\end{satz}
	\begin{proof}[Beweis der Funktionalgleichung]
		Durch den Variablenwechsel $y=n^2\pi t^2$ in Eulers Integraldarstellung der Gamma-Funktion erhalten wir f"ur $\Re(s)>1$
		\begin{align*}
			%\int_{0}^{\infty} e^{-n^2\pi t}t^{s/2} \frac{\dx[t]}{t} 
				%= \int_0^{\infty} e^{-y} \pi^{-s/2}n^{-s} y^{s/2} {\dx[t][y]}{y} 
				%= \frac{1}{n^s} \pi^{-s/2}\Gamma\left(\frac{s}{2}\right)
			\frac{1}{n^s} \pi^{-s/2}\Gamma\left(\frac{s}{2}\right) 
				= \frac{1}{n^s} \pi^{-s/2} \int_0^{\infty} e^{-n^2\pi t} n^s\pi^{s/2} t^{s/2} \frac{\dx[t]}{t} 
				= \int_{0}^{\infty} e^{-n^2\pi t}t^{s/2} \frac{\dx[t]}{t}.
		\end{align*}
		Anschließendes aufsummieren und ausnutzen von Fubinis Integralgleichung ergibt die Formel
		\begin{align*}
			\Gamma\left(\frac{s}{2}\right)\pi^{-s/2}\zeta(s) 
				= \int_{0}^{\infty} \sum_{n=1}^{\infty}\left(e^{-n^2\pi t}\right)t^{s/2} \frac{\dx[t]}{t}
		\end{align*}
		Die rechte Seite entspricht gerade der $\Xi$-Funktion. Wir f"uhren nun die \emph{Thetafunktion}
		\begin{align*}
			\Theta(t) = \sum_{n\in \Z} e^{-\pi t n^2} = 1 + 2 \sum_{n\in\N}e^{-\pi t n^2}
		\end{align*}
		ein.
		Damit k"onnen wir obige Gleichung etwas vereinfacht darstellen mit
		\begin{align*}
			\Xi(s) 
				= \frac{1}{2}\int_{0}^{\infty}(\Theta(t)-1)t^{s/2} \frac{\dx[t]}{t}.
		\end{align*}
		%\begin{align*}
				%\Gamma(\frac{s}{2}) \pi^{-\frac{s}{2}} \zeta(s)
					%&= \sum_{n=1}^{\infty}  \int_{0}^{\infty} n^{-s} \pi^{-s/2} t^{s/2} e^{-t} \frac{\dx[t]}{t}
					%= \sum_{n=1}^{\infty}  \int_{0}^{\infty} \frac{t}{n^2\pi}^{s/2} e^{-t} \frac{\dx[t]}{t}\\
					%&= \sum_{n=1}^{\infty}  \int_{0}^{\infty} t^{s/2} e^{-\pi n^2 t} \frac{\dx[t]}{t}
					%= \frac{1}{2} \int_{0}^{\infty} (\Theta(t) - 1) t^{s/2}  \frac{\dx[t]}{t} \\
		%\end{align*}
		Teilen wir nun das Integral auf in
		\begin{align}
			\int_{0}^{\infty} (\Theta(t) - 1) t^{s/2}  \frac{\dx[t]}{t} = \int_{0}^{1} (\Theta(t) - 1) t^{s/2}  \frac{\dx[t]}{t} + \int_{1}^{\infty} (\Theta(t) - 1) t^{s/2}  \frac{\dx[t]}{t}.\label{eq:einleitung:riemann1}
		\end{align}
		%Wir m"ochten nun
		%und erhalten mit der Substitution $t \mapsto 1/t$ im ersten Summanden
		%\begin{align*}
			%\int_{0}^{1} (\Theta(t) - 1) t^{s/2}  \frac{\dx[t]}{t} 
				%&= \int_{0}^{1} \Theta(t) t^{s/2}  \frac{\dx[t]}{t} - \frac{2}{s}.
		%\end{align*}
		Wie ben"otigen nun die \emph{Theta-Transformationsformel}
		\begin{align}
			\Theta(t) = t^{-1/2} \Theta(1/t).\label{eq:einleitung:thetatrafo}
		\end{align}
		Um deren G"ultigkeit einzusehen ben"otigen man Zweierlei: 
		Zuerst erinnern wir uns an die \emph{klassische Fouriertransformation} $\hat{f}:\R\to\Komplex$ einer $L^1$-Funktion $f:\R\to\Komplex$ definiert durch
		\begin{align*}
			\hat{f}(\xi) = \int_\R f(x)e^{-2\pi i x \xi} \dx.
		\end{align*}
		Mit dieser Abbildung haben wir folgenden
		\begin{satz}[Klassische Poisson-Summenformel]
			\label{satz:poisson}
			F"ur jedes geeignete $f:\R \to \Komplex$ und deren Fouriertransformation $\hat{f}:\R \to \Komplex$ gilt:
			\begin{align}
				\sum_{a \in \Z} f(a) = \sum_{a \in \Z} \hat{f}(a) \label{eq:einleitung:poisson}
			\end{align}
		\end{satz}
		\begin{proof}
			In dieser klassischen Form zum Beispiel in Deitmar \cite{deitmar2010} Proposition 5.4.10.
		\end{proof}
		Als Zweites halten wir fest, dass die \emph{Fouriertransformierte} von $f_t(x) = e^{-\pi t x^2}$ gleich $\hat{f}_t(x)= t^{-1/2}f_{\frac{1}{t}}(x)$ ist.
		Einsetzen in die Poisson-Summenformel \eqref{eq:einleitung:poisson} ergibt dann die Transformationsformel \eqref{eq:einleitung:thetatrafo}.
		Mit dieser kann man nun den ersten Summanden in Gleichung \eqref{eq:einleitung:riemann1} umformen zu
		\begin{align*}		
			\int_{0}^{1} \left(t^{-1/2}\Theta(1/t)-1\right) t^{s/2}  \frac{\dx[t]}{t} 
				= \int_{0}^{1}\Theta(1/t) t^{(s-1)/2}  \frac{\dx[t]}{t} - \frac{2}{s}
		\end{align*}
		Anschließend nutzen wir die Substitution $t \mapsto t^{-1}$ und rechnen
		\begin{align*}
			\int_{0}^{1}\Theta(1/t) t^{(s-1)/2}  \frac{\dx[t]}{t} - \frac{2}{s}
				&= \int_{1}^{\infty} \Theta(t) t^{(1-s)/2}  \frac{\dx[t]}{t} - \frac{2}{s}\\
				%&= \int_{1}^{\infty}  (\Theta(t) - 1) t^{(1-s)/2} + t^{(1-s)/2}  \frac{\dx[t]}{t} - \frac{2}{s}\\
				&= \int_{1}^{\infty} (\Theta(t) - 1) t^{(1-s)/2}  \frac{\dx[t]}{t} - \frac{2}{s} - \frac{2}{1-s}.
		\end{align*}
		Einsetzen in Gleichung von $\Xi(s)$ ergibt dann
		\begin{align*}
			\Xi(s)
				= \frac{1}{2} \left( \int_{1}^{\infty} (\Theta(t) - 1) t^{s/2}  \frac{\dx[t]}{t} + \int_{1}^{\infty} (\Theta(t) - 1) t^{(1-s)/2}  \frac{\dx[t]}{t} \right)  - \frac{1}{s} - \frac{1}{1-s}
		\end{align*}
		und wir sehen, dass dieser Ausdruck unver"andert bleibt wenn wir $s$ in $1-s$ verwandeln.
	\end{proof}

\subsection{Auf dem Weg zu Tate}
	Beide Beweise haben eine kleine Schw"ache: Sie starten bereits mit der Gamma-Funktion als einen notwendigen Faktor zur Bildung der Funktionalgleichung.
	Warum sie aber gerade so nahtlos zur Zeta-Funktion passt wird nicht ersichtlich. 
	Auftritt John Tate.
	
	Tate kommt aus einer langen Linie von Mathematikern deren Arbeit mehr oder weniger direkt durch Riemanns Ideen in  \emph{Ueber die Anzahl\dots} beeinflusst wurde.
	Unter der Aufsicht Emil Artins verfasste er 1950, fast 100 Jahre nach Riemann, seine Doktorarbeit \glqq Fourier Analysis in Number Fields and Hecke's Zeta-Functions\grqq{}\cite{tate}. Sie ist zum Beispiel in \cite{cassels1967algebraic} zu finden.
	In ihr bewies er die analytische Fortsetzung und Funktionalgleichung der Dedekind Zeta-Funktionen und Hecke L-Funktionen, eine Art Verallgemeinerung der Riemannschen Zeta-Funktion.
	Dieses Ergebnis war keineswegs neu und wurde 30 Jahre fr"uher bereits durch Erich Hecke gezeigt.
	Was Tates Doktorarbeit jedoch so besonders macht - und einer der Gr"unde daf"ur warum sie als  \glqq Tate's Thesis\grqq{} gewissen Kultstatus erreicht hat - ist die elegante Herangehensweise an Heckes Problemstellung in der globalen Sprache der \emph{Adele und Idele}.
	Sogar noch verbl"uffender: Ist Tates theoretischer Rahmen erstmal etabliert, so stimmen die einzelnen Beweisschritte gr"oßtenteils mit Riemanns klassischen zweiten Beweis überein.
	Es ergibt sich aber ein viel klareres Bild über das Zustandekommen der einzelnen Bestandteile der Funktionalgleichung.
	
	Ziel dieser Arbeit wird es nun sein Tates Doktorarbeit in ihrer einfachsten Form, d.h. im Fall des algebraischen K"orper $\Q$, Revue passieren zu lassen um nun die zentrale Frage
	\begin{quote}
		%\begin{flushright}
		\centering
		\textit{Woher kommt die Gamma-Funktion in der Funktionalgleichung der Riemannschen Zeta-Funktion?}
		%\end{flushright}
	\end{quote}
	zu beantworten. 
	%Wir werden daher Schritt f"ur Schritt die algebraischen, analytischen und topologischen Grundlagen f"ur das Verst"andnis von Tates Beweis erarbeiten.
	
	Wir beginnen dazu in Kapitel \ref{sec:topogroup} mit einer Einf"uhrung zu topologischen Gruppen, Lokalkompaktheit und Haar-Maße.
	Um den Rahmen dieser Arbeit nicht unn"otigen zu zerren werden wir allerdings Pontryagin Dualit"at und abstrakte Fourieranalyis - beides wichtige Grundlagen f"ur Tates Beweis in h"ohrer Allgemeinheit - nur in einem kurzen Ausblick behandeln.
	In Kapitel \ref{sec:padisch} wiederholen wir kurz die wichtigsten Begriffe und Eigenschaften zu Absolutbetr"agen. 
	Anschließend f"uhren wir die $p$-adischen Zahlen ein und untersuchen deren Eigenheiten.
	Um unser Auslassen der abstrakten Fourieranalysis wieder gut zu machen, definieren wir am Anfang von Kapitel \ref{sec:lokal} die Fouriertransformation auf den $p$-adischen Zahlen neu und zeigen, dass diese Definition der abstrakten entspricht. 
	Damit haben wir dann genug Grundlagen gesammelt um Tates erstes Ergebnis, die Funktionalgleichung lokaler Zeta-Funktion, zu beweisen.
	Wir schließen das Kapitel und die erste H"alfte der Arbeit mit der expliziten Berechnung f"ur den Beweis wichtiger Integrale.
	Es geht weiter mit der globale Betrachtung der Problemstellung in Kapitel \ref{sec:rdp} mit der wichtigen Theorie des eingeschr"ankten direkten Produkts.
	Wir stellen uns die Frage wie man auf dieser abstrakten Gruppe integriert und wie Charaktere aussehen.
	Daraufhin lernen wir in Kapitel \ref{sec:adeleidele} mit der Adele- und Idelegruppe $\A$ und $\I$ zwei konkrete Beispiele kennen.
	Kapitel \ref{sec:tateproof} behandelt nun die Fourieranalysis im globalen Kontext der Adele und Idele, beweisen die algebraische Variante des Satzes von Riemann-Roch und geben Tates vollen Beweis der Funktionalgleichung globaler Zeta-Funktion. 
	Zum Schluss runden wir die Arbeit ab und besprechen wie aus diesem Ergebnis direkt die Funktionalgleichung der Riemannschen Zeta-Funktion folgt.
	
	
	
	
	%Trotz der relativen K"urze von nur 10 Seiten beeinflussten Riemanns Ideen in \emph{Ueber die Anzahl\dots} die Arbeit vieler bedeutender Mathematiker, darunter Hadamard, von Mangoldt, de la Vallée Poussin, Landau, Hardy, Littlewood, Siegel, Polya, Jensen, Lindelöf, Bohr, Selberg, Artin und Hecke.
	%"Uber die letzten beiden kommen wir schließlich zu John Tate. 
	%Unter der Aufsicht Artins verfasste dieser 1950 seine Doktorarbeit\cite{tate} \glqq Fourier Analysis in Number Fields and Hecke's Zeta-Functions\grqq{}.
	%Das Ergebnis, die Funktionalgleichungen gewisser \glqq Hecke\grqq{} Zeta-Funktionen, wurde, wie der Name es vielleicht erahnen l"asst, bereits 30 Jahre fr"uher von Hecke gezeigt.
	%Tates Arbeit bietet dagegen einen elegante Umformulierung der Problemstellung von Hecke in einen m"achtigeren Kontext und erreichte damit als \glqq Tate's Thesis\grqq{}gewissen Kultstatus.
	


