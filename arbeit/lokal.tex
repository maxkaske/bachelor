\section{Lokale...}
\subsection{... K"orper}
	Wir betrachten im Folgenden alle 
	%Q_p topologischer Ring: done
	%Skalierung der Mengen durch multiplikation: done
	%Haarmass auf Q_p^\times: done
	%normierung des mult masses
	%annulus
	\begin{satz}\label{satz:QpIstLokalKompakt}
		F"ur alle Stellen $p\leq\infty$ gilt
		\begin{enumerate}[label=\emph{(\roman*)}]
		\item $(\Kp, +)$ ist eine lokalkompakte Gruppe.
		\item $(\K_p^\times, \cdot)$ ist eine lokalkompakte Gruppe.
		\end{enumerate}
	\end{satz}
	\begin{proof}
	%%% hier normierung des haar masses auf Qp
	
		Zu (i): Die Stetigkeit der Addition und der Negierung sind eigentlich direkte Folgen der Konstruktion der Vervollst"andigung.
		Als kleine Auffrischung zeigen wir sie trotzdem.
		Seien $x_n \to x$ und $y_n \to y$ konvergente Folgen in $\Kp$. 
		Es ist zu zeigen, dass $x_n+y_n$ gegen $x+y$ konvergiert. 
		Nach der Dreiecksungleichung gilt
		\begin{align*}
			\abs[(x_n+y_n) - (x+y)]_p = \abs[(x_n - x) + (y_n - y)]_p \leq \abs[(x_n - x)]_p + \abs[(y_n - y)]_p.
		\end{align*}
		Die rechte Seite konvergiert gegen $0$, also ist die linke eine Nullfolge und die Stetigkeit der Addition folgt.
		"Ahnlich zeigen wir $-x_n \to -x$, denn $\abs[(-x_n) - (-x)]_p = \abs[x_n-x]_p$.
		Damit ist $\Kp$ eine topologische Gruppe. 
		
		Da die Topologie von einer Metrik induziert wird, ist diese hausdorffsch.
		Wir m"ussen folglich nur noch die Lokalkompaktheit zeigen.
		Wegen der Stetigkeit der Addition reicht es dazu eine kompakte Umgebung der $0$ zu finden, denn sei $K$ eine kompakte Nullumgebung, so ist $a+K$ f"ur beliebige $a\in \Kp$ eine kompakte Umgebung von $a$.
		Wir behaupten, dass die Umgebung
		\begin{align*}
			\Zp = \{x \in \Kp : \abs[x]_p \leq 1\}
		\end{align*}
		kompakt ist.
		F"ur $p=\infty$ ist $\Z_\infty = [-1,1]$ also kompakt.
		F"ur $p<\infty$ gen"ugt es zu zeigen, dass $\Zp$ vollst"andig und totalbeschr"ankt ist. 
		Ersteres folgt daraus, dass $\Zp$ als abgeschlossene Menge des vollst"andigen Raumes $\Kp$ selber vollst"andig ist.
		Eine Menge heißt totalbeschr"ankt, wenn wir sie f"ur jedes $\varepsilon > 0$ mit endlich vielen $\varepsilon$-B"allen "uberdecken k"onnen.
		Wir k"onnen uns auf $\varepsilon$ der Form $p^{-k}$, $k\geq 0$ beschr"anken, da unsere Metrik nur den diskreten Wertebereich $p^\Z$ hat.
		Es ist $p^{k+1}\Zp$ eine Untergruppe von $\Zp$ und f"ur den Index gilt $[\Zp : p^{k+1}\Zp] = p^{k+1}$. Das bedeutet, dass die $p^{-k}$-B"alle
		\begin{align*}
			a + p^{k+1}\Zp = \{y\in \Zp : \abs[y-a]_p \leq p^{-k-1}\} = \{y\in \Zp : \abs[y-a]_p < p^{-k}\} = B_{p^{-k}}(a)
		\end{align*}
		mit $a=0\dots p^{k+1}-1$ die Menge $\Zp$ "uberdecken. Als vollst"andige und totalbeschr"ankte Menge ist $\Zp$ somit kompakt.
		
		
		Zu (ii): $\K_p^\times = \Kp\setminus \{0\}$ ist ein offener Teilraum von $\Kp$ und somit selbst wieder hausdorffsch und lokalkompakt.
		F"ur die Stetigkeit seien $x_n \to x$ und $y_n \to y$ zwei konvergente Folgen in $\K_p^\times$. 
		Wir zeigen, dass $x_ny_n$ gegen $xy$ konvergiert.
		Dies folgt aus der Absch"atzung
		\begin{align*}
			\abs[x_n y_n - xy]_p 
			&= \abs[x_n y_n -x_ny + x_n y- xy]_p  \\
			&\leq \abs[x_n y_n -x_ny]_p + \abs[x_n y- xy]_p \\
			&= \underbrace{\abs[x_n]_p}_{\text{beschr"ankt}} \underbrace{\abs[y_n -y]_p}_{\to 0} + \underbrace{\abs[x_n- x]_p}_{\to 0} \abs[y]_p \to 0
		\end{align*}
		Weiter zur Invertierung. Wegen
		\begin{align*}
			\abs[\frac{1}{x_n} - \frac{1}{x}]_p = \frac{\abs[x_n - x]_p}{\abs[x_nx]_p} \to 0
		\end{align*}
		folgt, dass $x_n^{-1}$ gegen $x^{-1}$ konvergiert und die Stetigkeit ist gezeigt.
	\end{proof}
	
	\begin{satz}\label{satz:multiplikativesHaarMass}
		F"ur jede messbare Menge $A\subset \Kp$ und jedes $x\in \Kp$ gilt
		\begin{align*}
			\mu(xA) = \abs[x]_p \mu(A).
		\end{align*}
		Insbesondere folgt f"ur jedes $f \in L^1(\Kp)$ und $x\not= 0$:
		\begin{align*}
			\int_\Kp f(x^{-1}y)d\mu(y) = \abs[x]_p \int_\Kp f(y)d\mu(y).
		\end{align*}
	\end{satz}
	\begin{proof}
		Sei $x\in \K_p^\times$. Die Funktion $\mu_x$ definiert durch
		\begin{align*}
			\mu_x (A) = \mu(xA)
		\end{align*}
		definiert wieder ein Haar-Maß auf $\Kp$ und unterscheidet sich daher nur durch Skalierung mit einer positiven Konstante $c>0$ von $\mu$.
		Ziel wird es nun sein $c=\abs[x]_p$ zu zeigen. 
		Dies ist klar im Fall $p=\infty$. 
		F"ur $p<\infty$ reicht es dank unserer Normierung $\mu(x\Zp) = \abs[x]_p$ zu zeigen.
		Sei dazu $\abs[x]_p = p^{-k}$.
		Dann ist $x=p^ky$ mit $y\in \Zp$ und $x\Zp = p^k\Zp$.
		Daher reduziert sich unsere Betrachtung auf $\mu(p^k\Zp) = p^{-k}$.
		Beginnen wir mit dem Fall $k\geq 0$. Dann ist $p^k\Zp$ eine Untergruppe von $\Zp$ und f"ur den Index gilt $[\Zp : p^k\Zp] = p^k$.
		Wir haben also eine disjunkte Zerlegung $\Zp = \bigsqcup_{a=0}^{p^{k}-1} a + p^k\Zp$.
		Aus der Translationsinvarianz folgern wir dann
		\begin{align*}
			1 = \mu (\Zp) = \sum_{a=0}^{p^{k}-1} \mu(a + p^k\Zp) =\sum_{a=0}^{p^{k}-1} \mu(p^k\Zp) = p^k \mu(p^k\Zp).
		\end{align*}
		Die Behauptung folgt dann durch einfaches Umformen. 
		Im anderen Fall $k<0$ ist umgekehrt $\Zp$ eine Untergruppe von $p^k\Zp$ mit Index $[p^k\Zp:\Zp]= p^{-k}$ und die Behauptung folgt analog.
	\end{proof}
	
	\begin{satz}\label{satz:lokal:multiplikativesmass}
		Ist $dx$ ein additives Haar-Maß auf $\Kp$, so definiert $\frac{dx}{\abs[x]_p}$ ein multiplikatives Haar-Maß $d^\times x$ auf $\K_p^\times$.
		Insbesondere gilt dann f"ur alle $g \in L^1(\K_p^\times)$
		\begin{align*}
			\int_{\K_p^\times} g(x) d^\times x = \int_{\Kp \setminus \{0\}} g(x) \frac{dx}{\abs[x]_p}
		\end{align*}
	\end{satz}
	\begin{proof}
		Wir haben bereits gezeigt, dass $\K_p^\times$ eine lokalkompakte Gruppe ist und folglich ein Haar-Maß besitzt.
		Wenn wir nun ein positives, lineares Funktional auf $C_c(\K_p^\times)$ angeben, erhalten wir nach Rieszschen Darstellungssatz ein Radonmaß, welches diesem Funktional entspricht. 
		Ist nun $g \in C_c(\K_p^\times)$, so ist $g\abs_p^{-1} \in C_c(\Kp\setminus\{0\})$. 
		Dies ist in der Tat eine eins-zu-eins Zuweisung.
		Wir definieren nun das Funktional
		\begin{align*}
			\Phi(g) = \int_{\Kp \setminus \{0\}} g(x) \frac{dx}{\abs[x]_p}.
		\end{align*}
		Dieses ist offensichtlich ein positives, nicht-triviales, lineares Funktional auf $ C_c(\K_p^\times)$. Es ist translationsinvariant, denn
		\begin{align*}
			\int_{\Kp \setminus \{0\}} g(y^{-1}x) \frac{dx}{\abs[x]_p} = \int_{\Kp \setminus \{0\}} g(x) \frac{\abs[y]_pdx}{\abs[yx]_p} = \int_{\Kp \setminus \{0\}} g(x) \frac{dx}{\abs[x]_p},
		\end{align*}
		folglich kommt es von einem Haar-Maß $d^\times x$. 
		Der zweite Teil der Behauptung folgt aus der Tatsache, dass die Funktionen in $C_c(\K_p^\times)$ dicht in $L^1(\K_p^\times)$ liegen.
		Beim "Ubergang zum Grenzwert erhalten wir die Gleichung auf Funktionen in $L^1$.
	\end{proof}
\subsection{... Fourieranalysis}
%hier den additiven charakter besprechen und kurz bedeutung erklaeren: done
%selbstdual:ist eigentlich fourierumkehrformel: done
%fouriertransformation: done
%schwartz-bruhat und deren form: done
%fourierumkehrformel: done
		F"ur die unendliche Stelle $p=\infty$ definieren wir die \emph{Schwartz-Bruhat Funktion} als eine komplexwertige, glatte Funktion $f$, die f"ur alle nicht-negativen ganzen Zahlen $n$ und $m$ die Bedingung
		\begin{align*}
			\sup_{x\in \K_\infty}\abs[x^n\frac{d^m}{dx^m}f(x)] < \infty
		\end{align*}
		erf"ullt.
		F"ur die endlichen Stellen $p<\infty$ definieren wir eine \emph{Schwartz-Bruhat Funktion} als eine lokal konstante Funktion mit kompakten Tr?ger.
		Die Menge aller solcher Funktionen bilden einen komplexen Vektorraum, den wir mit $\Sw(\K_p)$ bezeichnen. 
		Im Fall $p<\infty$ erkennt man leicht, dass $\Sw(\Kp)\subseteq L^1(\Kp)$. 
		F"ur $p=\infty$ gilt nach obiger Bedingung $(\abs[1]+\abs[x^2])\abs[f(x)] \leq C$, also $\abs[f(x)]\leq C(1+x^2)^{-1}$ und $(1+x^2)^{-1} \in L^1(\K\infty)$
		
		\begin{bsp}~ 
			\begin{enumerate}[label=(\roman*)]
				\item Im Fall $p=\infty$ ist die Funktion $f_k = x^k e^{-x^2}$ f"ur jedes $k\in\N_0$ in $\Sw(\K_\infty)$. 
				Die Ableitungen $\frac{d^m}{dx^m} f_k(x)$ sind von der Form $p(x)e^{-x^2}$, wobei $p(x)$ ein Polynom ist. 
				Aus der Analysis ist dann bekannt, dass $\abs[x^n p(x)e^{-x^2}]$ f"ur jedes $n\in \N_0$ beschr"ankt ist.
				\item Im Fall $p<\infty$ sind offensichtlich die charakteristischen Funktionen kompakter Mengen in $\Sw(\Kp)$. 
				Beispiele f"ur Kompakta sind Mengen der Form $a+p^k\Zp$ mit $a\in \K$ und $k\in \Z$.
			\end{enumerate}
		\end{bsp}
		
		\begin{lemma}\label{lemma:padischSBF}
			Jede Funktion $f\in \Sw(\Kp)$, $p<\infty$, ist eine endliche Linearkombination von charakteristischen Funktionen der Form $\ind_{a+p^k\Zp}$, wobei $a\in \K$ und $k\in \Z$
		\end{lemma}
		\begin{proof}
			Sei $f \in \Sw(\Kp)$. 
			Da $f$ lokal konstant ist, ist f"ur jedes $z\in\C$ das Urbild $f^{-1}(z)$ offen in $\Kp$. 
			Also ist $f^{-1}(0)$ offen, folglich $\Kp \setminus f^{-1}(0)$ abgeschlossen und daher schon $\text{supp}(f) = \Kp \setminus f^{-1}(0)$. 
			Per Definition hat die Schwartz-Bruhat Funktion $f$ kompakten Tr"ager, also ist $\Kp \setminus f^{-1}(0)$ kompakt. 
			Diese Menge wird von den offenen Mengen $f^{-1} (x)$ mit $x\not= 0$ "uberdeckt, wovon nach Kompaktheit schon endlich viele reichen.
			$f$ hat somit endliches Bild. Weiter ist jede offene Menge $f^{-1} (x)$ eine disjunkte Vereinigung offener B"allen in $\Kp$. 
			Diese haben aber genau die gesuchte Form $a+p^k\Zp$ wie oben. 
			Aufgrund der Kompaktheit, reichen wieder endliche viele solcher B"alle. Damit folgt auch schon das Lemma.
		\end{proof}
		
		Bevor wir mit etwas klassischer Fourieranalysis beginnen wollen wir den Kreis zur abstrakten Variante schließen.
		Daf"ur definieren wir z"unachst f"ur jede Stelle $p$ einen nicht-trivialen, unit"aren Charakter $e_p$ auf der additiven Gruppe $\K_p^+$ und zeigen anschließend, dass wir jeden beliebigen Charakter $\psi \in \widehat{\K}_p^+$ durch $e_p$ darstellen k"onnen.
		\begin{defi}
			Wir definieren den \emph{Standardcharakter} $e_p:\K_p^+ \to S^1$ wie folgt:
			\begin{itemize}[itemindent=3em]
				\item [$p=\infty$:] $e_p(x) = \exp(-2\pi i x)$
				\item [$p<\infty$:] Hier wird die Definition etwas aufwendiger. Zun"achst haben wir eine nat"urliche Projektion $\Kp \twoheadrightarrow \Kp/\Zp$.
				Die "Aquivalenzklassen von $\Kp/\Zp$ werden nach unseren "Uberlegungen zur Potenzreihendarstellung eindeutig repräsentiert durch Zahlen der Form $\sum_{k=-n}^{-1} a_kp^k$ mit $a_k \in \Z$ und $0\leq a_k\leq p-1$.
				Wir definieren den stetigen Homomorphismus $\Kp/\Zp \to \K/\Z$ indem wir diese Repräsentanten als summen in $\K$ interpretieren. 
				Zu guter Letzt schicken wir diese Summe in $\K/\Z$ durch $e: \$\K/\Z \to S^1, x \mapsto \exp(2\pi i x)$ in den Einheitskreis. 
				Alle diese Abbildungen sind stetige Gruppenhomomorphismen, bilden also selber wieder einen stetigen Gruppenhomomorphismus, den wir mit $e_p$ bezeichnen.
				Interpretieren wir die Potenzreihenentwicklung der $p$-adischen Zahlen als nicht unbedingt konvergente Summe in $\Q$, so k"onnen wir (etwas unsch"on) schreiben
				\begin{align*}
					e_p\left(\sum_{k=-n}^{\infty} a_kp^k\right) = \exp\left(2\pi i \sum_{k=-n}^{\infty} a_kp^k\right) = \exp\left(2\pi i \sum_{k=-n}^{-1} a_kp^k\right)
				\end{align*}
			\end{itemize}
		\end{defi}
		Sei $\psi$ ein unitärer Charakter auf $\K_p^+$, $p<\infty$ und $U$ eine offene Umgebung der $1 \in S^1$, die nur die triviale Untergruppe $1$ enth"alt. 
		Aufgrund der Stetigkeit von $\psi$ gibt es dann eine offene Umgebung $V$ der $0\in\K_p^+$ mit $\psi(V)\subseteq U$. 
		Ohne Einschr"ankung ist $V$ eine Untergruppe der Form $p^k\Zp$ f"ur ein $k\in\Z$. 
		Dann ist aber $\psi(V)$ eine Untergruppe von $S^1$ und somit gleich $1$.
		Die kleinste Untegruppe $p^k\Zp$ nennen wir den \emph{Konduktor} von $\psi$.
	
		
		\begin{lemma}
			Jeder unit"are Charakter $\psi: \Kp \to S^1$ ist von der Form $x \mapsto e_p(ax)$ f"ur ein $a \in \Kp$.
		\end{lemma}
		\begin{proof}
			Der Beweis teilt sich auf in die F"alle $p=\infty$ und $p<\infty$.
			Zuerst zu $p=\infty$: Sei $\psi: \K_\infty \to S^1$ ein Charakter.
			Wegen der Stetigkeit von $\psi$ existiert ein $\varepsilon > 0$, so dass $\psi((-\varepsilon, \varepsilon)) \subset \{z\in S^1: \Re(z)>0\}$.
			W"ahlen wir $\varepsilon$ noch etwas kleiner k"onnen wir sogar $\psi([-\varepsilon, \varepsilon)]) \subset \{z\in S^1: \Re(z)>0\}$ garantieren.
			Wir definieren nun $a$ als das eindeutig bestimmte Element aus $[-\frac{1}{4\varepsilon},\frac{1}{4\varepsilon}]$
			\footnote{Also der Logarithmuszweig, dass $-\pi/2<\varepsilon a<\pi/2$}, so dass $\psi (\varepsilon) = \exp(2\pi i a \varepsilon)$.
			Als n"achstes behaupten wir, dass auch
			\begin{align*}
				\psi \left(\frac{\varepsilon}{2}\right) =  \exp\left(2\pi i  a \frac{\varepsilon}{2}\right)
			\end{align*}
			gilt.
			Wegen $\psi (\frac{\varepsilon}{2})^2 = \psi (\varepsilon) = \exp(2\pi i a \varepsilon)$ ist $\psi (\frac{\varepsilon}{2}) = \pm \exp(2\pi i  a \frac{\varepsilon}{2})$.
			Da aber $\psi (\frac{\varepsilon}{2})$ nach der Wahl von $\varepsilon$ positiven Realteil hat, kommt nur $ \exp(2\pi i  a \frac{\varepsilon}{2})$ in Frage.
			Durch Iteration des Arguments erhalten wir $\psi (\frac{\varepsilon}{2^n}) = \exp(2\pi i  a \frac{\varepsilon}{2^n})$ f"ur $n\in \N_0$.
			
			Setzen wir jetzt $\varepsilon = 2^{-n_0}$ f"ur ein geeignetes  $n_0\in\N$.
			Dann ist $\varepsilon^{-1}$ eine nat"urliche Zahl und f"ur beliebige $k\in\Z$ haben wir
			\begin{align*}
				\psi \left(\frac{k} {2^{n}}\right) &= \psi \left(\frac{k\varepsilon} {2^{n}\varepsilon}\right) 
										= \psi \left(\frac{\varepsilon} {2^{n}}\right) ^{\frac{k}{\varepsilon}} 
										\\&= \exp\left(2\pi i  a \frac{\varepsilon}{2^n}\right) ^{\frac{k}{\varepsilon}}
										= \exp\left(2\pi i  a \frac{k}{2^n}\right)
			\end{align*}
			Die Menge aller $\frac{k}{2^n}$ mit $k\in \Z$ und $n\in \N_0$ liegt nun aber dicht in $\R$ und wir k"onnen aus der Stetigkeit $\chi(x) = \exp(2\pi i a x)$ schließen.
			Um die Eindeutigkeit von $a$ zu sehen, reicht es die Ableitung  von $x \mapsto \exp(2\pi i a x)$ zu berechnen und an der Stelle $x=0$ auszuwerten.
			Der Wert betr"agt gerade $2\pi i a$.
			
			Kommen wir nun zum Fall $p<\infty$. 
			Sei $\psi$ ein unit"arer Charakter auf $\K_p^+$ und sei $p^k\Zp$ dessen Konduktor.
			F"ur $k\leq0$ gilt offensichtlich $\psi(\Zp)=1$.
			Ohne Beschr"ankung der Allgemeinheit betrachten wir nur solche Charaktere, denn im Fall $k>0$ k"onnen wir auch den Charakter $x\mapsto \psi(p^kx)$ betrachten und die Aussage folgt aus $\psi(p^kx) = \psi(x)^{(p^k)}$.
			
			Wir suchen nun ein geeignetes $a\in\Kp$. 
			Dabei f"allt uns zun"achst auf, dass der Charakter bereits eindeutig durch seine Werte fur $p^{-k}$, $k \in \N$ bestimmt wird.
			Da $\psi(\Zp)=1$ gilt n"amlich
			\begin{align*}
				\psi \left( \sum_{k=-n}^\infty x_k p^k \right) = \sum_{k=-n}^{-1} \psi \left(p^k \right)^{x_k}.
			\end{align*}
			Es reicht also ein geeignetes $a$ f"ur diese Potenzen zu finden.
			Schauen wir uns $\psi{p^{-1}}$ genauer an, so erkennen wir, dass dies eine $p$-te Einheitswurzel sein muss.
			Damit ist $\psi(p^{-1}) = \exp(2\pi i \frac{a_1}{p})$ f"ur ein eindeutig bestimmtes nat"urliches $0\leq a_1 \leq p - 1$.
			Analog argumentieren wir auch $\psi(p^{-k}) = \exp(2\pi i \frac{a_k}{p^k})$ mit $0\leq a_k \leq p^k - 1$.
			Zudem gilt 
			\begin{align*}
				\exp\left(2\pi i \frac{a_{k+1}}{p^{k}}\right) = \exp\left(2\pi i \frac{a_{k+1}}{p^{k+1}}\right)^{p}= \psi(p^{-k-1})^p = \psi(p^{-k}) = \exp\left(2\pi i \frac{a_k}{p^k}\right).
			\end{align*}
			F"ur unsere Folge heißt das aber gerade $a_k \equiv a_{k+1} \pmod{p^k}$.
			Nach unseren "Uberlegunen zu den Potenzreihen definiert eine solche Folge aber gerade eine eindeutig bestimmte $p$-adische Zahl $a$ mit $a \equiv a_k \pmod{p^k}$ und es gilt
			\begin{align*}
				e_p\left( \frac{a}{p^{k}} \right) = \exp\left( 2\pi i \frac{a_k}{p^k} \right) = \psi \left( \frac{1}{p^k} \right)
			\end{align*}
			Damit sind wir dann aber auch schon fertig.
		\end{proof}
		Der Beweis liefert uns sogar einen kleinen Bonus.
		\begin{korollar}\label{kor:lokal:charTrivialZp}
			Wirkt $\chi:\Kp\to S^1$ im endlichen Fall trivial auf $\Zp$, so gilt $\chi(x) = e_p(ax)$ mit $a\in \Zp$.
		\end{korollar}
		\begin{proof}
			Wie wir im Beweis der Potenzreihendarstellung gesehen haben, konvergiert eine solche Folge von rationalen Zahlen $a_k$ gegen einen Wert aus $\Zp$.
		\end{proof}
		Somit k"onnen wir auch mit gutem Gewissen Folgendes definieren.
		\begin{defi}[Lokale Fouriertransformation]
			Sei $f\in L^1(\Kp)$. Wir definieren dann die \emph{Fouriertransformation} $\hat{f}: \Kp \to \C$ von $f$ durch die Formel
		\begin{align*}
			\hat{f}(\xi) = \int_{\Kp} f(x)e_p(-\xi x)  \dx
		\end{align*}
		f"ur alle $\xi \in \Kp$.
		\end{defi}
		Diese Definition entspricht im Fall $p=\infty$ gerade der klassischen Fouriertransformation bis auf eventuelle Normierung.
		Von daher m"ochten wir auch einige klassische Ergebnisse festhalten und beweisen.
		\begin{lemma}
			Sei $f \in L^1(\Kp)$.
			\begin{enumerate}[label=\emph{(\alph*)}]
				\item Ist $g(x)=f(x)e_p(ax)$ mit $a\in\Kp$, dann gilt $\hat{g}(x) = \hat{f}(x-a)$.
				\item Ist $g(x)=f(x-a)$ mit $a\in\Kp$, dann gilt $\hat{g}(x) = \hat{f}(x-a)e_p(-ax)$.
				\item Ist $g(x)=f(\lambda x)$ mit $\lambda \in\Kp^\times$, dann gilt $\hat{g}(x) =\frac{1}{\abs[\lambda]_p} \hat{f}(\frac{x}{\lambda})$.
			\end{enumerate}
		\end{lemma}
		\begin{proof}
			(a) und (b) sind einfache Folgerungen aus der Definition mit der Multiplikativit"at von $e_p$ und der Translationsinvarianz des Haar-Maß. 
			Bei (c) spielt unsere Normeriung des Absolutbetrags eine Rolle, denn mit dem Variablenwechsel $y\mapsto \lambda^{-1}y$ erhalten wir
			\begin{align*}
				\hat{g}(x) = \int_{\Kp} f(\lambda y) e_p(-xy)dy = \frac{1}{\abs[\lambda]_p} \int_{\Kp} f(y) e_p(-x\lambda^{-1}y)dy = \frac{1}{\abs[\lambda]_p} \hat{f}\left(\frac{x}{\lambda}\right)
			\end{align*}
		\end{proof}
		Beschr"anken wir uns jetzt auf die Fouriertransformation von Schwartz-Bruhat Funktionen.
		Diese besitzen, wie wir noch feststellen werden, besonders gutes Konvergenzverhalten. 
		So ist zum Beispiel folgendes Ergebnis bereits aus der Fourieranalysis in $\R$ bekannt und kann jetzt auf die $p$-adischen Zahlen "ubertragen werden.
		\begin{satz}\label{satz:lokal:umkehrformel}
			Ist $p\leq\infty$ und $f\in\Sw(\Kp)$, so ist $\hat{f} \in \Sw(\Kp)$ und es gilt die \emph{Umkehrformel}
			\begin{align*}
				\hat{\hat{f}}(x) = f(-x)
			\end{align*}
		\end{satz}
		\begin{proof}
			Betrachten wir zuerst den Fall $p<\infty$. Wie wir eben in Lemma \ref{lemma:padischSBF} gesehen haben haben, ist jede Funktion in $\Sw(\Kp)$ eine Linearkombination von Funktionen der Form $f = \ind_{a+p^k\Zp}$. Es reicht also die Aussage f"ur solche $f$ zu zeigen.
			Sei dazu $h:= \ind_{\Zp}$. Wir zeigen $\hat{h} = h$ durch folgende Rechnung
			\begin{align*}
				\hat{h}(x) = \int_\Kp h(y) e_p (-xy) dy_p = \int_\Zp e_p(-xy) dy_p.
			\end{align*}
			Nun ist $\chi(y):=e_p(-xy)$ ein Charakter auf $\Zp$ und genau dann trivial, wenn $x\in\Zp$. 
			Weiter ist $\Zp$ kompakt. 
			Nach Lemma \ref{Lemma:trivialerCharAufKompakt} und unserer Normierung von $dy_p$ folgt also
			\begin{align*}
				\hat(h)(x) = \text{Vol}(\Zp, dy_p) \ind_\Zp = \ind_\Zp = h(x)
			\end{align*}
			
			Wir f"uhren nun folgende Operatoren auf $\Sw(\Kp)$ ein
			\begin{align*}
				L_a f(x) = f(x-a), M_\lambda f(x) = f(\lambda x),
			\end{align*}
			wobei $a \in \Kp$ und $\lambda \in \Kp^\times$. 
			Nun k"onnen wir $f$ schreiben als $L_a M_{p^{-k}}h$. 
			Es folgt
			\begin{align*}
				\hat{f} = (L_a M_{p^{-k}}h)\widehat{\phantom{x}} = \Omega_{-a}p^{k}M_{p^k}\hat{h}=\Omega_{-a}p^{-k}M_{p^k}h.
			\end{align*}
			Also ist $\hat{f} (x) = e_p(-ax)p^k\ind_{p^{-k}\Zp}(x)$. 
			Der Charakter $e_p$ ist lokal konstant, $\hat{f}$ als das Produkt lokal konstanter Funktionen selbst wieder lokal konstant und damit in $\Sw(\Kp)$. 
			Damit haben wir den ersten Teil der Aussage gezeigt.
			
			F"ur den zweiten Teil sehen wir
			\begin{align*}
				\hat{\hat{f}} = (L_a M_{p^{-k}}h)\widehat{\widehat{\phantom{x}}} = L_{-a} (M_{p^k}h)\widehat{\widehat{\phantom{x}}}=L_{-a}M_{p^k}\hat{h} =L_{-a}M_{p^k}h,
			\end{align*}
			also $\hat{\hat{f}} (x) = \ind_{-a+p^k\Zp} (x) = \ind_{a+p^k\Zp} (-x) = f(-x)$. 
			Hier haben wir $p^k\Zp = - p^k\Zp$ ausgenutzt. 
			Damit haben wir die Umkehrformel f"ur den $p$-adischen Fall gezeigt. 
			F"ur $p=\infty$ ist die Formel bereits aus der klassischen Fourieranalysis bekannt.
		\end{proof}
		
\subsection{... Funktionalgleichung}
%unverzweigte charactere: done
%deren form: done
%lokale funktionalgleichung: done
%FILLER
	Die Einheiten $K_p^\times$ der lokalen K"orper $\Kp$ k"onnen dargestellt werden als direktes Produkt $\dedekind_p^\times \times V(\Kp)$, wobei $\dedekind_p^\times$ die Untergruppe der Elemente von $\K_p^\times$ mit Absolutbetrag $1$ und 
	\begin{align*}
		V(\Kp) := \abs[\K_p^\times]
	\end{align*}
	der Wertebereich des Absolutbetrags auf den Einheiten ist. 
	Wir haben n"amlich einen stetigen Homomorphismus $\tilde{\cdot}: x \mapsto \frac{x}{\abs[x]_p}$ von $\K_p^\times$ nach $\dedekind_p^\times$.\\
	F"ur $p=\infty$ ist $\dedekind_p^\times = \{-1, 1\}$ und $V(\K_p) = \R_+^\times$. 
	Jedes $x \in \Kp$ hat gerade Form $x=\text{sgn}(x)\abs[x]_p$, denn $\tilde{x}$ ist gerade die Signumsabbildung.\\
	Wenn $p<\infty$ ist $\dedekind_p^\times = \Z_p^\times$, $V(\K_p) = p^\Z$ und wir k"onnen jedes Element $x \in \K_p^\times$ schreiben als $x = \abs[x]_p\tilde{x}$.
	Es wird nun von Interesse sein, wie die multiplikativen Charaktere auf die Untergruppe $\dedekind_p^\times$ wirken. Dazu zun"achst eine kleine Definition.
	\begin{defi}
		Ein Charakter $\chi \in \text{Hom}_\text{cont}(\K_p^\times, \C^\times)$ ist \emph{unverzweigt}, wenn er trivial auf die Untergruppe $\dedekind_p^\times$ wirkt.
	\end{defi}
	Die unverzweigten Charaktere haben eine recht einfache Form, was folgendes Lemma zeigt.
	\begin{lemma}
		Jeder unverzweigte Charakter $\chi$ auf $\K_p^\times$ hat die Form $\chi(x) = \abs[x]_p^s$ mit $s\in\C$.
	\end{lemma}
	\begin{proof}
		Es ist klar, dass Funktionen dieser Form tats"achlich unverzweigte Quasi-Charaktere sind.
		Umgekehrt sei $\chi$ ein unverzweigter Quasi-Charakter. Dann gilt $\chi(x) = \chi(\abs[x]_p \tilde{x}) = \chi(\abs[x]_p)$.
		Dadurch induziert $\chi$ eine stetige Abbildung auf dem Wertebereich $V(\Kp)$. Wir zeigen, dass diese Abbildung gerade die Form $t\mapsto t^s$ hat.
		
		Sei zuerst $p=\infty$, also $V(\Kp) = \R_+^\times$. Wir definieren $s:= \log(\chi(e))$, also $\chi(e) = e^s$.
		Induktiv l"asst sich nun leicht $\chi(e^n) = e^{ns}$ f"ur ganze Zahlen $n\in\Z$ zeigen. 
		Analog zeigt man 
		\begin{align*}
			\chi(e^{\frac{n}{m}})^m = \chi(e^{m\frac{n}{m}}) =\chi(e^n) = e^{ns},
		\end{align*}
		woraus
		\begin{align*}
			\chi(e^{\frac{n}{m}}) = \left(\chi(e^{\frac{n}{m}})^m\right)^{\frac{1}{m}} = (e^{ns})^\frac{1}{m} = e^{\frac{n}{m}s}
		\end{align*}
		folgt, so dass wir $\chi(e^q) = e^{qs}$ f"ur alle rationalen Zahlen $q\in\Q$ haben. 
		Wegen Stetigkeit gilt nach "Ubergang zu Grenzwerten $\chi(e^r) = e^{rs}$ f"ur alle reellen $r \in \R$, also $\chi(t)=t^s$ f"ur alle $t\in \R_+^\times$.
		
		Der Fall $p<\infty$ ist etwas leichter. Wir definieren dieses mal $s:=\frac{\log(\chi(p))}{\log(p)}$, so dass $\chi(p) = p^s$. Da der Wertebereich aber gerade $p^\Z$ war, folgt die Behauptung sofort.
	\end{proof}
	%%% Filler: allgemeine charaktere sehen dann so aus blabla
	\begin{satz}\label{satz:lokal:stdchar}
		Jeder Charakter $\chi$ von $\K_p^\times$ hat die Form
		\begin{align*}
			\chi(x) = \mu(\tilde{x})\abs[x]_p^s,
		\end{align*}
		wobei $\mu$ ein unit"arer Charakter auf $\dedekind_p^\times$, $\tilde\cdot$ der stetige Homomorphismus von $\K_p^\times$ nach $\dedekind_p^\times$ und $s\in\C$ ist.
	\end{satz}
	\begin{proof}
		Es ist wieder klar, dass $\mu(\tilde{\cdot})\abs_p^s$ tats"achlich ein Charakter ist. 
		Betrachten wir nun einen beliebigen Charakter $\chi$ und definieren $\mu$ als die Einschr"ankung von $\chi$ auf $\dedekind_p^\times$. 
		Da die Untergruppe $\dedekind_p^\times$ kompakt und $\mu$ eine stetige Abbildung nach $\C^\times$ ist, muss $\mu(\dedekind_p^\times)$ eine kompakte Untergruppe von $\C^\times$ sein und ist damit in $S^1$ enthalten. 
		Folglich ist $\mu$ ein unit"arter Charakter auf $\dedekind_p^\times$.
		Damit definiert der stetige Homomorphismus $x\mapsto \chi(x)\mu(\tilde{x})^{-1}$ einen unverzweigten Charakter auf $\K_p^\times$, hat also nach vorherigem Lemma die Form $\chi(x)\mu(\tilde{x})^{-1} = \abs[x]_p^s$ f"ur ein $s\in\C$. Der Satz folgt sofort.
	\end{proof}
	Aus $\abs[\mu(\tilde{x})\abs[x]_p^s] = \abs[x]_p^\sigma$ folgt, dass der Realteil $\sigma=\text{Re}(s)$ eindeutig bestimmt ist. Er wird auch \emph{Exponent} des Charakters $\chi$ genannt.
		%%%Filler: wie sehen charaktere auf R aus (vllt auch Qp) wann sind sie aquivalent was ist eine klasse von charakteren
	Wir kommen zum ersten Ergbnis aus Tates Doktorarbeit
	
	\begin{satz}[Lokale Funktionalgleichung]
		Sei $f_p \in \Sw(K_p)$ und $\chi = \mu \abs_p^s$. Sei weiter $\sigma = \text{Re}(s)$. Dann gelten die folgenden Aussagen:
		\begin{enumerate}[label=\emph{(\roman*)}]
			\item $Z(f,\chi) = Z(f, \mu, s)$ ist holomorph und absolut konvergent f"ur $\sigma > 0$
			\item Auf dem Streifen $0 < \sigma < 1$ haben wir eine Funktionalgleichung
				\begin{align*}
					Z(\hat{f}, \check{\chi}) = \gamma(\chi, e_p, dx) Z(f,\chi),
				\end{align*}
				wobei $\gamma(\chi, \psi, dx)$ unabh"angig von $f$ und meromorph als Funktion in $s$ ist. Damit besitzt $Z(f,\chi)$ eine meromorphe Fortsetzung auf ganz $\C$
		\end{enumerate}
	\end{satz}
	\begin{proof}
		(i) Es reicht im Allgemeinen zu zeigen, dass das Integral
		\begin{align*}
			\int_{\Kp \setminus \{0\}} \abs[f] \cdot \abs[x]_p^{\sigma-1} dx
		\end{align*}
		endlich ist, denn das Haar-Maß $d^\times x$ ist ein konstantes vielfaches von $dx$.
		
		Sei zun"achst $p=\infty$. 
		Wir k"onnen $\abs[f_p]$  absch"atzen durch ein skalares Vielfaches von $\frac{1}{1+\abs[x]_\infty^n}$, wobei $n \in \N$ mit $n > \sigma$.
		Zusammen haben wir dann
		\begin{align*}
			\int_{\Kp \setminus \{0\}} \abs[f] \cdot \abs[x]_p^{\sigma-1} dx \leq C\cdot \int_{\Kp} \frac{\abs[x]_p^{\sigma-1}}{1+\abs[x]_\infty^n}dx < \infty.
		\end{align*}
		%%%EVTL TODO: Erklaerung warum endlich: Dazu aufteilen in kompaktum K und R\K. Einmal stetig kompakt also beschr, andere mal abschaetzen durch 1/x^2
		Im endlichen Fall
		
		%%%TODO
		
		(ii) Wir folgen Tate und beweisen ein kleines Lemma.
		\begin{lemma}
			F"ur alle Charaktere $\chi$ mit Exponenten $0<\sigma<1$ und beliebige Funktionen $f,g \in \Sw(\Kp)$ gilt:
			\begin{align*}
				Z(f, \chi) Z(\hat g, \check{\chi}) = Z(\hat f, \check{\chi}) Z(g, \chi) 
			\end{align*}
		\end{lemma}
		\begin{proof}
			Nach (i) haben wir absolute Konvergenz der Integrale f"ur Exponenten $\sigma > 0$. 
			Zudem ist $\check{\chi} = \abs[x]_p\chi^{-1} = \abs[x]_p^{1-s} \mu^{-1}$, also haben wir in diesem Fall Konvergenz f"ur $\sigma <1$.
			Damit sind die obigen Zeta-Funktionen wohldefiniert auf dem Streifen den wir betrachten.
			Wir schreiben das Produkt als Doppelintegral "uber $\K_p^\times \times \K_p^\times$
			\begin{align*}
				Z(f, \chi) Z(\hat g, \check{\chi}) 
				&= \iint\limits_{\K_p^\times \times \K_p^\times} f(x)\chi(x) \hat{g}(y)\chi^{-1}(y)\abs[y]_p d^\times (x,y) \\
				&= \iint\limits_{\K_p^\times \times \K_p^\times} f(x) \hat{g}(y)\chi(xy^{-1})\abs[y]_p d^\times (x,y)
			\end{align*}
			Das Integral ist invariant unter der Translation $(x,y)\mapsto (x,xy)$ und wir erhalten
			\begin{align*}
				\iint\limits_{\K_p^\times \times \K_p^\times} f(x) \hat{g}(xy)\chi(y^{-1})\abs[xy]_p d^\times (x,y).
			\end{align*}
			Nach Fubini ist das wiederum gleich
			\begin{align*}
				\int_{\K_p^\times} \left( \int_{\K_p^\times} f(x) \hat{g}(xy) d^\times x \right) \chi(y^{-1})\abs[y]_p d^\times y.
			\end{align*}
			Wir m"ussen also nur noch zeigen, dass das innere Integral symmetrisch in $f$ und $g$ ist.
			Dazu erinnern wir uns, dass $d^\times x = c \frac{dx}{\abs[x]_p}$ und nach der Definition der Fouriertransformation daher
			\begin{align*}
				\int_{\K_p^\times} f(x) \hat{g}(xy) d^\times x  
				= c \int_{\Kp}  \int_{\Kp} f(x) g(z) e_p(-xyz) dz dx = \int_{\K_p^\times} g(z) \hat{f}(zy) d^\times z,
			\end{align*}
			wobei wieder Fubini das Vertauschen der Reihenfolge bei der Integration erlaubt.
		\end{proof}
		Damit sind wir auch schon fast mit dem eigentlichen Beweis fertig. Wir fixieren eine geeignete Schwartz-Bruhat Funktion $g\in \Sw(\Kp)$ und setzen
		\begin{align*}
			\gamma(\chi, e_p, dx) := \frac{Z(\hat g, \check{\chi})}{Z(g, \chi)}.
		\end{align*}
		Aus dem Lemma folgt, dass dieser Quotient sicherlich unabh"angig von der Wahl von $g$ ist, und, dass
		\begin{align*}
			Z(\hat f, \check{\chi}) = \gamma(\chi, e_p, dx) Z(f, \chi).
		\end{align*}
		%%%%TODO: FILLER laber darueber dass es solche funktionen gibt, dass diese meromorph sind und dass wir die faktoren explizit ausrechnen
	\end{proof}
\subsection{... Berechnungen}

\subsubsection{Der Fall \texorpdfstring{$p = \infty$}{p gleich unendlich}}
	Wir betrachten zuerst die Klasse $\chi = \abs_\infty^s$ und nehmen die Schwartz-Bruhat Funktion
	\begin{align*}
		f(x) = e^{-\pi x^2}.
	\end{align*} 
	Wir behaupten, dass $f$ ihre eigene Fouriertransformierte ist. Dazu rechnen wir
	\begin{align*}
		\hat{f}(\xi) 	&= \int_{\K_\infty} e^{-\pi x^2} e_\infty(-x\xi)dx 
					= \int_{\K_\infty} e^{-\pi x^2} e^{2\pi ix\xi}dx
					\\&= \int_{\K_\infty} e^{-\pi (x^2 - 2ix\xi - \xi^2)} e^{-\pi \xi^2}dx
					= f(\xi) \int_{\K_\infty} e^{-\pi (x - i\xi)^2} dx.
	\end{align*}
	Es gen"ugt also zu zeigen, dass das Integral gleich $1$ ist. Dazu nutzen wir das bekannte Integral $\int_{\K_\infty} e^{-\pi x^2} dx = 1$ und etwas Kontourintegration.
	Sei $\gamma$ das Rechteck von $-r$ nach $r$ auf der reellen Achse, dann runter zu $r-i\xi$, horizontal zu $-r-i\xi$ und wieder zur"uck zu $-r$.
	Da $f$ eine ganze Funktion ist, gilt $\int_\gamma f(z) dz = 0$. 
	Die Integrale an der linken und rechten Seite des Rechtecks konvergieren gegen $0$ wenn $r$ anw"achst, denn f"ur $z = \pm r - iy$ und $0\leq y\leq \xi$ gilt
	\begin{align*}
		\abs[f(z)] = \abs[e^{-\pi (\pm r-iy)^2}] = e^{-\pi (r^2 - y^2)}
	\end{align*}
	und wir haben die Absch"atzung
	\begin{align*}
		\abs[\int_{\pm r}^{\pm r-i\xi} f(z)dz] = \abs[\int_{0}^{\xi} f(\pm r - iy)dy] \leq e^{-\pi r^2} \int_{0}^{\xi} e^{\pi y^2}dy \xrightarrow[]{r\to \infty} 0.
	\end{align*}
	Folglich muss schon $\int_{\K_\infty} f(x-i\xi) dx = \int_{\K_\infty} f(x)dx = 1$ gelten und wir sind fertig mit der Fouriertransformation.
	
	Nun zu den Zeta-Funktionen:
	\begin{align*}
		Z(f, \chi) = Z(f, \abs_\infty^s) = \int_{\Kinfx} f(x) \abs[x]_p^s \dxx 
							= \int_{\R^\times} e^{-\pi x^2} \abs[x]_\infty^s \dxx 
							= 2 \int_0^\infty e^{-\pi x^2} x^{s-1} \dx
	\end{align*}
	Wir benutzen den Trafo $u =\pi x^2 \Rightarrow du = 2\pi^{1/2}u^{1/2}$ und erhalten
	\begin{align*}
		Z(f, \chi) &= \int_0^\infty e^{-u}(u\pi^{-1})^\frac{s-1}{2} \pi^{-\frac{1}{2}} u^{-\frac{1}{2}} du\\	
							&= \pi^{-\frac{s}{2}} \int_0^\infty e^{-u} u^{\frac{s}{2} -1}du = \pi^{-\frac{s}{2}} \Gamma\left(\frac{s}{2}\right)
	\end{align*}
	Mit dem gleichen Argumentation rechnen wir auch
	\begin{align*}
		Z(\hat{f}, \check{\chi}) = Z(f, \abs_\infty^{1-s}) = \pi^{-\frac{1-s}{2}} \Gamma\left(\frac{1-s}{2}\right).
	\end{align*}
	Jetzt k"onnen wir endlich den versprochenen Faktor
	\begin{align*}
		\gamma(\abs_\infty^s, e_\infty, \dx) = \frac{\pi^{-\frac{1-s}{2}} \Gamma\left(\frac{1-s}{2}\right)}{\pi^{-\frac{s}{2}} \Gamma\left(\frac{s}{2}\right)}
	\end{align*}
	angeben und sehen, dass dieser auf dem Streifen $0<\sigma<1$ als Quotient holomorpher Funktionen meromorph ist.

	Nun zur zweiten und auch schon letzten Klasse $\chi = \sgn \abs_\infty^s$ von multiplikativen Charakteren auf $\Kinf$. 
	Wir w"ahlen die Funktion 
	\begin{align*}
		f_\pm (x) = x e^{-\pi x^2} \in \Sw(\K_\infty)
	\end{align*}
	und bemerken zun"achst die Beziehung $f_\pm(x) = (-2\pi)^{-1} f'(x) $.
	Damit k"onnen wir die Fouriertransformation schnell aus einem Ergebnis der klassischen Fourieranalysis gewinnen.
	Denn wir haben
	\begin{align*}
		\hat{f}_\pm(\xi) 	&= \int_{-\infty}^\infty f_\pm (x) e_\infty(-x\xi)\dx
							 = \int_{-\infty}^\infty (-2\pi)^{-1} f'(x) e_\infty(-x\xi)\dx\\
							&= \left[ (-2\pi^{-1} f(x) e_\infty(-x\xi)  \right]_{-\infty}^\infty 
								- \int_{-\infty}^\infty -2\pi^{-1} f(x) \cdot (-2\pi i \xi) e_\infty(-x\xi)\dx&\\
							&= 0 - i \xi \hat{f}(\xi) = -i \xi f(\xi) = -i f_\pm(\xi).
	\end{align*}
	Wir berechnen die Zeta-Funktionen
	\begin{align*}
		Z(f_\pm, \chi) 	&= Z(f, \sgn\abs_\infty^s) 
						= \int_{\Kinfx} f_\pm(x) \sgn(x)\abs[x]_\infty^s \dxx \\
						&= \int_{\Kinfx} x f(x) \sgn(x) \abs[x]_\infty^s \dxx
						= \int_{\Kinfx} f(x) \abs[x]_\infty^{s+1} \dxx \\
						&= Z(f, \abs_\infty^{s+1}) = \pi^{-\frac{s+1}{2}}\Gamma\left(\frac{s+1}{2}\right)
	\end{align*}
	und mit $\check{\chi} = \sgn^{-1}\abs_\infty^{1-s} = \sgn\abs_\infty^{1-s} $
	\begin{align*}
		Z(\hat{f}_\pm, \chi) 	&= Z(-i f_\pm, \sgn\abs_\infty^{1-s}) 
						= -i \int_{\Kinfx} f_\pm(x) \sgn(x)\abs[x]_\infty^{1-s} \dxx \\
						&= -i \int_{\Kinfx} x f(x) \sgn(x) \abs[x]_\infty^{1-s} \dxx
						= -i \int_{\Kinfx} f(x) \abs[x]_\infty^{2-s} \dxx \\
						&= -i Z(f, \abs_\infty^{2-s}) = -i \pi^{-\frac{2-s}{2}}\Gamma\left(\frac{2-s}{2}\right).
	\end{align*}
	Damit haben wir den Faktor
	\begin{align*}
		\gamma(\sgn\abs_\infty^s, e_\infty, \dx) = i\frac{\pi^{-\frac{2-s}{2}} \Gamma\left(\frac{2-s}{2}\right)}{\pi^{-\frac{s+1}{2}} \Gamma\left(\frac{s+1}{2}\right)}
	\end{align*}
	der nach der gleichen Begr"undung wie oben meromorph ist.
\subsubsection{Der Fall \texorpdfstring{$p < \infty$}{p kleiner unendlich}}
	Wir beginnen wieder mit dem unverzweigten Fall $\chi = \abs_p$ und definieren unsere Schwartz-Bruhat Funktion
	\begin{align*}
		f_0(x) = \ind_{\Zp}(x).
	\end{align*}
	Wie auch schon im archimedischen Fall werden wir feststellen, dass $f_0$ ihre eigene Fouriertransformierte ist.
	Wir rechnen
	\begin{align*}
		\hat{f}_0 (\xi) = \int_{\Kp} f_0(x) e_p(-x\xi) \dx = \int_{\Zp} e_p(-x\xi) \dx.
	\end{align*}
	Nun wissen wir, dass $\Zp$ kompakt ist und der Charakter $e_p(-x\xi)$ genau dann trivial auf $x\in\Zp$ wirkt, wenn auch $\xi \in \Zp$.
	In diesem Fall entpricht das Integral nach Lemma \ref{Lemma:trivialerCharAufKompakt} gerade dem Volumen von $\Zp$ bez"uglich $\dx$.
	Ansonsten verschwindet es.
	Mit unserer Normierung des Haar-Maßes folgt dann
	\begin{align*}
		\hat{f}_0 (\xi) = \int_{\Zp} e_p(-x\xi) \dx = \text{Vol}(\Zp, \dx) \ind_\Zp(\xi) = \ind_\Zp (\xi) = f_0(\xi).
	\end{align*}
	F"ur die Berechnungen der Zeta-Funktionen erinnern wir uns an die Beobachtung $\Zp\setminus\{0\} = \bigcup_{k=0}^\infty p^k\Zpx$ als disjunkte Vereinigung.
	\begin{align*}
		Z(f_0, \chi) 	&= Z(f_0, \abs_p^s) 
						= \int_{\Kpx} f_0(x) \abs[x]_p^s \dxx 
						= \int_{\Zp\setminus\{0\}} \abs[x]_p^s \dxx 
						\\&= \sum_{k=0}^{\infty} \int_{p^k\Zpx} \abs[x]_p^s \dxx
						= \sum_{k=0}^{\infty} \int_{\Zpx}  \abs[p^kx]_p^s \dxx
						\\&= \sum_{k=0}^{\infty} \int_{\Zpx}  p^{-ks} \dxx
						= \sum_{k=0}^{\infty} p^{-ks} \text{Vol}(\Zpx, \dxx)
						= \frac{1}{1-p^{-s}}
	\end{align*}
	und analog
	\begin{align*}
		Z(\hat{f}_0, \check{\chi}) 	= Z(f_0, \abs_p^{1-s})	= \frac{1}{1-p^{s-1}}.
	\end{align*}
	Der Faktor hat damit die Form
	\begin{equation*}
		\gamma(\abs_p^s, e_p, \dx) = \frac{1-p^{-s}}{1-p^{s-1}}
	\end{equation*}
	und ist insbesondere somit holomorph im betrachteten Streifen.
	
	Nun kommen wir zum verzweigten Fall $\chi = \mu \abs_p^s$.
	Bevor wir allerdings mit den eigentliche Berechnungen anfangen, schauen wir uns den unit"aren Charakter $\mu:\Zpx\to S^1$ etwas genauer an.
	W"ahlen wir eine offene Umgebung $U$ der $1 \in S^{1}$, die nur die triviale Untergruppe enth"alt, so finden wir aufgrund der Stetigkeit von $\mu$ eine offene Umgebung $V$ der $1 \in \Zpx$ mit $\mu(V)\subseteq U$.
	Diese enthalten aber stets eine Untergruppe der Form $1+p^n\Zp$.
	Da $\mu$ aber ein Gruppenhomomorphismus ist, muss diese Untergruppe bereits auf $1$ abgebildet werden.
	Es gibt also f"ur jeden Charakter $\chi = \mu \abs_p^s$ ein kleinstes $n\in\N$ mit $\mu(1+p^{n}\Zp) = 1$.
	Wir nennen dann $p^n$ den \emph{Konduktor von $\chi$}.
	Analog l"asst sich auch der Konduktor eines additiven Charakters definieren.
	Mit Hilfe des Konduktors $p^n$ definieren wir nun die Schwartz-Bruhat Funktion
	\begin{align*}
		f_n(x) = e_p(x)\ind_{p^{-n}\Zp}(x).
	\end{align*}
	Die Berechnung der Fouriertransformation erfolgt "ahnlich zum unverzweigten Fall:
	\begin{align*}
		\hat{f}_n(\xi) 	= \int_{\Kp} f_n(x) e_p(-x\xi)\dx 
						= \int_{p^{-n}\Zp} e_p\left(x(1-\xi)\right)\dx
	\end{align*}
	Der Charakter $\psi(x) = e_p(x(1-\xi))$ wirkt genau dann trivial auf $p^{-n}\Zp$, wenn $1-\xi \in p^n\Zp$, oder "aquivalent $\xi \in 1+p^n\Zp$.
	Es folgt 
	\begin{align*}
		\hat{f}_n(\xi) 	= \text{Vol}(p^{-n}\Zp, dx) \ind_{1+p^n\Zp}(\xi) =p^n \ind_{1+p^n\Zp}(\xi).
	\end{align*}
	blablabla zeta funktion
	\begin{align*}
		Z(f_n, \chi) &= Z(f_n, \mu\abs_p^s) 	
											= \int_{\Kpx} f_n(x) \mu(\tilde{x}) \abs[x]_p^s \dxx
											= \int_{p^{-n}\Zp\setminus\{0\}} e_p(x) \mu(\tilde{x}) \abs[x]_p^s \dxx
											\\&= \sum_{k=-n}^\infty  \int_{p^k\Zpx} e_p(x) \mu(\tilde{x}) \abs[x]_p^s \dxx
											= \sum_{k=-n}^\infty  \int_{\Zpx} e_p(p^k x) \mu(\widetilde{p^k x}) \abs[p^kx]_p^s \dxx
											\\&= \sum_{k=-n}^\infty p^{-ks} \int_{\Zpx} e_p(p^k x) \mu(x) \dxx.
	\end{align*}
	Ein Integral der Form $g(\omega, \lambda) = \int_{\Zpx} \omega(x)\lambda(x) \dxx$ mit multiplikativen Charakter $\omega: \Zpx \to S^1$ und additiven Charakter $\lambda: \Zp \to S^1$ wird \emph{Gauß-Summe} genannt.
	Mit $e_{p,k} (x) = e_p(p^kx)$ haben wir dann
	\begin{align}\label{eq:ZetaSumme}
		Z(f_n, \chi) = \sum_{k=-n}^\infty p^{-ks} g(\mu,e_{p,k}).
	\end{align}
	F"ur die weitere Berechnung definieren wir $U_k:= 1+p^k\Zp$ und beweisen ein kleines Lemma "uber Gauß-Summen.
	\begin{lemma}\label{lemma:gausssumme}
		Seien $\gamma$ und $\lambda$ wie oben. 
		Seien $n$ und $r$ die kleinsten Zahlen mit $\gamma(1+p^n\Zp) = 1$ und $\lambda(p^r\Zp)=1$, d.h. $p^n$ und $p^r$ sind die Konduktoren von $\gamma$ bzw. $\lambda$.
		Es gelten folgende Aussagen:  
		\begin{enumerate}[label=(\roman*)]
			\item Wenn $n>r$, dann $g(\omega,\lambda) = 0$. \label{lemma:gausssummei}
			\item Wenn $n=r$, dann $\abs[g(\omega,\lambda]^2 = c^2p^{-n}$.
			\item Wenn $n<r$, dann $\abs[g(\omega,\lambda]^2 = c^2\left[p^{-n} - p^{-r}\right]$.
		\end{enumerate}
	\end{lemma}
	%Wir werden Aussage (iii) hier nicht brauchen und verweisen daher f"ur den Beweis auf Ramakrishnan \cite{rama} Lemma 7-4.
	\begin{proof}
		F"ur (i) zerlegen wir $\Zpx$ in die Nebenklassen, die von $U_r=1+p^r\Zp$ erzeugt werden.
		Ein generisches Element aus $aU_r$ hat die Form $a(1+p^rb)$, wobei $a$ ein Repräsentant der Nebenklasse und $b$ aus $\Zp$ ist. 
		Wir haben $\lambda(a(1+p^rb)) = \lambda(a)\lambda(p^rab) = \lambda(a)$ nach der Definition des Konduktors.
		Daher %%%REpresentantensystem ueberlegen
		\begin{align*}
			g(\omega,\lambda) = \sum_{aU_r}  \int_{aU_r} \omega(x)\lambda(x) \dxx = \sum_{aU_r} \omega(a)\lambda(a) \int_{U_r} \omega(x)\dxx.
		\end{align*}
		Da aber $n>r$ wirkt $\omega$ nicht trivial auf $U_r$ und somit verschwindet das Integral.
		
		Weiter zur Aussage (ii) und (iii): Sei also $n\leq r$. Wir rechnen
		\begin{align*}
			\abs[g(\omega,\lambda)]^2 	&= \int_{\Zpx} \omega(x)\lambda(x)\dxx \cdot \conj{\int_{\Zpx} \omega(x)\lambda(x)\dxx}\\
										&= \int_{\Zpx} \int_{\Zpx} \omega(xy^{-1})\lambda(x-y) \dxx\dxx[y]\\
										&= \int_{\Zpx} \omega(x)h(x)\dxx
		\end{align*}
		wobei wir im letzten Schritt zum einen die Translation $x \mapsto xy$ und zum anderen die Funktion
		\begin{align*}
			h(x) = \int_{\Zpx} \lambda(xy-y)) \dxx[y] = c \int_{\Zpx} \lambda(y(x-1)) \frac{\dx[y]}{\abs[y]_p} = c \int_{\Zpx} \lambda(y(x-1))\dx[y]
		\end{align*}
		eingef"uhrt haben. Wegen $\Zpx = \Zp - p\Zp$ k"onnen wir das Integral weiter aufspalten.
		\begin{align*}
			h(x) =  c \int_{\Zp - p\Zp} \lambda(y(x-1))\dx[y] = c \int_{\Zp} \lambda(y(x-1))\dx[y] - c \int_{p\Zp} \lambda(y(x-1))\dx[y].
		\end{align*}
		Nun haben wir wieder den Fall von Lemma \ref{Lemma:trivialerCharAufKompakt}. 
		$y\mapsto \lambda(y(x-1))$ ist trivial auf $\Zp$ genau dann wenn $x-1 \in p^r\Zp$, d.h. wenn $x \in U_r$.
		"Ahnlich verh"alt es sich mit $y\mapsto \lambda(y(x-1))$ auf $p\Zp$, wobei dieser genau dann trivial ist, wenn $x\in U_{r-1}$.
		Es gilt also
		\begin{align*}
			h(x) 	=&  c \Vol(\Zp,\dx)\ind_{U_r} - c \Vol(p\Zp, \dx) \ind_{U_{r-1}} \\
					=& c \Vol(\Zp,\dx)\ind_{U_r} - c p^{-1} \Vol(\Zp, \dx) \ind_{U_{r-1}}
		\end{align*}
		Einf"ugen in $\abs[g(\omega,\lambda)]^2$ ergibt dann
		\begin{align*}
			\abs[g(\omega,\lambda)]^2 	&= \int_{\Zpx} \omega(x)h(x)\dxx \\
										&= c\Vol(\Zp,\dx) \int_{U_r} \omega(x)\dxx - c p^{-1}\Vol(\Zp, \dx) \int_{U_{r-1}} \omega(x)\dxx.
		\end{align*}
		Im Fall (ii) haben wir $n=r$. Damit ist der erste Integrand trivial, der zweite jedoch nicht.
		Folglich verschwindet das zweite Integral.
		Wir haben
		\begin{align*}
			\abs[g(\omega,\lambda)]^2 =  c\Vol(\Zp,\dx)\Vol(U_n,\dxx)
		\end{align*}
		und sind somit fertig.
		Im Fall (iii) ist $n<r$. Beide Integranden sind trivial und es folgt mit
		\begin{align*}
			\abs[g(\omega,\lambda)]^2 =  c\Vol(\Zp,\dx)\Vol(U_r,\dxx) - cp^{-1}\Vol(\Zp, \dx)\Vol(U_{r-1}, \dxx)
		\end{align*}
		die Behauptung. Damit ist das Lemma bewiesen.
	\end{proof}
	Zur"uck zur Berechnung der Zeta-Funktion.
	Der multiplikative Charakter $\mu$ hat den Konduktor $p^n$, w"ahrend die additiven Charaktere $e_{p,k}(x) = e_p(p^kx)$ offensichtlich den Konduktor $p^{-k}$ haben.
	Nach Lemma \ref{lemma:gausssumme} (i) verschwinden in \ref{eq:ZetaSumme} fast alle Summanden und wir erhalten
	\begin{align*}
		Z(f_n, \chi) = \sum_{k=-n}^\infty p^{-ks} g(\mu,e_{p,k}) = p^{ns} g(\mu,e_{p,-n})
	\end{align*}
	Die verbleibende Gauß-Summe konvergiert dann nach Aussage (ii) des Lemmas.
	
	F"ur die Berechnung der zweiten Zeta-Funktion bemerken wir zun"achst, dass $\mu^{-1}= 1/\mu = \conj{\mu}/(\mu \conj{\mu}) = \conj{\mu}$ den gleichen Konduktor wie $\mu$ hat.
	\begin{align*}
		Z(\hat{f}_n, \check{\chi}) 	&= Z(\hat{f}_n, \conj{\mu}\abs_p^{1-s})
									= p^n \int_{1+p^n\Zp}  \conj{\mu}(\tilde{x}) \abs[x]_p^{1-s} \dxx
									\\&= p^n \int_{1+p^n\Zp} \dxx
									= p^n c \int_{p^n\Zp} \dx
									= c.
	\end{align*}
	Zu guter Letzt der erhalten wir den holomorphen Faktor
	\begin{align*}
		\gamma(\mu\abs_p^s, e_p, \dx) = \frac{c}{p^{ns} g(\mu,e_{p,-n})} = \frac{cp^{-ns} \conj{g(\mu,e_{p,-n})}}{c^2p^{-n}} = c^{-1} p^{n(1-s)} \conj{g(\mu,e_{p,-n})}
	\end{align*}
	%%%TODO: volumen berechnunen fuer U_n einmal explizit am anfang.