\section{Lokale Betrachtungen}
\subsection{Lokale K"orper}
	Wir betrachten im Folgenden alle 
	%Q_p topologischer Ring: done
	%Skalierung der Mengen durch multiplikation: done
	%Haarmass auf Q_p^\times: done
	%normierung des mult masses
	%annulus
	\begin{satz}\label{satz:QpIstLokalKompakt}
		F"ur alle Stellen $p\leq\infty$ gilt
		\begin{enumerate}[label=(\roman*)]
		\item $(\Kp, +)$ ist eine lokalkompakte Gruppe.
		\item $(\K_p^\times, \cdot)$ ist eine lokalkompakte Gruppe.
		\end{enumerate}
	\end{satz}
	\begin{proof}
	%%% hier normierung des haar masses auf Qp
	
		Zu (i): Wir weisen zun?chst die Stetigkeit der Addition und der Negierung nach. 
		Seien $x_n \to x$ und $y_n \to y$ konvergente Folgen in $\Kp$. 
		Es ist zu zeigen, dass $x_n+y_n$ gegen $x+y$ konvergiert. 
		Nach der Dreiecksungleichung gilt
		\begin{align*}
			\abs[(x_n+y_n) - (x+y)]_p = \abs[(x_n - x) + (y_n - y)]_p \leq \abs[(x_n - x)]_p + \abs[(y_n - y)]_p.
		\end{align*}
		Die rechte Seite konvergiert gegen $0$, also ist die linke eine Nullfolge und die Stetigkeit der Addition folgt.
		"Ahnlich zeigen wir $-x_n \to -x$, denn $\abs[(-x_n) - (-x)]_p = \abs[x_n-x]_p$.
		Damit ist $\Kp$ eine topologische Gruppe. 
		
		Da die Topologie von einer Metrik induziert wird, ist diese hausdorffsch.
		Wir m"ussen folglich nur noch die Lokalkompaktheit zeigen.
		Wegen der Stetigkeit der Addition reicht es dazu eine kompakte Umgebung der $0$ zu finden, denn sei $K$ eine kompakte Nullumgebung, so ist $a+K$ f"ur beliebige $a\in \Kp$ eine kompakte Umgebung von $a$.
		Wir behaupten, dass die Umgebung
		\begin{align*}
			\Zp = \{x \in \Kp : \abs[x]_p \leq 1\}
		\end{align*}
		kompakt ist.
		F"ur $p=\infty$ ist $\Z_\infty = [-1,1]$ also kompakt.
		F"ur $p<\infty$ gen"ugt es zu zeigen, dass $\Zp$ vollst"andig und totalbeschr"ankt ist. 
		Ersteres folgt daraus, dass $\Zp$ als abgeschlossene Menge des vollst"andigen Raumes $\Kp$ selber vollst"andig ist.
		Eine Menge hei?t totalbeschr"ankt, wenn wir sie f"ur jedes $\epsilon > 0$ mit endlich vielen $\epsilon$-B"allen "uberdecken k"onnen.
		Wir k"onnen uns auf $\epsilon$ der Form $p^{-k}$, $k\geq 0$ beschr"anken, da unsere Metrik nur den diskreten Wertebereich $p^\Z$ hat.
		Es ist $p^{k+1}\Zp$ eine Untergruppe von $\Zp$ und f"ur den Index gilt $[\Zp : p^{k+1}\Zp] = p^{k+1}$. Das bedeutet, dass die $p^{-k}$-B"alle
		\begin{align*}
			a + p^{k+1}\Zp = \{y\in \Zp : \abs[y-a]_p \leq p^{-k-1}\} = \{y\in \Zp : \abs[y-a]_p < p^{-k}\} = B_{p^{-k}}(a)
		\end{align*}
		mit $a=0\dots p^{k+1}-1$ die Menge $\Zp$ "uberdecken. Als vollst"andige und totalbeschr"ankte Menge ist $\Zp$ somit kompakt.
		
		
		Zu (ii): $\K_p^\times = \Kp\setminus \{0\}$ ist ein offener Teilraum von $\Kp$ und somit selbst wieder hausdorffsch und lokalkompakt.
		F"ur die Stetigkeit seien $x_n \to x$ und $y_n \to y$ zwei konvergente Folgen in $\K_p^\times$. 
		Wir zeigen, dass $x_ny_n$ gegen $xy$ konvergiert.
		Dies folgt aus der Absch"atzung
		\begin{align*}
			\abs[x_n y_n - xy]_p 
			&= \abs[x_n y_n -x_ny + x_n y- xy]_p  \\
			&\leq \abs[x_n y_n -x_ny]_p + \abs[x_n y- xy]_p \\
			&= \underbrace{\abs[x_n]_p}_{\text{beschr"ankt}} \underbrace{\abs[y_n -y]_p}_{\to 0} + \underbrace{\abs[x_n- x]_p}_{\to 0} \abs[y]_p \to 0
		\end{align*}
		Weiter zur Invertierung. Wegen
		\begin{align*}
			\abs[\frac{1}{x_n} - \frac{1}{x}]_p = \frac{\abs[x_n - x]_p}{\abs[x_nx]_p} \to 0
		\end{align*}
		folgt, dass $x_n^{-1}$ gegen $x^{-1}$ konvergiert und die Stetigkeit ist gezeigt.
	\end{proof}
	
	\begin{satz}\label{satz:multiplikativesHaarMass}
		F"ur jede messbare Menge $A\subset \Kp$ und jedes $x\in \Kp$ gilt
		\begin{align*}
			\mu(xA) = \abs[x]_p \mu(A).
		\end{align*}
		Insbesondere folgt f"ur jedes $f \in L^1(\Kp)$ und $x\not= 0$:
		\begin{align*}
			\int_\Kp f(x^{-1}y)d\mu(y) = \abs[x]_p \int_\Kp f(y)d\mu(y).
		\end{align*}
	\end{satz}
	\begin{proof}
		Sei $x\in \K_p^\times$. Das Funktion $\mu_x$ definiert durch
		\begin{align*}
			\mu_x (A) = \mu(xA)
		\end{align*}
		definiert wieder ein Haar-Ma? auf $\Kp$ und unterscheidet sich daher nur durch Skalierung mit einer positiven Konstante $c>0$ von $\mu$.
		Ziel wird es nun sein $c=\abs[x]_p$ zu zeigen. 
		Dies ist klar im Fall $p=\infty$. 
		F"ur $p<\infty$ reicht es dank unserer Normierung $\mu(x\Zp) = \abs[x]_p$ zu zeigen.
		Sei dazu $\abs[x]_p = p^{-k}$.
		Dann ist $x=p^ky$ mit $y\in \Zp$ und $x\Zp = p^k\Zp$.
		Daher reduziert sich unsere Betrachtung auf $\mu(p^k\Zp) = p^{-k}$.
		Beginnen wir mit dem Fall $k\geq 0$. Dann ist $p^k\Zp$ eine Untergruppe von $\Zp$ und f"ur den Index gilt $[\Zp : p^k\Zp] = p^k$.
		Wir haben also eine disjunkte Zerlegung $\Zp = \bigsqcup_{a=0}^{p^{k}-1} a + p^k\Zp$.
		Aus der Translationsinvarianz folgern wir dann
		\begin{align*}
			1 = \mu (\Zp) = \sum_{a=0}^{p^{k}-1} \mu(a + p^k\Zp) =\sum_{a=0}^{p^{k}-1} \mu(p^k\Zp) = p^k \mu(p^k\Zp).
		\end{align*}
		Die Behauptung folgt dann durch einfaches Umformen. 
		Im anderen Fall $k<0$ ist umgekehrt $\Zp$ eine Untergruppe von $p^k\Zp$ mit Index $[p^k\Zp:\Zp]= p^{-k}$ und die Behauptung folgt analog.
	\end{proof}
	
	\begin{satz}
		Ist $dx$ ein additives Haar-Ha? auf $\Kp$, so definiert $\frac{dx}{\abs[x]_p}$ ein multiplikatives Haar-Ma? $d^\times x$ auf $\K_p^\times$.
		Insbesondere gilt dann f"ur alle $g \in L^1(\K_p^\times)$
		\begin{align*}
			\int_{\K_p^\times} g(x) d^\times x = \int_{\Kp \setminus \{0\}} g(x) \frac{dx}{\abs[x]_p}
		\end{align*}
	\end{satz}
	\begin{proof}
		Wir haben bereits gezeigt, dass $\K_p^\times$ eine lokalkompakte Gruppe ist und folglich ein Haar-Ma? besitzt.
		Wenn wir nun ein positives, lineares Funktional auf $C_c(\K_p^\times)$ angeben, erhalten wir nach Rieszschen Darstellungssatz ein Radonma?, welches diesem Funktional entspricht. 
		Ist nun $g \in C_c(\K_p^\times)$, so ist $g\abs_p^{-1} \in C_c(\Kp\setminus\{0\})$. 
		Dies ist in der Tat eine eins-zu-eins Zuweisung.
		Wir definieren nun das Funktional
		\begin{align*}
			\Phi(g) = \int_{\Kp \setminus \{0\}} g(x) \frac{dx}{\abs[x]_p}.
		\end{align*}
		Dieses ist offensichtlich ein positives, nicht-triviales, lineares Funktional auf $ C_c(\K_p^\times)$. Dieses ist translationsinvariant, denn
		\begin{align*}
			\int_{\Kp \setminus \{0\}} g(y^{-1}x) \frac{dx}{\abs[x]_p} = \int_{\Kp \setminus \{0\}} g(x) \frac{\abs[y]_pdx}{\abs[yx]_p}=\int_{\Kp \setminus \{0\}} g(x) \frac{dx}{\abs[x]_p},
		\end{align*}
		folglich kommt es von einem Haar-Ma? $d^\times x$. 
		Der zweite Teil der Behauptung folgt aus der Tatsache, dass die Funktionen in $C_c$ dicht in $L^1$ liegen.
		Daher ist es m"oglich "uber Grenzwerte die Gleichung $\int_{\K_p^\times} g(x) d^\times x = \int_{\Kp \setminus \{0\}} g(x) \frac{dx}{\abs[x]_p}$ auf Funktionen in $L^1$ zu erweitern.
	\end{proof}
\subsection{Lokale Fourieranalysis}
%Pontryagin-Duale Gruppe
%selbstdual
%fouriertransformation: done
%schwartz-bruhat und deren form: done
%fourierumkehrformel: done
		F"ur die unendliche Stelle $p=\infty$ definieren wir die \emph{Schwartz-Bruhat Funktion} als eine komplexwertige, glatte Funktion $f$, die f"ur alle nicht-negativen ganzen Zahlen $n$ und $m$ die Bedingung
		\begin{align*}
			\sup_{x\in \K_\infty}\abs[x^n\frac{d^m}{dx^m}f(x)] < \infty
		\end{align*}
		erf"ullt.\footnote{Hier ist mit $\abs$ der komplexe Absolutbetrag gemeint.}
		F"ur die endlichen Stellen $p<\infty$ definieren wir eine \emph{Schwartz-Bruhat Funktion} als eine lokal konstante Funktion mit kompakten Tr?ger.
		Die Menge aller solcher Funktionen bilden einen komplexen Vektorraum, den wir mit $\Sw(\K_p)$ bezeichnen. 
		Im Fall $p<\infty$ erkennt man leicht, dass $\Sw(\Kp)\subseteq L^1(\Kp)$. 
		F"ur $p=\infty$ gilt nach obiger Bedingung $(\abs[1]+\abs[x^2])\abs[f(x)] \leq C$, also $\abs[f(x)]\leq C(1+x^2)^{-1}$ und $(1+x^2)^{-1} \in L^1(\K\infty)$
		
		\begin{bsp}~ 
			\begin{enumerate}[label=(\roman*)]
				\item Im Fall $p=\infty$ ist die Funktion $f_k = x^k e^{-x^2}$ f"ur jedes $k\in\N_0$ in $\Sw(\K_\infty)$. 
				Die Ableitungen $\frac{d^m}{dx^m} f_k(x)$ sind von der Form $p(x)e^{-x^2}$, wobei $p(x)$ ein Polynom ist. 
				Aus der Analysis ist dann bekannt, dass $\abs[x^n p(x)e^{-x^2}]$ f"ur jedes $n\in \N_0$ beschr?nkt ist.
				\item Im Fall $p<\infty$ sind offensichtlich die charakteristischen Funktionen kompakter Mengen in $\Sw(\Kp)$. 
				Beispiele f"ur Kompakta sind Mengen der Form $a+p^k\Zp$ mit $a\in \K$ und $k\in \Z$.
			\end{enumerate}
		\end{bsp}
		
		\begin{lemma}\label{lemma:padischSBF}
			Jede Funktion $f\in \Sw(\Kp)$, $p<\infty$, ist eine endliche Linearkombination von charakteristischen Funktionen der Form $\ind_{a+p^k\Zp}$, wobei $a\in \K$ und $k\in \Z$
		\end{lemma}
		\begin{proof}
			Sei $f \in \Sw(\Kp)$. 
			Da $f$ lokal konstant ist, ist f"ur jedes $z\in\C$ das Urbild $f^{-1}(z)$ offen in $\Kp$. 
			Also ist $f^{-1}(0)$ offen, folglich $\Kp \setminus f^{-1}(0)$ abgeschlossen und daher schon $\text{supp}(f) = \Kp \setminus f^{-1}(0)$. 
			Per Definition hat die Schwartz-Bruhat Funktion $f$ kompakten Tr"ager, also ist $\Kp \setminus f^{-1}(0)$ kompakt. 
			Diese Menge wird von den offenen Mengen $f^{-1} (x)$ mit $x\not= 0$ "uberdeckt, wovon nach Kompaktheit schon endlich viele reichen.
			$f$ hat somit endliches Bild. Weiter ist jede offene Menge $f^{-1} (x)$ eine disjunkte Vereinigung offener B"allen in $\Kp$. 
			Diese haben aber genau die gesuchte Form $a+p^k\Zp$ wie oben. 
			Aufgrund der Kompaktheit, reichen wieder endliche viele solcher B"alle. Damit folgt auch schon das Lemma.
		\end{proof}
		\begin{lemma}
			Sei $f \in \Sw(\Kp)$.
			\begin{enumerate}[label=\emph{(\alph*)}]
				\item Ist $g(x)=f(x)e_p(ax)$ mit $a\in\Kp$, dann gilt $\hat{g}(x) = \hat{f}(x-a)$.
				\item Ist $g(x)=f(x-a)$ mit $a\in\Kp$, dann gilt $\hat{g}(x) = \hat{f}(x-a)e_p(-ax)$.
				\item Ist $g(x)=f(\lambda x)$ mit $\lambda \in\Kp^\times$, dann gilt $\hat{g}(x) =\frac{1}{\abs[\lambda]_p} \hat{f}(\frac{x}{\lambda})$.
			\end{enumerate}
		\end{lemma}
		\begin{proof}
			(a) und (b) sind einfache Folgerungen aus der Definition mit der Multiplikativit"at von $e_p$ und der Translationsinvarianz des Haar-Ma?. 
			Bei (c) spielt unsere Normeriung des Absolutbetrags eine Rolle, denn mit dem Variablenwechsel $y\mapsto \lambda^{-1}y$ erhalten wir
			\begin{align*}
				\hat{g}(x) = \int_{\Kp} f(\lambda y) e_p(-xy)dy = \frac{1}{\abs[\lambda]_p} \int_{\Kp} f(y) e_p(-x\lambda^{-1}y)dy = \frac{1}{\abs[\lambda]_p} \hat{f}\left(\frac{x}{\lambda}\right)
			\end{align*}
		\end{proof}
		\begin{satz}\label{Satz:fourierumkehrformel}
			Ist $p\leq\infty$ und $f\in\Sw(\Kp)$, so ist $\hat{f} \in \Sw(\Kp)$ und es gilt die Umkehrformel
			\begin{align*}
				\hat{\hat{f}}(x) = f(-x)
			\end{align*}
		\end{satz}
		\begin{proof}
			Betrachten wir zuerst den Fall $p<\infty$. Wie wir eben in Lemma \ref{lemma:padischSBF} gesehen haben haben, ist jede Funktion in $\Sw(\Kp)$ eine Linearkombination von Funktionen der Form $f = \ind_{a+p^k\Zp}$. Es reicht also die Aussage f"ur solche $f$ zu zeigen.
			Sei dazu $h:= \ind_{\Zp}$. Wir zeigen $\hat{h} = h$ durch folgende Rechnung
			\begin{align*}
				\hat{h}(x) = \int_\Kp h(y) e_p (-xy) dy_p = \int_\Zp e_p(-xy) dy_p.
			\end{align*}
			Nun ist $\chi(y):=e_p(-xy)$ ein Charakter auf $\Zp$ und genau dann trivial, wenn $x\in\Zp$. 
			Weiter ist $\Zp$ kompakt. 
			Nach Lemma \ref{Lemma:trivialerCharAufKompakt} und unserer Normierung von $dy_p$ folgt also
			\begin{align*}
				\hat(h)(x) = \text{Vol}(\Zp, dy_p) \ind_\Zp = \ind_\Zp = h(x)
			\end{align*}
			
			Wir f"uhren nun folgende Operatoren auf $\Sw(\Kp)$ ein
			\begin{align*}
				L_a f(x) = f(x-a), M_\lambda f(x) = f(\lambda x),
			\end{align*}
			wobei $a \in \Kp$ und $\lambda \in \Kp^\times$. 
			Nun k"onnen wir $f$ schreiben als $L_a M_{p^{-k}}h$. 
			Es folgt
			\begin{align*}
				\hat{f} = (L_a M_{p^{-k}}h)\widehat{\phantom{x}} = \Omega_{-a}p^{k}M_{p^k}\hat{h}=\Omega_{-a}p^{-k}M_{p^k}h.
			\end{align*}
			Also ist $\hat{f} (x) = e_p(-ax)p^k\ind_{p^{-k}\Zp}(x)$. 
			Der Charakter $e_p$ ist lokal konstant, $\hat{f}$ als das Produkt lokal konstanter Funktionen selbst wieder lokal konstant und damit in $\Sw(\Kp)$. 
			Damit haben wir den ersten Teil der Aussage gezeigt.
			
			F"ur den zweiten Teil sehen wir
			\begin{align*}
				\hat{\hat{f}} = (L_a M_{p^{-k}}h)\widehat{\widehat{\phantom{x}}} = L_{-a} (M_{p^k}h)\widehat{\widehat{\phantom{x}}}=L_{-a}M_{p^k}\hat{h} =L_{-a}M_{p^k}h,
			\end{align*}
			also $\hat{\hat{f}} (x) = \ind_{-a+p^k\Zp} (x) = \ind_{a+p^k\Zp} (-x) = f(-x)$. 
			Hier haben wir $p^k\Zp = - p^k\Zp$ ausgenutzt. 
			Damit haben wir die Umkehrformel f"ur den $p$-adischen Fall gezeigt. 
			F"ur $p=\infty$ ist die Formel bereits aus der klassischen Fourieranalysis bekannt.
		\end{proof}
	
