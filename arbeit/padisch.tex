\section{Exkurs: p-adische Zahlen}
\subsection{Absolutbetr"age}
	%padischen absolutbetraege:done
	Sei $\mathbb{K}$ ein beliebiger K"orper und $\R_+ = \{x\in \R: x\geq 0\}$ die Menge der nicht-negativen reellen Zahlen.
	\begin{defi}
		Ein \emph{Absolutbetrag} auf $\mathbb{K}$ ist eine Abbildung
		\begin{align*}
			\abs:\K \longrightarrow \R_+
		\end{align*}
		welche die folgenden Bedingungen erf"ullt:
		\begin{enumerate}[label=(\roman*),leftmargin=1.5cm]
			\item $\abs[x] = 0 \Leftrightarrow x=0$ (Definitheit)
			\item $\abs[xy] = \abs[x]\abs[y]$ f"ur alle $x, y \in \mathbb{K}$ (Multiplikativit"at)
			\item $\abs[x+y] \leq \abs[x] + \abs[y]$ f"ur alle $x,y \in \mathbb{K}$ (Dreiecksungleichung)
		\end{enumerate}
		Wir nennen den Absolutbetrag $\abs$ nicht-archimedisch, wenn er zus"atzlich die st"arkere Bedingung
		\begin{enumerate}[label=(\roman*)$'$,leftmargin=1.5cm]
			\setcounter{enumi}{2}
			\item $\abs[x+y] \leq \max\{\abs[x],\abs[y]\}$ f"ur alle $x, y \in \mathbb{K}$ (versch"arfte Dreiecksungleichung)
		\end{enumerate}
		erf"ullt. Anderenfalls sagen wir der Absolutbetrag ist archimedisch.
	\end{defi}
	Wir m"ochten zun"achst ein paar allgeimg"ultige Eigenschaften von Absolutbetr"agen im folgenden Lemma festhalten.
	\begin{lemma}
		F"ur beliebige Absolutbetr"age $\abs$ auf $\mathbb{K}$ und Elemente $x\in \K$ gilt:
		\begin{enumerate}[label=(\roman*),leftmargin=1.5cm]
			\item $\abs[1] = 1$
			\item $\abs[-1] = 1$
			\item Falls $\abs[x^n]=1$, dann $\abs[x]=1$
			\item $\abs[-x]$ = $\abs[x]$ 
		\end{enumerate}
	\end{lemma}
	
	Betrachten wir nun den K"orper $\K = \Q$ der rationalen Zahlen. Sei $x\in \Q^\times$ eine beliebige rationale Zahl. Dann existiert eine eindeutige (bis auf Reihenfolge) Primfaktorzerlegung
	\begin{align*}
		x = \prod_{p} p^{v_p},
	\end{align*}
	wobei das Produkt "uber alle Primzahlen $p \in \N$ geht und $v_p \in \Z$ f"ur fast alle $p$ gleich $0$ ist. Legen wir uns auf ein $p$ fest, so erm"oglicht sich die 
	\begin{defi}
		F"ur beliebige $x \in \Q$ sei der \emph{p-adische Absolutbetrag} von $x$ gegeben durch
		\begin{align*}
			\abs[x]_p = p^{-v_p}
		\end{align*}
		f"ur $x\neq 0$ und $v_p \in \Z$ wie oben. Durch $\abs[0]_p := 0$ vervollst"andigen wir die Definition.
	\end{defi}
	\begin{lemma}
		$\abs_p$ ist ein nicht-archimedischer Absolutbetrag auf $\Q$
	\end{lemma}
	\begin{proof}
		Die Definitheit folgt sofort aus der Definition. 
		F"ur die Multiplikativit"at schreiben wir $x=p^k \frac{m}{n}$ und $y=p^{k'} \frac{m'}{n'}$ mit $m,m',n,n'$ teilerfremd zu $p$.
		Dann ist
		\begin{align*}
			\abs[xy]_p = \abs[p^{k+k'}\frac{mm'}{nn'}]_p = p^{-(k+k')} = p^{-k} p^{-k'} = \abs[x]_p \abs[y]_p.
		\end{align*}
		Zuletzt zur versch"arften Dreiecksungleichung. 
		Mit $x$ und $y$ wie eben k"onnen wir ohne Einschr"ankung annehmen, dass $k\leq k'$. 
		Es gilt
		\begin{align*}
			x+y = p^k\frac{mn' + p^{k'-k}nm'}{nn'}.
		\end{align*}
		F"ur $\abs[x]_p \not=\abs[y]_p$ ist $k'>k$, also $mn' + p^{k'-k}nm'$ teilerfremd zu $p$ und es folgt $\abs[x+y]_p = p^{-k} = \max(\abs[x]_p,\abs[y]_p)$. 
		Ist dagegen $\abs[x]_p \not=\abs[y]_p$, so kann $p$ ein Teiler des Z"ahlers sein und wir erhalten $\abs[x+y]_p \leq p^{-k} = \max(\abs[x]_p,\abs[y]_p)$
	\end{proof}
	
	\begin{satz}[Ostrowski]
		Jeder nicht-triviale Absolutbetrag auf $\Q$ ist "aquivalent zu einem der Absolutbetr"age $\abs_p$, wobei $p$ entweder eine Primzahl ist oder $p=\infty$.
	\end{satz}
	\begin{proof}
		Sei $\abs$ ein beliebiger nicht-trivialer Absolutbetrag auf $\Q$. Wir untersuchen die zwei m"oglichen F"alle.
		\begin{enumerate}[align=left, leftmargin=0cm, labelsep=0cm, label=\alph*)\ ]
		\item $\abs$ ist archimedisch.
			Sei dann $n_0\in \N$ die kleinste nat"urliche Zahl mit $\abs[n_0] > 1$.
			Dann gibt es ein $\alpha \in \R^+$ mit $\abs[n_0]^{\alpha}=n_0$.
			Wir wollen nun zeigen, dass $\abs[n]=\abs[n]_\infty^\alpha$ f"ur alle $n \in \N$ gilt. Der allgemeine Fall f"ur $\Q$ folgt dann aus den Eigenschaften des Betrags.
			Dazu bedienen wir uns eines kleinen Tricks: F"ur $n \in \N$ nehmen wir die Darstellung zur Basis $n_0$, d.h.
			\begin{align*}
				n = \sum_{i=0}^{k} a_i n_0^i
			\end{align*}
			mit $a_i \in \{0,1,\dots,n_0-1\}$, $a_k \neq 0$ und $n_0^k\leq n < n_0^{k+1}$. Nehmen wir davon den Absolutbetrag und beachten, dass $\abs[a_i]\leq 1$ nach unserer Wahl von $n_0$ gilt, so erhalten wir
			\begin{align*}
				\abs[n] \leq \sum_{i=0}^{k} \abs[a_i] n_0^{i\alpha}
					\leq \sum_{i=0}^{k} n_0^{i\alpha}
					\leq n_0^{k\alpha}\sum_{i=0}^{k} n_0^{-i\alpha}
					\leq n_0^{k\alpha}\sum_{i=0}^{\infty} n_0^{-i\alpha}
					 = n_0^{k\alpha} \frac{n_0^\alpha}{n_0^\alpha - 1}.
			\end{align*}
			Setzt man nun $C:=\frac{n_0^\alpha}{n_0^\alpha - 1}>0$, so sehen wir
			\begin{align*}
				\abs[n]\leq C n_0^{k\alpha}\leq C n^\alpha
			\end{align*}
			f"ur beliebige $n \in \N$, also insbesondere auch
			\begin{align*}
				|n^N|\leq C n^{N\alpha}.
			\end{align*}
			Ziehen wir nun auf beiden Seiten die $N$-te Wurzel und lassen $N$ gegen $\infty$ laufen, so konvergiert $\sqrt[N]{C}$ gegen $0$ und wir erhalten 
			\begin{align*}
				\abs[n] \leq n^\alpha
			\end{align*}
			Damit w"are die erste H"alfte geschafft. Gehen wir nun z"uruck zu unserer Basisdarstellung
			\begin{align*}
				n = \sum_{i=0}^{k} a_i n_0^i.
			\end{align*}
			Da $n < n_0^{k+1}$ erhalten wir die Absch"atzung
			\begin{align*}
				n_0^{(k+1)\alpha}=|n_0^{k+1}| = |n + n_0^{k+1} - n| \leq \abs[n] + |n_0^{k+1} -n|.
			\end{align*}
			 mit dem Ergebnis aus der ersten H"alfte des Beweises und $n\geq n_0^k$ sehen wir
			\begin{align*}
				\abs[n] &\geq n_0^{(k+1)\alpha} - |n_0^{k+1} -n| 
					\geq n_0^{(k+1)\alpha} - (n_0^{k+1} -n)^\alpha
					\\&\geq n_0^{(k+1)\alpha} - (n_0^{k+1} -n_0^k)^\alpha
					=n_0^{(k+1)\alpha} \left(1 - \left(1 - \frac{1}{n_0}\right)\right)
					\\&> n^\alpha \left(1 - \left(1 - \frac{1}{n_0}\right)\right).
			\end{align*}
			Setzen wir wieder $C':=\left(1 - \left(1 - \frac{1}{n_0}\right)\right) >0$ folgt analog zum ersten Teil, dass
			\begin{align*} 
				\abs[n]\geq n^\alpha
			\end{align*}
			und daher $\abs[n]=n^\alpha$. Damit haben wir gezeigt, dass $\abs$ "aquivalent zum klassischen Absolutbetrag $\abs_\infty$ ist.
		\item $\abs$ ist nicht archimedisch.
			Dann ist $\abs[n_0]\leq 1$ f"ur alle $n \in \N$ und, da $\abs$ nicht-trivial ist, muss es eine kleinste Zahl $n_0$ geben mit $\abs[n_0]<1$. Insbesondere muss $n_0$ eine Primzahl sein, denn sei $p \in \N$ ein Primteiler von $n_0$, also $n_0=p \cdot n'$ mit $n' \in \N$ und $n' < n$, dann gilt nach unserer Wahl von $n_0$
			\begin{align*}
				|p| = |p|\cdot |n'| =|p \cdot n'| = \abs[n_0] < 1.
			\end{align*}
			Folglich muss schon $p=n_0$ gelten. Ziel wird es jetzt nat"urlich sein zu zeigen, dass $\abs$ "aquivalent zum $p$-adischen Absolutbetrag ist.
			Zun"achst finden wir ein $\alpha \in \R^+$ mit $|p| = |p|_p^{\alpha} = \frac{1}{p^{\alpha}}$ . Sei als n"achstes $n\in \Z$ mit $p \not | n$. Wir schreiben
			\begin{align*}
				n= rp + s, r \in \Z, 0<s<p
			\end{align*}
			Nach unserer Wahl von $p=n_0$ gilt $|s|=1$ und $|rp|<1$. Es folgt $\abs[n]=\text{max}\{|rp|,|s|\}=1$. Sei nun $n\in \Z$ beliebig. Wir schreiben $n=p^{v}n'$ mit $p\not | n'$ und sehen
			\begin{align*}
				\abs[n] = |p|^{v}|n'| = |p|^v = (|p|_p^{\alpha})^{v}=\abs[n]_p^{\alpha}.
			\end{align*}
			Mit den gleichen "Uberlegungen aus dem ersten Fall folgt damit die Behauptung.
		\end{enumerate}
	\end{proof}

\subsection{Vervollst"andigungen von $\K$}
	%offene Mengen in Q_p
%konstruktion erwaehnen
%eindeutigkeit
%Z_p
	

\subsection{Potenzreihen}
%Zeige potenzreihen konvergieren in Q_p: done
%Zeige jede zahl in Q_p darstellbar als Potenzreihe: done
	Wie auch schon im Fall der Vervollst"andigung von $\Q$ bez"uglich $\abs_\infty$ ist diese Konstruktion etwas schwierig zu handhaben.
	In $\R$ hatten wir die Dezimalbruchentwicklung  $x=\pm\sum_{k=-\infty}^{n}x_k 10^{k}$, $x_k\in\{0,\dots,9\}$ oder etwas allgemeiner die $b$-adische Darstellung $x =\pm \sum_{k=-\infty}^{n}x_k b^{k}$, $x_k\in\{0,\dots,b-1\}$ f"ur eine nat"urliche Zahl $b>1$, die zwar nicht unbedingt eindeutig ist ($0,999\dots = 1,000\dots$) aber leichter zu handhaben als Äquivalenzklassen von Cauchy-Folgen.
	Zu unserer Freude werden wir feststellen, dass auch in $\Q_p$ eine "ahnliche Darstellung gibt. Diese ist im Gegensatz zu der in $\R$ eindeutig und besitzt nur endlich viele Nachkommastellen, allerdings nicht unbedingt endlich viele Vorkommastellen.
	\begin{satz}
		Jede Element $x\in\Kp$ l"asst sich eindeutig als eine Potenzreihe 
		\begin{align*}
			x= \sum_{k=N}^{\infty} x_k p^k
		\end{align*}
		mit $x_k\in\{0,\dots,p-1\}$, $x_n\not= 0$ und $n\in\Z$ darstellen. 
		Insbesondere gilt $\abs[x]_p = p^{-n}$.
	\end{satz}
	\begin{proof}
		Wir zeigen zun"achst das die Folge der Partialsummen $S_n=\sum_{k=N}^{n} x_k p^k$ in $\Kp$ konvergiert.
		Sei dazu $\epsilon>0$. Wir finden ein $n_0 \in \Z$ mit $p^{-n_0} < \epsilon$. Dann gilt f"ur alle $m\geq n\geq n_0$
		\begin{align*}
			\abs[S_m - S_n]_p 
			= \abs[\sum_{k=n}^{m} x_k p^k]_p \leq \max_{n\leq k\leq m}\left(\abs[x_k p^k]_p\right) 
			= \max_{n\leq k\leq m}\left(p^{-k}\right) 
			= p^{-n} \leq p^{-n_0} <\epsilon
		\end{align*}
		und wir haben gezeigt, dass die Partialsummen eine Cauchy-Folge bilden.
		Da $\Kp$ vollst"andig ist konvergiert diese gegen einen Wert f"ur den wir $\sum_{k=N}^{\infty} x_k p^k$ schreiben.
		
		Als n"achstes sei $x\in\Zp$.
		Wir konstruieren nun eine Folge $(x_k)$ mit $x_k\in\{0,\dots,p-1\}$, so dass die Partialsummen $S_n = \sum_{k=0}^{n} x_k p^k$ gegen $x$ konvergieren.
		Da $\Q$ dicht in $\Kp$ liegt finden wir f"ur jedes $n\in\N_0$ einen vollst"andig gek"urzten Bruch $\frac{a}{b}\in\Q$ der 
		\begin{align*}
			\abs[x-\frac{a}{b}]_p \leq p^{-n} < 1
		\end{align*}
		In der Tat finden wir sogar eine ganze Zahl. 
		Denn f"ur $\frac{a}{b}$ wie oben haben wir
		\begin{align*}
			\abs[\frac{a}{b}]_p \leq \max\left\{\abs[x]_p, \abs[x-\frac{a}{b}]_p\right\} \leq 1,
		\end{align*}
		was zeigt, dass $p$ kein Teiler von $b$ ist. 
		Wir finden daher eine Bezout-Darstellung $1=b'b + p'p^{n}$, also $b'b \equiv 1 \pmod{p^n}$ f"ur  $b', p' \in \Z$.
		Es folgt
		\begin{align*}
			\abs[\frac{a}{b} - ab']_p 
			= \abs[\frac{a(1-bb')}{b}]_p 
			=\abs[\frac{a(p'p^n)}{b}]_p \leq p^{-n}
		\end{align*}
		und nat"urlich $ab' \in \Z$. Wir definieren nun $y_n\in\Z$ als die eindeutige Zahl, so dass
		\begin{align*}
			0\leq y_n \leq p^n-1  \text{ und }  y_n \equiv ab' \pmod{p^n}.
		\end{align*}
		Wir folgern analog zu eben
		\begin{align*}
			\abs[x-y_n]_p 
			= \abs[x-\frac{a}{b} + \frac{a}{b} - ab']_p 
			\leq \max\left\{\abs[x-\frac{a}{b}]_p, \abs[\frac{a}{b} - y_n]_p\right\}
			= p^{-n}
		\end{align*}
		und folglich konvergiert die Folge $(y_n)$ gegen $x$.
		Wir zeigen, dass diese Folge eindeutig ist. 
		Sei dazu $y_n'$ eine weiter Zahl mit $0\leq y_n' \leq p^n-1$ und $\abs[x-y_n']_p\leq p^{-n}$.
		Dann gilt aber wie eben
		\begin{align*}
			\abs[y_n - y_n']_p = \abs[y_n - x + x - y_n']_p \leq p^{-n},
		\end{align*}
		also ist $p^n$ ein Teiler von $y_n - y_n'$ und folglich $y_n -y_n' = 0$.
		Aus dieser Eindeutigkeit folgt unter anderem auch, dass $y_n \equiv y_{n-1} \pmod{p^{n-1}}$, denn
		\begin{align*}
			\abs[y_n - y_{n-1}]_p = \abs[y_n - x + x - y_{n-1}]_p \leq \max\left\{\abs[y_n-x]_p, \abs[x- y_{n-1}]_p\right\} = p^{-n+1},
		\end{align*}
		und folglich $p^{n-1}$ ein Teiler von $y_n - y_{n-1}$.
		Die eindeutigen $p$-adischen Entwicklungen dieser $y_n$ sind genau die gesuchten Partialsummen $S_n$ und definieren die $x_k$ daher induktiv.
		
		F"ur jedes beliebige $x\in \Q_p$ finden wir ein $n\in\N_0$, so dass $p^{n}x \in \Zp$ und damit eine Entwicklung $p^{n}x = \sum_{k=0}^{\infty} x_kp^k$. Also haben wir 
		\begin{align*}
			x = \sum_{k=-n}^{\infty} x_{k+n}p^k
		\end{align*}
	\end{proof}