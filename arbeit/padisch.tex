\section{Exkurs: p-adische Zahlen}
\subsection{Absolutbeträge}
	%padischen absolutbetraege:done
	Sei $\mathbb{K}$ ein beliebiger K"orper und $\R_+ = \{x\in \R: x\geq 0\}$ die Menge der nicht-negativen reellen Zahlen.
	\begin{defi}
		Ein \emph{Absolutbetrag} auf $\mathbb{K}$ ist eine Abbildung
		\begin{align*}
			\abs:\K \longrightarrow \R_+
		\end{align*}
		welche die folgenden Bedingungen erfüllt:
		\begin{enumerate}[label=(\roman*),leftmargin=1.5cm]
			\item $\abs[x] = 0 \Leftrightarrow x=0$ (Definitheit)
			\item $\abs[xy] = \abs[x]\abs[y]$ für alle $x, y \in \mathbb{K}$ (Multiplikativität)
			\item $\abs[x+y] \leq \abs[x] + \abs[y]$ für alle $x,y \in \mathbb{K}$ (Dreiecksungleichung)
		\end{enumerate}
		Wir nennen den Absolutbetrag $\abs$ nicht-archimedisch, wenn er zusätzlich die stärkere Bedingung
		\begin{enumerate}[label=(\roman*)$'$,leftmargin=1.5cm]
			\setcounter{enumi}{2}
			\item $\abs[x+y] \leq \max\{\abs[x],\abs[y]\}$ für alle $x, y \in \mathbb{K}$ (verschärfte Dreiecksungleichung)
		\end{enumerate}
		erfüllt. Anderenfalls sagen wir der Absolutbetrag ist archimedisch.
	\end{defi}
	Wir m"ochten zunächst ein paar allgeimgültige Eigenschaften von Absolutbeträgen im folgenden Lemma festhalten.
	\begin{lemma}
		Für beliebige Absolutbeträge $\abs$ auf $\mathbb{K}$ und Elemente $x\in \K$ gilt:
		\begin{enumerate}[label=(\roman*),leftmargin=1.5cm]
			\item $\abs[1] = 1$
			\item $\abs[-1] = 1$
			\item Falls $\abs[x^n]=1$, dann $\abs[x]=1$
			\item $\abs[-x]$ = $\abs[x]$ 
		\end{enumerate}
	\end{lemma}
	
	Betrachten wir nun den K"orper $\K = \Q$ der rationalen Zahlen. Sei $x\in \Q^\times$ eine beliebige rationale Zahl. Dann existiert eine eindeutige (bis auf Reihenfolge) Primfaktorzerlegung
	\begin{align*}
		x = \prod_{p} p^{v_p},
	\end{align*}
	wobei das Produkt über alle Primzahlen $p \in \N$ geht und $v_p \in \Z$ für fast alle $p$ gleich $0$ ist. Legen wir uns auf ein $p$ fest, so erm"oglicht sich die 
	\begin{defi}
		Für beliebige $x \in \Q$ sei der \emph{p-adische Absolutbetrag} von $x$ gegeben durch
		\begin{align*}
			\abs[x]_p = p^{-v_p}
		\end{align*}
		für $x\neq 0$ und $v_p \in \Z$ wie oben. Durch $\abs[0]_p := 0$ vervollständigen wir die Definition.
	\end{defi}
	\begin{lemma}
		$\abs_p$ ist ein nicht-archimedischer Absolutbetrag auf $\Q$
	\end{lemma}
	\begin{proof}
		Die Definitheit folgt sofort aus der Definition. 
		Für die Multiplikativität schreiben wir $x=p^k \frac{m}{n}$ und $y=p^{k'} \frac{m'}{n'}$ mit $m,m',n,n'$ teilerfremd zu $p$.
		Dann ist
		\begin{align*}
			\abs[xy]_p = \abs[p^{k+k'}\frac{mm'}{nn'}]_p = p^{-(k+k')} = p^{-k} p^{-k'} = \abs[x]_p \abs[y]_p.
		\end{align*}
		Zuletzt zur verschärften Dreiecksungleichung. 
		Mit $x$ und $y$ wie eben k"onnen wir ohne Einschränkung annehmen, dass $k\leq k'$. 
		Es gilt
		\begin{align*}
			x+y = p^k\frac{mn' + p^{k'-k}nm'}{nn'}.
		\end{align*}
		Für $\abs[x]_p \not=\abs[y]_p$ ist $k'>k$, also $mn' + p^{k'-k}nm'$ teilerfremd zu $p$ und es folgt $\abs[x+y]_p = p^{-k} = \max(\abs[x]_p,\abs[y]_p)$. 
		Ist dagegen $\abs[x]_p \not=\abs[y]_p$, so kann $p$ ein Teiler des Zählers sein und wir erhalten $\abs[x+y]_p \leq p^{-k} = \max(\abs[x]_p,\abs[y]_p)$
	\end{proof}
	
	\begin{satz}[Ostrowski]
		Jeder nicht-triviale Absolutbetrag auf $\Q$ ist äquivalent zu einem der Absolutbeträge $\abs_p$, wobei $p$ entweder eine Primzahl ist oder $p=\infty$.
	\end{satz}
	\begin{proof}
		Sei $\abs$ ein beliebiger nicht-trivialer Absolutbetrag auf $\Q$. Wir untersuchen die zwei m"oglichen Fälle.
		\begin{enumerate}[align=left, leftmargin=0cm, labelsep=0cm, label=\alph*)\ ]
		\item $\abs$ ist archimedisch.
			Sei dann $n_0\in \N$ die kleinste natürliche Zahl mit $\abs[n_0] > 1$.
			Dann gibt es ein $\alpha \in \R^+$ mit $\abs[n_0]^{\alpha}=n_0$.
			Wir wollen nun zeigen, dass $\abs[n]=\abs[n]_\infty^\alpha$ für alle $n \in \N$ gilt. Der allgemeine Fall für $\Q$ folgt dann aus den Eigenschaften des Betrags.
			Dazu bedienen wir uns eines kleinen Tricks: Für $n \in \N$ nehmen wir die Darstellung zur Basis $n_0$, d.h.
			\begin{align*}
				n = \sum_{i=0}^{k} a_i n_0^i
			\end{align*}
			mit $a_i \in \{0,1,\dots,n_0-1\}$, $a_k \neq 0$ und $n_0^k\leq n < n_0^{k+1}$. Nehmen wir davon den Absolutbetrag und beachten, dass $\abs[a_i]\leq 1$ nach unserer Wahl von $n_0$ gilt, so erhalten wir
			\begin{align*}
				\abs[n] \leq \sum_{i=0}^{k} \abs[a_i] n_0^{i\alpha}
					\leq \sum_{i=0}^{k} n_0^{i\alpha}
					\leq n_0^{k\alpha}\sum_{i=0}^{k} n_0^{-i\alpha}
					\leq n_0^{k\alpha}\sum_{i=0}^{\infty} n_0^{-i\alpha}
					 = n_0^{k\alpha} \frac{n_0^\alpha}{n_0^\alpha - 1}.
			\end{align*}
			Setzt man nun $C:=\frac{n_0^\alpha}{n_0^\alpha - 1}>0$, so sehen wir
			\begin{align*}
				\abs[n]\leq C n_0^{k\alpha}\leq C n^\alpha
			\end{align*}
			für beliebige $n \in \N$, also insbesondere auch
			\begin{align*}
				|n^N|\leq C n^{N\alpha}.
			\end{align*}
			Ziehen wir nun auf beiden Seiten die $N$-te Wurzel und lassen $N$ gegen $\infty$ laufen, so konvergiert $\sqrt[N]{C}$ gegen $0$ und wir erhalten 
			\begin{align*}
				\abs[n] \leq n^\alpha
			\end{align*}
			Damit wäre die erste Hälfte geschafft. Gehen wir nun züruck zu unserer Basisdarstellung
			\begin{align*}
				n = \sum_{i=0}^{k} a_i n_0^i.
			\end{align*}
			Da $n < n_0^{k+1}$ erhalten wir die Abschätzung
			\begin{align*}
				n_0^{(k+1)\alpha}=|n_0^{k+1}| = |n + n_0^{k+1} - n| \leq \abs[n] + |n_0^{k+1} -n|.
			\end{align*}
			 mit dem Ergebnis aus der ersten Hälfte des Beweises und $n\geq n_0^k$ sehen wir
			\begin{align*}
				\abs[n] &\geq n_0^{(k+1)\alpha} - |n_0^{k+1} -n| 
					\geq n_0^{(k+1)\alpha} - (n_0^{k+1} -n)^\alpha
					\\&\geq n_0^{(k+1)\alpha} - (n_0^{k+1} -n_0^k)^\alpha
					=n_0^{(k+1)\alpha} \left(1 - \left(1 - \frac{1}{n_0}\right)\right)
					\\&> n^\alpha \left(1 - \left(1 - \frac{1}{n_0}\right)\right).
			\end{align*}
			Setzen wir wieder $C':=\left(1 - \left(1 - \frac{1}{n_0}\right)\right) >0$ folgt analog zum ersten Teil, dass
			\begin{align*} 
				\abs[n]\geq n^\alpha
			\end{align*}
			und daher $\abs[n]=n^\alpha$. Damit haben wir gezeigt, dass $\abs$ äquivalent zum klassischen Absolutbetrag $\abs_\infty$ ist.
		\item $\abs$ ist nicht archimedisch.
			Dann ist $\abs[n_0]\leq 1$ für alle $n \in \N$ und, da $\abs$ nicht-trivial ist, muss es eine kleinste Zahl $n_0$ geben mit $\abs[n_0]<1$. Insbesondere muss $n_0$ eine Primzahl sein, denn sei $p \in \N$ ein Primteiler von $n_0$, also $n_0=p \cdot n'$ mit $n' \in \N$ und $n' < n$, dann gilt nach unserer Wahl von $n_0$
			\begin{align*}
				|p| = |p|\cdot |n'| =|p \cdot n'| = \abs[n_0] < 1.
			\end{align*}
			Folglich muss schon $p=n_0$ gelten. Ziel wird es jetzt natürlich sein zu zeigen, dass $\abs$ äquivalent zum $p$-adischen Absolutbetrag ist.
			Zunächst finden wir ein $\alpha \in \R^+$ mit $|p| = |p|_p^{\alpha} = \frac{1}{p^{\alpha}}$. 
			Sei als nächstes $n\in \Z$ mit $p \centernot\mid n$. Wir schreiben
			\begin{align*}
				n= rp + s, r \in \Z, 0<s<p
			\end{align*}
			Nach unserer Wahl von $p=n_0$ gilt $|s|=1$ und $|rp|<1$. 
			Es folgt 
			\begin{align*}
				\abs[n]=\text{max}\{|rp|,|s|\}=1.
			\end{align*}
			Sei nun $n\in \Z$ beliebig. 
			Wir schreiben $n=p^{v}n'$ mit $p \centernot\mid n'$ und sehen
			\begin{align*}
				\abs[n] = |p|^{v}|n'| = |p|^v = (|p|_p^{\alpha})^{v}=\abs[n]_p^{\alpha}.
			\end{align*}
			Mit den gleichen Überlegungen aus dem ersten Fall folgt damit die Behauptung.
		\end{enumerate}
	\end{proof}

\subsection{Vervollständigungen von \texorpdfstring{$\K$}{Q}}
	Eine M"oglichkeit die reellen Zahlen $\R$ zu definieren war über die \emph{Vervollständigung} der rationalen Zahlen $\K$ bezüglich des Absolutbetrags $\abs_\infty$.
	Wir werden diese Konstruktion für unsere $p$-adischen Beträge $\abs_p$ benutzen und geben daher im folgenden Abschnitt eine kurze Wiederholung der wichtigsten Konzepte.
	
	Ein \emph{metrischer Raum} ist ein Paar $(X, d)$ bestehen aus einer nichtleeren Menge $X$ und einer Abbildung $d: X\times X\to\R$, die
	\begin{itemize}
		\item $d(x,y) = 0 \Leftrightarrow x = y$
		\item $d(x,y) = d(y,x)$
		\item $d(x,y) \leq d(x,z) + d(z,y)$
	\end{itemize}
	erfüllt.
	Eine Folge $(x_n)$ in $X$ heißt \emph{konvergent} gegen $x \in X$, wenn die reelle Folge $(d(x_n,x))$ eine Nullfolge ist.
	Eine \emph{Cauchy-Folge} in dem metrischen Raum $X$ ist eine Folge $(x_n)$ in $X$, so dass für alle $\epsilon >0$ ein $n_0\in \N$ gibt, so dass $d(x_n,x_m) < \epsilon$ für alle $n,m\geq n_0$ gilt. 
	Im nicht-archimedischen Fall vereinfacht sich diese Definition etwas.
	\begin{lemma}
		Eine Folge $(x_n)$ rationaler Zahlen ist genau dann bezüglich eines nicht-archimedischen Absolutbetrags $\abs$ eine Cauchy-Folge, wenn
		\begin{align*}
			\lim_{n\to\infty}\abs[x_{n+1} - x_n] = 0
		\end{align*}
	\end{lemma}
	\begin{proof}
		Wenn $m  > n$, so haben wir
		\begin{align*}
			\abs[x_m - x_n] = \abs[\sum_{k=n}^{m-1} x_{k+1} - x_k] \leq \max\{\abs[x_m - x_{m-1}], \dots, \abs[x_{n+1} - x_{n}]\}
		\end{align*}
		und das Lemma folgt sofort aus den Definitionen.
	\end{proof}
	Man sieht leicht, dass jede konvergente Folge auch eine Cauchy-Folge ist. 
	Die Umkehrung gilt allerdings im Allgemeinen nicht.
	Ist jede Cauchy-Folge konvergent, so nennen wir den metrischen Raum $(X,d)$ \emph{vollständig}.
	
	Die Absolutbeträge $\abs_p$ induzieren durch $d_p(x,y)=\abs[x-y]_p$ eine Metrik auf $\K$. Es ist bereits bekannt, dass $(\K, d_\infty)$ nicht vollständig ist. Für $p<\infty$ haben wir folgendes
	\begin{lemma}
		Der metrische Raum $(\K, d_p)$ ist nicht vollständig. 
	\end{lemma}
	\begin{proof}
		Wir geben einen kurzen Beweis für $p>3$ und verweisen auf Gouvea \cite{gouv} Lemma 3.2.3 für die verbleibenden zwei Fälle.
		
		Betrachten wir die Folge $(x_n) = (a^{p^n})$, wobei $1<a<p-1$ eine natürliche Zahl ist.
		Es gilt
		\begin{align*}
			\abs[a^{p^{n+1}} - a^{p^n}]_p = \abs[a^{p^{n}} ( a^{p^n(p-1)} - 1)]_p.
		\end{align*}
		Nach dem kleinen Satz von Fermat wissen wir $a^{p^n(p-1)} - 1 \equiv 0 \pmod{p^n}$, also ist $p^n$ ein Teiler von $a^{p^n(p-1)} - 1$ und es folgt
		\begin{align*}
			\abs[x_{n+1} - x_n]_p = \abs[a^{p^{n}}]_p \cdot\abs[( a^{p^n(p-1)} - 1)]_p \leq p^{-n}\to 0.
		\end{align*}
		Damit ist $(x_n)$ eine Cauchy-Folge. Angenommen $(x_n)$ konvergiert gegen ein $x\in\K$. Dann haben wir
		\begin{align*}
			x = \lim_{n\to\infty} x_{n+1} = \lim_{n\to\infty} x_{n+1}^p = x^p
		\end{align*}
		und folglich $x=1$ oder $x=-1$.
		Für $n$ genügend groß gilt
		\begin{align*}
			\abs[x-a]_p 
			&= \abs[x-x_n + x_n -a]_p 
			\leq \max\{\abs[x-x_n]_p, \abs[x_n-a]_p\} \\
			&= \abs[x_n - a]_p 
			= \underbrace{\abs[a]_p}_{\leq 1}\cdot\abs[ a^{p^n-1} - 1]_p 
			\leq \abs[ a^{p^n-1} - 1]_p < 1,
		\end{align*}
		am Ende wieder nach dem kleinen Satz von Fermat. 
		Also ist $p$ ein Teiler von $x-a$. Wegen $x=\pm 1$ und der Wahl von $a$ ist aber $0<x-a<p$. Ein Widerspruch.
	\end{proof}

	\begin{satz}
		Für jedes $p\leq\infty$ auf $\K$ gibt es eine Vervollständigung $\Kp$ und ein Absolutbetrag $\abs_p$ mit folgenden Eigenschaften:
		\begin{enumerate}
			\item Es gibt eine Inklusion $\K \hookrightarrow \Kp$ und der durch $\abs_p$ auf $\K$ induzierte Betrag ist der $p$-adische bzw. reelle Absolutbetrag.
			\item Das Bild von $\K$ bezüglich diser Inklusion ist dicht in $\Kp$.
			\item $\Kp$ ist vollständig bezüglich des Absolutbetrags $\abs_p$.
		\end{enumerate}
		Der K"orper $\Kp$ mit diesen Eigenschaften ist eindeutig bis auf isometrische Isomorphie.
	\end{satz}
\subsection{Potenzreihen}
%Zeige potenzreihen konvergieren in Q_p: done
%Zeige jede zahl in Q_p darstellbar als Potenzreihe: done
	Wie auch schon im Fall der Vervollständigung von $\Q$ bezüglich $\abs_\infty$ ist diese Konstruktion etwas schwieriger zu handhaben.
	In $\R$ hatten wir die Dezimalbruchentwicklung  $x=\pm\sum_{k=-\infty}^{n}x_k 10^{k}$, $x_k\in\{0,\dots,9\}$ oder etwas allgemeiner die $b$-adische Darstellung $x =\pm \sum_{k=-\infty}^{n}x_k b^{k}$, $x_k\in\{0,\dots,b-1\}$ für eine natürliche Zahl $b>1$, die zwar nicht unbedingt eindeutig ist ($0,999\dots = 1,000\dots$) aber leichter zu handhaben als Äquivalenzklassen von Cauchy-Folgen.
	Zu unserer Freude werden wir feststellen, dass auch in $\Q_p$ eine ähnliche Darstellung gibt. Diese ist im Gegensatz zu der in $\R$ eindeutig und besitzt nur endlich viele Nachkommastellen, allerdings nicht unbedingt endlich viele Vorkommastellen.
	\begin{satz}
		Jede Element $x\in\Kp$ lässt sich eindeutig als eine Potenzreihe 
		\begin{align*}
			x= \sum_{k=N}^{\infty} x_k p^k
		\end{align*}
		mit $x_k\in\{0,\dots,p-1\}$, $x_n\not= 0$ und $n\in\Z$ darstellen. 
		Insbesondere gilt $\abs[x]_p = p^{-n}$.
	\end{satz}
	\begin{proof}
		Wir zeigen zunächst das die Folge der Partialsummen $S_n=\sum_{k=N}^{n} x_k p^k$ in $\Kp$ konvergiert.
		\begin{align*}
			\abs[S_{n} - S_{n-1}]_p 
			= \abs[x_{n}p^{n}]_p \leq p^{-n} \to 0
		\end{align*}
		und damit haben wir schon gezeigt, dass die Partialsummen eine Cauchy-Folge bilden.
		Da $\Kp$ vollständig ist konvergiert diese gegen einen Wert für den wir $\sum_{k=N}^{\infty} x_k p^k$ schreiben.
		
		Als nächstes sei $x\in\Zp$.
		Wir konstruieren nun eine Folge $(x_k)$ mit $x_k\in\{0,\dots,p-1\}$, so dass die Partialsummen $S_n = \sum_{k=0}^{n} x_k p^k$ gegen $x$ konvergieren.
		Da $\Q$ dicht in $\Kp$ liegt finden wir für jedes $n\in\N_0$ einen vollständig gekürzten Bruch $\frac{a}{b}\in\Q$ der 
		\begin{align*}
			\abs[x-\frac{a}{b}]_p \leq p^{-n} < 1
		\end{align*}
		In der Tat finden wir sogar eine ganze Zahl. 
		Denn für $\frac{a}{b}$ wie oben haben wir
		\begin{align*}
			\abs[\frac{a}{b}]_p \leq \max\left\{\abs[x]_p, \abs[x-\frac{a}{b}]_p\right\} \leq 1,
		\end{align*}
		was zeigt, dass $p$ kein Teiler von $b$ ist. 
		Wir finden daher eine Bezout-Darstellung $1=b'b + p'p^{n}$, also $b'b \equiv 1 \pmod{p^n}$ für  $b', p' \in \Z$.
		Es folgt
		\begin{align*}
			\abs[\frac{a}{b} - ab']_p 
			= \abs[\frac{a(1-bb')}{b}]_p 
			=\abs[\frac{a(p'p^n)}{b}]_p \leq p^{-n}
		\end{align*}
		und natürlich $ab' \in \Z$. Wir definieren nun $y_n\in\Z$ als die eindeutige Zahl, so dass
		\begin{align*}
			0\leq y_n \leq p^n-1  \text{ und }  y_n \equiv ab' \pmod{p^n}.
		\end{align*}
		Wir folgern analog zu eben
		\begin{align*}
			\abs[x-y_n]_p 
			= \abs[x-\frac{a}{b} + \frac{a}{b} - ab']_p 
			\leq \max\left\{\abs[x-\frac{a}{b}]_p, \abs[\frac{a}{b} - y_n]_p\right\}
			= p^{-n}
		\end{align*}
		und folglich konvergiert die Folge $(y_n)$ gegen $x$.
		Wir zeigen, dass diese Folge eindeutig ist. 
		Sei dazu $y_n'$ eine weiter Zahl mit $0\leq y_n' \leq p^n-1$ und $\abs[x-y_n']_p\leq p^{-n}$.
		Dann gilt aber wie eben
		\begin{align*}
			\abs[y_n - y_n']_p = \abs[y_n - x + x - y_n']_p \leq p^{-n},
		\end{align*}
		also ist $p^n$ ein Teiler von $y_n - y_n'$ und folglich $y_n -y_n' = 0$.
		Aus dieser Eindeutigkeit folgt unter anderem auch, dass $y_n \equiv y_{n-1} \pmod{p^{n-1}}$, denn
		\begin{align*}
			\abs[y_n - y_{n-1}]_p = \abs[y_n - x + x - y_{n-1}]_p \leq \max\left\{\abs[y_n-x]_p, \abs[x- y_{n-1}]_p\right\} = p^{-n+1},
		\end{align*}
		und folglich $p^{n-1}$ ein Teiler von $y_n - y_{n-1}$.
		Die eindeutigen $p$-adischen Entwicklungen dieser $y_n$ sind genau die gesuchten Partialsummen $S_n$ und definieren die $x_k$ daher induktiv.
		
		Für jedes beliebige $x\in \Q_p$ finden wir ein $n\in\N_0$, so dass $p^{n}x \in \Zp$ und damit eine Entwicklung $p^{n}x = \sum_{k=0}^{\infty} x_kp^k$. Also haben wir 
		\begin{align*}
			x = \sum_{k=-n}^{\infty} x_{k+n}p^k
		\end{align*}
	\end{proof}