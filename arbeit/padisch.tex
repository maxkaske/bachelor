\section{Lokale Betrachtungen}
	\subsection{Exkurs: $p$-adische Zahlen}
	Sei $\K$ ein beliebiger K"orper und $\R_+ = \{x\in \R: x\geq 0\}$ die Menge der nicht-negativen reellen Zahlen.
	\begin{defi}
		Ein \emph{Absolutbetrag} auf $\K$ ist eine Abbildung
		\begin{align*}
			\abs:\K \longrightarrow \R_+
		\end{align*}
		welche die folgenden Bedingungen erf"ullt:
		\begin{enumerate}[label=(\roman*),leftmargin=1.5cm]
			\item $\abs[x] = 0 \Leftrightarrow x=0$ (Definitheit)
			\item $\abs[xy] = \abs[x]\abs[y]$ f"ur alle $x, y \in \K$ (Multiplikativit"at)
			\item $\abs[x+y] \leq \abs[x] + \abs[y]$ f"ur alle $x,y \in \K$ (Dreiecksungleichung)
		\end{enumerate}
		Wir nennen den Absolutbetrag $\abs$ nicht-archimedisch, wenn er zus"atzlich die st"arkere Bedingung
		\begin{enumerate}[label=(\roman*)$'$,leftmargin=1.5cm]
			\setcounter{enumi}{2}
			\item $\abs[x+y] \leq \max\{\abs[x],\abs[y]\}$ f"ur alle $x, y \in \K$ (versch"arfte Dreiecksungleichung)
		\end{enumerate}
		erf"ullt. Anderenfalls sagen wir der Absolutbetrag ist archimedisch.
	\end{defi}
	Wir m"ochten zun"achst ein paar allgeimg"ultige Eigenschaften von Absolutbetr"agen im folgenden Lemma festhalten.
	\begin{lemma}
		F"ur beliebige Absolutbetr"age $\abs$ auf $\K$ und Elemente $x\in \K$ gilt:
		\begin{enumerate}[label=(\roman*),leftmargin=1.5cm]
			\item $\abs[1] = 1$
			\item $\abs[-1] = 1$
			\item Falls $\abs[x^n]=1$, dann $\abs[x]=1$
			\item $\abs[-x]$ = $\abs[x]$ 
		\end{enumerate}
	\end{lemma}
	
	Betrachten wir nun den K"orper $\K = \Q$ der rationalen Zahlen. Sei $x\in \Q^\times$ eine beliebige rationale Zahl. Dann existiert eine eindeutige (bis auf Reihenfolge) Primfaktorzerlegung
	\begin{align*}
		x = \prod_{p} p^{v_p},
	\end{align*}
	wobei das Produkt "uber alle Primzahlen $p \in \N$ geht und $v_p \in \Z$ f"ur fast alle $p$ gleich $0$ ist. Legen wir uns auf ein $p$ fest, so erm"oglicht sich die 
	\begin{defi}
		F"ur beliebige $x \in \Q$ sei der \emph{p-adische Absolutbetrag} von $x$ gegeben durch
		\begin{align*}
			\abs[x]_p = p^{-v_p}
		\end{align*}
		f"ur $x\neq 0$ und $v_p \in \Z$ wie oben. Durch $\abs[0]_p := 0$ vervollst"andigen wir die Definition.
	\end{defi}
	\begin{lemma}
		$\abs_p$ ist ein nicht-archimedischer Absolutbetrag auf $\Q$
	\end{lemma}
	\begin{proof}
		
	\end{proof}
	
	\subsection{Lokale Fourieranalysis}
		F"ur die unendliche Stelle $p=\infty$ definieren wir die \emph{Schwartz-Bruhat Funktion} als eine komplexwertige, glatte Funktion $f$, die f"ur alle nicht-negativen ganzen Zahlen $n$ und $m$ die Bedingung
		\begin{align*}
			\sup_{x\in \K_\infty}\abs[x^n\frac{d^m}{dx^m}f(x)] < \infty
		\end{align*}
		erf"ullt.\footnote{Hier ist mit $\abs$ der komplexe Absolutbetrag gemeint.}
		F"ur die endlichen Stellen $p<\infty$ definieren wir eine \emph{Schwartz-Bruhat Funktion} als eine lokal konstante Funktion mit kompakten Träger.
		Die Menge aller solcher Funktionen bilden einen komplexen Vektorraum, den wir mit $\Sw(\K_p)$ bezeichnen. 
		Im Fall $p<\infty$ erkennt man leicht, dass $\Sw(\Kp)\subseteq L^1(\Kp)$. 
		F"ur $p=\infty$ gilt nach obiger Bedingung $(\abs[1]+\abs[x^2])\abs[f(x)] \leq C$, also $\abs[f(x)]\leq C(1+x^2)^{-1}$ und $(1+x^2)^{-1} \in L^1(\K\infty)$
		
		\begin{bsp}~ 
			\begin{enumerate}[label=(\roman*)]
				\item Im Fall $p=\infty$ ist die Funktion $f_k = x^k e^{-x^2}$ f"ur jedes $k\in\N_0$ in $\Sw(\K_\infty)$. 
				Die Ableitungen $\frac{d^m}{dx^m} f_k(x)$ sind von der Form $p(x)e^{-x^2}$, wobei $p(x)$ ein Polynom ist. 
				Aus der Analysis ist dann bekannt, dass $\abs[x^n p(x)e^{-x^2}]$ f"ur jedes $n\in \N_0$ beschränkt ist.
				\item Im Fall $p<\infty$ sind offensichtlich die charakteristischen Funktionen kompakter Mengen in $\Sw(\Kp)$. 
				Beispiele f"ur Kompakta sind Mengen der Form $a+p^k\Zp$ mit $a\in \K$ und $k\in \Z$.
			\end{enumerate}
		\end{bsp}
		
		\begin{lemma}\label{lemma:padischSBF}
			Jede Funktion $f\in \Sw(\Kp)$, $p<\infty$, ist eine endliche Linearkombination von charakteristischen Funktionen der Form $\ind_{a+p^k\Zp}$, wobei $a\in \K$ und $k\in \Z$
		\end{lemma}
		\begin{proof}
			Sei $f \in \Sw(\Kp)$. 
			Da $f$ lokal konstant ist, ist f"ur jedes $z\in\C$ das Urbild $f^{-1}(z)$ offen in $\Kp$. 
			Also ist $f^{-1}(0)$ offen, folglich $\Kp \setminus f^{-1}(0)$ abgeschlossen und daher schon $\text{supp}(f) = \Kp \setminus f^{-1}(0)$. 
			Per Definition hat die Schwartz-Bruhat Funktion $f$ kompakten Tr"ager, also ist $\Kp \setminus f^{-1}(0)$ kompakt. 
			Diese Menge wird von den offenen Mengen $f^{-1} (x)$ mit $x\not= 0$ "uberdeckt, wovon nach Kompaktheit schon endlich viele reichen.
			$f$ hat somit endliches Bild. Weiter ist jede offene Menge $f^{-1} (x)$ eine disjunkte Vereinigung offener B"allen in $\Kp$. 
			Diese haben aber genau die gesuchte Form $a+p^k\Zp$ wie oben. 
			Aufgrund der Kompaktheit, reichen wieder endliche viele solcher B"alle. Damit folgt auch schon das Lemma.
		\end{proof}
		\begin{lemma}
			Sei $f \in \Sw(\Kp)$.
			\begin{enumerate}[label=\emph{(\alph*)}]
				\item Ist $g(x)=f(x)e_p(ax)$ mit $a\in\Kp$, dann gilt $\hat{g}(x) = \hat{f}(x-a)$.
				\item Ist $g(x)=f(x-a)$ mit $a\in\Kp$, dann gilt $\hat{g}(x) = \hat{f}(x-a)e_p(-ax)$.
				\item Ist $g(x)=f(\lambda x)$ mit $\lambda \in\Kp^\times$, dann gilt $\hat{g}(x) =\frac{1}{\abs[\lambda]_p} \hat{f}(\frac{x}{\lambda})$.
			\end{enumerate}
		\end{lemma}
		\begin{proof}
			(a) und (b) sind einfache Folgerungen aus der Definition mit der Multiplikativit"at von $e_p$ und der Translationsinvarianz des Haar-Maß. 
			Bei (c) spielt unsere Normeriung des Absolutbetrags eine Rolle, denn mit dem Variablenwechsel $y\mapsto \lambda^{-1}y$ erhalten wir
			\begin{align*}
				\hat{g}(x) = \int_{\Kp} f(\lambda y) e_p(-xy)dy = \frac{1}{\abs[\lambda]_p} \int_{\Kp} f(y) e_p(-x\lambda^{-1}y)dy = \frac{1}{\abs[\lambda]_p} \hat{f}\left(\frac{x}{\lambda}\right)
			\end{align*}
		\end{proof}
		\begin{satz}\label{Satz:fourierumkehrformel}
			Ist $p\leq\infty$ und $f\in\Sw(\Kp)$, so ist $\hat{f} \in \Sw(\Kp)$ und es gilt die Umkehrformel
			\begin{align*}
				\hat{\hat{f}}(x) = f(-x)
			\end{align*}
		\end{satz}
		\begin{proof}
			Betrachten wir zuerst den Fall $p<\infty$. Wie wir eben in Lemma \ref{lemma:padischSBF} gesehen haben haben, ist jede Funktion in $\Sw(\Kp)$ eine Linearkombination von Funktionen der Form $f = \ind_{a+p^k\Zp}$. Es reicht also die Aussage f"ur solche $f$ zu zeigen.
			Sei dazu $h:= \ind_{\Zp}$. Wir zeigen $\hat{h} = h$ durch folgende Rechnung
			\begin{align*}
				\hat(h)(x) = \int_\Kp h(y) e_p (-xy) dy_p = \int_\Zp e_p(-xy) dy_p.
			\end{align*}
			Nun ist $\chi(y):=e_p(-xy)$ ein Charakter auf $\Zp$ und genau dann trivial, wenn $x\in\Zp$. 
			Weiter ist $\Zp$ kompakt. 
			Nach Lemma \ref{Lemma:trivialerCharAufKompakt} und unserer Normierung von $dy_p$ folgt also
			\begin{align*}
				\hat(h)(x) = \text{Vol}(\Zp, dy_p) \ind_\Zp = \ind_\Zp = h(x)
			\end{align*}
			
			Wir f"uhren nun folgende Operatoren auf $\Sw(\Kp)$ ein
			\begin{align*}
				L_a f(x) = f(x-a), M_\lambda f(x) = f(\lambda x),
			\end{align*}
			wobei $a \in \Kp$ und $\lambda \in \Kp^\times$. 
			Nun k"onnen wir $f$ schreiben als $L_a M_{p^{-k}}h$. 
			Es folgt
			\begin{align*}
				\hat{f} = (L_a M_{p^{-k}}h)\widehat{\phantom{x}} = \Omega_{-a}p^{k}M_{p^k}\hat{h}=\Omega_{-a}p^{-k}M_{p^k}h.
			\end{align*}
			Also ist $\hat{f} (x) = e_p(-ax)p^k\ind_{p^{-k}\Zp}(x)$. 
			Der Charakter $e_p$ ist lokal konstant, $\hat{f}$ als das Produkt lokal konstanter Funktionen selbst wieder lokal konstant und damit in $\Sw(\Kp)$. 
			Damit haben wir den ersten Teil der Aussage gezeigt.
			
			F"ur den zweiten Teil sehen wir
			\begin{align*}
				\hat{\hat{f}} = (L_a M_{p^{-k}}h)\widehat{\widehat{\phantom{x}}} = L_{-a} (M_{p^k}h)\widehat{\widehat{\phantom{x}}}=L_{-a}M_{p^k}\hat{h} =L_{-a}M_{p^k}h,
			\end{align*}
			also $\hat{\hat{f}} (x) = \ind_{-a+p^k\Zp} (x) = \ind_{a+p^k\Zp} (-x) = f(-x)$. 
			Hier haben wir $p^k\Zp = - p^k\Zp$ ausgenutzt. 
			Damit haben wir die Umkehrformel f"ur den $p$-adischen Fall gezeigt. 
			F"ur $p=\infty$ ist die Formel bereits aus der klassischen Fourieranalysis bekannt.
		\end{proof}
	
	\begin{satz}[Ostrowski]
		Jeder nicht-triviale Absolutbetrag auf $\Q$ ist "aquivalent zu einem der Absolutbetr"age $\abs_p$, wobei $p$ entweder eine Primzahl ist oder $p=\infty$.
	\end{satz}
	\begin{proof}
		Sei $\abs$ ein beliebiger nicht-trivialer Absolutbetrag auf $\Q$. Wir untersuchen die zwei m"oglichen F"alle.
		\begin{enumerate}[align=left, leftmargin=0cm, labelsep=0cm, label=\alph*)\ ]
		\item $\abs$ ist archimedisch.
			Sei dann $n_0\in \N$ die kleinste nat"urliche Zahl mit $\abs[n_0] > 1$.
			Dann gibt es ein $\alpha \in \R^+$ mit $\abs[n_0]^{\alpha}=n_0$.
			Wir wollen nun zeigen, dass $\abs[n]=\abs[n]_\infty^\alpha$ f"ur alle $n \in \N$ gilt. Der allgemeine Fall f"ur $\Q$ folgt dann aus den Eigenschaften des Betrags.
			Dazu bedienen wir uns eines kleinen Tricks: F"ur $n \in \N$ nehmen wir die Darstellung zur Basis $n_0$, d.h.
			\begin{align*}
				n = \sum_{i=0}^{k} a_i n_0^i
			\end{align*}
			mit $a_i \in \{0,1,\dots,n_0-1\}$, $a_k \neq 0$ und $n_0^k\leq n < n_0^{k+1}$. Nehmen wir davon den Absolutbetrag und beachten, dass $\abs[a_i]\leq 1$ nach unserer Wahl von $n_0$ gilt, so erhalten wir
			\begin{align*}
				\abs[n] \leq \sum_{i=0}^{k} \abs[a_i] n_0^{i\alpha}
					\leq \sum_{i=0}^{k} n_0^{i\alpha}
					\leq n_0^{k\alpha}\sum_{i=0}^{k} n_0^{-i\alpha}
					\leq n_0^{k\alpha}\sum_{i=0}^{\infty} n_0^{-i\alpha}
					 = n_0^{k\alpha} \frac{n_0^\alpha}{n_0^\alpha - 1}.
			\end{align*}
			Setzt man nun $C:=\frac{n_0^\alpha}{n_0^\alpha - 1}>0$, so sehen wir
			\begin{align*}
				\abs[n]\leq C n_0^{k\alpha}\leq C n^\alpha
			\end{align*}
			f"ur beliebige $n \in \N$, also insbesondere auch
			\begin{align*}
				|n^N|\leq C n^{N\alpha}.
			\end{align*}
			Ziehen wir nun auf beiden Seiten die $N$-te Wurzel und lassen $N$ gegen $\infty$ laufen, so konvergiert $\sqrt[N]{C}$ gegen $0$ und wir erhalten 
			\begin{align*}
				\abs[n] \leq n^\alpha
			\end{align*}
			Damit w"are die erste H"alfte geschafft. Gehen wir nun z"uruck zu unserer Basisdarstellung
			\begin{align*}
				n = \sum_{i=0}^{k} a_i n_0^i.
			\end{align*}
			Da $n < n_0^{k+1}$ erhalten wir die Absch"atzung
			\begin{align*}
				n_0^{(k+1)\alpha}=|n_0^{k+1}| = |n + n_0^{k+1} - n| \leq \abs[n] + |n_0^{k+1} -n|.
			\end{align*}
			 mit dem Ergebnis aus der ersten H"alfte des Beweises und $n\geq n_0^k$ sehen wir
			\begin{align*}
				\abs[n] &\geq n_0^{(k+1)\alpha} - |n_0^{k+1} -n| 
					\geq n_0^{(k+1)\alpha} - (n_0^{k+1} -n)^\alpha
					\\&\geq n_0^{(k+1)\alpha} - (n_0^{k+1} -n_0^k)^\alpha
					=n_0^{(k+1)\alpha} \left(1 - \left(1 - \frac{1}{n_0}\right)\right)
					\\&> n^\alpha \left(1 - \left(1 - \frac{1}{n_0}\right)\right).
			\end{align*}
			Setzen wir wieder $C':=\left(1 - \left(1 - \frac{1}{n_0}\right)\right) >0$ folgt analog zum ersten Teil, dass
			\begin{align*} 
				\abs[n]\geq n^\alpha
			\end{align*}
			und daher $\abs[n]=n^\alpha$. Damit haben wir gezeigt, dass $\abs$ "aquivalent zum klassischen Absolutbetrag $\abs_\infty$ ist.
		\item $\abs$ ist nicht archimedisch.
			Dann ist $\abs[n_0]\leq 1$ f"ur alle $n \in \N$ und, da $\abs$ nicht-trivial ist, muss es eine kleinste Zahl $n_0$ geben mit $\abs[n_0]<1$. Insbesondere muss $n_0$ eine Primzahl sein, denn sei $p \in \N$ ein Primteiler von $n_0$, also $n_0=p \cdot n'$ mit $n' \in \N$ und $n' < n$, dann gilt nach unserer Wahl von $n_0$
			\begin{align*}
				|p| = |p|\cdot |n'| =|p \cdot n'| = \abs[n_0] < 1.
			\end{align*}
			Folglich muss schon $p=n_0$ gelten. Ziel wird es jetzt nat"urlich sein zu zeigen, dass $\abs$ "aquivalent zum $p$-adischen Absolutbetrag ist.
			Zun"achst finden wir ein $\alpha \in \R^+$ mit $|p| = |p|_p^{\alpha} = \frac{1}{p^{\alpha}}$ . Sei als n"achstes $n\in \Z$ mit $p \not | n$. Wir schreiben
			\begin{align*}
				n= rp + s, r \in \Z, 0<s<p
			\end{align*}
			Nach unserer Wahl von $p=n_0$ gilt $|s|=1$ und $|rp|<1$. Es folgt $\abs[n]=\text{max}\{|rp|,|s|\}=1$. Sei nun $n\in \Z$ beliebig. Wir schreiben $n=p^{v}n'$ mit $p\not | n'$ und sehen
			\begin{align*}
				\abs[n] = |p|^{v}|n'| = |p|^v = (|p|_p^{\alpha})^{v}=\abs[n]_p^{\alpha}.
			\end{align*}
			Mit den gleichen "Uberlegungen aus dem ersten Fall folgt damit die Behauptung.
		\end{enumerate}
	\end{proof}