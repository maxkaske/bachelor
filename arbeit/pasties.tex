%
	%% Gaussfunktion Fourier %%%
	\begin{satz}
		Die (archimedische) Gaussche Funktion
		\begin{align}
			g_\infty(\xinf) \coloneqq  e^{-\pi |\xinf|^2}
		\end{align}
		ist ihre eigene Fouriertransformierte.
	\end{satz}

	\begin{proof}
		Die Fouriertransformation von $g_\infty(x)$ ist definiert als
		\begin{align*}
			\hat{g}_\infty (\xi) = \int_{-\infty}^{\infty}{g(x)e^{-2\pi i x \xi}dx}
		\end{align*}
		Betrachten wir zunächst den Integranden etwas genauer sehen wir, dass wir dank
		\begin{align*}
			g(x)e^{-2\pi i x \xi} = e^{-\pi(x^2 +2 i x \xi - \xi ^2)}e^{-\pi \xi^2} = e^{-pi (x + i \xi)^2} g(\xi)
		\end{align*}
		die Fouriertransformierte $\hat{g} (\xi)$ umschreiben können zu
		\begin{align*}
			\hat{g}(\xi) = g(\xi) \int_{-\infty}^{\infty} {e^{-\pi(x+i\xi)^2}dx}
		\end{align*}
		Fuer den Beweis reicht es also zu zeigen, dass das verbleibende Integral gleich $1$ ist.
		Wir berechnen zun"achst
		\begin{align*}
			g(x)e^{-2\pi i x \xi} = e^{-\pi(x^2 +2 i x \xi - \xi ^2)}e^{-\pi \xi^2} = e^{-pi (x + i \xi)^2} g(\xi)
		\end{align*}
		und stellen erfreut fest, dass die Fouriertransformierte von $g$ gerade
		\begin{align*}
			\hat{g}(\xi) = g(\xi) \int_{-\infty}^{\infty} e^{-\pi(x+i\xi)^2}dx
		\end{align*}
		ist. Es reicht also zu zeigen, dass das zweite Integral 1 ist.\\
		Sei zun"achst $\gamma$ eine Kurve entlang des Rechtecks mit den Ecken $-R$, $R$, $R+i\eta$ und $-R+i\eta$. 
		Nach dem Cauchy Integralsatz gilt f"ur unsere ganze Funktion $g(z)$
		\begin{align*}
			0 = \int_{-R}^{R} {g(z)dz} + \int_{R}^{R+i\eta} {g(z)dz}  + \int_{R+i\eta}^{-R+i\eta} {g(z)dz}  + \int_{-R+i\eta}^{-R} {g(z)dz} 
		\end{align*}
		Weiter gilt $|g(z)|=e^{-\pi (R^2 - y^2)}$ f"ur $z=\pm R + i y$ und $0\leq y \leq \eta$ und so verschwinden das zweite und vierte Integral f"ur $R\rightarrow \infty$. 
		Nach Umstellen der verbleibenden Integrale und genauen hinsehen stellen wir fest, dass
		\begin{align*}
			\int_\R{e^{-\pi (x + i\xi)^2}} = \int_\R {e^{-\pi x^2} = 1}
		\end{align*}
	\end{proof}