\section{Das Eingeschr"ankte direkte Produkt abstrakter Gruppen}\label{sec:rdp}
	Ziel dieses Kapitels ist es, die n"otigen Grundlagen zu schaffen, um alle bisherigen lokalen fourieranalytischen Berechnungen in einer einzigen globalen fourieranalytischen Berechnung zu vereinen. 
	Wir werden erkl"aren, warum der Ansatz über das direkte Produkt aller lokalen K"orper schiefgeht und führen darauf das eingeschr"ankte direkte Produkt mit der dazugeh"origen eingeschr"ankten Produkttopologie ein.
	Anschließend schauen wir uns die Besonderheiten der Quasi-Charaktere und der Integration auf diesem neuen Produkt an. 
	Dabei halten wir uns an Ramakrishnans und Valenzas Definitionen in \cite{rama} Kapitel 5.1, die sich wiederum an Tates Doktorarbeit \cite{tate} orientieren.
	Für einen etwas alternativen Ansatz bei der Definition des eingeschr"ankten direkten Produkts verweisen wir auf \textcite{deitmar2010} Kapitel 5.1.
\subsection{Definitionen}\label{kapitel:RDP}
		Wir beginnen mit einer (endlichen, abz"ahlbaren, überabz"ahlbaren) Indexmenge $I$ und für alle $\nu\in I$ sei $G_\nu$  eine lokalkompakte Gruppe.
		Gesucht ist zun"achst ein Objekt $G$, welches alle $G_\nu$ umfasst und auf dem wir wieder Integration und Fouriertransformation definieren k"onnen.
		$G$ sollte also idealerweise eine lokalkompakte Gruppe sein.
		Ein erster Ansatz über das direkte Produkt der Gruppen $G_\nu$ wird allerdings, wie folgendes Lemma zeigt, im Allgemeinen scheitern.
		\begin{lemma}\label{Lemma:lokalkompaktProd}
			Sei $I$ eine Indexmenge und $X_\nu$ ein lokalkompakter Hausdorff-Raum für alle $\nu \in I$. Der Raum $X\coloneqq \prod_{\nu \in I} X_\nu$ ist genau dann lokalkompakt, wenn fast alle $X_\nu$ kompakt sind.
		\end{lemma}
		Bevor wir den Beweis nach \textcite{deitmar2010} Lemma 5.1.1 geben, noch eine kurze Beobachtung: 
		Ist $X$ kompakt, so ist auch jedes $X_\nu$ als Bild von $X$ unter den stetigen Projektionen $\pi_\nu:X \to X_\nu$ kompakt.
		\begin{proof}
			Sei $E \subseteq I$ eine endliche Teilmenge und für jedes $\nu\in E$ sei $U_\nu \in X_\nu$ eine offene Menge. 
			Wir betrachten die offenen Rechtecke
			\begin{align*}
				\prod_{\nu\in E} U_\nu \times \prod_{\nu\in I\setminus E} X_\nu,
			\end{align*}
			welche eine Basis der Produkttopologie bilden. 
			Ist $X$ lokalkompakt, so gibt es ein offenes Rechteck, dessen Abschluss kompakt ist. 
			Folglich sind fast alle $X_\nu$ kompakt. 
			Die Rückrichtung ist eine Folgerung des Satzes von Tychonoff (das direkte Produkt beliebiger Familien kompakter Mengen ist wieder kompakt) und Lemma \ref{satz:topo:lcaproduct} (das endliche direkte Produkt lokalkompakter Räume ist wieder lokalkompakt).
		\end{proof}
		Das direkte Produkt lokalkompakter Gruppen liefert uns daher im Allgemeinen keine neue lokalkompakte Gruppe. 
		Wir sehen aber, welche Einschr"ankung n"otig ist, um doch eine lokalkompakte Gruppe zu erhalten.
		%%%%%%%%%%%%%%%%%%
		%  DEFINITION  %
		%%%%%%%%%%%%%%%%%%
		\begin{defi}[Eingeschränkte direkte Produkt]
			Sei $I=\{\nu\}$ eine Indexmenge und für jedes $\nu \in I$ sei $G_v$ eine lokalkompakte Gruppe. 
			Sei weiter $I_\infty \subseteq I$ eine endliche Teilmenge von $I$ und für jedes $\nu \notin I_\infty$ sei $H_\nu\leq G_\nu$ eine kompakte offene (und damit abgeschlossene) Untergruppe. 
			Das \emph{eingeschränkte direkte Produkt} der Gruppen $G_\nu$ bezüglich $H_\nu$ ist definiert als 
			\begin{align*}
				G = \rdprod{\nu \in I} G_\nu \coloneqq \{ (x_\nu): x_\nu \in G_\nu \text{ und } x_\nu \in H_\nu \text{ für alle bis auf endlich viele } \nu \}.
			\end{align*}
			mit komponentenweiser Verknüpfung. 
			Die Topologie auf $G$ ist durch die \emph{eingeschränkte Produktopologie}  gegeben. 
			Diese wird durch die Basis der \emph{eingeschränkten offenen Rechtecke}
			\begin{align*}
				\prod_{\nu\in E} U_\nu \times \prod_{\nu\in I\setminus E} H_\nu
			\end{align*}
			erzeugt, wobei $E \subset I$ eine endliche Teilmenge mit $I_\infty \subset E$ und $U_\nu \in G_\nu$ offen für alle $\nu\in E$ ist.
		\end{defi}
		G ist damit (gruppentheoretisch) zwischen der direkten Summe und dem direkten Produkt der Komponenten $G_\nu$ anzusiedeln.
		Topologisch entspricht die eingeschränkte Produkttopologie jedoch nicht der vom direkten Produkt induzierten Teilraumtopologie.
		Sie ist im Allgemeinen feiner, weshalb die Projektionen
		\begin{align*}
			\pi_\nu: G &\to G_\nu\\
					g &\mapsto g_\nu
		\end{align*}
		auf die $\nu$-te Komponenten von $G$ stetige Abbildungen sind.
		
		Betrachten wir einige Untergruppen des eingeschr"ankten direkten Produkts.
		Die Gruppen $G_\nu$ k"onnen über die stetige Inklusion 
		\begin{align*}
			\iota_\nu: G_\nu &\to G \\
					g &\mapsto (\dots,1,g,1,\dots)
		\end{align*}
		auf natürliche Weise in $G$ eingebettet werden und bilden damit eine Familie von abgeschlossenen Untergruppen.
		Sei $S$ eine endliche Teilmenge von $I$, die $I_\infty$ enth"alt. 
		Wir definieren die offene Untergruppe
		\begin{align*}
			G_S \coloneqq \prod_{\nu\in S}G_\nu \times \prod_{\nu\in I\setminus S} H_\nu
		\end{align*}
		von $G$. 
		Bezüglich der Produkttopologie sind diese $G_S$, nach Lemma \ref{lemma:topogroup:directproduct} und Lemma \ref{Lemma:lokalkompaktProd}, selbst wieder lokalkompakte Gruppen.
		Die Produkttopologie stimmt aber mit der durch $G$ induzierten Teilraumtopologie überein und die $G_S$ bilden eine Familie von lokalkompakten Untergruppen in $G$.
		Nun liegt jeder Punkt $x \in G$ in einer Untergruppe dieser Form und es folgt sofort, dass $G$ selbst eine lokalkompakte Gruppe ist.
		Damit haben wir einen geeignet Kandidaten gefunden, der uns hoffentlich die globale Kalkulation erm"oglicht.
		
		Bevor wir uns im n"achsten Abschnitt den Quasi-Charakteren auf $G$ widmen, richten wir unseren Blick noch auf die kompakten Mengen des eingeschr"ankten direkten Produkts.
		\begin{satz}%Kompakte Mengen in G_S
			Eine Teilmenge $Y$ des eingeschr"ankten direkten Produkts $G$ hat genau dann kompakten Abschluss, wenn $Y \subseteq \prod_\nu{K_\nu}$ für eine Familie von kompakten Teilmengen $K_\nu \subseteq G_\nu$ und $K_\nu = H_\nu$ für fast alle Indizes $\nu\in I$.
		\end{satz}
		\begin{proof}
			Die Rückrichtung ist klar, denn jede abgeschlossene Teilmenge eines kompakten Raumes ist wieder kompakt. 

			Für die Hinrichtung sei $K$ der Abschluss von $Y$ und damit kompakt in $G$. 
			Da die Untergruppen $G_S$ eine offene Überdeckung von $G$ bilden, gibt es eine endliche Familie $\{G_{S_n}\}$, die $K$ überdeckt. 
			Wir können sogar noch mehr sagen. 
			Da die die $S_k$ endlich sind, ist $S = \bigcup_k S_k$ endlich, also wird $K$ sogar von nur einem $G_S$ überdeckt. 
			Sei $K_\nu$ das kompakte Bild von $K$ unter der natürlichen Projektion $\pi_\nu$. 
			Es ist $K_\nu \subseteq H_\nu$ für alle $\nu\notin S$, sodass wir hier $K_\nu$ durch die kompakten $H_\nu$ ersetzen können. 
			Dann ist $Y\subseteq K \subseteq \prod_\nu{K_\nu}$ und wir sind fertig.
		\end{proof}
		\begin{korollar}\label{kor:rdp:kompakt}
			Jede kompakte Teilmenge $K$ von $G$ liegt in einer der Untergruppen $G_S$.
		\end{korollar}
 
\subsection{Quasi-Charaktere}
		Jede stetige Abbildung $f$ auf $G$ induziert durch die Inklusion $\iota_\nu: G_\nu \to G$ eine stetige Abbildung $f_\nu = f \circ \iota_\nu$ auf der Komponente $G_\nu$.
		Analog ist $f_\nu$ ein Homomorphismus auf $G_\nu$, wenn $f$ selbst ein Homomorphismus auf $G$ ist.
		Damit lassen sich die Charaktere und Quasi-Charaktere auf $G$ wie folgt charakterisieren.
		\begin{lemma}\label{lemma:rdp:char}
			Sei $\chi:G \to \Komplex^\times$ ein (Quasi-)Charakter. 
			Dann sind alle $\chi_\nu: G_\nu \to \Komplex^\times$ (Quasi-)Charaktere und wirken fast alle trivial auf $H_\nu$.
			Folglich haben wir fast überall $\chi_\nu (g_\nu) = 1$ mit $g=(g_\nu)\in G$ und es gilt die Produktformel
			\begin{align*}
				\chi(g) = \prod_\nu \chi_\nu(g_\nu).
			\end{align*}
		\end{lemma}
		\begin{proof}
			Das alle $\chi_\nu$ (Quasi-)Charaktere sind folgt aus unseren Vorüberlegungen.
			Wir müssen also nur noch zeigen, dass fast alle $\chi_\nu$ jeweils trivial auf die Untergruppen $H_\nu$ wirken. 
			Dazu wählen wir uns eine offene Umgebung $U$ der $1$ in $\Komplex^\times$, die nur die triviale Untergruppe $\{1\}$ enthält. 
			Aufgrund der Stetigkeit von $\chi$ finden wir eine offene Umgebung $V=\prod_\nu V_\nu$ des neutralen Elements $1$ in $G$ mit $V_\nu = H_\nu$ für alle $\nu$ außerhalb einer endlichen Indexmenge $S$ und $\chi(V)\subseteq U$.
			Dann gilt aber
			\begin{align*}
				(\prod_{\nu\in S} 1) \times (\prod_{\nu \notin S} H_\nu) \subseteq V 
			\end{align*}
			und daher
			\begin{align*}
				\chi\Bigl((\prod_{\nu\in S} 1) \times (\prod_{\nu \notin S} H_\nu)\Bigr) \subseteq U .
			\end{align*}
			Die linke Seite ist, als Bild einer Gruppe unter einem Homomorphismus, selbst wieder eine Gruppe. 
			Nach unserer Wahl von $U$ folgt also
			\begin{align*}
				\chi\Bigl((\prod_{\nu\in S} 1) \times (\prod_{\nu \notin S} H_\nu)\Bigr) = \{1\}
			\end{align*}
			und daher $\chi_\nu (H_\nu) = \{1\}$ für alle $\nu\notin S$. 
			Damit ist aber klar, dass für jedes $g \in G$ das Produkt $\prod_\nu \chi_\nu(g_\nu)$ endlich ist und, wegen $g = \prod_{\nu} \iota_\nu(g_\nu)$, genau $\chi(g)$ entspricht.
			
		\end{proof}
		Wir k"onnen das Lemma aber auch umdrehen und so Quasi-Charaktere auf $G$ konstruieren.
		\begin{lemma}\label{lemma:rdp:char2}
			Seien $\chi_\nu: G_\nu \to \Komplex^\times$ (Quasi-)Charaktere und nehmen wir an, dass fast alle trivial auf $H_\nu$ wirken.
			Dann ist
			\begin{align*}
				\chi(g) = \prod_\nu \chi_\nu(g_\nu)
			\end{align*}
			ein (Quasi-)Charakter auf $G$.
		\end{lemma}
		\begin{proof}
			Das Produkt $\chi$ ist wohldefiniert, da fast alle $g_\nu \in H_\nu$, und bildet einen Gruppenhomomorphismus auf $G$. 
			Es bleibt noch zu zeigen, dass $\chi$ stetig ist. 
			Da $G$ und $\Komplex^\times$ topologische Gruppen sind, genügt es, sich offene Umgebungen der $1$ anzuschauen. 
			Sei daher $U$ eine offene Umgebung der $1 \in \Komplex^\times$.
			Sei $S$ die endliche Menge aller Indizes $\nu$, so dass $\chi_\nu$ nicht trivial auf $H_\nu$ wirkt, und setze $n = \abs[S]$. 
			Wir finden eine weitere Umgebung $W$ der $1$ in $\Komplex^\times$, so dass das Produkt von $n$ beliebigen Elementen aus $W$ wieder in $U$ liegt. 
			Aufgrund der Stetigkeit von $\chi_\nu$, finden wir für jedes $\nu\in S$ eine offene Umgebungen $V_\nu$ der $1 \in G_\nu$ mit $\chi_\nu(V_\nu) \subseteq W$. 
			Für $\nu\notin S$ setzen wir $V_\nu = H_\nu$. 
			Dann ist $V = \prod_\nu V_\nu$ eine offene Umgebung der $1 \in G$ und für jedes $g \in V$ ist $\chi(g)$ das endliche Produkt von $n$ Faktoren aus $W$, also $\chi(g) \in V$.		
		\end{proof}
	
\subsection{Integration}
		Wie wir gesehen haben ist $G=\rdprod{\nu\in I} G_\nu$ eine lokalkompakte Gruppe, besitzt also nach Satz \ref{satz:topo:haarmeasure} ein Haar-Maß.
		Dieses m"ochten, wir abh"angig von den lokalen Maßen, normalisieren.
		\begin{satz}
			Sei $G$ das eingeschränkte direkte Produkt einer Familie lokalkompakter Gruppen $G_\nu$ bezüglich der kompakten Untergruppen $H_\nu \leq G_\nu$.
			Bezeichne $dg_\nu$ das Haar-Maß auf $G_\nu$ mit der Normalisierung
			\begin{align*}
				\Vol(H_\nu, dg_\nu) = \int_{H_\nu} dg_\nu = 1
			\end{align*}
			für fast alle $\nu\in I$. 
			Dann gibt es ein eindeutiges Haar-Maß $dg$ auf G, so dass für jede endliche Teilmenge $S\supseteq I_\infty$ der Indexmenge $I$ die Einschränkung $dg_S$ von $dg$ auf $G_S$ genau das (Radon-)Produktmaß ist.
		\end{satz}
		Die im Satz angesprochene Normalisierung der $dg_\nu$ ist möglich, da die Untergruppen $H_\nu$ per Definition offen und kompakt sind und daher ein nicht verschwindendes, endliches Volumen besitzen.
		\begin{proof}
			Sei $S$ eine beliebige Indexmenge,  die $I_\infty$ enthält, und definiere $dg_S$ als das Produktmaß $dg_S \coloneqq \left(\prod_{\nu \in S}dg_\nu\right) \times dg^S$, wobei $dg^S$ das Haar-Maß auf der kompakten Gruppe $G^S\coloneqq \prod_{i \notin S} H_\nu$ mit $\Vol(G^S, dg^S) = 1$ bezeichnet. 
			Für die Existenz von $dg^S$ siehe zum Beispiel \textcite{folland} Satz 7.28. 
			Als endliches Produkt von Haar-Maßen ist $dg_S$ selbst wieder ein Haar-Maß und wir können $dg$ auf $G$ so normieren, dass dessen Einschränkung auf $G_S$ mit $dg_S$ übereinstimmt.
			
			Damit haben wir eine Normierung, von der wir allerdings noch nicht wissen, ob sie unabh"angig von unserer Wahl der Indexmenge $S$ ist.
			Sei daher $T\supseteq S$ eine weitere endliche Indexmenge. 
			Per Definition ist $G_S$ eine Untergruppe von $G_T$. 
			Wir wollen jetzt zeigen, dass das Maß $dg_T$ mit $dg_S$ auf $G_S$ übereinstimmt.
			Zerlegt man $G^S = \prod_{\nu\in T \setminus S} H_\nu \times G^T$, so bildet $\prod_{\nu\in T \setminus S} dg_\nu\times dg^T$ ein Haar-Maß, welches der kompakten Gruppe $G^S$ das oben geforderte Maß $1$ zuweist.
			Aus der Eindeutigkeit des Haar-Maßes auf (lokal)kompakten Gruppen folgt somit die Gleichheit zu $dg^S$ und daher
			\begin{align*}
				dg_S 
					= \prod_{\nu \in S}dg_\nu \times dg^S 
					= \prod_{\nu \in S}dg_\nu \times \prod_{\nu\in T \setminus S} dg_\nu \times dg^T
					= \prod_{\nu \in T}dg_\nu \times dg^T = dg_T
			\end{align*}
			auf $G_S$. 
			Sei $S'$ eine weitere beliebige Indexmenge, die $I_\infty$ enthält. 
			Das normierte Maß $dg$ wird auf $G_{S\cup S'}$ eingeschränkt zu einem Maß, welches ein konstantes Vielfaches von $dg_{S\cup S'}$ ist. 
			Da aber $G_S \subseteq G_{S\cup S'}$, muss diese Konstante $1$ sein, denn nach obigen Überlegungen ist die Einschränkung von $dg_{S\cup S'}$ auf $G_S$ gerade $dg_{S}$.
			Umgekehrt ist aber $dg_{S'}$ die die Einschränkung von $dg_{S\cup S'}$ auf $G_{S'}$, also ist die Normalisierung unabhängig von der Wahl unserer Indexmenge $S$.
		\end{proof}
		%Wir schreiben manchmal $\prod_{i} dg_\nu$ für das wie oben normierte Maß $dg$.
		Als n"achstes lernen wir, wie man einfache Funktionen auf $G$ bezüglich $dg$ integriert.
		\begin{proposition}\label{prop:rdp:integrieren}
			Sei $G$ das eingeschränkte direkte Produkt mit dem induzierten Maß $dg$
			\begin{enumerate}[label=(\roman*)]
				\item Sei $f \in L(G)$ eine integrierbare Funktion auf $G$. Dann gilt
					\begin{align*}
						\int_G f(g)dg = \lim_S \int_{G_S} f(g_S) dg_S.
					\end{align*}
					Nehmen wir nur an, dass $f$ stetig ist, so gilt diese formale Identität immer noch, wenn wir unendliche Werte erlauben.
				\item Sei $S_0$ ein beliebige endliche Indexmenge, die $I_\infty$ und alle Indizes $\nu$ enthält, für die $\Vol(H_\nu, dg_\nu) \not= 1$ gilt. 
					Für jeden Index $\nu$ sei $f_\nu$ eine stetige Funktion auf $G_\nu$, so dass $f_\nu |_{H_\nu} = 1$ für alle $\nu \notin S_0$. 
					Wir definieren
					\begin{align*}
						f(g) = \prod_{\nu}f_\nu(g_\nu)
					\end{align*} 
					für alle $g\in G$. Dann ist $f$ wohldefiniert und stetig auf $G$. 
					Ist $S$ eine beliebige endliche Indexmenge, die $S_0$ enthält, so haben wir
					\begin{align}\label{eq:satz:integration}
						\int_{G_S} f(g_S) dg_S = \prod_{\nu\in S}\Bigl(\int_{G_\nu} f_\nu (g_\nu)dg_\nu\Bigr).
					\end{align}
					Nimmt ferner das Produkt
					\begin{align}\label{eq:rdp:produkt}
						\prod_{\nu\in I}\Bigl(\int_{G_\nu} f_\nu (g_\nu)dg_\nu\Bigr)
					\end{align}
					sogar nur einen endlichen Wert an, dann ist $f$ insbesondere integrierbar und es gilt
					\begin{align*}
						\int_{G} f(g) dg = \prod_{\nu\in I}\Bigl(\int_{G_\nu} f_\nu (g_\nu)dg_\nu\Bigr).
					\end{align*}	
			\end{enumerate}
		\end{proposition}
		Bevor wir zum Beweis kommen, halten wir zünachst fest, was überhaupt mit dem Ausdruck $\lim\limits_S \phi(S) = \phi_0$ für eine Funktion $\phi$ auf den endlichen Indexmengen $S$, mit Werten in einem abgeschlossenen topologischen Raum, gemeint ist: 
		Gegeben sei eine beliebige Umgebung $V$ von $\phi_0$, dann gibt es eine Indexmenge $S(V)$, sodass für alle weiteren Indexmengen $S \supseteq S(V)$ der Wert $\phi(S)$ in $V$ liegt. 
		Intuitiv ist also $\lim\limits_S \phi(S)$ der Limes von $\phi(S)$ über gr"oßer und gr"oßere $S$.
			
		\begin{proof}
			(i) Aus der Integrationstheorie (vgl. \textcite{folland} Korollar 7.13) ist bekannt, dass 
			\begin{align*}
				\int_{G} f(g) = \sup_{K \text{ kompakt}} \left\{ \int_{K} f(g)dg \right\}.
			\end{align*}
			Da aber jedes solche $K$ nach Korollar \ref{kor:rdp:kompakt} in einer der Untergruppen $G_S$ liegt, folgt die Gleichung sofort.
			
			(ii) Aus der Bedingung $f_\nu |_{H_\nu} = 1$ folgt, dass das Produkt $f(g) = \prod_{\nu}f_{\nu}(g_\nu)$ für alle Elemente $g \in G$ endlich und die Funktion damit wohldefiniert ist. 
			Um die Stetigkeit zu zeigen, betrachten wir die offenen Umgebungen von $g$.
			Die offenen Rechtecke $\prod N_\nu \times \prod H_\nu$ bilden eine Umgebungsbasis, wobei in dem ersten endlichen Produkt die $N_\nu$ offene Umgebungen von $g_\nu$ sind. 
			Dies bleibt auch eine Basis wenn wir zus"atzlich noch verlangen, dass das erste Produkt zus"atzlich auch alle Indizes enth"alt, für die $f_\nu$ nicht trivial auf $H_\nu$ wirkt.
			Auf diesen Rechtecken ist $f$ lokal betrachtet dann nichts weiter als ein endliches Produkt stetiger Funktionen und daher selbst stetig.
			
			Für den zweiten Teil der Behauptung sei $S$ eine Indexmenge nach den Bedingungen von (ii). 
			Unter der Annahme $\Vol(H_\nu, dg_\nu) = 1$, für alle $\nu \notin S$, haben wir auf $G_S$ das durch $dg$ induzierte Produktmaß $dg_S =\left(\prod_{s \in S}dg_\nu\right) \times dg^S$.
			Weiter hat $f$ wegen $f_\nu|_{H_\nu} = 1$ für $\nu \notin S$ dort die Form $f(g) = \prod_{\nu\in S}f_\nu(g_\nu)$. 
			Es folgt:
			\begin{align*}
				\int_{G_S} f(g_S) dg_S = \prod_{\nu\in S}\Bigl(\int_{G_\nu} f_\nu (g_\nu)dg_\nu\Bigr) \cdot \int_{G^S} dg^S = \prod_{\nu\in S}\Bigl(\int_{G_\nu} f_\nu (g_\nu)dg_\nu\Bigr).
			\end{align*}
			
			Nehmen wir nun an, dass das Produkt \eqref{eq:rdp:produkt} endlich ist. 
			Dann gilt nach Aussage (i) und Gleichung \eqref{eq:satz:integration}
			\begin{align*}
				\prod_{\nu\in I}\Bigl(\int_{G_\nu} f_\nu (g_\nu)dg_\nu\Bigr) = \lim_S \int_{G_S} f(g_S) dg_S = \int_{G} f(g) dg
			\end{align*}
			und der Rest der Behauptung folgt.
		\end{proof}
	%Damit haben wir die wichtigsten theoretischen Grundlagen etabliert.
	%Im n"achsten Kapitel lernen wir explizit zwei Beispiele kennen.