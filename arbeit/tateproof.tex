\section{Tates Beweis der Funktionalgleichung}\label{sec:tateproof}
	Wir haben nun alle Werkzeuge zusammen um die Ideen aus Kapitel \ref{sec:lokal} und Riemanns Beweis auf die Adele zu "ubertragen.
	Dazu f"uhren wir die Fouriertransformation auf den Adelen ein und beweisen das "Aquivalent zu Riemanns Thetaformel.
	Daf"ur weichen wir etwas von Tates Doktorarbeit um eine leichter vest"andliche Argumentation zu geben, die sich zwar nicht so leicht auf den Allgemeineren Fall "ubertragen l"asst, jedoch weitaus weniger Vorwissen ben"otigt.
	Anschließend werden wir aber Tates Beweis der globalen Funktionalgleichung in seiner ganzen F"ulle geben und zeigen zum Schluss des Kapitels und dieser Arbeit, wie dieses Resultat die Funktionalgleichung der Riemannschen Zeta-Funktion liefert.
	
	
\subsection{Globale Fourieranalysis}
	Wir spiegeln den Abschnitt "uber die lokale Fourieranalysis und beginnen mit der Definition eines Standardcharakters auf $\A$.
	\begin{defi}
		Der \emph{globale Standardcharakter} $e:\A \to S^1$ ist durch
		\begin{align*}
			e(x) = \prod_{p\leq \infty} e_p(x_p)
		\end{align*}
		definiert als Produkt aller lokalen Standardcharaktere $e_p:\Kp \to S^1$.
	\end{defi}
	Nach Korollar \ref{kor:lokal:charTrivialZp} wirken fast alle $e_p$ trivial auf $\Zp$ und von daher folgt mit Lemma \ref{lemma:rdp:char} "uber die Charaktere des eingeschr"ankten direkten Produkts direkt, dass diese Abbildung wohldefiniert und ebenfalls ein Charakter ist.
	%Der Standardcharakter verh"alt sich besonders sch"on auf $\K$. 
	%Wir haben n"amlich folgenden
	\begin{satz}\label{satz:global:stdcharTrivialAufK}
		Der Standardcharakter $e: \A \to S^1$ wirkt trivial auf $\K$.
	\end{satz}
	F"ur den Beweis ben"otigen wir noch ein kleines Lemma aus der Zahlentheorie.
	\begin{lemma}
		Jede rationale Zahl kann als endliche Summe von Zahlen der Form $\pm \frac{1}{p^m}$ geschrieben werden.
	\end{lemma}
	\begin{proof}
		F"ur die Zahl $0$ entspricht das der leeren Summe und gilt trivialerweise.
		Sei nun $r = \pm a/b$ mit $a,b\in\N$ eine nicht verschwindende rationale Zahl. 
		Wir k"onnen $r$ zun"achstmal als Summe von $a$ Kopien von $\sgn{r}/b$ geschrieben werden.
		Von daher reicht also rationale Zahlen der Form $1/b$ zu betrachten. 
		Sei $b=\prod_{k=1}^{n} p_k^{m_k}$ die Primfaktorzerlegung von $b$.
		Dann gibt es eine Bezout-Darstellung $1 = u p_1^{m_1} + v p_2^{m_2}\dots p_n^{m_n}$ mit ganzen Zahlen $u$ und $v$.
		Es folgt
		\begin{align*}
			\frac{1}{b} = \frac{u p_1^{m_1} + v p_2^{m_2}\dots p^{m_n}}{p_1^{m_1}\dots p_n^{m_n}} =  \frac{u}{p_2^{m_2}\dots p_n^{m_n}} + \frac{v}{p_1^{m_1}}.
		\end{align*}
		Induktiv erhalten wir damit ganze Zahlen $u_1,\dots,u_n$, so dass
		\begin{align*}
			\frac{1}{b} = \frac{u_1}{p_1^{m_1}} + \dots + \frac{u_n}{p_n^{m_n}}.
		\end{align*}
		Mit dem gleichen Argument vom Anfang des Beweises ist jeder dieser Summand wieder eine endliche Summe von $\abs[u_k]$ Kopien von $\pm 1/p_k^{m_k}$ und es folgt das Lemma.
	\end{proof}
	Kommen wir damit zum eigentlichen Beweis des Satzes.
	\begin{proof}[Beweis von Satz \ref{satz:global:stdcharTrivialAufK}] 
		Da es sich bei dem Standardcharakter um einen additiven Charakter handelt, reicht es nach obigen Lemma sich auf rationale Zahlen der Form $1/p^m$ zu beschr"anken.
		F"ur jede endliche Stelle $q\not=p$ ist aber $1/p^m \in \Z_q$, also $e_q(1/p^m) =1$.
		Es folgt
		\begin{align*}
			e\left( \frac{1}{p^m} \right) 	= \prod_{q\leq \infty} e_q\left( \frac{1}{p^m} \right) 
											&= e_\infty \left( \frac{1}{p^m} \right) e_p\left( \frac{1}{p^m} \right)\\
											&= \exp(-2\pi i p^{-m}) \exp(2\pi i p^{-m}) = 1
		\end{align*}
	\end{proof}
	Wir wollen wieder den Bezug zur abstrakten harmonischen Analysis herstellen und beginnen damit, das globale "Aquivalent von Lemma \ref{lemma:lokal:genericchar} zu beweisen
	\begin{lemma}
		Ist $\psi: \A \to S^1$ ein Charakter, so gibt es ein eindeutig bestimmtes Adel $a\in\A$ mit $\psi(x) = e(ax)$.
	\end{lemma}
	\begin{proof}
		Den Großteil der Arbeit haben wir bereits in Lemma \ref{lemma:rdp:char} gemacht. 
		Demnach ist $\psi(x) = \prod_{p\leq\infty} \psi_p (x_p)$, wobei die $\psi_p: \Kp \to S^1$ ein lokale Charaktere sind und fast alle $\psi_p$ trivial auf $\Zp$ wirken.
		Nach Lemma \ref{lemma:lokal:genericchar} gilt $\psi_p (x_p) = e_p(a_p x_p)$ f"ur eindeutige $a_p \in \Kp$ und nach Korollar \ref{kor:lokal:charTrivialZp} liegen schon fast alle $a_p$ bereits in $\Zp$.
		Folglich ist $a = (a_p)$ ein eindeutig bestimmtes Adel und es gilt
		\begin{align*}
			\psi(x) = \prod_{p\leq\infty} \psi_p (x_p) = \prod_{p\leq\infty} e_p (a_px_p) = e(ax).
		\end{align*}
	\end{proof}
	%----------------------%
	% DEFINITION : FOURIER %
	%----------------------%
	Damit haben wir nun alles zusammen und kommen zur zentralen Definition dieses Abschnitts.
	\begin{defi}[Globale Fouriertransformation]
		Sei $f\in L^1(\A)$. Wir definieren dann die \emph{adelische Fouriertransformation} $\hat{f}: \A \to \Komplex$ von $f$ durch die Formel
	\begin{align*}
		\hat{f}(\xi) = \int_{\A} f(x)e(-\xi x)  \dx
	\end{align*}
	f"ur alle $\xi \in \A$.
	\end{defi}
	
	Lokal haben wir gesehen, dass die Schwartz-Bruhat Funktionen $\Sw(\Kp)$ sich besonders f"ur die Fourieranalysis geeignet haben. 
	Was ist das "Aquivalent im Globalen?
	Wir definieren zun"achst die \emph{faktorisierbaren Schwartz-Bruhat Funktion} $f=\prod_{p\leq \infty} f_p$ als Produkte lokaler Schwartz-Bruhat Funktionen $f_p$, wobei f"ur fast alle Stellen $p<\infty$ dieser Faktor gleich der charakteristischen Funktion von $\Zp$ ist.
	Damit sind sie insbesondere im Sinne von Kapitel \ref{sec:adeleidele} einfache Funktionen auf $\A$ und k"onnen nach Satz \ref{satz:adeleidele:intA} vergleichsweise schmerzlos integriert werden.
	Mit der Charakterisierung der lokalen Schwartz-Bruhat Funktionen in Lemma \ref{lemma:lokal:sw} "uber allen endlichen Stellen k"onnen die faktorisierbaren Funktionen auch geschrieben werden als
	\begin{align}\label{eq:tateproof:fsw1}
		f(x) = f_\infty(x) \prod_{p<\infty} \ind_{a_p+p^{k_p}\Zp}
	\end{align}
	mit $a_p \in \Zp$ und $k_p = 0$ f"ur fast alle Stellen.
	Die \emph{adelischen Schwartz-Bruhat Funktion} sind endliche Linearkombinationen faktorisierbarer Schwartz-Bruhat Funktionen "uber  $\Komplex$ und notieren sie wieder mit $\Sw(\A)$.
	Es l"asst sich wieder folgende Satz beweisen.
	
	\begin{satz}\label{satz:tateproof:umkehrformel}
		F"ur jede Funktion $f\in\Sw(\A)$ ist die Fouriertransformation wohldefiniert und liegt wieder in $\Sw(\A)$.
		Insbesondere gilt die Umkehrformel
		\begin{align*}
			\hat{\hat{f}}(x) = f(-x)
		\end{align*}
		f"ur alle Adele $x\in \A$.
	\end{satz}
	\begin{proof}
		F"ur den Beweis reicht es nur faktorisierbare $f$ zu betrachten. 
		Mit Satz \ref{satz:adeleidele:intA} "uber die Integration einfacher Funktionen auf den Adelen folgt
		\begin{align*}
			\hat{f} (\xi) 	&= \int_{\A} f(x)e(-\xi x)  \dx
							%= \int_{\A} \prod_{p\leq \infty} f_p(x_p) e_p(-\xi_p x_p) \dx \\
							= \prod_{p\leq \infty} \int_{\Kp}  f_p(x_p) e_p(-\xi_p x_p) \dx_p
							= \prod_{p\leq \infty} \hat{f}_p (\xi_p).
		\end{align*}
		Die Umkehrformel ergibt sich damit direkt aus den lokalen Umkehrformeln.
	\end{proof}
	Mit der gleichen Argumentation zeigt man auch leicht:
	\begin{korollar}\label{kor:tateproof:fourierids}
		Sei $f \in S^1(\A)$.
		\begin{enumerate}[label=\emph{(\roman*)}]
			\item Ist $g(x)=f(x)e(ax)$ mit $a\in\A$, dann gilt $\hat{g}(x) = \hat{f}(x-a)$.
			\item Ist $g(x)=f(x-a)$ mit $a\in\A$, dann gilt $\hat{g}(x) = \hat{f}(x-a)e(-ax)$.
			\item Ist $g(x)=f(\lambda x)$ mit $\lambda \in \I$, dann gilt $\hat{g}(x) =\frac{1}{\abs[\lambda]_\A} \hat{f}(\frac{x}{\lambda})$.
		\end{enumerate}
	\end{korollar}
	
	
\subsection{Adelische Poisson-Summenformel und der Satz von Riemann-Roch}
%poisson done
%riemann-roch (einfach) done
%beweisskizze fuer beweis nach tate: anhang

	Werfen wir anfangs einen Blick auf die faktorisierbaren Schwartz-Bruhat Funkionen.
	\begin{lemma}\label{lemma:global:sbf}
		Jede faktorisierbare adelische Schwartz-Bruhat Funktion $f\in\Sw(\A)$ hat die Form
		\begin{align*}
			f = f_\infty\prod_{p<\infty} \ind_{a+p^{k_p}\Zp},
		\end{align*}
		wobei $a\in\K$ und $k_p\in \Z$ f"ur fast alle Stellen $p$ verschwindet.
	\end{lemma}
	Der Unterschied zur Darstellung der faktorisierbaren Schwartz-Bruhat Funktionen wie in Gleichung \eqref{eq:tateproof:fsw1} liegt in der Tatsache, dass der Faktor $a$ jetzt unabh"angig von der jeweiligen Stelle ist.
	\begin{proof}
		F"ur fast alle Stellen $p$ ist $a\in \Zp$ und $k_p=0$, d.h. $\ind_{a+p^{k_p}\Zp} = \ind_{\Zp}$ fast "uberall.
		Damit ist klar, dass Funktionen dieser Form in $\Sw(\A)$ liegen. 
		
		Sei nun $f = \prod_{p\leq\infty} f_p\in\Sw(\A)$ eine faktorisierbare Schwartz-Bruhat Funktion und $S$ die Menge der Stellen mit $f_p\not=\ind_\Zp$.
		F"ur alle $p \in S$ gilt $f_p=\ind_{a_p+p^{k_p}\Zp}$ f"ur ein $a_p \in \K$ und $k\in \Z$.
		Sind wir vorerst optimistisch und nehmen an, dass $k_p>0$ und die $a_p$ bereits ganze Zahlen sind.
		Wir suchen nun eine ganze Zahl $a$, die $a \equiv a_p \pmod{p^{k_p}}$ f"ur alle $p\in S$ erf"ullt.
		Das liefert uns aber gerade der chinesische Restsatz.
		Damit ist$a_p+p^{k_p}\Zp = a+p^{k_p}\Zp$ und wir w"aren fertig.
		
		Bleiben wir weiter optimistisch und versuchen die Funktionen f"ur den allgemeinen Fall geeignet umzuformen.
		Sei dazu $N$ der Hauptnenner der rationalen Zahlen $a_p$, $p\in S$ und sei $M=\prod_{p\in S} p^{\abs[k_p]+1}$.
		Dann sind $N$ und $M$ nat"urliche Zahlen und wir betrachten die Funktion 
		\begin{align*}
			f\left((NM)^{-1} x\right) 
				=   \prod_{p\leq\infty} f_p\left((NM)^{-1} x_p\right).
		\end{align*}
		An fast alle Stellen ist $f_p\left((NM)^{-1} x_p\right)$ wieder gleich $\ind_\Zp$, da $(NM)^{-1} \in \Zp$ fast "uberall.
		F"ur die restlichen endlichen Stellen dagegen entspricht sie der charakteristischen Funktion von $NM(a_p+p^k_p\Zp)$.
		Hier m"ussen wir nur aufpassen, da dies nicht mehr nur die Stellen in $S$ betrifft. 
		Wegen $NM\in\Z$ macht dies jedoch keine Probleme.
		Mit den bekannten Eigenschaften der $p$-adischen Zahlen erhalten wir dort $NM(a_p+p^k_p\Zp) = a_p'+p^{k'_p}\Zp$ mit $a_p'\in \Z$ und $k_p'>0$.
		Somit finden wir ein $a' \in \Z$ mit $NM(a_p+p^{k_p}\Zp) = a'+p^{k_p'}\Zp$.
		%Jetzt setzen wir alles zusammen durch
		%\begin{align*}
			%a_p+p^{k_p}\Zp &= (NM)^{-1}NM(a_p+p^{k_p}\Zp) 
			%\\&= (NM)^{-1}(a'+p^{k_p'}\Zp)  = (NM)^{-1}a'+p^{k_p}\Zp
		%\end{align*}
		%f"ur alle Stellen $p$.
		Setzen wir nun $a\coloneqq (NM)^{-1}a'$ folgt die Behauptung.
	\end{proof}

	Diese sch"one Form der globalen Schwartz-Bruhat Funktionen erlaubt es uns nun die Poisson Summenformel auf den Adelen zu beweisen.
	\begin{satz}[Poisson Summenformel]
	\label{satz:tateproof:poisson}
		Sei $f \in S(\A)$. Dann konvergieren folgende Summen absolut und wir haben die Gleichung
		\begin{align}
			\sum_{\gamma \in \K} {f(\gamma)} = \sum_{\gamma \in \K}{\hat{f}(\gamma)} \label{eq:tateproof:fourier}
		\end{align}
	\end{satz}
	Der folgende Beweis orientiert sich an Deitmar \cite{deitmar2010} Satz 5.4.9 und wurde etwas auf unsere Definition der Adele angepasst.
	\begin{proof}
		Es gen"ugt wieder die Aussage f"ur faktorisierbare Schwartz-Bruhat Funktionen $f \in S(\A)$ zu zeigen.
		Nach vorherigem Lemma \ref{lemma:global:sbf} haben diese gerade die Form
		\begin{align}
		\label{eq:schwartzform}
			f = f_\infty\prod_{p<\infty} \ind_{a+p^{k_p}\Zp}.
		\end{align}
		f"ur ein $a \in \K$ und $k_p = 0$ f"ur fast alle Stellen $p$.
		Mit $N\coloneqq \prod_{p<\infty} p^{-k_p} \in \K$ haben wir dann
		\begin{align*}
			\sum_{\gamma \in \K} {f(\gamma)} 	&= 	\sum_{\gamma \in \K} f(\gamma - a)
												= 	\sum_{\gamma \in \K} f(N(\gamma - a)) \\
												&= 	\sum_{\gamma \in \K} \left[ f_\infty (N(\gamma - a))  \prod_{p< \infty} \ind_{a+p^{k_p}\Zp}(N(\gamma - a)) \right] \\
												&= 	\sum_{\gamma \in \K} \left[ f_\infty (N(\gamma - a))  \prod_{p< \infty} \ind_{\Zp}(\gamma) \right]
												=	\sum_{\gamma \in \Z} f_\infty (N(\gamma - a))
		\end{align*}
		Die Summe am Ende konvergiert aber absolut, da $f_\infty \in \Sw(\R)$.
		
		Mit der Konvergenz aus dem Weg besch"aftigen wir uns nun mit der Summenformel.
		Dabei werden sich die obigen Umformungen als sehr hilfreich erweisen.
		Erinnern wir uns zun"achst an die Operatoren aus dem Beweis der lokalen Umkehrformeln in Satz \ref{satz:lokal:umkehrformel}.
		Dort haben wir die Notation $L_a f(x) = f(x-a)$ und $M_\lambda f(x) = f(\lambda x)$ eingef"uhrt.
		Mit ihnen k"onnen wir die Funktion \eqref{eq:schwartzform} umschreiben zu
		\begin{align*}
			f = L_a M_{N}\left( \left(M_{1/N} L_{-a} f_\infty\right) \prod_{p<\infty}\ind_{\Zp}\right).
		\end{align*}
		Damit er"offnet sich nun folgende Beweisidee. 
		Wir zeigen zuerst, dass Gleichung \eqref{eq:tateproof:fourier} (falls sie denn tats"achlich gilt) stabil unter den Operatoren $L_a$ und $M_\lambda$ ist.
		Anschließend zeigen wir sie f"ur den einfacheren Fall $f=f_\infty \prod_{p<\infty}\ind_{\Zp}$.
		Beides kombiniert ergibt dann den Beweis.
		
		Gelte also $\sum_{\gamma \in \K} {f(\gamma)} = \sum_{\gamma \in \K}{\hat{f}(\gamma)}$ und sei $a\in\K$.
		Dann haben wir 
		\begin{align*}
			\sum_{\gamma \in \K} L_a {f(\gamma)} = \sum_{\gamma \in \K} {f(\gamma -a)} = \sum_{\gamma \in \K} {f(\gamma)}.
		\end{align*}
		Da der Standardcharakter $e$ allerdings trivial auf $\K$ wirkt, ist dies gleich
		\begin{align*}
			\sum_{\gamma \in \K} {\hat{f}(\gamma)} 	&= \sum_{\gamma \in \K} e(-a\gamma)\hat{f}(\gamma)
													= \sum_{\gamma \in \K} \Omega_{-a}\hat{f}(\gamma)
													= \sum_{\gamma \in \K} (L_a f)\widehat{\phantom{x}}(\gamma).
		\end{align*}
		F"ur den zweiten Operator erinnern wir uns daran, dass $\abs[\lambda]_\A = 1$ f"ur alle $\lambda \in \Kx$.
		Es folgt
		\begin{align*}
			\sum_{\gamma \in \K} M_\lambda {f(\gamma)} 	&= \sum_{\gamma \in \K} {f(\lambda\gamma)} 
														= \sum_{\gamma \in \K} \hat{f}(\lambda\gamma) \\
														&= \sum_{\gamma \in \K} \frac{1}{\abs[\lambda]_\A} \hat{f}(\lambda^{-1}\gamma) 
														= \sum_{\gamma \in \K}  (M_\lambda f)\widehat{\phantom{x}} (\gamma).
		\end{align*}	
		Damit haben wir auch schon den ersten Teil der Beweisidee verwirklicht.
		
		Der zweite Teil ist weniger Arbeit.
		F"ur $f=f_\infty \prod_{p<\infty}\ind_{\Zp}$ wissen wir aus den lokalen Berechnungen und Lemma \ref{satz:adeleidele:intA}, dass $\hat{f} = \hat{f}_\infty \prod_{p<\infty}\ind_{\Zp}$. 
		Zum anderen ist
		\begin{align*}
			\sum_{\gamma \in \K}{f(\gamma)} = \sum_{\gamma \in \K} f_\infty(\gamma) \prod_{p<\infty}\ind_{\Zp}(\gamma) =\sum_{\gamma \in \Z} f_\infty(\gamma),
		\end{align*}
		denn das Produkt "uber alle endlichen Stellen verschwindet genau dann nicht, wenn $\gamma$ eine ganze Zahl ist.
		Wir m"ussen also nur $\sum_{\gamma \in \Z} f_\infty(\gamma) = \sum_{\gamma \in \Z} \hat{f}_\infty(\gamma)$ zeigen.
		Aber das sagt uns gerade die klassische Poisson Summenformel aus Satz \ref{satz:einleitung:poisson}.
	\end{proof}
	Ausger"ustet mit dieser Summenformel k"onnen wir jetzt Tates Variante der Thetaformel beweisen.
	\begin{satz}[Riemann-Roch]
	\label{satz:global:riemannroch}
		Sei $x \in \I$ ein Idel und $f\in S(\A)$. Dann
		\begin{align*}
			\sum_{\gamma \in \K} {f(\gamma x)} = \frac{1}{|x|_{\A}}\sum_{\gamma \in \K} {\hat{f}(\gamma x^{-1})}.
		\end{align*}
		Insbesondere konvergieren die Summen absolut.
	\end{satz}
	\begin{proof}
		Sei $x \in \I$ beliebig aber fest. 
		F"ur beliebige $y \in \A$ definieren wir eine Funktion $h(y)\coloneqq f(yx)$. Diese ist wieder in $S(\A)$, erf"ullt damit die Poisson Summenformel
		\begin{align*}
			\sum_{\gamma \in \K}h(\gamma) = \sum_{\gamma \in \K} \hat{h}(\gamma)
		\end{align*}
		und beide Summen konvergieren absolut.
		Nach Korollar \ref{kor:tateproof:fourierids} gilt aber
		%Berechnen wir allerdings die Fouriertransformation von $h$ erhalten wir mit einer Translation um $x^{-1}$
		\begin{align*}
			\hat{h}(\gamma) 
				%&= \int_{\A}h(y)\Psi(\gamma y)dy \\
				%&= \int_{\A}f(yx)\Psi(\gamma y)dy \\
				%&= \frac{1}{|x|_{\A}} \int_{\A}f(y)\Psi(\gamma y x^{-1})dy \\
				= \frac{1}{|x|_{\A}} \hat{f}(\gamma x^{-1}).
		\end{align*}
		Damit sind wir auch schon fertig.
	\end{proof}
	F"ur eine Interpretation dieses Resultats, die einen Bezug zum namensgebenden Satz von Riemann-Roch in der algebraischen Geometrie herstellt, verweisen wir auf \textcite{rama} Kapitel 7.2.
	
\subsection{Die globale Funktionalgleichung}
%Hauptsatz gegliedert wie Riemann beweis
%Zeta funktion und idele class char einfuehren
%%%
%%%	Fubini und das Volumen von des Fundamentalbereichs F
%%%
	In Satz \ref{satz:adeleidele:ideleiso} haben wir gesehen, dass  sich die Idele $\I$  als direktes Produkt $\I^1\times\R_+^\times$ schreiben lassen.
	Um nun auf $\I^1$ ein (multiplikatives) Haar-Maß $\dxs[b]$ zu fixieren, nehmen wir auf $\R_+^\times$ das Maß $\frac{\dx[t]}{t}$ und verlangen $\dxx = \dxs[b] \times \frac{\dx[t]}{t}$.
	F"ur die Berechnungen haben wir dann ganz im Sinne von Fubini
	\begin{align*}
		\int_\I f(x) \dxx = \int_0^\infty \left[\int_{\I^1} f(t\cdot b) \dxs[b]\right]  \frac{\dx[t]}{t} =  \int_{\I^1} \left[\int_0^\infty f(t\cdot b) \frac{\dx[t]}{t}\right] \dxs[b].
	\end{align*}
	Die Notation $t\cdot b$ ist dabei als Multiplikation $\iota_\infty(t) \cdot b$ zu verstehen.
	Es folgt sofort, dass $\abs[t\cdot b]_\A = \abs[t]_\infty$ gilt. 
	Weiter haben wir in Satz \ref{satz:adeleidele:ideleiso} auch den kompakten Fundamentalbereich $F=\{1\}\times \prod_{p<\infty}\Zpx$ der Wirkung von $\Kx$ auf $\I^1$ kennengelernt.
	Das Volumen von $F$ bez"uglich $\dxs[b]$ wird f"ur unsere Berechnung eine Rolle spielen.
	Berechnen wir es schnell.
	Mit der Normierung des Maßes $\dxx$ haben wir
	\begin{align*}
		1 = \Vol\left((1,e)\times \prod_{p<\infty}\Zpx, \dxx\right) = \Vol\left(F, \dxs[b]\right) \cdot \Vol\left( (1,e), \frac{\dx[t]}{t}\right) = \Vol(F, \dxs[b]) 
	\end{align*}
	
	Bevor wir jetzt die die globalen Zeta-Funktionen einf"uhren ben"otigen wir noch geeignete multiplikative Charaktere.
	Es wird sich als sinnvoll erweisen die \emph{Idelklassencharaktere} zu betrachten.
	Dies sind, wie der Name tr"ugerisch verschweigt, Quasi-Charaktere auf $\I$, die trivial auf die Untergruppe $\Kx$ wirken\footnote{Der Name erkl"art sich aus der Tatsache, dass sie gerade Quasi-Charaktere auf der sogenannten \emph{Idelklassengruppe $\I/\Kx$} sind (vgl. \cite{neukirch} S.375)}.
	Aus dieser Definition wird klar, dass die Idelkassencharaktere $\chi$ das Vorzeichen nicht beachten.
	Es gilt n"amlich $\chi(-x) = \chi(-1) \chi(x) = \chi(x)$ f"ur jedes Idel $x \in I$.
	Wir k"onnen aber noch mehr "uber sie aussagen.
	\begin{satz}
		Jeder Idelklassencharakter $\chi$ hat die Form 
		\begin{align*}
			\chi(x) = \mu(\tilde{x}) \abs[x]_\A^s,
		\end{align*}
		wobei $\mu$ ein Charakter auf $\I^1$ und $\tilde{\cdot}:\I\to \I^1$ der stetige Homomorphismus nach Satz \ref{satz:adeleidele:ideleiso} (iv) ist.
	\end{satz}
	\begin{proof}
		Wir schreiben wie oben $x\in \I$ als eindeutiges Produkt $t\cdot b$ mit $t=\abs[x]_\A\in \R^\times_+$ und $b\in \I^1$.
		Dies ergibt die Zerlegung $\chi(x) = \chi_\infty(t) \cdot \chi(b)$.
		Da $\chi$ trivial auf $\Kx$ wirkt, induziert er einen Charakter auf der kompakten Gruppe $\I^1/\Kx$, den wir wieder mit $\chi$ bezeichnen.
		%$\chi(\I^1/\Kx)$ liegt als kompakte Untergruppe von $\Komplex^\times$ in $S^1$.
		Folglich ist (mit etwas Missbrauch der Notation) $\mu\coloneqq \chi|_{\I^1} = \chi|_{\I^1/\Kx}$ nach Lemma \ref{Lemma:trivialerCharAufKompakt} ein Charakter auf $\I^1$.
		Weiter ist nach Lemma \ref{lemma:lokal:unverzweigterChar} $\chi_\infty(t) = \abs[t]_\infty^s$ und wegen $\abs[t]_\infty = \abs[t]_\A$ folgt die Behauptung.
	\end{proof}
	Auch hier ist in dieser Darstellung im Allgemeinen nur der \emph{Exponenten} $\sigma=\Re(s)$ des Quasi-Charakters $\chi = \mu\abs_\A^s$ eindeutig.
	Ein Idelklassencharakter soll nun \emph{unverzweigt} heißen, wenn er trivial auf $\I^1$ wirkt. 
	\begin{korollar}\label{kor:tateproof:iccOnF}
		Jeder Idelklassencharakter $\chi$ ist genau dann unverzweigt, wenn er  trivial auf $F$ ist.
	\end{korollar}
	\begin{proof}
		Das folgt sofort aus $\I^1/\Kx \cong F$.
	\end{proof}
	Nach obigen Satz h"angt die Verzweigtheit also nur von dem Charakter $\mu$ ab.
	\begin{lemma}\label{lemma:tateproof:unverzweigterChar}
		Jeder unverzweigte Idelklassencharakter $\chi$ hat die Form $\chi = \abs_\A^s$.
	\end{lemma}
	\begin{proof}[Beweisidee]
		Daf"ur m"ussen wir im wesentlichen den Beweis von Lemma \ref{lemma:lokal:unverzweigterChar} kopieren
	\end{proof}
	
	Die Definition der Zeta-Funktionen auf $\A$ kann jetzt fast eins zu eins aus dem Lokalen "ubernommen werden. 
	\begin{defi}\label{def:tateproof:zeta}
		Sei $f\in \Sw(\A)$ eine Schwartz-Bruhat Funktion und $\chi=\mu\abs_p^s$ Idelklassencharakter.
		Die \emph{globale Zeta-Funktion} von $f$ und $\chi$ ist definiert als das Integral
		\begin{align*}
			Z(f, \chi) = \int_{\I} f(x) \chi(x) \dxx.
		\end{align*}
		Wir schreiben wieder $Z(f, \mu, s)$ f"ur $Z(f, \mu\abs_p^s)$.
	\end{defi}
	
	%Nun haben wir alles zusammen f"ur den großen Beweis:
	%\begin{itemize}
		%\item Integration auf den Adelen und Idelen
		%\item Fouriertransformation
		%\item Riemann-Roch als Ersatz f"ur die Theta-Transformationsformel
		%\item Zeta-Funktionen als Analogon zur Mellin-Transformation
	%\end{itemize}
	%
	Damit haben wir nun alles zusammen was wir brauchen um Riemanns Beweis in die Welt der Adele zu "ubertragen.
	%Zusammen mit dem, was wir aus unseren ersten Trocken"ubungen mit den lokalen Funktiongleichungen gelernt haben, werden wir jetzt eine globale Funktionalgleichung etablieren indem wir die Beweisideen aus Riemanns Beweis "ubertragen.
%---------------------------------%
% Die Globale Funktionalgleichung %
%---------------------------------%
	\begin{satz}[Globale Funktionalgleichung]
		Sei $\chi=\mu\abs_\A^s$ ein Idelklassencharakter und sei $f \in S(\A)$ eine Schwartz-Bruhat Funktion. 
		Dann konvergiert die globale Zeta-Funktion $Z(f,\mu,s)$  f"ur $\Re(s) > 1$ absolut und gleichmäßig auf kompakten Teilmengen und definiert dort eine holomorphe Funktion, die zu einer meromorphen Funktion auf ganz $\Komplex$ fortgesetzt werden kann, welche die \emph{globale Funktionalgleichung}
		\begin{align*}
			Z(f,\chi) = Z(\hat{f}, \check{\chi})
		\end{align*}
		bzw.
		\begin{align*}
			Z(f,\mu,s) = Z(\hat{f}, 1/\mu, 1-s)
		\end{align*}
		erf"ullt.
		Diese Funktion ist "uberall holomorph, außer wenn $\mu = \abs_\A^{-i\tau}$, $\tau \in \R$. 
		Dann besitzt sie einen einfachen Pol bei $s= i\tau$ mit Residuum $-f(0)$ und einen einfachen Pol bei $s=1+i\tau$ mit Residuum $\hat{f}(0)$.
	\end{satz}
	\begin{proof}
		Starten wir mit der Konvergenz. 
		Wie bisher gen"ugt es faktorisierbare Schwartz-Bruhat Funktionen $f$ zu betrachten.
		%F"ur alle endlichen Stellen $p$ ist dann $f_p$ die charakteristische Funktion von $p^k\Z_p$ mit $k\in\Z$, wobei $k=0$ f"ur fast alle Stellen.
		Ziel ist es dann zu zeigen, dass das Integral 
		\begin{align}\label{eq:zetaproduct}
			\int_\I \abs[f(x)\chi(x)] d^\times x = \int_\I \abs[f(x)] \cdot \abs[x]_\A^\sigma d^\times x = \prod_{p\leq\infty} \int_{\K_p^\times} \abs[f_p(x_p)] \cdot \abs[x_p]_p^\sigma d^\times x_p
		\end{align}
		endlich ist.
		Dazu teilen wir das Produkt auf.
		Es gibt eine Primzahl $p_0$, so dass $f_p$ f"ur alle $p_0\leq p <\infty$ gleich der charakteristischen Funktion $\ind_\Zp$ ist.
		Wir k"onnen Gleichung \eqref{eq:zetaproduct} also schreiben als 
		\begin{align*}
			\int_{\K_\infty^\times} \abs[f_\infty(x_\infty)] \cdot \abs[x_\infty]_\infty^\sigma d^\times x_\infty \cdot \prod_{p < p_0} \int_{\K_p^\times} \abs[f_p(x_p)] \cdot \abs[x_p]_p^\sigma d^\times x_p \cdot \prod_{p_0\leq p < \infty} \int_{\Z_p \setminus \{0\}} \abs[x_p]_p^\sigma d^\times x_p .
		\end{align*}
		Das Integral setzt sich also als unendliches Produkt lokaler Zeta-Funktionen $Z_p(\abs[f_p], \abs[x]_p^\sigma)$ zusammen.
		Der erste Faktor und das endliche Produkt in der Mitte konvergieren damit nach dem Satz "uber die lokalen Funktionalgleichungen f"ur $\sigma >0$, also sicherlich auch f"ur $\sigma >1$. 
		Die Konvergenz der Zeta-Funktion h"angt also nur noch von der des unendlichen Produkts 
		\begin{align*}
			\prod_{p_0\leq p < \infty} \int_{\Z_p \setminus \{0\}} \abs[x_p]_p^\sigma d^\times x_p
		\end{align*}
		ab. 
		Der aufmerksame Leser wird sich hier vielleicht an die genauen Berechnungen der lokalen Funktionalgleichungen erinnern und einen Bezug zur Riemannschen Zeta-Funktion erahnen. 
		Um die sp"atere "Uberraschung aber nicht zu verderben, werden wir nur anmerken, dass dies ein Teilprodukt einer bekannten Produktformel ist, welche gleichm"aßig auf jedem Kompaktum der gegeben Halbebene konvergiert.
		
		%%%Ende Konvergenz
		%%%
		%%%Anfang Fortsetzung
		
		Weiter zur meromorphen Fortsetzung auf ganz $\Komplex$. 
		Aufgrund absoluter Konvergenz auf der Halbebene $\Re(s)>1$ haben wir
		\begin{align*}
			Z(f,\chi) 	&= \int_{\I} f(x) \chi(x) d^\times x \\
							&= \iint_{\R^\times_+ \times \I^1} f(t\cdot b) \chi(t\cdot b) (d^\times t\times \dxs[b])\\
							&= \int_0^\infty \left[\int_{\I^1} (f(t\cdot b) \chi(t\cdot b) \dxs[b]\right] \frac{dt}{t}
		\end{align*}
		Um uns etwas Schreibarbeit zu sparen, definieren wir das innere Integral als
		\begin{align*}
			Z_t(f,\chi) \coloneqq  \int_{\I^1} f(t\cdot b) \chi(t\cdot b) \dxs[b].
		\end{align*}
		Wie in Riemanns Beweis wollen wir nun eine auf ganz $\Komplex$ g"ultige meromorphe Darstellung der Zeta-Funktion finden.
		Dazu teilen wir, ganz analog, das Integral in die zwei Summanden
		\begin{align}\label{eq:tateproof:intSplit}
			Z(f,\chi) = \int_0^1 Z_t(f,\chi) \frac{dt}{t} 
							+ \int_1^\infty Z_t(f,\chi) \frac{dt}{t}.
		\end{align}
		
		Das Integral $\int_1^\infty$ macht uns keine Probleme.
		Wir haben festgestellt, dass
		\begin{align*}
			\int_1^\infty \abs[Z_t(f,\chi)] \frac{dt}{t} 
				= \int_{\{x\in \I: \abs[x]_\A\geq 1\}} \abs[f(x)] \abs[x]_\A^\sigma \dxx,
		\end{align*}
		f"ur $\sigma>1$ konvergiert.
		Da $\abs[x]_\A \geq 1$ verbessert sich das Konvergenzverhalten des Integranden umso kleiner $\sigma$ ist und es folgt die Konvergenz auf ganz $\Komplex$.
		
		Widmen wir uns also dem Integral $\int_0^1$ zu.
		Zuerst "uberlegen wir uns, dass wir $\I^1$ nach Satz \ref{satz:adeleidele:ideleiso} als disjunkte Vereinigung $\bigcup_{a \in \Kx} aF$ darstellen konnten, 
			wobei $F= \{1\} \times \prod_{p<\infty}\Z_p$ der Fundamentalbereich der Gruppenwirkung war.
		Kombiniert mit der Translationsinvarianz von $\dxs[b]$ und der Tatsache, dass $\chi$ als Ideleklassencharakter trivial auf $\Kx$ wirkt, ergibt sich
		\begin{align*}
			Z_t(f,\chi)	&= \int_{\I} (f(t\cdot b) \chi(t\cdot b) \dxs[b] 
							%= \int_{\bigsqcup_{q \in \Kx} qF} (f(t\cdot b) \chi(t\cdot b) \dxs[b]
							= \sum_{a \in \Kx} \int_{aF} (f(t\cdot b) \chi(t\cdot b) \dxs[b]\\
							&= \sum_{a \in \Kx} \int_{F} (f(at\cdot b) \chi(t\cdot b) \dxs[b]
							= \int_{F} \left(\sum_{a \in \Kx}  f(at\cdot b)\right) \chi(t\cdot b) \dxs[b],
		\end{align*}
		wobei wir auf $\sigma>1$ wegen absoluter Konvergenz Summe und Integral vertauschen d"urfen.
		Die Summe "uber $a\in\Kx$ erinnert dabei den Ausdruck $\Theta(x)-1$ in Riemanns Beweis, welcher gerade $\sum_{n\in\Z\setminus\{0\}} \exp(-\pi x n^2)$ entspricht. 
		Wir sind also dazu verleitet hier den Satz von Riemann-Roch anzuwenden, wof"ur wir allerdings eine Summe "uber ganz $\K$ ben"otigen. 
		Dieses Problem l"asst sich leicht beheben.
		\begin{lemma}
			\begin{align*}
				Z_t(f,\chi) = Z_{t^{-1}}(\hat{f},\check{\chi}) + \hat{f}(0) \int_F \check{\chi} (x/t)\dxs[b] - f(0)\int_F \chi(tx)\dxs[b].
			\end{align*}
		\end{lemma}
		\begin{proof}
			Die Idee ist klar. 
			Wir addieren $f(0)\int_F \chi(tx)\dxs[b]$ auf $Z_t(f,\chi)$, erhalten
			\begin{align*}
				Z_t(f,\chi) + f(0)\int_F \chi(tb)\dxs[b]= \int_{F} \left(\sum_{a \in \K}  (f(at\cdot b)\right) \chi(t\cdot b) \dxs[b]
			\end{align*}
			und k"onnen jetzt unsere Version von Riemann-Roch anwenden:
			\begin{align*}
				\int_{F} \left(\sum_{a \in \K}  f(at\cdot b)\right) \chi(t\cdot b) \dxs[b] 
					&= \int_{F} \left(\sum_{a \in \K}  \hat{f}(a t^{-1} b^{-1}) \right) \frac{\chi(t\cdot b)}{\abs[tb]_{\A}} \dxs[b]\\
					&= \int_{F} \left(\sum_{a \in \K}  \hat{f}(a t^{-1} b) \right) \abs[t^{-1}b]_{\A} \chi(t\cdot b) \dxs[b]\\
					&= \int_{F} \left(\sum_{a \in \K}  \hat{f}(a t^{-1} b) \right) \check{\chi}(b/t)\dxs[b] + \hat{f}(0) \int_F \check{\chi} (x/t)\dxs[b]\\
					&= Z_{t^{-1}}(\hat{f},\check{\chi}) + \hat{f}(0) \int_F \check{\chi} (x/t)\dxs[b]
			\end{align*}
			wobei wir im zweiten Schritt den Variablenwechsel $b\mapsto b^{-1}$ und im dritten Schritt $\chi(x^{-1}) = \chi(x)^{-1}$ ausgenutzt haben.
		\end{proof}
		Mit diesem Lemma l"asst sich das Integral $\int_0^1$ also umformen zu
		\begin{align*}
			\int_0^1 Z_t(f,\chi) \frac{dt}{t} 
				= \int_0^1 \left( Z_{t^{-1}}(\hat{f},\check{\chi}) 
					+ \hat{f}(0) \check{\chi}(t) ^{-1}\int_F \check{\chi} (x)\dxs[b] 
					- f(0)\chi(t)\int_F \chi(x)\dxs[b] \right)\frac{dt}{t}
		\end{align*}
		Ein Variablenwechsel $t\mapsto t^{-1}$ im ersten Summanden ergibt dann
		\begin{align*}
			\int_0^1  Z_{t^{-1}}(\hat{f},\check{\chi}) \frac{dt}{t} = \int_1^\infty  Z_{t}(\hat{f},\check{\chi}) \frac{dt}{t}
		\end{align*}
		was nach dem gleichen Argument wie bei $\int_1^\infty$ auf ganz $\Komplex$ konvergiert. 
		Ganz nach Riemanns Vorbild haben wir nun zwei Integrale auf dem Bereich von $1$ bis $\infty$.
		Es bleibt noch der Term 
		\begin{align*}
			E(f,\chi)\coloneqq  \int_0^1  \hat{f}(0) \check{\chi}(t)^{-1} \left(\int_F \check{\chi} (x)\dxs[b]\right) \frac{dt}{t}
					- \int_0^1 f(0)\chi(t)\left(\int_F \chi(x)\dxs[b] \right)\frac{dt}{t}.
		\end{align*}
		Dieser entspricht wiederum den beiden Termen $\frac{1}{s}$ und $\frac{1}{1-s}$ aus Riemanns Beweis.
		Ist der Idelklassencharakter $\chi$ nicht unverzweigt, so wirkt er nach Korollar \ref{kor:tateproof:iccOnF} auch nicht trivial auf dem Kompaktum $F$.
		Folglich verschwinden beide Integrale und $E(f,\chi) = 0$ ist offensichtlich eine ganze Funktion.
		%\begin{align*}
				%Z(f,\chi) =  \int_1^\infty Z_t(\hat{f}, \check{\chi}) \frac{dt}{t} 
							%+ \int_1^\infty Z_t(f,\chi) \frac{dt}{t}.
		%\end{align*}
		Ist $\chi = \mu \abs_\A^s$ dagegen unverzweigt, so haben wir in Lemma \ref{lemma:tateproof:unverzweigterChar} gezeigt, dass $\chi = \abs_\A^{s'}$, wobei $s'=s-i\tau$ f"ur ein $\tau \in \R$. 
		Wegen $\chi(t) = \abs[t]_\A^{s'}$ ist damit
		\begin{align*}
			E(f,\chi) 	&= \text{Vol}(F,\dxs[b]) \hat{f}(0) \int_0^1  t^{s'-1} \frac{dt}{t}  -\text{Vol}(F,\dxs[b]) f(0) \int_0^1  t^{s'}\frac{dt}{t}\\
						&= \frac{\hat{f}(0)}{s' - 1} - \frac{f(0)}{s'} = \frac{\hat{f}(0)}{s - (i \tau + 1)} - \frac{f(0)}{s-i\tau}
		\end{align*}
		und wir sehen, dass $E$ in diesem Fall eine rationale Funktion in $s$ ist. 
		Eingesetzt in Gleichung \eqref{eq:tateproof:intSplit} erh"alt man
		\begin{align*}
			Z(f,\chi) =  \int_1^\infty Z_t(\hat{f}, \check{\chi}) \frac{dt}{t} 
							+ \int_1^\infty Z_t(f,\chi) \frac{dt}{t} + E(f,\chi),
		\end{align*}
		einen auf ganz $\Komplex$ g"ultigen Ausdruck der Zeta-Funktion und damit eine meromorphe Fortsetzung  der Funktion. 
		Diese ist f"ur $\mu \not=\abs_\A^{-i\tau}$ sogar ganz und hat im Fall $\mu =\abs_\A^{-i\tau}$ nur zwei einfache Pole bei $s=i\tau$ und $s=1+i\tau$ mit Residuen $-f(0)$ bzw. $\hat{f}(0)$.
		
		%%%
		%%%Funktionalgleichung
		%%%
		Zum Schluss kommen wir noch zur Funktionalgleichung. 
		Es gilt zun"achst $E(f,\chi) = E(\hat{f},\check{\chi})$.
		Im unverzweigten Fall folgt dies sofort, und im verzweigten Fall $\chi=\abs_\A^{s'}$ haben wir 
		\begin{align*}
			E(f,\chi) = \frac{\hat{f}(0)}{s' - 1} - \frac{f(0)}{s'} = - \frac{\hat{f}(0)}{1 - s'} + \frac{\hat{\hat{f}}(0)}{1- (1-s')} = E(\hat{f},\check{\chi}),
		\end{align*}
		wobei wir hier die Umkehrformel
		$
			\hat{\hat{f}}(x) = f(-x) 
		$
		nach Satz \ref{satz:tateproof:umkehrformel} benutzt haben. Mit $\check{\check{\chi}} = \chi$folgt
		\begin{align*}
			Z(\hat{f},\check{\chi}) 
				&=  \int_1^\infty Z_t(\hat{\hat{f}}, \check{\check{\chi}}) \frac{dt}{t} 
					+ \int_1^\infty Z_t(\hat{f},\check{\chi}) \frac{dt}{t} + E(\hat{f},\check{\chi})\\
				&= \int_1^\infty \int_{\I^1}f(-t \cdot b)\chi(t \cdot b)\dxs[b]  \frac{dt}{t} 
					+ \int_1^\infty \int_{\I^1}\hat{f}(t \cdot b)\check{\chi}(t \cdot b)\dxs[b]  \frac{dt}{t} +E(f,\chi)\\
				&= \int_1^\infty \int_{\I^1}f(t \cdot b)\chi(t \cdot b)\dxs[b]  \frac{dt}{t} 
					+ \int_1^\infty \int_{\I^1}\hat{f}(t \cdot b)\check{\chi}(t \cdot b)\dxs[b]  \frac{dt}{t} +E(f,\chi) = Z(f,\chi)
		\end{align*}
		wobei wir im letzten Schritt im ersten Integral die Translationsinvarianz der Haar-Maßes $\dxs[b]$ und die Eigenschaft des Idelklassencharakters $\chi(-tx) = \chi(tx)$ ausgenutzt haben.
	\end{proof}
	
\subsection{Globale Berechnung: Die Riemannsche Zeta-Funktion}
	Wir wollen jetzt zeigen, dass unsere Berechnungen im letzten Abschnitt nicht ohne Inhalt sind.
	Immerhin besteht immer noch die Gefahr, dass wir gerade $0 = 0$ gezeigt haben.
	Betrachten wir also die adelische Schwartz-Bruhat Funktion
	\begin{align*}
		f(x) = \exp(-2\pi i x_\infty) \prod_{p<\infty} \ind_{\Zp}(x_p).
	\end{align*}
	Dank unserer Berechnung der lokalen Faktoren wissen wir, dass
	\begin{align*}
		\hat{f}(x) = \exp(-2\pi i x_\infty) \prod_{p<\infty} \ind_{\Zp}(x_p),
	\end{align*}
	also $f$ ihre eigene Fouriertransformierte ist.
	
	%Auf zu den Zeta-Funktionen!
	F"ur jede Stelle $p\leq \infty$ ist der Faktor $f_p$ die Funktion, die wir bei den Berechnungen der lokalen Zeta-Funktionen $Z_p(f_p,1,\abs_p^s)$ im unverzweigten Fall verwendet haben.
	Bleiben wir also auch hier unverzweigt und betrachten den Charakter $\abs_\A^s$.
	Damit vereinfacht sich die Berechnung der globalen Zeta-Funktion zu
	\begin{align*}
		Z(f, 1, \abs_\A^s) 	= \int_\I f(x) \abs[x]_\A^s \dxx 
							= \prod_{p\leq \infty} Z(f_p, 1 , \abs_p^s)
							= \pi^{-\frac{s}{2}} \Gamma\left(\frac{s}{2}\right) \prod_{p<\infty} \frac{1}{1-p^{-s}}.
	\end{align*}
	Das Produkt am Ende sollte uns aber bekannt vorkommen.
	Es ist genau die Darstellung der Riemannschen Zeta-Funktion als Euler-Produkt.
	Damit ist also
	\begin{align*}
		Z(f, 1, \abs_\A^s) = \pi^{-\frac{s}{2}} \Gamma\left(\frac{s}{2}\right) \zeta(s) = \Xi(s)
	\end{align*}
	genau Riemanns bequemer Ausdruck $\Xi$.
	Analog ist
	\begin{align*}
		Z(\hat{f}, 1, \abs_\A^{1-s})  	%= \pi^{-\frac{1-s}{2}} \Gamma\left(\frac{1-s}{2}\right) \prod_{p<\infty} \frac{1}{1-p^{s-1}} 
										= \pi^{-\frac{1-s}{2}} \Gamma\left(\frac{1-s}{2}\right) \zeta(1-s) = \Xi(1-s)
	\end{align*}
	und die globale Funktionalgleichung liefert uns damit gerade
	\begin{align*}
		\Xi(s) = Z(f, 1, \abs_\A^s) = Z(\hat{f}, 1, \abs_\A^{1-s}) =\Xi(1-s),
	\end{align*}
	die Funktionalgleichung der Riemannschen Zeta-Funktion.
	Nun ist aber auch sofort klar woher der Faktor $\pi^{-\frac{s}{2}} \Gamma\left(\frac{s}{2}\right)$ r"uhrt.
	Er ist gerade der Anteil, den die archimedische Stelle $p=\infty$ bei den Berechnungen liefert.
	
	B"ose Zungen m"ogen jetzt behaupten, dass wir durch die Wahl unserer Funktion $f$, vor allem durch die Wahl des Faktors $f_\infty (x_\infty) =  exp(-2\pi i x_\infty)$,
	die Form dieser Gleichung letztendlich beeinflusst haben um gerade auf das klassische Ergebnis zu kommen.
	Das stimmt, denn immerhin h"atte ein anderer Faktor $g_\infty$ eine andere Funktionalgleichung ergeben.
	Diese beiden Funktionalgleichungen unterscheiden sich dann aber nur um einen meromorphen Faktor, den wir dank
	\begin{align*}
		  Z_\infty(g_\infty, \abs_\infty^s)Z_\infty(\hat{f}_\infty,\abs_\infty^{s-1}) =Z_\infty(f_\infty, \abs_\infty^s) Z_\infty(\hat{g}_\infty, \abs_\infty^{s-1}),
	\end{align*}
	mit
	\begin{align*}
		 \frac{ Z_\infty(g_\infty, \abs_\infty^s)}{Z_\infty(f_\infty, \abs_\infty^s)} = \frac{Z_\infty(\hat{g}_\infty, \abs_\infty^{s-1})} {Z_\infty(\hat{f}_\infty,\abs_\infty^{s-1})}
	\end{align*}
	angeben k"onnen.
	Unsere Wahl war also nur durch die besonders einfache Form der Faktoren und den damit verbundenen Berechnungen beeinflusst.
	