\section{Tates Beweis der Funktionalgleichung}\label{sec:tateproof}
	Wir kommen nun zum Kernst"uck dieser Arbeit.
\subsection{Globale Fourieranalysis}
	Um auch bei unserer Definition der Fouriertransformation im Globalen ein gutes Gewissen zu haben geben wir den Adelen einen Standardcharakter.
	Dieser wird definiert durch
	\begin{align*}
		e(x) = \prod_{p\leq \infty} e_p(x_p),
	\end{align*}
	was, wie wir bereits in Lemma \ref{lemma:rdp:char} gesehen haben, wohldefiniert ist.
	Der Standardcharakter verh"alt sich besonders sch"on auf $\K$. 
	Wir haben n"amlich folgenden
	\begin{satz}\label{satz:global:stdcharTrivialAufK}
		Der Standardcharakter $e: \A \to S^1$ wirkt trivial auf $\K$.
	\end{satz}
	Bevor wir das aber beweisen k"onnen brauchen wir noch ein kleines Lemma aus der Zahlentheorie.
	\begin{lemma}
		Jede rationale Zahl kann geschrieben werden als endliche Summe von Zahlen der Form $\pm \frac{1}{p^m}$.
	\end{lemma}
	\begin{proof}
		Jedes $r = \pm a/b$ mit nat"urlichen Zahlen $a$ und $b$ kann zun"achstmal als Summe von $a$ Kopien von $\sgn{r}/b$ geschrieben werden.
		Es reicht also rationale Zahlen der Form $1/b$ zu betrachten. 
		Sei $b=\prod_{k=1}^{n} p_k^{m_k}$ die Primfaktorzerlegung von $b$.
		Wir haben eine Bezoutdarstellung $1 = u p_1^{m_1} + v p_2^{m_2}\dots p_n^{m_n}$ mit ganzen Zahlen $u$ und $v$.
		Es zeigt sich
		\begin{align*}
			\frac{1}{b} = \frac{u p_1^{m_1} + v p_2^{m_2}\dots p^{m_n}}{p_1^{m_1}\dots p_n^{m_n}} =  \frac{u}{p_2^{m_2}\dots p_n^{m_n}} + \frac{v}{p_1^{m_1}}
		\end{align*}
		Induktiv erhalten wir damit ganze Zahlen $u_1,\dots,u_n$, so dass
		\begin{align*}
			\frac{1}{b} = \frac{u_1}{p_1^{m_1}} + \dots + \frac{u_n}{p_n^{m_n}}
		\end{align*}
		Mit dem gleichen Argument vom Anfang des Beweises ist jeder Summand wieder eine endliche Summe von $\abs[u_k]$ Kopien von $1/p_k^{m_k}$ und es folgt das Lemma
	\end{proof}
	Nun kommen wir zum eigentlichen Beweis des Satzes.
	\begin{proof}[Beweis von Satz \ref{satz:global:stdcharTrivialAufK}]
		Da es sich um einen additiven Charakter handelt, erlaubt uns das obige Lemma uns auf rationale Zahlen der Form $1/p^m$ zu beschr"anken.
		F"ur jede endliche Stelle $q\not=p$ ist aber $1/p^m \in \Z_q$, also $e_q(1/p^m) =1$.
		Es folgt
		\begin{align*}
			e\left( \frac{1}{p^m} \right) 	= \prod_{q\leq \infty} e_q\left( \frac{1}{p^m} \right) 
											&= e_\infty \left( \frac{1}{p^m} \right) e_p\left( \frac{1}{p^m} \right)\\
											&= \exp(-2\pi i p^{-m}) \exp(2\pi i p^{-m}) = 1
		\end{align*}
	\end{proof}
	
	Schlag auf Schlag geht es weiter. 
	Als n"achstes wollen wir sicher stellen, dass die Definition eines Standardcharakters auf den Adelen "uberhaupt Sinn macht.
	Dazu zeigen wir das Ergebnis, was wir mit Satz \ref{satz:lokal:stdchar} im Lokalen bewiesen haben, nun auch im Globalen.
	\begin{satz}
		Ist $\psi: \A \to S^1$ ein Charakter, so gibt es ein eindeutig bestimmtes Adel $a\in\A$ mit $\psi(x) = e(ax)$.
		Weiter ist wirkt $e$ trivial auf $\K$.
	\end{satz}
	\begin{proof}
		Den Großteil der Arbeit haben wir bereits in Lemma \ref{lemma:rdp:char} gemacht. Demnach ist $\psi(x) = \prod_{p\leq\infty} \psi_p (x_p)$, wobei $\psi_p: \Kp \to S^1$ ein lokaler Charakter ist und fast alle $\psi_p$ wirken trivial auf $\Zp$.
		Wir haben bereits gesehen, dass $\psi_p (x_p) = e_p(a_p x_p)$ f"ur eindeutige $a_p \in \Kp$.
		Nach Korollar \ref{kor:lokal:charTrivialZp} liegen zudem fast alle $a_p$ in $\Zp$
		Folglich ist $a = (a_p)$ ein eindeutig bestimmtes Adel und es gilt
		\begin{align*}
			\psi(x) = \prod_{p\leq\infty} \psi_p (x_p) = \prod_{p\leq\infty} e_p (a_px_p) = e(ax).
		\end{align*}
	\end{proof}
	Damit haben wir nun alles zusammen und kommen zur zentralen Definition dieses Abschnitts.
	
	\begin{defi}[Globale Fouriertransformation]
		Sei $f\in L^1(\A)$. Wir definieren dann die \emph{adelische Fouriertransformation} $\hat{f}: \A \to \Komplex$ von $f$ durch die Formel
	\begin{align*}
		\hat{f}(\xi) = \int_{\A} f(x)e(-\xi x)  \dx
	\end{align*}
	f"ur alle $\xi \in \A$.
	\end{defi}
	
	Lokal haben wir gesehen, dass die Schwartz-Bruhat Funktionen $\Sw(\Kp)$ sich besonders f"ur die Fourieranalysis geeignet haben. 
	Was ist das "aquivalent im Globalen?
	Wir definieren eine \emph{faktorisierbare Schwartz-Bruhat Funktion} $f=\prod_{p\leq \infty} f_p$ als ein Produkt lokaler Schwartz-Bruhat Funktionen $f_p$, wobei f"ur fast alle Stellen $p<\infty$ der Faktor die charakteristische Funktion von $\Zp$ ist.
	Mit unserer Charakterisierung der lokalen Schwartz-Bruhat Funktionen "uber den endlichen Stellen k"onnen die Faktorisierbaren auch schreiben als
	\begin{align}
		f(x) = f_\infty(x) \prod_{p<\infty} \ind_{a_p+p^{k_p}\Zp}
	\end{align}
	mit $a_p \in \Zp$ (oder alternativ: f"ur ein $a \in \A$) und $k_p =0$ f"ur fast alle Stellen.
	Die \emph{adelischen Schwartz-Bruhat Funktion} sind endliche Linearkombinationen dieser "uber $\Komplex$.
	Wir notieren sie wieder mit $\Sw(\A)$.
	Damit k"onnen wir wieder folgenden Satz beweisen.
	
	\begin{satz}
		F"ur jede Funktion $f\in\Sw(\A)$ ist die Fouriertransformation wohldefiniert und liegt wieder in $\Sw(\A)$.
		Insbesondere gilt die Umkehrformel.
	\end{satz}
	\begin{proof}
		Es reicht wieder nur faktorisierbare $f$ zu betrachten. 
		Diese haben Form $f = \prod_{p\leq \infty} f_p$ mit $f_p \in \Sw(\Kp)$ und fast "uberall $f_p = \ind_{\Zp}$.
		Mit unserem Wissen "uber die Integration auf den Adelen folgt
		\begin{align*}
			\hat{f} (\xi) 	&= \int_{\A} f(x)e(-\xi x)  \dx
							= \int_{\A} \prod_{p\leq \infty} f_p(x_p) e_p(-\xi_p x_p) \dx \\
							&= \prod_{p\leq \infty} \int_{\Kp}  f_p(x_p) e_p(-\xi_p x_p) \dx_p
							= \prod_{p\leq \infty} \hat{f}_p (\xi_p).
		\end{align*}
		Die Umkehrformel folgt damit direkt aus den lokalen Umkehrformeln.
	\end{proof}
	
	\begin{korollar}
		Sei $f \in S^1(\A)$.
		\begin{enumerate}[label=\emph{(\alph*)}]
			\item Ist $g(x)=f(x)e(ax)$ mit $a\in\A$, dann gilt $\hat{g}(x) = \hat{f}(x-a)$.
			\item Ist $g(x)=f(x-a)$ mit $a\in\A$, dann gilt $\hat{g}(x) = \hat{f}(x-a)e(-ax)$.
			\item Ist $g(x)=f(\lambda x)$ mit $\lambda \in \I$, dann gilt $\hat{g}(x) =\frac{1}{\abs[\lambda]_\A} \hat{f}(\frac{x}{\lambda})$.
		\end{enumerate}
	\end{korollar}
	
	
\subsection{Adelische Poisson-Summenformel und der Satz von Riemann-Roch}
%poisson done
%riemann-roch (einfach) done
%beweisskizze fuer beweis nach tate: anhang
	\begin{lemma}
	\label{lemma:global:sbf}
		Jede faktorisierbare adelische Schwartz-Bruhat Funktion $f$ hat die Form
		\begin{align*}
			f = f_\infty\prod_{p<\infty} \ind_{a+p^{k_p}\Zp},
		\end{align*}
		wobei $a\in\K$ und $k_p\in \Z$ f"ur fast alle Stellen $p$ verschwindet.
	\end{lemma}
	\begin{proof}
		Der Unterschied zur Definition der faktorisierbaren Schwartz-Bruhat Funktionen liegt in der Tatsache, dass der Faktor $a$ jetzt unabh"angig von der jeweiligen Stelle ist.
		F"ur fast alle $p$ ist $a\in \Zp$ und $k_p=0$, also $\ind_{a+p^k_p\Zp} = \ind_{\Zp}$ fast "uberall.
		Damit ist klar, dass Funktionen dieser Form in $\Sw(\A)$ liegen. 
		Sei nun $f = \prod_{p\leq\infty} f_p\in\Sw(\A)$ und $S$ die Menge der Stellen mit $k_p\not=0$.
		F"ur alle $p \in S$ ist der Faktor $f_p$ gleich $\ind_{a_p+p^k_p\Zp}$ f"ur ein $a_p \in \K$.
		Sind wir vorerst optimistisch und nehmen an, dass $k_p>0$ und die $a_p$ bereits ganze Zahlen sind.
		Ziel ist es jetzt eine ganze Zahl $a$ zu finden, die $a \equiv a_p \pmod{p^{k_p}}$ f"ur alle $p\in S$ erf"ullt.
		Das liefert uns aber gerade der chinesische Restsatz.
		Damit w"are $a_p+p^k_p\Zp = a+p^k_p\Zp$ und wir w"aren fertig.
		
		Bleiben wir optimistisch und versuchen die Funktionen geeignet umzuformen.
		Sei dazu $N$ der Hauptnenner der nun wieder rationalen Zahlen $a_p$ und sei $M=\prod_{p\in S} p^{\abs[k_p]+1}$.
		Dann ist $NM(a_p+p^k_p\Zp) = a_p'+p^{k_p'}\Zp$ mit $a_p'\in \Z$ und $k'>0$.
		Folglich finden wir ein $a' \in \Z$ mit $NM(a_p+p^{k_p}\Zp) = a'+p^{k_p'}\Zp$.
		Jetzt setzen wir alles zusammen durch
		\begin{align*}
			a_p+p^{k_p}\Zp &= (NM)^{-1}NM(a_p+p^{k_p}\Zp) 
			\\&= (NM)^{-1}(a'+p^{k_p'}\Zp)  = (NM)^{-1}a'+p^{k_p}\Zp
		\end{align*}
		f"ur alle $p\in S$.
		Setzen wir also $a\coloneqq (NM)^{-1}a'$ folgt die Behauptung.
	\end{proof}


	\begin{satz}[Poisson Summenformel]
	\label{satz:global:poisson}
		Sei $f \in S(\A)$. Dann konvergieren folgende Summen absolut und wir haben die Gleichung
		\begin{align}
			\sum_{\gamma \in \K} {f(\gamma)} = \sum_{\gamma \in \K}{\hat{f}(\gamma)}
		\end{align}
	\end{satz}
	Wir geben es zu. Der folgende Beweis kann nicht auf den allgemeineren Fall "ubertragen.
	\begin{proof}
		Es gen"ugt wieder die Aussage f"ur faktorisierbare Schwartz-Bruhat Funktionen $f \in S(\A)$ zu zeigen.
		Nach vorherigem Lemma \ref{lemma:global:sbf} haben diese gerade die Form
		\begin{align}
		\label{eq:schwartzform}
			f = f_\infty\prod_{p<\infty} \ind_{a+p^{k_p}\Zp}.
		\end{align}
		f"ur ein $a \in \K$ und $k_p = 0$ f"ur fast alle Stellen $p$.
		Mit $N\coloneqq \prod_{p<\infty} p^{-k_p} \in \K$ haben wir dann
		\begin{align*}
			\sum_{\gamma \in \K} {f(\gamma)} 	&= 	\sum_{\gamma \in \K} f(\gamma - a)
												= 	\sum_{\gamma \in \K} f(N(\gamma - a)) \\
												&= 	\sum_{\gamma \in \K} \left[ f_\infty (N(\gamma - a))  \prod_{p< \infty} \ind_{a+p^{k_p}\Zp}(N(\gamma - a)) \right] \\
												&= 	\sum_{\gamma \in \K} \left[ f_\infty (N(\gamma - a))  \prod_{p< \infty} \ind_{\Zp}(\gamma) \right]
												=	\sum_{\gamma \in \Z} f_\infty (N(\gamma - a))
		\end{align*}
		Die Summe am Ende konvergiert aber absolut, da $f_\infty \in \Sw(\R)$.
		
		Mit der Konvergenz aus dem Weg besch"aftigen wir uns nun mit der Summenformel.
		Dabei werden sich die obigen Umformungen als sehr hilfreich erweisen.
		Erinnern wir uns zun"achst an die Operatoren aus dem Beweis der lokalen Umkehrformeln in Satz \ref{satz:lokal:umkehrformel}.
		Dort haben wir die Notation $L_a f(x) = f(x-a)$ und $M_\lambda f(x) = f(\lambda x)$ eingef"uhrt.
		Definieren wir nun wieder $N= \prod_{p<\infty} p^{-k_p}$, so k"onnen wir die Funktion \ref{eq:schwartzform} umschreiben als
		\begin{align*}
			f = L_a M_{N}\left( \left(M_{1/N} L_{-a} f_\infty\right) \prod_{p<\infty}\ind_{\Zp}\right).
		\end{align*}
		Damit er"offnet sich nun folgende Beweisidee. 
		Wir zeigen zuerst, dass die obige Gleichung (falls sie denn tats"achlich gilt) stabil unter den Operatoren $L_a$ und $M_\lambda$ ist.
		Anschließend zeigen wir sie f"ur den einfacheren Fall $f=f_\infty \prod_{p<\infty}\ind_{\Zp}$.
		Beides kombiniert ergibt dann den Beweis.
		
		Gelte also $\sum_{\gamma \in \K} {f(\gamma)} = \sum_{\gamma \in \K}{\hat{f}(\gamma)}$ und sei $a\in\K$.
		Dann haben wir 
		\begin{align*}
			\sum_{\gamma \in \K} L_a {f(\gamma)} = \sum_{\gamma \in \K} {f(\gamma -a)} = \sum_{\gamma \in \K} {f(\gamma)}.
		\end{align*}
		Da unser additiver Charakter $e$ allerdings trivial auf $\K$ wirkt, ist dies gleich
		\begin{align*}
			\sum_{\gamma \in \K} {\hat{f}(\gamma)} 	&= \sum_{\gamma \in \K} e(-a\gamma)\hat{f}(\gamma)\\
													&= \sum_{\gamma \in \K} \Omega_{-a}\hat{f}(\gamma)
													= \sum_{\gamma \in \K} (L_a f)\widehat{\phantom{x}}(\gamma).
		\end{align*}
		Nutzen wir aus, dass $\abs[\lambda]_\A = 1$ f"ur alle $\lambda \in \Kx$, so erhalten wir
		\begin{align*}
			\sum_{\gamma \in \K} M_\lambda {f(\gamma)} 	&= \sum_{\gamma \in \K} {f(\lambda\gamma)} 
														= \sum_{\gamma \in \K} \hat{f}(\lambda\gamma) \\
														&= \sum_{\gamma \in \K} \frac{1}{\abs[\lambda]_\A} \hat{f}(\lambda^{-1}\gamma) 
														= \sum_{\gamma \in \K}  (M_\lambda f)\widehat{\phantom{x}} (\gamma).
		\end{align*}	
		Damit haben wir auch schon den ersten Teil der Beweisidee verwirklicht.
		
		Der zweite Teil ist weniger Arbeit.
		F"ur $f=f_\infty \prod_{p<\infty}\ind_{\Zp}$ haben wir zum einen, dass $\hat{f} = \hat{f}_\infty \prod_{p<\infty}\ind_{\Zp}$. 
		Zum anderen ist
		\begin{align*}
			\sum_{\gamma \in \K}{f(\gamma)} = \sum_{\gamma \in \K} f_\infty(\gamma) \prod_{p<\infty}\ind_{\Zp}(\gamma) =\sum_{\gamma \in \Z} f_\infty(\gamma),
		\end{align*}
		denn das Produkt "uber alle verschwindet genau dann nicht, wenn $\gamma$ eine ganze Zahl ist.
		Wir m"ussen also nur $\sum_{\gamma \in \Z} f_\infty(\gamma) = \sum_{\gamma \in \Z} \hat{f}_\infty(\gamma)$ zeigen.
		Aber das sagt uns gerade die klassische Poisson-Summenformel.
	\end{proof}
	Ausger"ustet mit der adelischen Summenformel kommen wir zu einem weiteren wichtigen Eckpunkt von Tates Beweis.
	\begin{satz}[Riemann-Roch]
	\label{satz:global:riemannroch}
		Sei $x \in \I$ ein Idel und $f\in S(\A)$. Dann
		\begin{align*}
			\sum_{\gamma \in \K} {f(\gamma x)} = \frac{1}{|x|_{\A}}\sum_{\gamma \in \K} {\hat{f}(\gamma x^{-1})}.
		\end{align*}
		Insbesondere konvergieren die Summen absolut.
	\end{satz}
	\begin{proof}
		Sei $x \in \I$ beliebig aber fest. 
		F"ur beliebige $y \in \A$ definieren wir eine Funktion $h(y)\coloneqq f(yx)$. Diese ist wieder in $S(\A)$, erf"ullt damit die Poisson-Summenformel
		\begin{align*}
			\sum_{\gamma \in \K}h(\gamma) = \sum_{\gamma \in \K} \hat{h}(\gamma)
		\end{align*}
		und beide Summen konvergieren absolut.
		Berechnen wir allerdings die Fouriertransformation von $h$ erhalten wir mit einer Translation um $x^{-1}$
		\begin{align*}
			\hat{h}(\gamma) &= \int_{\A}h(y)\Psi(\gamma y)dy \\
							 &= \int_{\A}f(yx)\Psi(\gamma y)dy \\
							 &= \frac{1}{|x|_{\A}} \int_{\A}f(y)\Psi(\gamma y x^{-1})dy \\
							 &= \frac{1}{|x|_{\A}} \hat{f}(\gamma x^{-1}).
		\end{align*}
		Damit sind wir auch schon fertig.
	\end{proof}
	
\subsection{Die globale Funktionalgleichung}
%Hauptsatz gegliedert wie Riemann beweis
%Zeta funktion und idele class char einfuehren
%%%
%%%	Fubini und das Volumen von des Fundamentalbereichs F
%%%
	In Satz \ref{satz:global:ideleiso} haben wir gesehen, dass $\I$ sich schreiben l"asst als direktes Produkt $\I^1\times\R_+^\times$.
	Um nun auf $\I^1$ ein (multiplikatives) Haar-Maß $\dxx[b]$ zu fixieren, nehmen wir auf $\R_+^\times$ das Maß $\frac{\dx[t]}{t}$ und verlangen $\dxx = \dxx[b] \times \frac{\dx[t]}{t}$.
	F"ur die Berechnungen haben wir dann ganz im Sinne von Fubini
	\begin{align*}
		\int_\I f(x) \dxx = \int_0^\infty \left[\int_{\I^1} f(tb) \dxx[b]\right]  \frac{\dx[t]}{t} =  \int_{\I^1} \left[\int_0^\infty f(tb) \frac{\dx[t]}{t}\right] \dxx[b].
	\end{align*}
	Die Notation $tb$ ist dabei als Multiplikation $\iota_\infty(t) \cdot b$ zu verstehen. 
	Es folgt sofort, dass $\abs[tb]_\A = \abs[t]_\infty$ gilt. 
	Weiter haben wir in Satz \ref{satz:global:ideleiso} auch den Fundamentalbereich $F=\{1\}\times \prod_{p<\infty}\Zpx$ der Wirkung von $\Kx$ auf $\I^1$ kennengelernt.
	Das Volumen von $F$ bez"uglich $\dxx[b]$ wird f"ur unsere Berechnung eine Rolle spielen.
	Berechnen wir es schnell.
	Mit der Normierung des Maßes $\dxx$ haben wir
	\begin{align*}
		1 = \Vol\left((0,e)\times \prod_{p<\infty}\Zpx, \dxx\right) = \Vol\left(F, \dxx[b]\right) \cdot \Vol\left( (0,e), \frac{\dx[t]}{t}\right) = \Vol(F, \dxx[b]) 
	\end{align*}
	
	Bevor wir jetzt die die globalen Zeta-Funktionen einf"uhren ben"otigen wir noch geeignete multiplikative Charaktere.
	Es wird sich als sinnvoll erweisen die sogenannten \emph{Idelklassencharaktere}\footnote{Der Name r"uhrt aus der Tatsache, dass diese die Charaktere auf der \emph{Idelklassengruppe $\I/\Kx$} sind (vgl. \cite{neukirch} S.375)} zu betrachten.
	Dies sind, wie der Name tr"ugerisch verschweigt, Quasi-Charaktere auf $\I$, die trivial auf die Untergruppe $\Kx$ wirken.
	Im Gegensatz zu unseren lokalen multiplikativen Charakteren fangen wir zuerst mit der Form an.
	\begin{satz}
		Jeder Idelklassencharakter $\chi$ hat die Form 
		\begin{align*}
			\chi(x) = \mu(\tilde{x}) \abs[x]_\A^s
		\end{align*}
		wobei $\mu$ ein unit"arer Charakter auf $\I^1$ und $\tilde{\cdot}:\I\to \I^1$ der stetige Homomorphismus nach Satz \ref{satz:global:ideleiso} (d) ist.
	\end{satz}
	\begin{proof}
		Wir schreiben wie oben $x\in \I$ als eindeutiges Produkt $tb$ mit $t=\abs[x]_\A\in \R^\times_+$ und $b\in \I^1$.
		Dies ergibt die Zerlegung $\chi(x) = \chi_\infty(t) \cdot \chi(b)$.
		Da $\chi$ trivial auf $\Kx$ wirkt induziert er einen Charakter auf der kompakten Gruppe ansehen $\I^1/\Kx$ den wir wieder mit $\chi$ bezeichnen.
		Wir haben das Argument, dass $\chi(\I^1/\Kx)$ als kompakte Untergruppe von $\Komplex^\times$ in $S^1$ liegt.
		Folglich ist mit etwas Missbrauch der Notation $\mu\coloneqq \chi|_{\I^1} = \chi|_{\I^1/\Kx}$ ein Charakter auf $\I^1$.
		Weiter ist nach Lemma \ref{lemma:lokal:unverzweigterChar} $\chi_\infty(t) = \abs[t]_\infty^s$ und mit $\abs[t]_\infty = \abs[t]_\A$ folgt die Behauptung.
	\end{proof}
	Wie auch im Lokalen, k"onnen wir wieder von dem \emph{Exponenten} $\sigma=\Re{s}$ von $\chi = \mu\abs_\A^s$ reden.
	Ein Idelklassencharakter soll nun \emph{unverzweigt} heißen, wenn er trivial auf $\I^1$ wirkt. 
	Nach obigen Satz h"angt die Verzweigtheit also nur von dem Charakter $\mu$ ab.
	\begin{korollar}
		Jeder Idelklassencharakter $\chi$ ist genau dann unverzweigt, wenn er  trivial auf $F$ ist.
	\end{korollar}
	\begin{proof}
		Das folgt sofort aus $\I/\Kx \cong F$.
	\end{proof}
	
	Damit kommen wir zur
	\begin{defi}
		F"ur jede globale Schwartz-Bruhat Funktion $f\in \Sw(\A)$ und Idelklassencharakter $\chi$ mit Exponenten $\sigma > 0$ sei
		\begin{align*}
			Z_p(f, \chi) = \int_{\Kpx} f(x) \chi(x) \dxx_p
		\end{align*}
		eine \emph{globale Zeta-Funktion} auf $\A$.
	\end{defi}
	
	Nun haben wir alles zusammen f"ur den großen Beweis:
	\begin{itemize}
		\item Integration auf den Adelen und Idelen
		\item Fouriertransformation
		\item Riemann-Roch als Ersatz f"ur die Theta-Transformationsformel
		\item Zeta-Funktionen als Analogon zur Mellin-Transformation
	\end{itemize}
	
	Zusammen mit dem was wir aus unseren ersten Trocken"ubungen mit den lokalen Funktiongleichungen gelernt haben, werden wir jetzt eine globale Funktionalgleichung etablieren indem wir die Beweisideen aus Riemanns Beweis "ubertragen.
%%%
%%% Die Globale Funktionalgleichung
%%%
	\begin{satz}[Globale Funktionalgleichung]
		Sei $\chi=\mu\abs_\A^s$ ein Idelklassencharakter. Sei $f \in S(\A)$. 
		Dann konvergiert die globale Zeta-Funktion $Z(f,\mu,s)$  f"ur $\text{Re}(s) > 1$ absolut und gleichmäßig auf kompakten Teilmengen und definiert dort eine holomorphe Funktion, die zu einer meromorphen Funktion auf ganz $\Komplex$ fortgesetzt werden kann, welche die \emph{globale Funktionalgleichung}
		\begin{align*}
			Z(f,\mu,s) = Z(\hat{f}, \frac{1}{\mu}, 1-s)
		\end{align*}
		erf"ullt.
		Diese Funktion ist "uberall holomorph, außer wenn $\mu = \abs_\A^{-i\tau}$, $\tau \in \R$. 
		Dann besitzt sie einen einfachen Pol bei $s= i\tau$ mit Residuum $-f(0)$ und einen einfachen Pol bei $s=1+i\tau$ mit Residuum $\hat{f}(0)$.
	\end{satz}
	\begin{proof}
		Wir beweisen zun"achst die Konvergenz. 
		Dazu gen"ugt es faktorisierbare Schwartz-Bruhat Funktionen $f$ zu betrachten.
		%F"ur alle endlichen Stellen $p$ ist dann $f_p$ die charakteristische Funktion von $p^k\Z_p$ mit $k\in\Z$, wobei $k=0$ f"ur fast alle Stellen.
		Wir m"ussen also zeigen, dass das Integral 
		\begin{align}\label{eq:zetaproduct}
			\int_\I \abs[f(x)\chi(x)] d^\times x = \int_\I \abs[f(x)] \cdot \abs[x]_\A^\sigma d^\times x = \prod_{p\leq\infty} \int_{\K_p^\times} \abs[f_p(x_p)] \cdot \abs[x_p]_p^\sigma d^\times x_p
		\end{align}
		endlich ist.
		Dazu teilen wir das Produkt auf.
		Es gibht eine Primzahl $p_0$, so dass $f_p$ f"ur alle $p_0\leq p <\infty$ die charakteristische Funktion $\ind_\Zp$ ist.
		Wir k"onnen Gleichung \ref{eq:zetaproduct} also schreiben als 
		\begin{align*}
			\int_{\K_\infty^\times} \abs[f_\infty(x_\infty)] \cdot \abs[x_\infty]_\infty^\sigma d^\times x_\infty \cdot \prod_{p < p_0} \int_{\K_p^\times} \abs[f_p(x_p)] \cdot \abs[x_p]_p^\sigma d^\times x_p \cdot \prod_{p_0\leq p < \infty} \int_{\Z_p \setminus \{0\}} \abs[x_p]_p^\sigma d^\times x_p .
		\end{align*}
		Der erste Faktor und das Produkt in der Mitte sind endlich nach dem Satz "uber die lokalen Funktionalgleichungen. 
		In der Tat haben wir hier ein endliches Produkt lokaler Zeta-Funktionen $Z(\abs[f_p], \abs[x]_p^\sigma)$. 
		Diese konvergieren f"ur $\sigma >0$, also sicherlich auch f"ur $\sigma >1$.
		Konvergenz h"angt also nur von der Konvergenz des unendlichen Produkts 
		\begin{align*}
			\prod_{p_0\leq p < \infty} \int_{\Z_p \setminus \{0\}} \abs[x_p]_p^\sigma d^\times x_p
		\end{align*}
		ab. 
		Der aufmerksame Leser wird sich hier vielleicht an die genauen Berechnungen der lokalen Funktionalgleichungen erinnern und einen Bezug zur Riemannschen Zeta-Funktion erkennen. 
		Um die sp"atere "Uberraschung aber nicht zu verderben werden wir nur anmerken, dass dies ein Teilprodukt eines bekannten unendlichen Produkts, welches gleichm"aßig auf jedem Kompaktum der gegeben Halbebene konvergiert.
		
		%%%Ende Konvergenz
		%%%
		%%%Anfang Funktionalgleichung
		
		Nun zur Funktionalgleichung. 
		Aufgrund absoluter Konvergenz auf der Halbebene $\Re(s)>1$ haben wir
		\begin{align*}
			Z(f,\chi) 	&= \int_{\I} f(x) \chi(x) d^\times x \\
							&= \iint_{\R^\times_+ \times \I^1} f(t\cdot b) \chi(t\cdot b) (d^\times t\times \dxx[b])\\
							&= \int_0^\infty \left[\int_{\I^1} (f(t\cdot b) \chi(t\cdot b) \dxx[b]\right] \frac{dt}{t}
		\end{align*}
		Um uns etwas Schreibarbeit zu sparen definieren wir
		\begin{align*}
			Z_t(f,\chi) \coloneqq  \int_{\I^1} f(t\cdot b) \chi(t\cdot b) \dxx[b].
		\end{align*}
		Wie in Riemanns Beweis teilen wir das Integral auf durch
		\begin{align*}
			Z(f,\chi) = \int_0^1 Z_t(f,\chi) \frac{dt}{t} 
							+ \int_1^\infty Z_t(f,\chi) \frac{dt}{t}.
		\end{align*}
		Das Integral $\int_1^\infty$ macht uns keine Probleme.
		Wir haben festgestellt, dass
		\begin{align*}
			\int_1^\infty \abs[Z_t(f,\chi)] \frac{dt}{t} 
				= \int_{\{x\in \I: \abs[x]_\A\geq 1\}} \abs[f(x)] \abs[x]_\A^\sigma \dxx,
		\end{align*}
		f"ur $\sigma>1$ konvergiert.
		Da $\abs[x]_\A \geq 1$ verbessert sich das Konvergenzverhalten des Integranden umso kleiner $\sigma$ ist und es folgt die Konvergenz auf ganz $\Komplex$.
		Als n"achstes erinnern wir uns daran, dass wir $\I^1$ als disjunkte Vereinigung $\bigsqcup_{a \in \K^\times} aF$ darstellen konnten, 
			wobei $F= \{1\} \times \prod_{p<\infty}\Z_p$.
		Kombiniert mit der Translationsinvarianz von $\dxx[b]$ und der Tatsache, dass $\chi$ trivial auf $\K^\times$ wirkt, ergibt sich
		\begin{align*}
			Z_t(f,\chi)	&= \int_{\I} (f(t\cdot b) \chi(t\cdot b) \dxx[b] 
							%= \int_{\bigsqcup_{q \in \K^\times} qF} (f(t\cdot b) \chi(t\cdot b) \dxx[b]
							= \sum_{a \in \K^\times} \int_{aF} (f(t\cdot b) \chi(t\cdot b) \dxx[b]\\
							&= \sum_{a \in \K^\times} \int_{F} (f(at\cdot b) \chi(t\cdot b) \dxx[b]
							= \int_{F} \left(\sum_{a \in \K^\times}  (f(at\cdot b)\right) \chi(t\cdot b) \dxx[b]
		\end{align*}
		Die Summe "uber $a$ verleitet uns dazu Riemann-Roch anzuwenden, allerdings ben"otigen wir hierf"ur eine Summe "uber $\K$. Das Problem l"asst sich jedoch leicht beheben.
		\begin{lemma}
			\begin{align*}
				Z_t(f,\chi) = \zeta_{t^{-1}}(f,\check{\chi}) + \hat{f}(0) \int_F \check{\chi} (x/t)db - f(0)\int_F \chi(tx)db.
			\end{align*}
		\end{lemma}
		\begin{proof}
			Die Idee ist klar. 
			Wir f"ugen $f(0)\int_F \chi(tx)db$ zu $Z_t(f,\chi)$ hinzu, erhalten
			\begin{align*}
				Z_t(f,\chi) + f(0)\int_F \chi(tb)db= \int_{F} \left(\sum_{a \in \K}  (f(at\cdot b)\right) \chi(t\cdot b) \dxx[b]
			\end{align*}
			und k"onnen jetzt unsere Version von Riemann-Roch anwenden:
			\begin{align*}
				\int_{F} \left(\sum_{a \in \K}  f(at\cdot b)\right) \chi(t\cdot b) \dxx[b] 
					&= \int_{F} \left(\sum_{a \in \K}  \hat{f}(a t^{-1} b^{-1}) \right) \frac{\chi(t\cdot b)}{\abs[tx]_{\A}} \dxx[b]\\
					&= \int_{F} \left(\sum_{a \in \K}  \hat{f}(a t^{-1} b) \right) \abs[t^{-1}b]_{\A} \chi(t\cdot b) \dxx[b]\\
					&= \int_{F} \left(\sum_{a \in \K}  \hat{f}(a t^{-1} b) \right) \check{\chi}(b/t)db + \hat{f}(0) \hat{f}(0) \int_F \check{\chi} (x/t)db\\
					&= Z_{t^{-1}}(f,\check{\chi}) + \hat{f}(0) \int_F \check{\chi} (x/t)db
			\end{align*}
			wobei wir im zweiten Schritt den Variablenwechsel $b\mapsto b^{-1}$ und im dritten Schritt $\chi(x^{-1}) = \chi(x)^{-1}$ ausgenutzt haben.
		\end{proof}
		Wir widmen uns nun dem Integral $\int_0^1$. Dank Riemann-Roch k"onnen wir es umformen zu
		\begin{align*}
			\int_0^1 Z_t(f,\chi) \frac{dt}{t} 
				= \int_0^1 \left( Z_{t^{-1}}(\hat{f},\check{\chi}) 
					+ \hat{f}(0) \check{\chi}(t^{-1}) \int_F \check{\chi} (x)db 
					- f(0)\chi(t)\int_F \chi(x)db \right)\frac{dt}{t}
		\end{align*}
		Mit einem Variablenwechsel $t\mapsto t^{-1}$ im ersten Summanden ergibt sich
		\begin{align*}
			\int_0^1  Z_{t^{-1}}(\hat{f},\check{\chi}) \frac{dt}{t} = \int_1^\infty  Z_{t}(\hat{f},\check{\chi}) \frac{dt}{t}
		\end{align*}
		was nach dem gleichen Argument wie oben auf ganz $\Komplex$ konvergiert. 
		Damit haben wir ganz nach Riemann beide Integrale von $1$ bis $\infty$.
		Es verbleibt noch der \glqq Fehler\grqq{}-Term 
		\begin{align*}
			E(f,\chi)\coloneqq  \int_0^1  \hat{f}(0) \check{\chi}(t^{-1}) \left(\int_F \check{\chi} (x)db\right) \frac{dt}{t}
					- \int_0^1 f(0)\chi(t)\left(\int_F \chi(x)db \right)\frac{dt}{t}.
		\end{align*}
		Dieser entspricht wiederum den beiden Termen $\frac{1}{s}$ und $\frac{1}{1-s}$ aus Riemanns Beweis.
		Ist $\chi$ nicht trivial auf $\I^1$, so wirkt $\chi$ auch nicht trivial auf dem Kompaktum $F$.
		Folglich verschwinden beide Integrale und $E(f,\chi) = 0$
		%\begin{align*}
				%Z(f,\chi) =  \int_1^\infty Z_t(\hat{f}, \check{\chi}) \frac{dt}{t} 
							%+ \int_1^\infty Z_t(f,\chi) \frac{dt}{t}.
		%\end{align*}
		Ist $\chi = \mu \abs^s$ dagegen trivial auf $\I^1$, dann wissen wir, dass $\chi = \abs^{s'}$, wobei $s'=s-i\tau$ f"ur ein $\tau \in \R$. Also,
		\begin{align*}
			E(f,\chi) 	&= \int_0^1  \hat{f}(0) t^{s'-1} \text{Vol}(F,db) - f(0) t^{s'}\text{Vol}(F,db)\frac{dt}{t}\\
						&= \frac{\hat{f}(0)}{s' - 1} - \frac{f(0)}{s'}
		\end{align*}
		und wir sehen, dass $E$ in diesem Fall eine rationale Funktion ist. Damit ist
		\begin{align*}
			Z(f,\chi) =  \int_1^\infty Z_t(\hat{f}, \check{\chi}) \frac{dt}{t} 
							+ \int_1^\infty Z_t(f,\chi) \frac{dt}{t} + E(f,\chi)
		\end{align*}
		Eine meromorphe Erweiterung der Funktion auf ganz $\Komplex$. 
		Zudem haben wir gezeigt, dass f"ur $\mu \not=\abs^{-i\tau}$ die Funktion $\zeta$ sogar ganz ist und im Fall $\mu =\abs^{-i\tau}$ ihre einzigen Pole bei $s=i\tau$ und $s=1+i\tau$ liegen mit den Residuen $-f(0)$ bzw. $\hat{f}(0)$.\\
		%%%
		%%%Funktionalgleichung
		%%%
		Zum Schluss kommen wir noch zur Funktionalgleichung. Aus
		\begin{align*}
			\hat{\hat{f}}(x) = f(-x) \text{ und } \check{\check{\chi}} = \chi
		\end{align*}
		folgt
		\begin{align*}
			Z(\hat{f},\check{\chi}) 
				&=  \int_1^\infty Z_t(\hat{\hat{f}}, \check{\check{\chi}}) \frac{dt}{t} 
					+ \int_1^\infty Z_t(\hat{f},\check{\chi}) \frac{dt}{t} + E(\hat{f},\check{\chi})\\
				&= \int_1^\infty \int_{\I^1}f(-tb)\chi(tb)db  \frac{dt}{t} 
					+ \int_1^\infty \int_{\I^1}\hat{f}(tb)\check{\chi}(tb)db  \frac{dt}{t} +E(f,\chi)\\
				&= \int_1^\infty \int_{\I^1}f(tb)\chi(tb)db  \frac{dt}{t} 
					+ \int_1^\infty \int_{\I^1}\hat{f}(tb)\check{\chi}(tb)db  \frac{dt}{t} +E(f,\chi) = Z(f,\chi)
		\end{align*}
		wobei wir im letzten Schritt im ersten Integral die Translationsinvarianz der Haar-Maßes $db$ und die Eigenschaft des Idele-Klassencharakters $\chi(-tx) = \chi(tx)$ ausgenutzt haben.
	\end{proof}
\subsection{Globale Berechnung: Die Riemannsche Zeta-Funktion}
	Wir wollen jetzt zeigen, dass unsere Berechnungen im letzten Abschnitt nicht ohne Inhalt sind.
	Immerhin besteht immer noch die Gefahr, dass wir gerade $0 = 0$ gezeigt haben.
	Betrachten wir also die adelische Schwartz-Bruhat Funktion
	\begin{align*}
		f(x) = \exp(-2\pi i x_\infty) \prod_{p<\infty} \ind_{\Zp}(x_p).
	\end{align*}
	Dank unserer Berechnung der lokalen Faktoren wissen wir, dass
	\begin{align*}
		\hat{f}(x) = \exp(-2\pi i x_\infty) \prod_{p<\infty} \ind_{\Zp}(x_p),
	\end{align*}
	also $f$ ihre eigene Fouriertransformierte ist.
	
	Auf zu den Zeta-funktionen!
	F"ur jede Stelle $p\leq \infty$ ist der Faktor $f_p$ die Funktion, die wir bei den Berechnungen der lokalen Zeta-Funktionen $Z_p(f_p,1,\abs_p^s)$ im unverzweigten Fall verwendet haben.
	Bleiben wir also auch hier im unverzweigten und betrachten den Charakter $\abs_\A^s$.
	Damit vereinfacht sich die berechnung der globalen Zeta-Funktion zu
	\begin{align*}
		Z(f, 1, \abs_\A^s) 	= \int_\I f(x) \abs[x]_\A^s \dxx 
							= \prod_{p\leq \infty} Z(f_p, 1 , \abs_p^s)
							= \pi^{-\frac{s}{2}} \Gamma\left(\frac{s}{2}\right) \prod_{p<\infty} \frac{1}{1-p^{-s}}.
	\end{align*}
	Das Produkt am Ende sollte uns bekannt vorkommen.
	Es ist genau die Darstellung der Riemannschen Zeta-Funktion als Euler-Produkt.
	Damit ist also
	\begin{align*}
		Z(f, 1, \abs_\A^s) = \pi^{-\frac{s}{2}} \Gamma\left(\frac{s}{2}\right) \zeta(s) = \xi(s)
	\end{align*}
	die vervollständigte Riemannsche Zeta-Funktion.
	Analog ist
	\begin{align*}
		Z(\hat{f}, 1, \abs_\A^{1-s})  	%= \pi^{-\frac{1-s}{2}} \Gamma\left(\frac{1-s}{2}\right) \prod_{p<\infty} \frac{1}{1-p^{s-1}} 
										= \pi^{-\frac{1-s}{2}} \Gamma\left(\frac{1-s}{2}\right) \zeta(1-s) = \xi(1-s)
	\end{align*}
	und die globale Funktionalgleichung liefert uns gerade
	\begin{align*}
		\xi(s) = Z(f, 1, \abs_\A^s) = Z(\hat{f}, 1, \abs_\A^{1-s}) =\xi(1-s),
	\end{align*}
	die Funktionalgleichung der Riemannschen Zeta-Funktion.
	Nun wird aber auch klar woher der Faktor $\pi^{-\frac{s}{2}} \Gamma\left(\frac{s}{2}\right)$ r"uhrt.
	Es ist gerade der Beitrag, den die Stelle $p=\infty$ bei den Berechnungen liefert.
	
	B"ose Zungen m"ogen jetzt behaupten, dass wir durch die Wahl unserer Funktion $f$, vor allem durch die Wahl des Faktors $f_\infty (x_\infty) =  exp(-2\pi i x_\infty)$,
	die Form dieser Gleichung letztendlich beeinflusst haben um gerade auf das klassische Ergebnis zu kommen.
	Sie haben recht, denn immerhin h"atte ein anderer Faktor $g_\infty$ eine andere Funktionalgleichung ergeben.
	Diese beiden Funktionalgleichungen unterscheiden sich dann aber nur um einen meromorphen Faktor, den wir dank
	\begin{align*}
		  Z_\infty(g_\infty, \abs_\infty^s)Z_\infty(\hat{f}_\infty,\abs_\infty^{s-1}) =Z_\infty(f_\infty, \abs_\infty^s) Z_\infty(\hat{g}_\infty, \abs_\infty^{s-1}),
	\end{align*}
	mit
	\begin{align*}
		 \frac{ Z_\infty(g_\infty, \abs_\infty^s)}{Z_\infty(f_\infty, \abs_\infty^s)} = \frac{Z_\infty(\hat{g}_\infty, \abs_\infty^{s-1})} {Z_\infty(\hat{f}_\infty,\abs_\infty^{s-1})}
	\end{align*}
	angeben k"onnen.
	Unsere Wahl war also nur durch die besonders einfache Form der Faktoren und den damit verbundenen Berechnungen beeinflusst.
	