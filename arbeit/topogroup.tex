\section{Lokalkompakte Gruppen}\label{sec:topogroup}
	Auch wenn es der Beweis Riemanns nicht direkt vermuten lässt, die Fouriertransformation von Funktionen wird eine entscheidende Rolle im Beweis der Funktionalgleichung spielen.
	Wir beschäftigen uns daher zunächst mit den Grundlagen der (abstrakten) harmonischen Analysis, einer Verallgemeinerung der klassischen Fourieranalysis auf $\R$.
	Im Mittelpunkt stehen hier die lokalkompakten Gruppen.
	Auf diesen lässt sich sehr natürlich ein zum klassischen Lebesgue-Maß analoges Maß definieren, wodurch wir Integral und auch Fouriertransformation definieren können.
	Wir werden in dieser Arbeit jedoch nicht den vollen Weg zu abstrakten Fourieranalysis gehen.
	Stattdessen definieren wir in Kapitel \ref{sec:lokal} eine eigene Fouriertransformation und geben im letzten Abschnitt dieses Kapitels nur einen kurzen Ausblick.
	Für die Behandlung des Stoffes halten wir uns dabei an \textcite{rama} Kapitel 1 und 3.
\subsection{Lokalkompakte Gruppen}
%topologische Gruppe: done
%charaktere:partly
%haar masse: done
%fouriertransformation:done
%pontryagin erwaehnen: done
	Am Anfang steht immer eine
	\begin{defi}
		Eine \emph{topologische Gruppe} ist eine (nicht unbedingt abelsche) Gruppe $G$ zusammen mit einer Topologie, welche die folgenden Eigenschaften erfüllt:
			\begin{enumerate}[label=(\roman*)] % leftmargin=*, align=left, labelsep=1pt]
				\item Die Gruppenoperation
					\begin{align*}
						G \times G &\longrightarrow G\\
						(g,h) &\longmapsto gh
					\end{align*}
				ist stetig auf der Produkttopologie von $G \times G$.
				\item Die Umkehrabbildung
					\begin{align*}
						G &\longrightarrow G\\
						g &\longmapsto g^{-1}
					\end{align*}
					ist stetig auf G.
			\end{enumerate}
		Ein \emph{Homomorphismus topologischer Gruppen} von $G_1$ auf $G_2$ ist ein stetiger Gruppenhomomorphismus $G_1 \to G_2$.
		Ist dieser bijektiv und die Umkehrabbildung wieder ein Homomorphismus topologischer Gruppen, so sprechen wir von einem \emph{Isomorphismus topologischer Gruppen}.
	\end{defi}
	Man sieht sofort, dass jede Translation um ein beliebiges Gruppenelement einen Homöomorphismus $G \to G$ bildet.
	Die Topologie ist also \emph{translationsinvariant} in dem Sinne, dass für alle $g \in G$ und jede Mengen $U$ die Äquivalenzen
	\begin{align*}
		U \text{ ist offen} \Leftrightarrow gU = \{gu \in G: u\in U\} \text{ ist offen} \Leftrightarrow Ug = \{ug \in G: u\in U\} \text{ ist offen}
	\end{align*}
	gelten.
	Analog verhält es sich für die Umkehrabbildung. 
	Sie ist ebenso ein Homöomorphismus und $U$ ist genau dann offen, wenn $U^{-1}=\{u^{-1}\in G: u \in U\}$ offen ist.
	
	Die Translationsinvarianz der Topologie hat einige Vorteile.
	Zum Beispiel wird die gesamte Topologie bereits durch eine Umgebungsbasis des neutralen Elements definiert.
	Durch Translation erhalten wir Umgebungsbasen beliebiger anderer Elemente und damit zwangsläufig die komplette Topologie.
	Für ein weiteres Beispiel erinnern wir uns an die Definition der Stetigkeit in topologischen Räumen.
	Eine Abbildung $f: G_1 \to G_2$ zwischen zwei topologischen Räumen heißt stetig, wenn für alle $g\in G_1$ und jede offene Umgebung $U$ von $f(g)$ eine offene Umgebung $V$ von $g$ existiert, sodass $f(V)\subseteq U$.
	Sind $G_1$ und $G_2$ topologische Gruppen und ist $f$ ein (nicht unbedingt topologischer) Gruppenhomomorphismus, so reicht es die Stetigkeit in dem neutralen Element $e_1$ der Gruppe $G_1$ nachzuweisen.
	Denn ist $f$ stetig in $e_1$ und sei $g$ ein beliebiges weiteres Element der Gruppe, so ist für jede offene Umgebung U von $f(g)$ die Menge $U'\coloneqq f(g)^{-1}U$ eine offene Umgebung des neutralen Elements $e_2$. 
	Aufgrund der Stetigkeit in $e_1$ gibt es dann eine offene Umgebung $V'$ mit $f(V') \subseteq U'$.
	Es ist aber $V\coloneqq gV'$ eine offene Umgebung von $g$ und, da $f$ ein Gruppenhomomorphismus ist, erhalten wir
	\begin{align*}
		f(V) = f(gV') = f(g)f(V') \subseteq f(g) U' =f(g)f(g)^{-1} U = U.
	\end{align*}
	Folglich ist die Abbildung überall stetig.
	
	Betrachten wir wieder zwei topologische Gruppen $G_1$ und $G_2$, so ist deren direktes Produkt $G_1\times G_2$, wie man leicht feststellen kann, wieder eine topologische Gruppe.
	Dieses Ergebnis lässt sich auch auf endliche, abzählbare und sogar beliebige Mengen von Gruppen übertragen. 
	\begin{lemma}\label{lemma:topogroup:directproduct}
		Sei $I$ eine beliebige Indexmenge und $G_\nu$ eine topologische Gruppe für alle $\nu\in I$. 
		Das direkte Produkt $G = \prod_{\nu\in I} G_\nu$, versehen mit der Produkttopologie und komponentenweiser Gruppenverknüpfung, ist wieder eine topologische Gruppe.
	\end{lemma}
	\begin{proof}
		Wir erinnern uns daran, dass eine Basis der Produkttopologie durch Rechtecke der Form
		\begin{align*}
			\prod_{\nu\in E} U_\nu \times \prod_{\nu\in I\setminus E} G_\nu
		\end{align*}
		gegeben ist, wobei $E$ eine endliche Teilmenge von $I$ und jedes $U_\nu$ offen in $G_\nu$ ist. 
		Ohne Einschränkung sei also 
		\begin{align*}
			W = \prod_{\nu\in E} W_\nu \times \prod_{\nu\in I\setminus E} G_\nu
		\end{align*}
		eine offene Umgebung von $gh = (g_\nu h_\nu)$. 
		Da alle $G_\nu$ topologische Gruppen sind, finden wir für jeden Index $\nu\in E$ eine offene Umgebungen $U_\nu$ und $V_\nu$ von $g_\nu$ bzw. $h_\nu$, sodass $U_\nu V_\nu \subseteq W_\nu$. Wir behaupten , dass
		\begin{align*}
			(\prod_{\nu\in E} U_\nu \times \prod_{\nu\in I\setminus E} G_\nu) \times (\prod_{\nu\in E} V_\nu \times \prod_{\nu\in I\setminus E} G_\nu) \subseteq G \times G
		\end{align*}
		eine offene Umgebung von $(g, h) \in G \times G$ ist, deren Bild in $W$ liegt. 
		Der erste Aussage ist klar, beide Faktoren des Produkts sind offene Basiselemente der Topologie und enthalten $g$ bzw. $h$.
		Weiter ist das Bild unter der Gruppenoperation durch
		\begin{align*}
			\prod_{\nu\in E} U_\nu V_\nu \times \prod_{\nu\in I\setminus E} G_\nu
		\end{align*}
		gegeben, was wiederum nach den obigen Überlegungen in $W$ liegt.
		Somit folgt die Stetigkeit der Gruppenverknüpfung.
		Der Beweis für die Stetigkeit der Umkehrabbildung funktioniert analog.
	\end{proof}
	%Hat man eine Topologie so kann man sich auch die Frage nach Kompaktheit stellen.
	%Leider sind die meisten topologischen Gruppen, die wir betrachten werden, jedoch nicht kompakt.
	
	\begin{defi}
		Ein topologischer Raum heißt \emph{lokalkompakt}, wenn jeder Punkt des Raumes eine kompakte Umgebung hat. 
		Eine \emph{lokalkompakte Gruppe} ist eine topologische Gruppe, die sowohl lokalkompakt, als auch hausdorffsch ist. 
	\end{defi}
	Wir kennen bereits einige lokalkompakte Gruppen aus der.
	\begin{bsp}~ 
		\begin{enumerate}[label=(\alph*)]
			\item Jede diskrete topologische Gruppe $G$, also eine Gruppe in der alle Teilmengen offen sind, ist lokalkompakt. 
				Für jedes $x \in G$ ist $\{x\}$ eine kompakte offene Umgebung von $x$.
			\item Die additive Gruppe $\R^+$ mit der gewohnten Topologie ist lokalkompakt. 
				Denn ist $x\in\R^+$, so ist $[x-\varepsilon, x+\varepsilon]$ für $\varepsilon>0$ eine kompakte Umgebung von $x$. "
				Ahnlich verhält es sich für die multiplikative Gruppe $\R^\times = \R \setminus\{0\}$.
			\item Analog kann man sich überlegen, dass die Gruppen $\Komplex^+$ und $\Komplex^\times$ lokalkompakt sind, wobei in diesem Fall die abgeschlossenen Bälle $\conj{B_\varepsilon(x)}$ die kompakten Umgebung von $x\in\Komplex$ bilden.
		\end{enumerate}
	\end{bsp}
	Wieder können wir das direkte Produkt zweier lokalkompakter Gruppen betrachten. 
	\begin{lemma}\label{satz:topo:lcaproduct}
		Seien $G_1$ und $G_2$ zwei lokalkompakte Gruppen. 
		Dann ist $G_1\times G_2$ wieder lokalkompakt. 
		Insbesondere ist also jedes endliche direkte Produkt lokalkompakter Gruppen lokalkompakt.
	\end{lemma}
	\begin{proof}
		Sei $(g_1,g_2) \in G_1\times G_2$. Wegen der Lokalkompaktheit von $G_1$ und $G_2$ finden wir kompakte Umgebungen $K_1$ bzw. $K_2$ von $g_1$ bzw. $g_2$.
		Dann ist aber $K_1 \times K_2$ eine kompakte Umgebung von $(g_1,g_2)$. 
		Weiter ist das direkte Produkt zweier Hausdorff-Räume wieder hausdorffsch, wodurch $G_1\times G_2$ zu einer lokalkompakten Gruppe wird.
	\end{proof}
	Wie wir später in Lemma \ref{Lemma:lokalkompaktProd} sehen werden, kann diese Aussage nicht ohne Weiteres auf beliebig große direkte Produkte übertragen werden.

\subsection{Das Haar-Maß}
	Weiter mit etwas Maßtheorie. 
	Wir beginnen mit einer kleinen Auffrischung der wichtigsten Konzepte. 
	Eine \emph{$\sigma$-Algebra} auf einer Menge $X$ ist eine Teilmenge $\Omega$ von $\mathcal{P}(X)$, so dass
	\begin{enumerate}[label=(\roman*)]
		\item $X \in \Omega$
		\item Wenn $A \in \Omega$, dann $X\setminus A \in \Omega$.
		\item $\Omega$ ist abgeschlossen unter abzählbarer Vereinigung, d.h. $\bigcup_{k=1}^{\infty} A_k \in \Omega$, falls $A_k \in \Omega$ für alle $k$.
	\end{enumerate}
	Die Teilmengen in $\Omega$ werden \emph{messbar} genannt. 
	Aus den Axiomen lässt sich leicht folgern, dass die leere Menge und abzählbare Schnitte von messbaren Mengen wiederum messbar sind. 
	Weiter ist der Schnitt $\bigcap_n \Omega_n$ beliebiger Familien $\{\Omega_n\}$ von $\sigma$-Algebren auf X selbst wieder eine $\sigma$-Algebra.
	
	Zusammen bilden eine Menge $X$ und eine $\sigma$-Algebra $\Omega$ den \emph{messbaren Raum} $(X, \Omega)$. 
	Ist X ein topologischer Raum, so können wir die kleinste $\sigma$-Algebra $\Borel$ betrachten, die alle offenen Mengen von $X$ enthält. 
	Die Elemente von $\Borel$ werden \emph{Borelmengen} von $X$ genannt.
	
	Ein \emph{Maß} auf einem beliebigen messbaren Raum $(X, \Omega)$ ist eine nicht-negative Abbildung $\mu: \Omega \to [0, \infty]$, die \emph{$\sigma$-additiv} ist und $\mu(\emptyset) = 0$ erfüllt. 
	Das bedeutet
	\begin{align*}
		\mu\left( \bigcup_{k=1}^{\infty} A_k\right) = \sum_{k=1}^{\infty} \mu (A_k)
	\end{align*}
	für beliebige Familien $\{A_n\}_1^\infty$ von paarweise disjunkten Mengen in $\Omega$.
	Zusammen definiert dies den \emph{Maßraum} $(X, \Omega, \mu)$.
	Ist dieses Maß auf der $\sigma$-Algebra der Borelmengen definiert, so nennen wir es ein \emph{Borel-Maß}.
	
	Ein wichtiges Ziel der Maßtheorie war es, den Begriff des Integrals zu verallgemeinern.
	Wir geben eine kurze, stark vereinfachte Variante der Grundkonzepte der Integrationstheorie und verweisen auf \textcite{folland} Kapitel 2 für eine vollständige Einführung.
	
	Für einen beliebigen Maßraum $(X, \Omega, \mu)$ geschieht dies über die so genannten \emph{Treppenfunktionen} auf $X$
	\begin{align*}
		f(x) = \sum_{k=1}^n \alpha_k \ind_{A_k} (x)
	\end{align*}
	mit $\alpha \in \R$ und der \emph{charakteristischen Funktion} der messbaren Menge $A_k$
	\begin{align*}
		\ind_{A_k}(x) =
			\begin{cases}
				1 &\text{falls } x\in A_k\\
				0 &\text{sonst}.
			\end{cases}
	\end{align*}
	Das Integral dieser Funktionen wird definiert durch
	\begin{align*}
		\int_G f d\mu = \sum_{k=1}^n \alpha_k \mu(A_k),
	\end{align*}
	mit der Konvention $0 \times \infty = 0$. 
	Eine Möglichkeit, dies auf beliebige andere Funktionen $f$ zu erweitern, ist es Folgen $(f_n)$ von solchen Treppenfunktionen zu betrachten.
	Diese nennen wir \emph{$L^1$-Cauchy-Folge}, wenn sie eine Cauchy-Folge bezüglich der Norm $\norm*{g}_{L^1}\coloneqq \int_X \abs[g] d\mu$ ist. 
	Konvergiert die Folge zusätzlich fast überall punktweise gegen eine Funktion $f:X \to \R$, so definieren wir durch
	\begin{align*}
		\int f d\mu = \lim_{n\to \infty} \int f_n d\mu
	\end{align*}
	das Integral von $f$ über $X$. 
	Komplexwertige Funktionen $h = f + i g$ können dann durch
	\begin{align*}
		\int h d\mu = \int f d\mu + i \int g d\mu
	\end{align*}
	integriert werden.
	
	Ist $Y \subseteq X$ messbar in X, so setzen wir $\int_Y f d\mu \coloneqq \int \ind_Y f d\mu$ und definieren
	\begin{align*}
		\Vol(Y,d\mu) = \int_Y d\mu = \int \ind_Y d\mu = \mu(Y).
	\end{align*}
	Wir nennen eine Funktion $f$ integrierbar auf Y, wenn das Integral $\int_Y \abs[f]d\mu$ endlich ist.
	Allgemeiner heißt eine Funktion integrierbar, wenn sie auf X integrierbar ist und wir schreiben\footnote{Hier missbrauchen wir etwas die Notation des $L^1$-Raumes. (vgl. Folland \cite{folland} Seite 53)} dann $f\in L^1(X,\Omega, \mu)$. 
	Wenn es klar ist, über welchen Maßraum wir reden, lassen wir häufig auch die $\sigma$-Algebra und das Maß weg und schreiben $\dx$ für das Maß, $\int_X f(x) \dx = \int f d\mu$ für das Integral und $L^1(X)$ für die integrierbaren Funktionen auf $X$.
	Damit endet unsere kurze Wiederholung.
	
	Sei $\mu$ ein Borel-Maß auf einem lokalkompakten, hausdorffschen Raum X und sei $E$ ein eine beliebige Borelmenge von $X$.
	Wir nennen $\mu$ von \emph{innen regulär} auf E, falls
	\begin{align*}
		\mu(E) = \sup\{\mu(K): K \subseteq E,\; K \text{ kompakt}\}.
	\end{align*}
	Umgekehrt heißt $\mu$ von \emph{außen regulär} auf E, wenn
	\begin{align*}
		\mu(E) = \inf\{\mu(U): E \subseteq U,\; U \text{ offen}\}.
	\end{align*}
	
	\begin{defi}
		Ein \emph{Radon-Maß} auf $X$ ist ein Borel-Maß, das endlich auf kompakten Mengen, von innen regulär auf allen offenen Mengen und von außen regulär auf allen Borelmengen ist.
	\end{defi}
	Radon-Maße stehen in einem engen Zusammenhang mit positiven linearen Funktionalen auf dem Vektorraum der stetigen Funktionen mit kompakten Träger $C_c(X)$.
	Dies sind lineare Abbildungen $I$ von $C_c(X)$ nach $\R$, so dass $I(f) \geq 0$, wenn $f\geq 0$.
	\begin{satz}[Rieszscher Darstellungssatz]
		Sei $I$ solch ein positives lineares Funktional auf dem Raum der stetigen Funktionen mit kompakten Träger $C_c(X)$.
		Dann gibt es ein eindeutiges Radon-Maß $\mu$ auf $X$, so dass $I(f) = \int f d\mu$ für alle $C_c(X)$.
	\end{satz}
	\begin{proof}[Beweisskizze]
		Für die Konstruktion des Maßes definiert man die Abbildungen
		\begin{align*}
			\mu(U) = \sup\{I(f): f\in C_c(X), 0\leq f\leq 1 \text{ und } \text{supp}(f) \subseteq U\}
		\end{align*}
		für alle $U$ offen und
		\begin{align*}
			\mu^*(E) = \inf \{ \mu(U): U\supseteq E \text{ und } U \text{ offen}\}
		\end{align*}
		für beliebige Teilmengen $E\subseteq X$. 
		Anschließend zeigt man
		\begin{enumerate}[label=(\roman*)]
			\item $\mu^*$ ist ein äußeres Maß.
			\item Jede offene Menge ist $\mu^*$-messbar.
			\item $\mu(K) = \inf\{I(f): f\in C_c(X), f\geq \ind_K\}$ für alle kompakten Mengen $K \subseteq X$.
			\item $I(f) = \int f d\mu$ für alle $f\in C_c(X)$.
		\end{enumerate}
		(i) und (ii) implizieren zusammen mit dem Satz von Caratheodory, dass $\mu$ ein von außen reguläres Borel-Maß ist.
		(iii) liefert uns die Endlichkeit auf kompakten Mengen und die Regularität von innen auf offenen Mengen und (iv) vollendet den Satz.
		Für den vollständigen Beweis verweisen wir auf \textcite{folland} Satz 7.2.
	\end{proof}
	Dieser Satz ist ein wichtiger Grundstein für viele Aussagen über Radon-Maße.
	Sind zum Beispiel $X$ und $Y$ zwei lokalkompakte Gruppen mit dazugehörigen Radon-Maßen $\mu$ und $\nu$, so ist im Allgemeinen das klassische Produktmaß $\mu \times \nu$ kein Borel- und daher kein Radon-Maß auf $X \times Y$.
	Wir definieren daher das Radonprodukt von $\mu$ und $\nu$ als das Radon-Maß, welches durch das positive Funktional $I(f) = \int f d(\mu \times \nu)$ nach dem Rieszschen Darstellungssatz gegeben wird.
	Dieses Produkt wird in Kapitel \ref{sec:rdp} eine Rolle spielen.
	Danach m"ussen wir es nicht weiter beachten, denn erfüllen $X$ und $Y$ das zweite Abzählbarkeitsaxiom, entspricht das Radonprodukt genau dem Produktmaß.
	Für eine ausführliche Behandlung dieser Konzepte verweisen wir auf \textcite{folland} Kapitel 7.
	
	Damit kommen wir zum großen Ziel dieses Abschnitts.
	Sei $G$ eine lokalkompakte Gruppe und $\mu$ ein beliebiges Borel-Maß.
	Wir können uns dann fragen, wie sich das Maß einer beliebigen Borelmenge $E$ bez"uglich $\mu$ nach der Translation durch beliebige Gruppenelemente $g\in G$ verhält. 
	Gilt $\mu(gE) = \mu(E)$ für jede Borelmenge, so nennen wir $\mu$ \emph{linksinvariant}.
	Analog heißt $\mu$ \emph{rechtsinvariant}, falls $\mu(Eg) = \mu(E)$.
	Ist $G$ abelsch, fallen diese beiden Begriffe natürlich zusammen und wir sprechen nur von einem \emph{translationsinvarianten} Maß $\mu$.
	
	Nun haben wir alle wichtigen Grundlagen zusammen für folgende wichtige
	\begin{defi}
		Ein \emph{linkes} bzw. \emph{rechtes} \emph{Haar-Maß} auf $G$ ist ein linksinvariantes bzw. rechtsinvariantes Radon-Maß, das auf nichtleeren offenen Mengen positiv ist. 
	\end{defi}
	Wir haben bereits einige solcher Haar-Maße kennengelernt:
	\begin{bsp}~ 
		\begin{enumerate}[label=(\alph*)]
			\item Sei $G$ wieder eine diskrete Gruppe, dann ist das Zählmaß ein Haar-Maß.
			\item Für $G=\R^+$ definiert das klassische Lebesgue-Maß $\dx$ ein Haar-Maß.
			\item Für $G=\R^\times$ wird durch $\mu(E)\coloneqq \int_{\R^\times} \ind_{E} \frac{1}{\abs[x]}dx$ ein Haar-Maß definiert, wie wir später in Satz \ref{satz:lokal:multiplikativesmass} sehen werden.
		\end{enumerate}
	\end{bsp}
	
	Diese Beispiele sind keine Ausnahmen.
	Einer der Hauptgründe, warum wir uns mit lokalkompakten Gruppen beschäftigen, ist der folgende Satz.
	\begin{satz}[Existenz und Eindeutigkeit des Haar-Maßes]
	\label{satz:topo:haarmeasure}
		Auf jeder lokalkompakten Gruppe $G$ existiert ein linksinvariantes Haar-Maß. Dieses ist eindeutig bis auf skalares Vielfaches.
	\end{satz}
	\begin{proof}
		Der etwas längere Beweis befindet sich in \textcite{rama} Theorem 1-8 und benutzt den oben vorgestellten Rieszschen Darstellungssatz.
		%Ziel ist es, ein links-invariantes lineares Funktional auf $C_c(G)$ zu konstruieren.
	\end{proof}
	Dieser Satz stellt gewissermaßen die bestmögliche Situation dar, die man sich erhoffen kann. 
	Ist nämlich $\mu$ ein Haar-Maß und $c>0$ eine Konstante, so erhalten wir durch $c\cdot \mu$ wieder ein Haar-Maß. 
	Wir beenden diesen Abschnitt mit einem kleinen Lemma, welches uns einige bekannte Eigenschaften des Lebesgue-Maß auf Haar-Maße verallgemeinert.
	\begin{lemma}
		Sei $\mu$ ein Haar-Maß auf einer lokalkompakten abelschen Gruppe. 
		Für jede integrierbare Funktion $f\in L^{1}(G)$ gilt:
		\begin{multicols}{2}
			\begin{enumerate}[label=(\roman*)]
				\item $\int_{G} f(yx)d\mu(x) = \int_{G} f(x)d\mu(x)$.
				\item $\int_{G} f(x^{-1})d\mu(x) = \int_{G} f(x)d\mu(x)$.
			\end{enumerate}
		\end{multicols}
	\end{lemma}
	\begin{proof}
		(i) folgt leicht aus der Definition des Integrals und der Translationsinvarianz des Maßes, denn $\ind_A(yx) = \ind_{y^{-1}A}(x)$ und $\mu({y^{-1}A}) = \mu(A)$.
		
		Für (ii) überlegen wir uns zunächst, dass $\tilde{\mu}(E)\coloneqq \mu(E^{-1})$ ein weiteres Haar-Maß auf $G$ definiert. 
		Nach der Eindeutigkeit unterscheiden sich beide Maße nur um eine Konstante $c > 0$. Wir wollen zeigen, dass $c=1$ ist.
		Sei dazu $K$ eine kompakte Umgebung der $1$. Dann gibt es eine offene Umgebung $U$ der $1$ mit $U \subseteq K$. 
		Definieren wir $S \coloneqq KK^{-1}$, so ist $S$ kompakt, $U \subseteq S$ und es gilt $0 < \mu(U) \leq \mu(S)<\infty$. 
		Wir folgern 
		\begin{align*}
			c\cdot \mu(S) = \tilde{\mu}(S) = \mu(S^{-1}) =\mu(S)
		\end{align*}
		und damit $c=1$. 
		Das Haar-Maß ist also invariant unter der Umkehrabbildung.
		Der Rest folgt dann aus der Definition des Integrals.
	\end{proof}
	
\subsection{Charaktere und Quasi-Charaktere}
	Im Rest dieser Arbeit beschränken wir uns auf abelsche lokalkompakte Gruppen.
	Um die Fouriertransformation auf diesen Gruppen definieren zu können, müssen wir uns erst mit Charakteren beschäftigen.
	Auch wenn wir sie in diesem Kontext nicht (explizit) brauchen, werden sie eine wichtige Rolle im Beweis der Funktionalgleichung spielen.
	\begin{defi}
		Ein \emph{Quasi-Charakter} einer topologischen abelschen Gruppe $G$ ist ein Homomorphismus topologischer Gruppen von $G$ in die multiplikative Gruppe $\Komplex^\times$ der komplexen Zahlen.
		Ein \emph{Charakter} ist ein Quasi-Charakter, dessen Bild auf dem komplexen Einheitskreis $S^1 =\{z\in\Komplex: \abs[z]=1\}$ liegt.
	\end{defi}
	In der Literatur werden Quasi-Charaktere häufig auch einfach nur als Charaktere bezeichnet. 
	Was wir in dieser Arbeit unter einem Charakter verstehen, wird dann unitärer Charakter genannt.
	Beginnen wir mit ein paar Beispielen.
	\begin{bsp}~
		\begin{enumerate}[label=(\alph*)]
			\item Für jede topologische Gruppe $G$ ist die Abbildung $g\mapsto 1$ ein Charakter, der sogenannte \emph{triviale Charakter}. 
				Er ist der einzige konstante Charakter, denn für jeden Gruppenhomomorphismus $\chi$ gilt bekanntlich $\chi(1) = 1$.
			\item Ein nicht-triviales Beispiel für einen Charakter ist die Abbildung $t \mapsto \exp(i t)$ von der additiven Gruppe $\R^+$ in den Einheitskreis $S^1$.
			\item Die bekannte Abbildung $\exp: \Komplex^+ \to \Komplex^\times$ von der additiven Gruppe in die multiplikative Gruppe der komplexen Zahlen ist ein Quasi-Charakter, jedoch kein Charakter.
		\end{enumerate}
	\end{bsp}
	
	Werfen wir einen Blick auf die Quasi-Charaktere kompakter Gruppen.
	\begin{lemma}\label{Lemma:trivialerCharAufKompakt}
		Sei $K$ eine kompakte Gruppe mit Haar-Maß $\dx$ und $\chi: K \to \Komplex^\times$ ein Quasi-Charakter. Dann gilt
		\begin{enumerate}[label=(\roman*)]
				\item $\chi$ ist bereits ein Charakter.
				\item Für das Integral von $\chi$ über $K$ gilt
					\begin{align*}
						\int_K \chi(x) \dx = 
							\begin{cases}
								\Vol(K, \dx),	& \text{falls } \chi \text{ trivial}\\
								0,					& \text{sonst.}
							\end{cases}
					\end{align*}
		\end{enumerate}
	\end{lemma}
	\begin{proof}
		(i) Sei $x$ ein beliebiges Element von $K$. 
		Sei $H$ der Abschluss der von $x$ erzeugten Untergruppe von $K$.
		Damit ist $H$ selbst wieder eine Untergruppe von $K$ und als abgeschlossene Teilmenge eines Kompaktums kompakt.
		Da $\chi$ ein stetiger Gruppenhomomorphismus ist, muss $\chi(H)$ eine kompakte Untergruppe von $\Komplex^\times$ sein.
		Diese liegen aber gerade alle auf $S^1$ und die Behauptung folgt.
		
		(ii) Der erste Fall ist klar. 
		Im zweiten Fall gibt es ein $x_0 \in K$ mit $\chi(x_0) \not=1$ und mit Translationsinvarianz daher
		\begin{align*}
			\int_K \chi(x)\dx = \int_K\chi(x_0x)\dx = \chi(x_0)\int_K\chi(x)\dx.
		\end{align*}
		Umstellen und Division durch $\chi(x_0) - 1$ ergibt $\int_K \chi(x)dx = 0$.
	\end{proof}
	Im weiteren Verlauf dieser Arbeit werden wir Quasi-Charaktere noch gründlicher untersuchen.
	
\subsection{Ausblick: Pontryagin-Dualität}
	Zum Ende dieses Kapitels blicken wir etwas über den Tellerrand hinaus.
	Dieser Abschnitt ist für das Verständnis der kommenden Beweise der Arbeit absolut optional, bietet aber Einblick in die nötigen Abstraktionen, die in Tates eigener Argumentation zum tragen kamen.
	Wir werden daher die für Tates Doktorarbeit wichtigsten Grundlagen der abstrakten harmonischen Analysis nur anschneiden und versprechen in den späteren Kapiteln zu zeigen, dass unsere eigenen Definitionen mit denen in diesem Abschnitt verträglich sind.
	
	Beginnen wir mit der Charaktergruppe auf einer beliebigen topologischen Gruppe $G$.
	Die Charaktere auf $G$ werden mit punktweiser Multiplikation $(\chi\psi) (x) \coloneqq \chi(x) \psi (x)$ selbst wieder eine Gruppe, welche als die \emph{duale Gruppe $\hat{G}$} bezeichnet wird.
	Wir statten $\hat{G}$ mit einer Topologie aus.
	Sei $K$ eine kompakte Teilmenge von $G$ und $V\subseteq S^1$ eine Umgebung der $1\in S^1$.
	Dann wird durch die Mengen
	\begin{align*}
		W(K, V) = \{ \chi\in \hat{G}: \chi(K) \subseteq V \}
	\end{align*}
	eine Umgebungsbasis des trivialen Charakters definiert und induziert damit eine Topologie: die sogenannte \emph{kompakt-offen Topologie}.
	Damit wird $\hat{G}$ zu einer topologischen Gruppe.
	Es ergibt sich folgender Satz.
	\begin{satz} 
		Sei $G$ eine abelsche topologische Gruppe. 
		Dann gelten die folgenden Aussagen:
		\begin{enumerate}[label=(\roman*)]
			\item Ist $G$ diskret, so ist $\hat{G}$ kompakt.
			\item Ist $G$ kompakt, so ist $\hat{G}$ diskret.
			\item Ist $G$ lokalkompakt, so ist auch $\hat{G}$ lokalkompakt.
		\end{enumerate}
	\end{satz}
	\begin{proof}
		Siehe \textcite{rama} Proposition 3-2.
	\end{proof}
	Aussage (iii) sieht verdächtig aus.
	Eine erste Vermutung wäre, dass $G$ isomorph zu $\hat{G}$ ist.
	Dieser Verdacht stimmt im Allgemeinen leider nicht. 
	Dafür gilt aber das nächstbeste Ergebnis.
	\begin{satz}[Pontryagin Dualität]
		Jede lokalkompakte Gruppe $G$ ist kanonisch isomorph zu ihrem Doppel-Dual $\hat{\hat{G}}$.
		Der Isomorphismus topologischer Gruppen $\alpha$ ist durch die Auswertungsabbildung $\alpha(y)(\chi) = \chi(y)$ gegeben.
	\end{satz}
	\begin{proof}
		Der etwas längere Beweis ist zum Beispiel zu finden in \textcite{rama} Theorem 3-20.
	\end{proof}
	
	F"ur diesen Beweis wird ein klassisches Konzept verallgemeinert.
	Wir wissen bereits, dass jede abelsche lokalkompakte Gruppe $G$ ein eindeutiges Haar-Maß $\dx$ besitzt.
	Damit können wir auf $G$ integrieren und eine abstrakte Variante der klassischen Fouriertransformation definieren.
	\begin{defi}[Abstrakte Fouriertransformation]
		Sei $f\in L^1(G)$. 
		Wir definieren die \emph{Fouriertransformation} $\hat{f}: \hat{G} \to \Komplex$ von $f$ durch die Formel
		\begin{align*}
			\hat{f}(\chi) = \int_G f(x)\conj{\chi(x)} \dx
		\end{align*}
		für alle $\chi\in \hat{G}$.
	\end{defi}
	Diese Formel ist wohldefiniert, denn für alle $x \in G$ hat $\chi(x)$ den Betrag $1$. 
	Ist also $f$ integrierbar, so ist es auch das Produkt im Integranden.
	Da die Gruppe $\hat{G}$ selbst wieder lokalkompakt ist, besitzt sie ein Haar-Maß $\dx[\chi]$ und es macht Sinn, die Fouriertransformation auf $\hat{G}$ betrachten.
	Durch geeignete Normierung der verwendeten Maße gelangen wir zu folgendem Satz, der unter Anderem in Tates Beweis der verallgemeinerten Poisson Summenformel, eine wichtige Rolle spielt.
	\begin{satz}[Umkehrformel für lokalkompakte Gruppen]\label{satz:topogroup:umkehrformel}
		Es gibt ein Haar-Maß $\dx[\chi]$ auf $\hat{G}$, so dass alle $f \in L^1(G)$ stetig mit $\hat{f} \in L^1(\hat{G})$, die Gleichung
		\begin{align*}
			f(x) = \int_{\hat{G}} \hat{f}(\chi) \chi(x) \dx[\chi]
		\end{align*}
		erfüllen. 
		Insbesondere ist also $\hat{\hat{f}}(x) = f(-x)$.
	\end{satz}
	\begin{proof}
		Siehe \textcite{rama} Theorem 3-9.
	\end{proof}

	
	
