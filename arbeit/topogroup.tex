\section{Lokalkompakte Gruppen}
%topologische Gruppe: done
%charaktere:partly
%haar masse: done
%fouriertransformation:done
%pontryagin erwaehnen: done
\begin{defi}
	Eine \highlight{topologische Gruppe} ist eine Gruppe $G$ zusammen mit einer Topologie, welche die folgenden Eigenschaften erf"ullt:
		\begin{enumerate}[label=(\roman*), leftmargin=*, align=left, labelsep=1pt]
			\item Die Gruppenoperation
				\begin{align*}
					G \times G &\to G\\
					(g,h) &\mapsto gh
				\end{align*}
			stetig auf der Produkttopologie von $G \times G$.
			\item Die Umkehrabbildung
				\begin{align*}
					G &\to G\\
					g &\mapsto g^{-1}
				\end{align*}
				ist stetig.
		\end{enumerate}
	Ein \emph{Homomorphismus topologischer Gruppen} $G_1$ und $G_2$ ist eine stetiger Gruppenhomomorphismus $G_1 \to G_2$.
	Ist dieser bijektiv und die Umkehrabbildung wieder ein Homomorphismus topologischer Gruppen, so sprechen wir von einem \emph{Isomorphismus topologischer Gruppen}.
\end{defi}
	
	Man sieht sofort, dass die Translationen um ein beliebiges Gruppenelement ein Homeomorphismus $G \to G$ bilden.
	Die Topologie ist also \emph{translationsinvariant} in dem Sinne, dass f"ur alle $g \in G$ und jede Mengen $U$ die "Aquivalenzen
	\begin{align*}
		U \text{ ist offen} \Leftrightarrow gU = \{gu \in G: u\in U\} \text{ ist offen} \Leftrightarrow Ug = \{ug \in G: u\in U\} \text{ ist offen}
	\end{align*}
	gelten.
	Analog verh"alt es sich f"ur die Umkehrabbildung. Sie ist ein Homeomorphismus und $U$ ist genau dann offen, wenn $U^{-1}=\{u^{-1}\in G: u \in U\}$ offen ist.
	
	Die Translationsinvarianz der Topologie hat nun einige nette Vorteile.
	Zum Beispiel wird bereits die ganze Topologie durch die Umgebungsbasis des neutralen Elements definiert.
	Durch Translation erhalten wir Umgebungsbasen beliebiger anderer Elemente und damit zwangsl"aufig die komplette Topologie.
	F"ur ein weiteres Beispiel erinnern wir uns an die Definition der Stetigkeit in topologischen R"aumen.
	Eine Abbildung $f: G_1 \to G_2$ zwischen zweier topologischer R"aume heißt stetig, wenn f"ur alle $g\in G_1$ und jede offene Umgebung $U$ von $f(g)$ eine offene Umgebung $V$ von $g$ existiert, sodass $f(V)\subseteq U$.
	Sind $G_1$ und $G_2$ nun topologische Gruppen und ist $f$ ein (nicht unbedingt topologischer) Gruppenhomomorphismus reicht es jetzt die Stetigkeit in dem neutralen Element $e_1$ der Gruppe $G_1$ nachzuweisen.
	Denn ist $f$ stetig in $e_1$ und sei $g$ ein beliebiges weiteres Element der Gruppe, so ist f"ur jede offene Umgebung U von $f(g)$ ist $U':=f(g)^{-1}U$ eine offene Umgebung des neutralen Elements $e_2$. 
	Wegen Stetigkeit gibt es dann eine offene Umgebung $V'$ mit $f(V') \subseteq U'$.
	Nun ist aber $V:= gV'$ eine offene Umgebung von $g$ und da $f$ ein Gruppenhomomorphismus ist, haben wir
	\begin{align*}
		f(V) = f(gV') = f(g)f(V') \subseteq f(g) U' =f(g)f(g)^{-1} U = U,
	\end{align*}
	also ist die Abbildung stetig.
	
	Haben wir wieder zwei topologische Gruppen $G_1$ und $G_1$, so ist deren direktes Produkt $G_1\times G_2$, wie man leicht sieht, wieder eine topologische Gruppe.
	Dieses Ergebnis l"asst sich nuch auf endliche, abz"ahlbare und sogar beliebige Indexmengen "ubertragen. 
	\begin{lemma}
	\label{lemma:direktesProduktTopologischerGruppen}
		Sei $I$ eine Indexmenge und $G_i$ eine topologische Gruppe f"ur alle $i \in I$. 
		Das direkte Produkt $G = \prod_{i \in I} G_i$ versehen mit der Produkttopologie ist und komponentenweiser Gruppenverkn"upfung ist wieder eine topologische Gruppe.
	\end{lemma}
	\begin{proof}
		Wir erinnern uns daran, dass eine Basis der Produkttopologie gegeben ist durch Rechtecke der Form
		\begin{align*}
			\prod_{i \in E} U_i \times \prod_{i \in I\setminus E}  G_i,
		\end{align*}
		wobei $E$ eine endliche Teilmenge von $I$ und jedes $U_i$ offen in $G_i$ ist. 
		Ohne Einschr"ankung sei also 
		\begin{align*}
			W = \prod_{i \in E} W_i \times \prod_{i \in I\setminus E}  G_i
		\end{align*}
		eine offene Umgebung von $gh = (g_i h_i)$. 
		Da die $G_i$ topologische Gruppen sind, finden wir f"ur  alle $i\in E$ offene Umgebungen $U_i$ und $V_i$ von $g_i$ und $h_i$,  sodass  $U_i V_i \subseteq W_i$. Wir behaupten nun, dass
		\begin{align*}
			(\prod_{i \in E} U_i \times \prod_{i \in I\setminus E}  G_i) \times (\prod_{i \in E} V_i \times \prod_{i \in I\setminus E}  G_i) \subseteq G \times G
		\end{align*}
		eine offene Umgebung von $(g, h) \in G \times G$ ist, deren Bild in $W$ liegt. 
		Der erste Aussage ist klar, beide Faktoren des Produkts sind offene Basiselemente der Topologie und enthalten jeweils $g$ bzw. $h$.
		Weiter ist das Bild unter Gruppenoperation gegeben durch
		\begin{align*}
			\prod_{i \in E} U_i V_i \times \prod_{i \in I\setminus E}  G_i,
		\end{align*}
		was nach obigen "Uberlegungen in $W$ liegt.
		Somit folgt die Stetigkeit der Gruppenverkn"upfung.
		Der Beweis f"ur die Stetigkeit der Umkehrabbildung funktioniert analog.
	\end{proof}
	
	\begin{defi}
		Ein topologischer Raum heißt \highlight{lokalkompakt}, wenn jeder Punkt des Raumes eine kompakte Umgebung hat. 
		Eine \highlight{lokalkompakte Gruppe} ist eine topologische Gruppe, die lokalkompakt und hausdorffsch ist. 
	\end{defi}
	
	\begin{lemma}
		Seien $G_1$ und $G_2$ zwei lokalkompakte Gruppen. Dann ist $G_1\times G_2$ wieder lokalkompakt. Insbesondere ist also jedes endliche direkte Produkt lokalkompakter Gruppen lokalkompakt.
	\end{lemma}
	\begin{proof}
		Sei $(g_1,g_2) \in G_1\times G_2$. Wegen der Lokalkompaktheit von $G_1$ und $G_2$ finden wir kompakte Umgebungen $K_1$, $K_2$ von $g_1$ bzw. $g_2$. Dann ist aber $K_1 \times K_2$ eine kompakte Umgebung von $(g_1,g_2)$. Weiter ist das direkte Produkt zweier Hausdorff-R"aume wieder hausdorffsch, wodurch $G_1\times G_2$ zu einer lokalkompakten Gruppe wird.
	\end{proof}
	Wie wir sp"ater in Lemma \ref{Lemma:lokalkompaktProd} sehen werden, kann diese Aussage nicht ohne Weiteres auf beliebig gro\ss e direkte Produkte "ubertragen werden.

\subsection{Das Haar-Maß}
	Nun zu etwas Maßtheorie. Wir beginnen mit einer kleinen Auffrischung der wichtigsten Objekte. 
	Eine \emph{$\sigma$-Algebra} auf einer Menge $X$ ist eine Teilmenge $\Omega$ von $P(X)$, so dass
	\begin{enumerate}[label=(\roman*)]
		\item $X \in \Omega$
		\item Wenn $A \in \Omega$, dann $A^c \in \Omega$, wobei hier $A^c := X\setminus A$ das Komplement von $A$ in $X$ notiert.
		\item $\Omega$ ist abgeschlossen unter abz"ahlbarer Vereinigung, d.h. $\bigcup_{k=0}^{\infty} A_k \in \Omega$, falls $A_k \in \Omega$ f"ur alle $k$.
	\end{enumerate}
	Die Elemente in $\Omega$ werden \emph{messbar} genannt. 
	Aus den Axiomen l"asst sich leicht folgern, dass die leere Menge und abz"ahlbare Schnitte von messbaren Mengen wiederum messbar sind. 
	Weiter ist der Schnitt $\bigcap_n \Omega_n$ beliebiger Familien $\{\Omega_n\}$ von $\sigma$-Algebren auf X selbst wieder eine $\sigma$-Algebra.
	
	
	Eine Menge $X$ zusammen mit einer $\sigma$-Algebra $\Omega$ bilden den \emph{messbaren Raum} $(X, \Omega)$. 
	Ist X ein topologischer Raum, so k"onnen wir die kleinste $\sigma$-Algebra $\Borel$ betrachten, die alle offenen Mengen von $X$ enth"alt. 
	Die Elemente von $\Borel$ werden \emph{Borelmengen} von $X$ genannt.
	
	
	Nun zum eigentlichen Messen der messbaren Mengen. 
	Ein \emph{Maß} auf einem beliebigen messbaren Raum $(X, \Omega)$ ist eine Funktion $\mu: \Omega \to [0, \infty]$ mit $\mu(\emptyset) = 0$ und die \emph{$\sigma$-additiv} ist. 
	Das bedeutet
	
	\begin{align*}
		\mu( \bigcup_{n=1}^{\infty}) = \sum_{n=1}^{\infty} \mu (A_n)
	\end{align*}
	f"ur beliebige Familien $\{A_n\}_1^\infty$ von paarweise disjunkten Mengen in $\Omega$.
	
	%%TODO: Borelmass
	
	Sei nun $\mu$ ein Borelmaß auf einem lokalkompakten hausdorffschen Raum X und sei $E$ ein eine beliebige Borelmenge von $X$.
	Wir nennen $\mu$ von \emph{innen regul"ar} auf E, falls
	\begin{align*}
		\mu(E) = \sup\{\mu(K): K \subseteq E, K \text{ kompakt}\}
	\end{align*}
	Umgekehrt heißt $\mu$ von \emph{außen regul"ar} auf E, wenn
	\begin{align*}
		\mu(E) = \inf\{\mu(U): E \subseteq U, U \text{ offen}\}.
	\end{align*}
	
	
	\begin{defi}
		Ein \emph{Radonmaß} auf $X$ ist ein Borelmaß, das endlich auf kompakten Mengen, von innen regul"ar auf allen offenen Mengen und von außen regul"ar auf allen Borelmengen ist.
	\end{defi}
	
	\begin{satz}[Rieszscher Darstellungssatz]
		Sei $I$ ein positives lineares Funktional auf dem Raum der stetigen Funktionen mit kompakten Trager $C_c(G)$. Dann gibt es ein eindeutiges Radonmaß $\mu$ auf $G$, so dass $I(f) = \int f d\mu$ f"ur alle $C_c(G)$.
	\end{satz}
	Dieser Satz ist ein wichtiger Grundstein f"ur viele Sätze "uber Radonmaße.
	Sind zum Beispiel $X$ und $Y$ zwei lokalkompakte Gruppen mit dazugeh"origen Radonmaßen $\mu$ und $\nu$ so ist im Allgemeinen das Produktmaß $\mu \times \nu$ kein Borel- und daher Radonmaß auf $X \times Y$.
	Wir definieren daher das Radonprodukt von $\mu$ und $\nu$ als das Radonmaß welches durch das positive Funktional $I(f) = \int f d(\mu \times \nu)$ nach Rieszschen Darstellungssatz gegeben wird.
	Dieses Produkt wird in Kapitel \ref{kapitel:RDP} eine Rolle spielen. F"ur unsere sp"ateren Berechnungen m"ussen wir uns allerdings keine Sorgen machen, denn erf"ullen $X$ und $Y$ das zweite Abzählbarkeitsaxiom so entspricht das Radonprodukt genau dem Produktmaß. F"ur eine ausf"urhliche Behandlung dieser Konzepte siehe Folland \cite{folland} Kapitel 7.
	
	
	Sei nun $G$ eine topologische Gruppe und $\mu$ ein Borelmaß auf G. 
	Wir k"onnen untersuchen, wie sich das Maß bez"uglich der Translation durch beliebige Gruppenelemente $g\in G$ verh"alt. 
	Gilt $\mu(gE) = \mu(E)$ f"ur jede Borelmenge, so nennen wir $\mu$ \emph{linksinvariant}.
	Analog heißt $\mu$ \emph{rechtsinvariant}, falls $\mu(Eg) = \mu(E)$.
	Diese beiden Begriffe fallen nat"urlich zusammen, wenn $G$ abelsch ist. 
	
	Nun haben wir alle wichtigen Konzepte zusammen f"ur folgende wichtige
	\begin{defi}
		Sei $G$ eine lokalkompakte Gruppe. 
		Ein \emph{linkes} (beziehungsweise \emph{rechtes}) \emph{Haar-Maß} auf $G$ ist ein linksinvariantes (beziehungsweise rechtsinvariantes) Radon-Maß, das auf nichtleeren offenen Mengen positiv ist. 
	\end{defi}
	\begin{bsp}~ 
		\begin{enumerate}[label=(\roman*)]
			\item Ist $G$ eine diskrete Gruppe, dann ist das Z"ahlmaß ein Haar-Maß.
			\item F"ur $G=\R^+$ definiert das Lebesgue-Maß dx ein Haarsches Maß.
			\item F"ur $G=\R^\times$ wird durch $\mu(E):=\int_{\R^\times} \ind_{E} \frac{1}{\abs[x]}dx$ ein Haar-Maß, wie wir in Satz \ref{satz:lokal:multiplikativesmass}sehen werden.
		\end{enumerate}
	\end{bsp}
	
	Einer der Hauptgr"unde, warum wir uns mit lokalkompakten Gruppen besch"aftigen ist folgender Satz.
	\begin{satz}[Existenz und Eindeutigkeit des Haar-Maß]
	\label{satz:topo:haarmeasure}
		Sei $G$ eine lokalkompakte Gruppe. Dann existiert ein linksinvariantes Haar-Maß auf $G$. Dieses ist eindeutig bis auf skalares Vielfaches.
	\end{satz}
	\begin{proof}
		Einen ausf"urhlichen Beweis befindet sich in \cite{rama} Kapitel 1.
	\end{proof}
	Dies ist die bestm"ogliche Situation die man erwarten kann, denn offensichtlich ist f"ur jedes Haar-Maß $\mu$ und eine Konstante $c>0$ das Maß $c\cdot \mu$ wieder ein Haar-Maß.
	Da wir nun auf jeder lokalkompakten Gruppe ein sch"ones Maß haben k"onnen wir auch etwas Integrationstheorie betreiben.
	
	
	\begin{lemma}F"ur jede Abbildung $f\in L^{1}(G,\mu)$ gilt
		\begin{enumerate}[label=\emph{(\alph*)}]
			\item $\int_{G} f(zx)d\mu(x) = \int_{G} f(x)d\mu(x)$
			\item $\int_{G} f(x^{-1})d\mu(x) = \int_{G} f(x)d\mu(x)$
		\end{enumerate}
	\end{lemma}
	\begin{proof}
		(a) folgt leicht aus der Definition des Integrals und der Translationsinvarianz des Maßes.
		
		F"ur (b) "uberlegen wir uns zun"achst, dass $\tilde{\mu}(E):= \mu(E^{-1})$ ein weiteres Haar Maß auf $G$ definiert. Nach der Eindeutigkeit unterscheiden sich beide Maße nur um eine Konstante $c > 0$. Wir wollen zeigen, dass $c=1$ ist.
		Sei dazu $K$ eine kompakte Umgebung der $1$. Dann gibt es eine offene Umgebung $U$ der $1$ mit $G \subseteq K$. Definieren wir nun $S := KK^{-1}$, so ist $S$ kompakt, $U \subseteq S$ und es gilt $0 < \mu(U) \leq \mu(S)<\infty$. 
		Es folgt $ c\cdot \mu(S) = \tilde{\mu}(S) = \mu(S^{-1}) =\mu(S)$ und damit $c=1$. Das Haar-Maß ist also invariant unter der Umkehrabbildung. Der Rest folgt dann aus der Definition des Integrals.
	\end{proof}
	
	
\subsection{Charaktere}
	\begin{defi}
		Ein \emph{Charakter} einer topologischen Gruppe $G$ ist ein stetiger Gruppenhomomorphismus von $G$ in die multiplikative Gruppe $\C^\times$ der komplexen Zahlen.
		Ein Charakter heißt \emph{unit"ar}, wenn dessen Bild auf dem Einheitskreis $S^1 =\{z\in\C: \abs[z]=1\}$ liegt, und \emph{trivial}, wenn er konstant ist.
	\end{defi}
	In der Literatur werden Charaktere auch h"aufiger \emph{Quasi-Charaktere} genannt. Unit"are Charaktere verlieren ihren Zusatz und werden nur Charakter genannt.
	Beginnen wir mit ein paar Beispielen.
	\begin{bsp}~
		\begin{itemize}
			\item F"ur jede topologische Gruppe $G$ ist die Abbildung $g\mapsto 1$ ist ein trivialer (unit"arer) Charakter. 
				Er ist sogar der einzige triviale Charakter, denn f"ur jeden Gruppenhomomorphismus $\chi$ gilt $\chi(1) = 1$.
			\item Ein nicht-triviales  Beispiel f"ur einen unit"aren Charakter ist die Abbildung $t \mapsto \exp(i t)$ von der additiven Gruppe $\R^+$ in den Einheitskreis $S^1$.
			\item  Die bekannte Abbildung $\exp: \C^+ \to \C^\times$ von der additiven Gruppe in die multiplikative Gruppe der komplexen Zahlen ist ein Charakter, jedoch nicht unit"ar.
			\item Im $p$-adischen Fall haben wir mit $x \mapsto \abs[x]_p^s$ einen sogenannten \emph{multiplikativen Charakter}, d.h. einen Charakter auf der multiplikativen Gruppe $\Kpx$.
		\end{itemize}
	\end{bsp}
	
	\begin{lemma}\label{Lemma:trivialerCharAufKompakt}
		Sei $K$ eine kompakte Gruppe mit Haar-Maß dx und $\chi: K \to S^1$ ein Charakter. Dann gilt
		\begin{align*}
			\int_K \chi(x) dx = 
				\begin{cases}
					\text{\emph{Vol}}(K, dx),	& \text{\emph{falls} } \chi\equiv 1\\
					0,					& \text{\emph{ansonsten.}}
				\end{cases}
		\end{align*}
	\end{lemma}
	\begin{proof}
		Der erste Fall ist klar. Im zweiten Fall gibt es ein $x_0 \in K$ mit $\chi(x_0) \not=1$ und mit Translationsinvarianz daher
		\begin{align*}
			\int_K \chi(x)dx = \int_K\chi(x_0x)dx = \chi(x_0)\int_K\chi(x)dx.
		\end{align*}
		Umstellen und Division durch $\chi(x_0) - 1 \not=0$ ergibt $\int_K \chi(x)dx = 0$.
	\end{proof}
	
	
\subsection{Abstrakte Fourieranalysis}
	 Die unit"aren Charaktere einer topologischen Gruppe $G$ bilden mit der punktweisen Multiplikation $\chi\psi (x) = \chi(x) \psi (x)$ selbst wieder eine Gruppe, welche als die \emph{duale Gruppe $\hat{G}$} bezeichnet wird.
	Ausgestattet mit der kompakt-offen Topologie wird diese selbst wieder zu einer topologischen Gruppe und man kann sogar zeigen, dass $\hat{G}$ genau dann lokalkompakt ist, wenn $G$ selber lokalkompakt ist.
	\begin{defi}[Fouriertransformation]
		Sei $f\in L^1(G)$. Wir definieren dann die \emph{Fouriertransformation} $\hat{f}: \hat{G} \to \C$ von $f$ durch die Formel
		\begin{align*}
			\hat{f}(\chi) = \int_G f(x)\conj{\chi}(x) dx
		\end{align*}
		f"ur alle $\chi \in \hat{G}$.
	\end{defi}
	Diese Formel macht Sinn, denn f"ur alle $x \in G$ hat $\chi(x)$ den Betrag $1$. 
	Ist also $f$ integrierbar, so ist es auch das Produkt im Integranden.
	Da $\hat{G}$ selber wieder lokalkompakt ist, besitzt das Duale ein Haar-Maß und es macht Sinn die Fouriertransformation auf $\hat{G}$ betrachten.
	Dabei stoßen wir auf folgenden Satz der f"ur den Tates Beweis der adelischen Poisson-Summenformel eine wichtige Rolle spielt.
	Wir  werden ihn auf etwas niedrigerem Abstraktionsgrad in Kapitel TODO beweisen.
	\begin{satz}[Fourier-Umkehrformel f"ur lokalkompakte Gruppen]
		Es gibt ein Haar-Maß $\dx[\chi]$ auf $\hat{G}$, so dass alle $f \in L^1(G)$ stetig mit $\hat{f} \in L^1(\hat{G})$, die Gleichung
		\begin{align*}
			f(x) = \int_{\hat{G}} \hat{f}(\chi) \chi(x) \dx[\chi]
		\end{align*}
		erf"ullen. Insbesondere ist also $\hat{\hat{f}}(x) = f(-x)$.
	\end{satz}
	\begin{proof}
		Siehe zum Beispiel \cite{folland} Kapitel 4, Satz 4.32 oder \cite{rama} Kapitel 3, Satz 3-9.
	\end{proof}

	\begin{satz}[Pontryagin Dualit"at]
		Jede lokalkompakte Gruppe $G$ ist kanonisch isomorph zu ihrem Doppel-Dual $\hat{\hat{G}}$ 
	\end{satz}
	
